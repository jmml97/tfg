%&tfgheader

\documentclass[
  a4paper,
  12pt,
  spanish,
  dvipsnames,
  footinclude,
  headinclude,
  %oneside,
]{scrbook}

\usepackage{expl3}
\usepackage{xparse}

%-------------------------------------------------------------------------------
%	ENTORNOS MATEMÁTICOS
%-------------------------------------------------------------------------------

\usepackage{amsmath, amsthm, amssymb}

\newtheoremstyle{theorem-style}  % Nombre del estilo
{\topsep}                                  % Espacio por encima
{\topsep}                                  % Espacio por debajo
{\itshape}                                  % Fuente del cuerpo
{0pt}                                  % Identación
{\scshape}                      % Fuente para la cabecera
{}                                 % Puntuación tras la cabecera
{5pt plus 1pt minus 1pt}                              % Espacio tras la cabecera
{\lsc{{\thmname{#1}\thmnumber{ #2}}.\thmnote{ (#3.)}}}  % Especificación de la cabecera
\theoremstyle{theorem-style}
\newtheorem{theorem}{Teorema}[section]
\newtheorem{corollary}[theorem]{Corolario}
\newtheorem{lemma}[theorem]{Lema}
\newtheorem{proposition}[theorem]{Proposición}
\newtheorem{question}{Pregunta}
\newtheorem{conjecture}[theorem]{Conjetura}
\newtheorem{remark}[theorem]{Nota}
\newtheoremstyle{definition-style}  % Nombre del estilo
{\topsep}                                  % Espacio por encima
{\topsep}                                  % Espacio por debajo
{}                                  % Fuente del cuerpo
{0pt}                                  % Identación
{}                      % Fuente para la cabecera
{.}                                 % Puntuación tras la cabecera
{5pt plus 1pt minus 1pt}                              % Espacio tras la cabecera
{\lsc{{\thmname{#1}\thmnumber{ #2}}\thmnote{ (#3)}}}  % Especificación de la cabecera
\theoremstyle{definition-style}
\newtheorem{definition}[theorem]{Definición}
\newtheorem{example}[theorem]{Ejemplo}
\newtheorem{notation}[theorem]{Notación}
\newtheorem{exercise}[theorem]{Ejercicio}

% Matemáticas

\usepackage{mathtools}
\usepackage{commath}

% Enlaces y colores

\usepackage{hyperref}
\usepackage{xcolor}
\hypersetup{
  colorlinks=true,
  citecolor=,
}

% Otros elementos de página

\usepackage[inline]{enumitem}
\setlist[itemize]{ noitemsep, leftmargin=*}
\setlist[enumerate]{noitemsep, leftmargin=*}

\usepackage{adjustbox}

\usepackage{multirow}

\DeclareRobustCommand{\textacr}[1]{\textls{\small #1}}

% Código

\usepackage{listings}
\lstset{
	basicstyle=\ttfamily,%
	breaklines=true,%
	captionpos=b,                    % sets the caption-position to bottom
  tabsize=2,	                   % sets default tabsize to 2 spaces
  frame=none,
  numbers=left,
  xleftmargin=18pt,
  stepnumber=1,
  aboveskip=12pt,
  showstringspaces=false,
  keywordstyle=\bfseries,
  commentstyle=\itshape,
  numberstyle=\scriptsize\bfseries
}
\renewcommand{\lstlistingname}{Listado}

%\endofdump

%-------------------------------------------------------------------------------
%	PAQUETES
%-------------------------------------------------------------------------------

% Idioma

\usepackage[es-noindentfirst, es-lcroman, es-tabla]{babel}
\spanishdashitems

% Citas de texto en línea/bloque

\usepackage[autostyle]{csquotes}

% Tikz

\usepackage{tikz}
\usetikzlibrary{babel, cd}
\usepackage{float}

% Bibliografía

\usepackage[sorting=none, style=apa, isbn=true]{biblatex}
\addbibresource{bibliografia.bib}

% Lorem ipsum

\usepackage{blindtext}

% Algoritmos

\usepackage[linesnumbered, onelanguage, ruled]{algorithm2e}
\SetAlCapFnt{\sffamily}
\SetAlCapNameFnt{\sffamily}
\newcommand\mycommentfont[1]{\sffamily\textcolor{darkgray}{#1}}
\SetCommentSty{mycommentfont}
%\SetAlFnt{\sffamily}
%\SetKwSty{sffamily}
\SetAlCapSkip{.5em}
\SetAlgoCaptionLayout{raggedright}
\SetAlgorithmName{Algoritmo}{Algoritmo}{Lista de algoritmos}
%\SetAlgoSkip{medskip}
\SetAlgoInsideSkip{medskip}

% Entorno algoritmo con líneas pero caption debajo
\makeatletter
\newenvironment{Ualgorithm}[1][htpb]{\def\@algocf@post@ruled{\color{gray}\hrule  height\algoheightrule\kern6pt\relax}%
\def\@algocf@capt@ruled{under}%
\setlength\algotitleheightrule{0pt}%
\SetAlgoCaptionLayout{raggedright}%
\begin{algorithm}[#1]}
{\end{algorithm}}
\makeatother

\makeatletter
\newcommand{\setalgotoprulecolor}[1]{\colorlet{toprulecolor}{#1}}
\let\old@algocf@pre@ruled\@algocf@pre@ruled % Adjust top rule colour
\renewcommand{\@algocf@pre@ruled}{\textcolor{toprulecolor}{\old@algocf@pre@ruled}}

\newcommand{\setalgobotrulecolor}[1]{\colorlet{bottomrulecolor}{#1}}
\let\old@algocf@post@ruled\@algocf@post@ruled % Adjust middle rule colour
\renewcommand{\@algocf@post@ruled}{\textcolor{bottomrulecolor}{\old@algocf@post@ruled}}

\newcommand{\setalgomidrulecolor}[1]{\colorlet{midrulecolor}{#1}}
\renewcommand{\algocf@caption@ruled}{%
  \box\algocf@capbox{\color{midrulecolor}\kern\interspacetitleruled\hrule
    width\algocf@ruledwidth height\algotitleheightrule depth0pt\kern\interspacealgoruled}}
\makeatother

\setalgotoprulecolor{gray}
\setalgomidrulecolor{gray}
\setalgobotrulecolor{gray}

%-------------------------------------------------------------------------------
%	NOTSOCLASSICTHESIS & FUENTES
%-------------------------------------------------------------------------------

\usepackage[
  drafting=true,
  %tocaligned=true,
  dottedtoc=true,
  parts,
  floatperchapter,
  pdfspacing,
  beramono=false,
  palatino=false,
]{notsoclassicthesis}

\setmainfont[%
Contextuals=Alternate,
SmallCapsFeatures={LetterSpace=1}]%
{EBGaramond}
\setsansfont[Scale=0.8]{Open Sans} 
\renewfontface\chapterNumber[Scale=7, Color=000000]{EBGaramond}
\setmonofont[Scale=0.75]{Go Mono}

\usepackage[math-style=TeX, bold-style=ISO]{unicode-math}
\setmathfont{Garamond Math}[StylisticSet={3}]

\usepackage{caption}
\captionsetup{font=sf, labelfont=bf, justification=raggedright,singlelinecheck=false}

\linespread{1.1}
\setlength{\parindent}{1.5em}

% Operadores

\let\deg\relax
\DeclareMathOperator{\deg}{gr}
\DeclareMathOperator{\lc}{cl}
\DeclareMathOperator{\rank}{rg}

%-------------------------------------------------------------------------------
%	TÍTULO
%-------------------------------------------------------------------------------

\title{Algoritmo de Peterson-Gorenstein-Zierler para códigos cíclicos sesgados}
\renewcommand{\myVersion}{0.8} 

\author{José María Martín Luque}

\date{\normalsize\today}

% Portada

\usepackage{eso-pic}
\newcommand\BackgroundPic{%
	\put(0,0){%
		\parbox[b][\paperheight]{\paperwidth}{%
			\vfill
			\centering
      % Indicar la imagen de fondo en el siguiente comando
			\includegraphics[width=\paperwidth,height=\paperheight,%
			keepaspectratio]{assets/portada-ugr-color}%
			\vfill
    }
  }
}

%-------------------------------------------------------------------------------
%	CONTENIDO
%-------------------------------------------------------------------------------

%\includeonly{teoria-codigos}

\begin{document}

%\maketitle

\begin{titlepage}
  \AddToShipoutPicture*{\BackgroundPic}
  \phantomsection
  \pdfbookmark[1]{Title}{title}

  % Para que el título esté centrado en la página.
  % Los valores numéricos deberán elegirse de acuerdo con el diseño de
  % página (sobre todo si se cambia la opción BCOR o DIV).
  \begin{addmargin}[2.05cm]{0cm}
  \begin{flushleft}
    \Large
    \hfill\vfil

    \vfill\vfill

    \textsf{Facultad de Ciencias}\\[.5em]
    \textsf{E.T.S. de Ingenierías Informática y de Telecomunicación} \\
    \vfill\vfill

    \textsf{\Large Doble Grado en Ingeniería Informática y Matemáticas} \vfill


    %\textsc{trabajo de fin de grado}
    \spacedlowsmallcaps{Trabajo de Fin de Grado}

    \begingroup
    \Huge{Algoritmo de Peterson-Gorenstein-Zierler para códigos cíclicos sesgados} \\ \bigskip
    \endgroup

    [\,\today\ a las \thistime\ -- \myVersion\,]

    \vfill\vfill\vfill\vfill

    \textsf{\normalsize{Presentado por}}\\[6pt]
    {\huge\textrm{José María Martín Luque}}

    \vspace*{0.7cm}

    \textsf{\normalsize{Tutorizado por}}\\[6pt]
    {\huge\textrm{Gabriel Navarro Garulo}}

    \vfill
    \textsf{Curso académico 2019--2020}
  \end{flushleft}
  \end{addmargin}

\end{titlepage}

\newpage

\chapter*{Resumen}

El objetivo de este trabajo es estudiar y presentar el algoritmo de Peterson-Gorenstein-Zierler para códigos cíclicos sesgados.  

\chapter*{Summary}

The main objetive of this project is to present, study and implement the Peterson-Gorenstein-Zierler algorithm for Skew Cyclic Codes.

\paragraph{Chapter 1} This chapter, sets the foundations of mathematic knowledge that's needed to understand the latter parts of the project. Skew Cyclic Codes require knowledge of two concepts: Ore Polynomial Rings and Cyclic Codes. On the one hand, Ore Polynomial Rings require the theory of Rings and Field Automorphisms. On the other hand, Cyclic Codes require the theory of Finite Fields and Rings. We introduce all the concepts needed.

\paragraph{Chapter 2} In this chapter, we introduce the concept of a code and the most studied class of codes: linear codes.
We also explain that linear codes over a finite field are just a vector subspace, so they conform to a mathematical structure that is well known and easy to study.
The elements in a code are called codewords.
We also use this chapter to introduce some concepts that we will refer to throughout the project and that are fundamental to the further development of coding theory.
To that extent, we present the definition of a generator matrix: a base of the vector subspace that a linear code ultimately is.
As we shall see, this matrix is fundamental in the most common procedure of message encoding.
Another important concept is the distance of a code, that is, the minimum number of coordinates that differentiate one codeword from another.
Finally, some simple families of codes are presented, such as repetition codes or Hamming codes.


\paragraph{Chapter 3} In this chapter, we shift our focus to the class of Cyclic Codes. 
As the name may suggest, these codes have the interesting property that cyclic shifts of codewords are also codewords of said code.
We'll see that codewords of these kinds of codes can be mapped to certain polynomials, and cyclic codes are simply the ideals of a polynomial ring quotient over the ideal generated by the polynomial \(x^n - 1\).
That's why we dedicate the first part of this chapter to the study of the factorization of this polynomial on polynomial rings over finite fields.
Once we know how to do this, we can properly describe all cyclic codes of any length.
We also show how to encode messages in cyclic codes.
An alternate way of describing cyclic codes is also explained, using what we call generating idempotents.
Finally, we explain the concept of zeroes of cyclic codes, a notion we'll need in the next chapter.

\paragraph{Chapter 4} In this chapter, we describe the family of BCH codes and the ``classic'' version of the Peterson-Gorenstein-Zierler algorithm for BCH codes.
We also take an opportunity to briefly introduce the family of RS codes, as we'll refer to them in later chapters.
To properly introduce BCH codes, we explain the BCH bound, a result that links the concept of zeroes of a cyclic code and the minimum distance of said code.
Then BCH codes are defined to take advantage of the BCH bound.
This means that it's possible to design BCH codes with any error correction capability, although their length will vary accordingly.
To end this chapter we take on the Peterson-Gorenstein-Zierler algorithm for BCH codes: we explain the decoding procedure and justify why it works.

\paragraph{Chapter 5} In this chapter, we move onto Ore Polynomial Rings.
Since we are working with finite fields with codes we'll limit our Ore Polynomial Rings study over them.
We introduce the main concepts as well as algorithms to calculate division from left or right and an extented Euclid algorithm.
We also state the fact that factorization of this kind of polynomials is not unique in the common sense, but there's a similar concept.

\paragraph{Chapter 6} In this chapter, we describe the family of Skew Cyclic Codes.
Throughout the chapter we set the basis for the introduction of Skew RS Codes at the end.
This is the family of codes that we will be using with the algorithm that is explained in the following chapter.

\paragraph{Chapter 7} Finally, in this chapter, we introduce the algorithm that is the main objective of this project: the Peterson-Gorenstein-Zierler algorithm for Skew Cyclic Codes.
As we did with the analogous algorithm for BCH codes we explain the decoding procedure as well as provide proof of why it works.

As part of this project we've developed various classes for Sage.
\begin{itemize}
  \item A decoder class for Sage's BCH codes that uses the Peterson-Gorenstein-Zierler algorithm described here.
  \item A skeleton class for Skew Cyclic Codes. That means we have not developed methods for this class but rather left a framework for other classes to be inherited by this one.
  \item A Skew RS Code class that implements the definition we study in this project, as well as a simple encoder class for them.
  \item Finally, a decoder class for the Skew RS Codes that we implemented that uses the Peterson-Gorenstein-Zierler algorithm for Skew Cyclic Codes that is the main goal of this project.
\end{itemize}

This classes allow us to work with these structures in Sage in a very natural form.
Throughout the project we will present some examples of the concepts we explain and we use Sage for that, either with the pre-existing classes or the classes we created.
We've also used some other helpful functions that we've described in annex \ref{annex:sage-gen-idemp}.

\paragraph{Keywords}
\begin{itemize*}[label=,itemsep=4em,itemjoin=\hspace{2em}]
  \item cyclic codes 
  \item skew polynomials
\end{itemize*}

{\hypersetup{hidelinks}
\tableofcontents
}

\newpage

\chapter*{Introducción}
 
El artículo de Claude Shannon \textit{Mathematical Theory of Cryptography} \parencite{shannon_mathematical_1945} y su ampliación posterior, \textit{Mathematical Theory of Communication} \parencite{shannon_mathematical_1948} dieron a luz a dos disciplinas hoy plenamente establecidas, la teoría de información y la teoría de códigos.
El objetivo principal de estas disciplinas es el de establecer mecanismos de comunicación que sean \emph{eficientes} y \emph{fiables} en ambientes posiblemente hostiles.
La eficiencia requiere que la transmisión de la información no necesite de demasiados recursos, sean materiales o temporales.
Por otro lado, la fiabilidad requiere que el mensaje recibido en una comunicación sea lo más parecido posible, dentro de unos márgenes de tolerancia, al mensaje original.
La teoría de la información se encarga del estudio tanto de la representación de la información como de la capacidad que tienen los sistemas para transmitir y procesar la información. 
Por otra parte, la teoría de códigos se basa en los resultados de la teoría de la información para el diseño y desarrollo de modelos de transmisión de información mediante herramientas algebraicas.

A pesar de que digamos que Shannon fue el padre de estas disciplinas, el problema de codificar la información no surge, ni mucho menos, entonces.
De hecho, el filósofo inglés Francis Bacon ya afirmó en el año 1623 que únicamente son necesarios dos símbolos para codificar toda la comunicación.
\blockquote[{\cite[30]{dyson_catedral_2015}}]{La transposición de dos letras en cinco emplazamientos bastará para dar 32 diferencias [y] por este arte se abre un camino por el que un hombre puede expresar y señalar las intenciones de su mente, a un lugar situado a cualquier distancia, mediante objetos ... capaces solo de una doble diferencia.}
Efectivamente, hoy en día la codificación binaria forma parte de prácticamente todos los aspectos nuestras vidas.

Sin embargo, el problema más relevante —y el que nos ocupa— es el de codificar la información para que se produzca una correcta transmisión y recepción de los mensajes a través de un canal.
Como parte de su teoría Shannon introdujo lo que posteriormente se conocería como \emph{modelo de comunicación} de Shannon, un esquema simplificado de cómo se produce la transmisión de información entre emisor y receptor a través de un canal.
\begin{figure}
  \includegraphics[width=\textwidth]{assets/shannon-communication-model.pdf}
  \caption*{Modelo de comunicación de Shannon}
\end{figure}
Al enviar la información a través del canal es posible que, debido al ruido que pueda haber en el mismo, se produzcan interferencias que alteren el mensaje enviado.
Por supuesto recibir un mensaje alterado no es deseable, pues es posible que la alteración sea tal que sea ilegible.
La teoría de códigos trata por tanto de diseñar estructuras algebraicas que permitan la corrección de errores por parte del receptor del mensaje.
Estas estructuras son los códigos, y son el objeto matemático sobre el que orbita todo este trabajo.


\section*{Enfoque}

El título de este trabajo es «Algoritmo de Peterson-Gorenstein-Zierler para códigos cíclicos sesgados», y como tal, expresa el objetivo último del mismo: estudiar y exponer este algoritmo.
En consecuencia, dirigiremos nuestra atención a los conceptos y herramientas necesarios para ello.
En cuanto a los fundamentos matemáticos necesarios, hablaremos de cuerpos finitos, anillos de polinomios sobre cuerpos finitos y anillos de polinomios de Ore sobre cuerpos finitos.
La teoría de códigos se centrará en la exposición de los códigos lineales y de los códigos cíclicos, así como de las familias de códigos concretas sobre las que trabajarán los algoritmos que vamos a describir.
Para comprender mejor el citado algoritmo estudiaremos también la versión original, de mitad del siglo pasado, para los conocidos como códigos BCH.
A lo largo del trabajo ilustraremos algunos ejemplos con el sistema algebraico computacional \href{https://www.sagemath.org/}{Sage}.

\section*{Motivación}

Los códigos cíclicos son muy utilizados porque son muy sencillos de implementar en sistemas digitales utilizando lo que se conoce como registros de desplazamiento.
El uso de anillos de polinomios de Ore nos permite obtener una mayor cantidad de códigos cíclicos.
El algoritmo Peterson-Gorenstein-Zierler para códigos BCH es interesante porque es el que, en general, tiene un menor coste computacional para un número de errores pequeño.
La adaptación para códigos cíclicos sesgados ofrece.

\section*{Objetivos}

Los objetivos de este trabajo son los siguientes. \begin{itemize}
  \item Estudio de las nociones básicas sobre Teoría de Códigos lineales.
  \item Estudio de las extensiones de Ore y de sus cocientes.
  \item Exponer el algoritmo de Peterson-Gorenstein-Zierler para códigos cíclicos sesgados.
  \item Implementación de sistemas de decodificación en Python usando Sage.
\end{itemize}

\section*{Esquema}

En los tres primeros capítulos introduciremos todos los fundamentos matemáticos necesarios, así como las definiciones y propiedades fundamentales de los códigos lineales y de los códigos cíclicos.
El capítulo cuarto ahondaremos en la familia de los códigos BCH y presentaremos el algoritmo original de Peterson-Gorenstein-Zierler para esta familia de códigos.
En el capítulo quinto explicamos las estructuras matemáticas que constituyen la base de los códigos cíclicos sesgados, los anillos de polinomios de Ore, para ya introducirlos propiamente en el capítulo sexto.
Finalmente, en el capítulo séptimo describimos el funcionamiento del algoritmo de Peterson-Gorenstein-Zierler para códigos cíclicos sesgados.





\chapter{Preliminares}

En este capítulo se detallan algunos conceptos básicos de Álgebra que son necesarios para la comprensión más adelante de la teoría de códigos. Según \parencite{cohn_algebra_1982} y \parencite{cohn_algebra_1989} y el libro de Jacobson.

\section{Monoides y grupos}

\begin{definition}
  Un \textit{monoide} es un conjunto \(S\) con un elemento \(e\) y una aplicación \(\mu: S^2 \to S\) tal que si \(\mu(x, y)\) es el resultado de aplicar \(\mu\) a la pareja de elementos \(x, y \in S\), se verifican: \begin{enumerate}
    \item \(\mu(x, \mu(y, z)) = \mu(\mu(x, y), z)\) para todo \(x, y, z \in S\).
    \item \(\mu(e, x) = \mu(x, e) = x\) para todo \(x \in S\).
  \end{enumerate}
\end{definition}

Obsérvese que, por definición, un monoide siempre tiene al menos un elemento. A la aplicación \(\mu\) que actúa sobre parejas de elementos se le llama \textit{operación binaria} y al elemento \(e\), elemento neutro de \(\mu\).

Un \textit{grupo} es un monoide en el que todo elemento tiene inverso.
Un grupo que además verifica la propiedad conmutativa es un \textit{grupo abeliano}.

\begin{definition}
  Un \textit{grupo} \(G\) es un conjunto junto a una operación binaria \(xy\) definida sobre él que verifica las siguientes propiedades.
  \begin{enumerate}
    \item Propiedad asociativa: \((xy)z = x(yz)\) para todo \(x, y, z \in G\).
    \item Existencia de elemento neutro \(1\): \(1x = x1 = x\) para todo \(x \in G\).
    \item Existencia de elemento inverso: cada elemento \(x\) tiene un inverso \(x^{-1}\) tal que \(xx^{-1} = x^{-1}x = 1\). 
  \end{enumerate}
\end{definition}

\section{Anillos}

\begin{definition}
  Un \textit{anillo} es un conjunto junto a dos operaciones: la suma \((+)\) y la multiplicación \((\cdot)\), que verifican las siguientes propiedades.
  \begin{itemize}[itemsep=0pt]
    \item Propiedad asociativa:
    \[(x+y)+z = x + (y+z), \qquad (xy)z = x(yz).\]
    \item Propiedad conmutativa para la suma:
    \[x + y = y + x.\]
    \item Existencia de elemento neutro:
    \[x + 0 = x, \qquad x1 = x.\]
    \item Existencia de elemento inverso para la suma:
    \[x + (-x) = 0.\]
    \item Propiedad distributiva para la multiplicación sobre la suma:
    \[x(y+z) = xy + xz.\]
  \end{itemize}

  Si un anillo verifica la propiedad conmutativa para la multiplicación, es decir, \(xy = yx\), se dice que es un \textit{anillo conmutativo}.
\end{definition}

Dos anillos son \textit{isomorfos} si hay un \textit{isomorfismo} entre ellos, es decir, existe una biyección que preserva todas las operaciones.
Decimos que los anillos isomorfos son idénticos, pues intrínsecamente son iguales.

En  cualquier anillo \(R\) se verifica que \(0x = x0 = 0\) para todo \(x \in R\), ya que \(x0 = x(0+0) = x0 + x0\), de donde concluimos que \(x0 = 0\) e igualmente, \(0x = 0\).

El \textit{anillo trivial} es aquel que solo tiene un elemento. 
Necesariamente entonces \(1 = 0\), pero este es el único caso en el que ocurre. 
Supongamos que \(1 = 0\). 
Entonces, para cada elemento \(x\) del anillo se tiene que \(x = x1 = x0 = 0\), luego tiene un único elemento.

% MAYBE: abreviar na = a + a + a (Cohn 137 [147])

Un elemento \(a\) de un anillo se dice que es \textit{invertible} si existe un elemento \(a'\) en el anillo tal que \(aa' = a'a = 1\).
A este elemento, que es único lo llamamos \textit{elemento inverso} de \(a\) y lo denotamos por \(a^{-1}\).
% TODO: por qué es único (anillo está formado por monoide multiplicativo de R)
El elemento \(0\) no puede tener inverso porque ya hemos visto que siempre que se multiplique por él se obtiene el \(0\).
Los anillos en los que todo elemento distinto de 0 es invertible se llaman \textit{anillos de división}.

% Dados dos anillos \(R\) y \(S\), decimos que \(S\) es un \textit{subanillo} de \(R\) si \(S\) está contenido en \(R\) y forma un anillo con las mismas operaciones que \(R\) (y con los mismos elemento neutro e identidad).

% CITEME: definciiones de Niederreiter (13)

\begin{definition}[Subanillo]
  Un subconjunto \(S\) de un anillo \(R\) se denomina \textit{subanillo} si \(S\) es cerrado bajo las operaciones de suma y producto de \(R\) y forma un anillo con estas operaciones.
\end{definition}

\begin{definition}[Ideal]
  Un subconjunto \(J\) de un anillo \(R\) se denomina \textit{ideal} si \(J\) es un subanillo de \(R\) y para todo \(a \in J\) y \(r \in R\) se verifica que \(ar \in J\) y \(ra \in J\).
\end{definition}

\begin{example}\hfill
  \begin{itemize}
    \item Sea \(R\) el cuerpo \(\mathbb Q\) de números racionales. Entonces, el conjunto \(\mathbb Z\) de los enteros es un subanillo de \(\mathbb Q\) pero no es un ideal porque, por ejemplo, \(1 \in \mathbb Z, 1/2 \in \mathbb Q\), pero \(1/2 \cdot 1 = 1/2 \notin \mathbb Z\). 
    \item Sea \(R\) un anillo conmutativo, \(a \in R\) y sea \(J = \{ra : r \in R\}\). Entonces, \(J\) es un ideal.
    % TODO: algún ejemplillo más?
  \end{itemize}
\end{example}

\begin{definition}
  Sea \(R\) un anillo conmutativo. Un ideal \(J\) de \(R\) se dice que es \textit{principal} si existe un elemento \(a \in R\) tal que \(J = (a) = \{ra : r \in R\}\).
\end{definition}

% MAYBE: si el anillo no es conmutativo, ideales por la izquierda y por la derecha.

Decimos que un ideal \(J\) es \textit{minimal} si no existe ningún otro ideal entre \(\{0\}\) y \(J\).

Dado un elemento del anillo distinto de cero, podemos clasificarlo en dos tipos. 
Sea \(a \neq 0\). 
Si existe \(b \neq 0\) tal que \(ab\) o \(ba\) es cero, entonces \(a\) es un elemento \textit{divisor de cero}, y en caso contrario, un elemento \textit{regular}.

Un anillo no trivial sin divisores de cero se dice que es \textit{entero}, un anillo entero conmutativo se denomina \textit{dominio de integridad}.

Una propiedad importante de los elementos regulares es la ley de cancelación.
\begin{proposition}
  Si \(c\) es un elemento regular de un anillo \(R\) entonces para cada \(a, b \in R\), tales que \(ca = cb\) o bien \(ac = bc\), se tiene que \(a = b\).
\end{proposition}

\begin{definition}
  Sea \(R\) un anillo. La \textit{característica} del anillo es el menor natural \(n\) tal que \(n1 = 0\).
  Si no existe tal número, la característica del anillo es \(0\).  
\end{definition}

\begin{definition}
  Un elemento \(e\) de un anillo tal que \(e^2 = e\) se dice que es \textit{idempotente}.
\end{definition}

\section{Cuerpos finitos}

\begin{definition}
  Un \textit{cuerpo} es un anillo de división conmutativo.
  Se dice que un cuerpo es \textit{finito} si tiene un número finito de elementos, al que llamamos \textit{orden} del cuerpo.
\end{definition}

% MAYBE: todo dominio de integridad finito es un cuerpo (Niederreiter, 12 [21])

Sea \(F\) un cuerpo. Un subconjunto \(K\) de \(F\) que es por sí mismo un cuerpo bajo las operaciones de \(F\) se denomina \textit{subcuerpo} de \(F\).
También podemos decir que \(F\) es una \textit{extensión} de \(K\).

Las extensiones de cuerpos se suelen notar de la forma \(F/K\).
Podemos decir que \(F\) es implícitamente un espacio vectorial sobre \(K\).
Esto es, los elementos de \(F\) pueden ser vistos como vectores sobre el cuerpo de escalares \(K\), con las operaciones de suma \(\alpha + \beta\), para \(\alpha, \beta \in F\) y multiplicación por escalares \(a\alpha\), para \(a \in K\) y \(\alpha \in F\), dadas por las propias operaciones de suma y multiplicación en \(E\).

De hecho todas las nociones que hemos definido para anillos (característica, ...) son válidas para los cuerpos, pues un cuerpo no deja de ser un anillo.

\begin{theorem}
  % Cohn Vol 2, 63 [74]
  \label{th:cuerpo-subcuerpo-primo-caracteristica}
  Todo cuerpo \(F\) tiene al menos un subcuerpo \(P\), el subcuerpo primo de \(F\) que está contenido en cada subcuerpo de \(F\).
  O bien \(F\) tiene característica 0 y \(P \cong \mathbb Q\) o bien \(F\) tiene característica \(p\), un número primo, y entonces \(P \cong \mathbb F_p\).
\end{theorem}


% CITEME ? Cohn Algebra Vol 2

Mención especial merecen los cuerpos finitos.

Un cuerpo con un número finito de elementos se denomina \textit{cuerpo finito} o \textit{cuerpo de Galois}, por su descubridor.

Sea \(V\) un espacio vectorial \(n\)-dimensional sobre \(\mathbb F_p\), el cuerpo de \(p\) elementos.
Si \(u_1, \dots, u_n\) es una base de \(V\), entonces cada elemento de \(V\) se escribe de forma única en la forma \(\sum \alpha_iu_i\), donde \(\alpha_i \in \mathbb F_p\).
Como cada coeficiente puede tener hasta \(p\) valores distintos, obtenemos un total de \(p^n\) elementos.

\begin{lemma}
  Un espacio vectorial \(n\)-dimensional sobre \(\mathbb F_p\) tiene \(p^n\) elementos.
\end{lemma}

Todo cuerpo finito \(F\) tiene claramente característica \(p\) en virtud del teorema \ref{th:cuerpo-subcuerpo-primo-caracteristica} y su subcuerpo primo es \(\mathbb F_p\).

\subsection{Anillos de polinomios sobre cuerpos finitos}

Para cualquier anillo \(R\) podemos definir un anillo de polinomios en \(x\) con coeficientes en \(R\).

Trabajaremos con anillos de polinomios en cuerpos finitos.

Denotamos el anillo de le los polinomios con coeficientes en \(\mathbb F_q\) por \(\mathbb F_q[x]\).
Es un anillo conmutativo con las operaciones habituales de suma y multiplicación de polinomios.
De hecho, es un dominio de integridad.

Un polinomio en \(\mathbb F_q[x]\) viene dado por \(f(x) = \sum_{i=0}^n a_ix^i\), donde \(a_i\) son los coeficientes del término de grado \(i\) y pertenecen a \(\mathbb F_q\).

El grado de un polinomio es el mayor grado de cualquier término con coeficiente distinto de cero.

\begin{proposition}
  Grado de sumas y productos
  % TODO: (Cohn, 149 [159]) (Ejercicio 158 de ECC)
\end{proposition}

\begin{proposition}
  \label{prop:raices-factores-pol-Fq}
  % TODO: Ejercicio 159 de ECC
\end{proposition}

El coeficiente del término de mayor grado se denomina \textit{coeficiente líder}.

Un polinomio es \textit{mónico} si su coeficiente líder es 1.
Sean \(f(x)\) y \(g(x)\) polinomios en \(\mathbb F_q[x]\).
Decimos que \(f(x)\) \textit{divide a} \(g(x)\), denotado por \(f(x) | g(x)\), si existe un polinomio \(h(x) \in \mathbb F_q[x]\) tal que \(g(x) = f(x)h(x)\).
El polinomio \(f(x)\) se llama \textit{divisor} o \textit{factor} de \(g(x)\).
El \textit{máximo común divisor} de \(f(x)\) y \(g(x)\), siendo al menos uno de ellos distinto de cero, es el polinomio mónico de \(\mathbb F_q[x]\) de mayor grado que divida tanto a \(f(x)\) como a \(g(x)\).
Lo denotamos por \(\operatorname{mcd}(f(x), g(x))\).
Dos polinomios son \textit{primos relativos} si su máximo común divisor es 1.

% TODO: polinomios irreducibles

% Algoritmo de división

\begin{theorem}[Algoritmo de división]
  \label{th:algoritmo-division}
  Sean \(f(x)\) y \(g(x)\) polinomios de \(\mathbb F_q[x]\), siendo \(g(x)\) distinto de cero.
  \begin{enumerate}
    \item \label{thi:algoritmo-division-division} Existen polinomios únicos, \(h(x)\), \(r(x) \in \mathbb F_q[x]\) tales que \[
      f(x) = g(x)h(x) + r(x), \quad \text{donde } \operatorname{gr} r(x) < \operatorname{gr} g(x) \text{ o } r(x) = 0. 
    \]
    \item \label{thi:algoritmo-division-mcd} Si \(f(x) = g(x)h(x) + r(x)\), entonces \[\operatorname{mcd}(f(x), g(x)) = \operatorname{mcd}(g(x), r(x)).\]
  \end{enumerate}
\end{theorem}

% TODO: algoritmo de euclides

Podemos utilizar el algoritmo de división del apartado \ref{thi:algoritmo-division-division} del teorema \ref{th:algoritmo-division} junto al apartado \ref{thi:algoritmo-division-mcd} de ese mismo teorema para hallar el máximo común divisor de dos polinomios.
Este procedimiento se conoce como \textit{algoritmo de Euclides} y es muy parecido a su homólogo para números enteros.

\begin{theorem}[Algoritmo de Euclides]
  Sean \(f(x)\) y \(g(x)\) dos polinomios en \(\mathbb F_q[x]\) con \(g(x)\) distinto de cero.
  \begin{enumerate}
    \item Realiza los siguientes pasos hasta que \(r_n(x) = 0\) para algún \(n\):
    \begin{align*}
      f(x) &= g(x)h_1(x) + r_1(x), \qquad \text{donde } \operatorname{gr} r_1(x) < \operatorname{gr} g(x),\\
      g(x) &= r_1(x)h_2(x) + r_2(x), \qquad \text{donde } \operatorname{gr} r_2(x) < \operatorname{gr} r_1(x),\\
      r_1(x) &= r_2(x)h_3(x) + r_3(x), \qquad \text{donde } \operatorname{gr} r_2(x) < \operatorname{gr} r_3(x),\\
        &\,\vdots \\
      r_{n-3}(x) &= r_{n-2}(x)h_{n-1}(x) + r_{n-1}(x), \ \text{donde } \operatorname{gr} r_{n-1}(x) < \operatorname{gr} r_{n-2}(x),\\
      r_{n-2}(x) &= r_{n-1}(x)h_n(x) + r_n(x), \qquad \text{donde } r_n(x) = 0.
    \end{align*} 
    Entonces, \(\operatorname{mcd}(f(x), g(x)) = cr_{n-1}(x)\), donde \(c \in \mathbb F_q\) se escoge para que \(cr_{n-1}(x)\) sea mónico.
    \item Existen polinomios \(a(x), b(x) \in \mathbb F_q[x]\) tales que 
    \[
      a(x)f(x) + b(x)g(x) = \operatorname{mcd}(f(x), g(x)).
    \]
  \end{enumerate}
\end{theorem}

La secuencia de pasos descrita termina porque en cada paso el grado del resto se reduce al menos en 1.

A continuación vamos a comentar un resultado que nos será útil en el futuro.

\begin{proposition}
  Sean \(f(x)\) y \(g(x)\) polinomios en \(\mathbb F_q[x]\).
  \begin{enumerate}
    \item Si \(k(x)\) es un divisor de \(f(x)\) y de \(g(x)\), entonces \(k(x)\) es divisor de \(a(x)f(x) + b(x)g(x)\) para todo \(a(x), b(x) \in \mathbb F_q[x]\).
    \item Si \(k(x)\) es un divisor de \(f(x)\) y de \(g(x)\) entonces \(k(x)\) es divisor de \(\operatorname{mcd}(f(x), g(x))\).
  \end{enumerate}
\end{proposition}

\begin{proof}
  % TODO
\end{proof}

% 3.3 Primitive elements

\subsection{Elementos primitivos}

El conjunto \(\mathbb F_q^*\) --de los elementos de \(\mathbb F_q\) distintos de cero-- es un grupo.

\begin{theorem}
  \label{th:Fq-ast-cilcico}
  Se verifican las siguientes afirmaciones.
  \begin{enumerate}
    \item El grupo \(\mathbb F_q^*\) es cíclico de orden \(q - 1\) con la multiplicación de \(\mathbb F_q\).
    \item Si \(\gamma\) es un generador de este grupo cíclico entonces
    \[
      \mathbb F_q = \{0, 1 = \gamma^0, \gamma, \gamma^2, \dots, \gamma^{q-2}\},
    \] y \(\gamma^i = 1\) si y solo si \((q-1) \mid i\).
  \end{enumerate}
\end{theorem}

\begin{proof}
  % TODO
\end{proof}

Cada generador \(\gamma\) de \(\mathbb F_q^*\) se llama \textit{elemento primitivo} de \(\mathbb F_q\).
Cuando los elementos distintos de cero de un cuerpo finito se expresan como potencias de \(\gamma\) podemos multiplicar de forma sencilla teniendo en cuenta que \(\gamma^i\gamma^j = \gamma^{i+j} = \gamma^s\), donde \(0 \leq s \leq q-2\) e \(i + j \equiv s \bmod q - 1\).

\begin{theorem}
  \label{th:el-Fq-raices-pol}
  Los elementos de \(\mathbb F_q\) son las raíces del polinomio \(x^q - x\).
\end{theorem}

\begin{proof}
  Sea \(\gamma\) un elemento primitivo de \(\mathbb F_q\).
  Entonces, \(\gamma^{q-1} = 1\) por definición.
  Por tanto, \((\gamma^i)^{q-1} = 1\) para todo \(i\) tal que \(0 \leq i \leq q - 2\).
  En consecuencia, los elementos de \(\mathbb F_q^*\) son las raíces de \(x^{q-1}-1 \in \mathbb F_p[x]\) y en consecuencia, de \(x^q - x\).
  Como \(0\) es raíz de \(x^q - x\), por la proposición \ref{prop:raices-factores-pol-Fq} sabemos que los elementos de \(\mathbb F_q\) son las raíces de \(x^q - x\), como queríamos.
\end{proof}

Un elemento \(\xi \in \mathbb F_q\) es una raíz enésima de la unidad si \(\xi^n = 1\), y es una raíz enésima primitiva de la unidad si además \(\xi^s \neq 1\) para todo s tal que \(0 < s < n\).
Un elemento primitivo \(\gamma\) de \(\mathbb F_q\) es por tanto una raíz \((q-1)\)-ésima de la unidad.
Se deduce del teorema \ref{th:Fq-ast-cilcico} que el cuerpo \(\mathbb F_q\) contiene una raíz enésima primitiva de la unidad si y solo si \(n \mid (q - 1)\), en cuyo caso \(\gamma^{(q-1)/n}\) es dicha raíz.

% 3.4 Constructing finite fields

\subsection{Construcción de cuerpos finitos}

Un polinomio no constante \(f(x) \in \mathbb F_q[x]\) es \textit{irreducible sobre} \(\mathbb F_q\) si no es posible factorizarlo como producto de dos polinomios de \(\mathbb F_q[x]\) de grado menor.

\begin{theorem}
  Sea \(f(x)\) un polinomio no constante. Entonces, 
  \[
    f(x) = p_1(x)^{a_1}p_2(x)^{a_2}\dots p_k(x)^{a_k},
  \]
  donde cada \(p_i(x)\) es irreducible, los polinomios \(p_i(x)\) son únicos salvo producto por escalares y los elementos \(a_i\) son únicos.
\end{theorem}

Esto nos dice que \(\mathbb F_q[x]\) es lo que se conoce como \textit{dominio de factorización única}.
Pero es además un dominio de ideales principales.

\begin{proof}
  % TODO: demostración de que F_q[x] es un dominio de ideales principales.
  % Ejercicio 153 (p. 102 [121]), F_q[x] es un anillo conmutativo con unidad y un dominio de integridad
  % Ejercicio 166 (p. 106 [125]), cada ideal de F_q[x] es un ideal principal
\end{proof}

Para construir un cuerpo de característica \(p\) comenzamos con un polinomio \(f(x) \in \mathbb F_q[x]\) que es irreducible sobre \(\mathbb F_q\) y que tiene grado \(m\).
Usando el algoritmo de euclides podemos demostrar que el anillo cociente dado por \(\mathbb F_p[x]/(f(x))\) es un cuerpo, y de hecho, un cuerpo finito con \(q = p^m\) elementos.

\begin{proof}
  % TODO: demostración de que F_p[x]/(f(x)) es un cuerpo de q = p^m elementos
  % Ejercicio 167 (p. 107 [126])
\end{proof}

Cada elemento de dicho anillo cociente es una clase lateral de la forma \(g(x) + (f(x))\), donde \(g(x)\) es único y tiene grado ocmo mucho \(m-1\).

Escribiremos cada clase lateral como un vector en \(\mathbb F_p^m\) siguiendo la correspondencia:
\[
  g_{m-1}x^{m-1} + g_{m-2}x^{m-2}+ \dots + g_1x + g_0 + (f(x)) \iff g_{m-1}g_{m-2}\dots g_1g_0.
\]

Esta notación de vectorial nos permite realizar la suma en el cuerpo utilizando la suma habitual de los vectores.
Multiplicar es una tarea a priori más complicada.
Para multiplicar \(g_1(x) + (f(x))\) por \(g_2(x) + (f(x))\) primero utilizamos al algoritmo de división para escribir
\[
  g_1(x)g_2(x) = f(x)h(x) + r(x),
\]
donde como sabemos o bien \(\operatorname{gr} r(x) \leq m -1\) o bien \(r(x) = 0\).
Puesto que estamos en el anillo cociente \(\mathbb F_p[x]/(f(x))\) nos queda
\[
  (g_1(x) + (f(x)))(g_2(x) + (f(x))) = r(x) + (f(x)).
\]
Esta notación es engorrosa, por lo que habitualmente operaremos en \(\alpha\) en vez de en \(x\) suponiendo que \(f(\alpha) = 0\).
Así, \(g_1(\alpha)g_2(\alpha) = r(\alpha)\).
% TODO: correspondencia vectorial de los polinomios en alpha (3.4)
En consecuencia, multiplicamos los polinomios en \(\alpha\) de la forma habitual y utilizamos la ecuación \(f(\alpha) = 0\) para reducir las potencias de \(\alpha\) de grado mayor a \(m-1\) a polinomios en \(\alpha\) de grado menor que \(m\).

El conjunto \(\{0\alpha^{m-1} + 0\alpha^{m-2} + \dots + 0\alpha + a_0 \mid a_0 \in \mathbb F_p\} = \{a_0 \mid a_0 \in \mathbb F_p\}\) es el subcuerpo primo de \(\mathbb F_q\).

Decimos que obtenemos \(\mathbb F_q\) a partir de \(\mathbb F_p\) yuxtaponiendo una raíz \(\alpha\) de \(f(x)\) a \(\mathbb F_p\).
Esta raíz viene dada formalmente por \(\alpha = x + (f(x))\) en el anillo cociente \(\mathbb F_p[x]/(f(x))\).
Por tanto, ya hemos visto antes que \(g(x) + (f(x)) = g(\alpha)\) y \(f(\alpha) = f(x + (f(x))) = f(x) + (f(x)) = 0 + (f(x))\).

Un polinomio irreducible sobre \(\mathbb F_p\) de grado \(m\) es \textit{primitivo} si tiene una raíz que es un elemento primitivo de \(\mathbb F_q = \mathbb F_{p^m}\).
Puede probarse que existen polinomios irreducibles de cualquier grado.
% TODO: probar? referencia prueba? mejorar afirmación?

\begin{theorem}
  Para cualquier primo \(p\) y cualquier entero positivo \(m\), existe un cuerpo finito, único salvo isomorfismos, con \(q = p^m\) elementos.
\end{theorem}

% 3.7 Cyclotomic cosets and minimal polynomials

\subsection{Clases ciclotómicas y polinomios minimales}

Sea \(\mathbb F_{q^t}/\mathbb F\) una extensión de cuerpos.
Por el teorema \ref{th:el-Fq-raices-pol} cada elemento de \(\mathbb F_{q^t}\) es raíz del polinomio \(x^{q^t} - x\).
Existe por tanto un polinomio mónico \(M_{\alpha}\) en \(\mathbb F-q[x]\) de grado mínimo que tiene a \(\alpha\) como raíz.
Este polinomio se conoce como \textit{polinomio minimal} de \(\alpha\) sobre \(\mathbb F_q\).
En el siguiente teorema vamos a estudiar algunas propiedades de los polinomios minimales.

\begin{theorem}
  Sea \(\mathbb F_{q^t}/\mathbb F\) una extensión de cuerpos y sea \(\alpha\) un elemento de \(\mathbb F_{q^t}\) cuyo polinomio minimal es \(M_{\alpha} \in \mathbb F_q[x]\).
  Se verifica:
  \begin{enumerate}
    \item El polinomio \(M_{\alpha}(x)\) es irreducible sobre \(\mathbb F_q\).
    \item Si \(g(x)\) es cualqueir polinomio en \(\mathbb F_q[x]\) tal que \(g(\alpha) = 0\) entonces \(M_{\alpha}(x) \mid g(x)\).
    \item El polinomio \(M_{\alpha}(x)\) es único.
  \end{enumerate}
\end{theorem}

\begin{proof}
  % TODO
\end{proof}

Si partimos de \(f(x)\), un polinomio irreducible sobre \(\mathbb F_q\) de grado \(r\), podemos considerar la extensión generada por una de las raíces de \(f(x)\) y obtendremos el cuerpo \(\mathbb F_{q^r}\).
De hecho, el siguiente teorema afirma que en ese caso todas las raíces de \(f(x)\) estarán en \(\mathbb F_{q^r}\).

\begin{theorem}
  Sea \(f(x)\) un polinomio irreducible mónico sobre \(\mathbb F_q\) de grado \(r\).
  Entonces:
  \begin{enumerate}
    \item Todas las raíces de \(f(x)\) están en \(\mathbb F_{q^r}\) y en cualquier extensión de cuerpos de \(\mathbb F_q\) generada por una de sus raíces.
    \item Podemos expresar \(f(x)\) como \(f(x) = \prod_{i=1}^r (x - \alpha_i)\), donde \(\alpha_i \in \mathbb F_{q^r}\) para \(1 \leq i \leq r\).
    \item El polinomio \(f(x)\) divide a \(x^{q^r} - x\).
  \end{enumerate}
\end{theorem}

\begin{proof}
  % TODO (ECC 113 [132] theorem 3.7.2)
\end{proof}

En particular este teorema se verifica para los polinomios minimales \(M_{\alpha}(x)\) sobre \(\mathbb F_q\), pues son mónicos irreducibles.

\begin{theorem}
  \label{th:prop-pol-minimal}
  Sea \(\mathbb F_{q^t}/\mathbb F_q\) una extensión de cuerpos y sea \(\alpha\) un elemento de \(\mathbb F_{q^t}\) con polinomio minimal \(M_{\alpha}\) en \(\mathbb F_q[x]\).
  Se verifican las siguientes afirmaciones.
  \begin{enumerate}
    \item El polinomio \(M_{\alpha}(x)\) divide a \(x^{q^t} - x\).
    \item El polinomio \(M_{\alpha}(x)\) tiene raíces distintas dos a dos, todas en \(\mathbb F_{q^t}\).
    \item El grado de \(M_{\alpha}(x)\) divide a \(t\).
    \item Podemos expresar \(x^{q^t}- x = \prod_{\alpha}M_{\alpha}(x)\), donde \(\alpha\) varía entre los elementos de un subconjunto de \(\mathbb F_{q^t}\) que enumera los polinomios minimales de todos los elementos de \(\mathbb F_{q^t}\) una sola vez.
    \item Podemos expresar \(x^{q^t}- x = \prod_{f}f(x)\), donde \(f\) varía entre todos los mónicos irreducibles sobre \(\mathbb F_q\) cuyo grado divide a \(t\).
  \end{enumerate}
\end{theorem}

\begin{proof}
  % TODO (ECC teorema 3.7.3, p 113 [132])
\end{proof}

Dos elementos de \(\mathbb F_{q^t}\) que tienen el mismo polinomio minimal en \(\mathbb F_q[x]\) se llaman \textit{conjugados sobre} \(\mathbb F_q\).
Es importante encontrar todos los conjugados de \(\alpha \in \mathbb F_q\), es decir, todas las raíces de \(M_{\alpha}(x)\).
Sabemos por el teorema \ref{th:prop-pol-minimal} que las raíces de \(M_{\alpha}(x)\) son todas distintas dos a dos y que se encuentran en el cuerpo \(\mathbb F_{q^t}\).
Podemos encontrar estas raíces con ayuda del siguiente teorema.

\begin{theorem}
  Sea \(f(x)\) un polinomio en \(\mathbb F_q[x]\) y sea \(\alpha\) una raíz de \(f(x)\) en una extensión \(\mathbb F_{q^t}/\mathbb F_q\).
  Entonces se verifican las siguientes afirmaciones.
  \begin{enumerate}
    \item Evaluando el polinomio obtenemos que \(f(x^q) = f(x)^q\).
    \item El elemento \(\alpha^q\) es también una raíz de \(f(x)\) en \(\mathbb F_q\).
  \end{enumerate}
\end{theorem}

\begin{proof}
  % TODO (ECC teorema 3.7.4, p114 [133])
\end{proof}

Si aplicamos este teorema de forma consecutiva podremos obtener todas las raíces de \(M_{\alpha}(x)\), que serán de la forma \(\alpha, \alpha^q, \alpha^{q^2},\)\,etc.; secuencia que terminará tras \(r\) términos, cuando \(\alpha^{q^r} = \alpha\).

Supongamos ahora que \(\gamma\) es un elemento primitivo de \(\mathbb F_{q^t}\).
Entonces sabemos que \(\alpha = \gamma^s\) para algún \(s\).
Por tanto, \(\alpha^{q^r} = \alpha\) si y solo si \(\gamma^{sq^r - s} = 1\).
Por el teorema \ref{th:Fq-ast-cilcico} se tiene que \(sq^r \equiv s \bmod q^t - 1\).
Basándonos en esta idea podemos definir la \textit{clase q-ciclotómica de s módulo} \(q^t - 1\) como el conjunto
\[
  C_s = \{s, sq, \dots, sq^{r-1}\} \bmod q^t - 1, 
\]
donde \(r\) es el menor entero positivo tal que \(sq^r \equiv s \bmod q^t - 1\).
Los conjuntos \(C_s\) dividen el conjunto de enteros \(\{0, 1, 2, \dots, q^t - 2\}\) en conjuntos disjuntos.

% TODO: justificación de este teorema

\begin{theorem}
  Si \(\gamma\) es un elemento primitivo de \(\mathbb F_{q^t}\) entonces el polinomio minimal de \(\gamma^s\) sobre \(\mathbb F_q\) es
  \[
    M_{\gamma^s}(x) = \prod_{i \in C_s}(x - \gamma^i).
  \]
\end{theorem}

\begin{proof}
  % TODO ECC theorem 3.7.6 p.115 [133]
\end{proof}

\section{Automorfismos}

\chapter{Fundamentos de teoría de códigos}

En la introducción ya hemos visto cuáles son los objetivos de la teoría de códigos, así como el medio principal del que se sirve: el álgebra.
Esta sección vamos a comentar algunos de los conceptos y resultados fundamentales de la teoría de códigos.
Daremos la definición más sencilla de código para posteriormente estudiar otras estructuras más complejas, los códigos lineales y los códigos cíclicos, así como resultados básicos asociados a ellos.
Finalmente, estudiaremos la versión original del algoritmo de Peterson-Gorenstein-Zierler, diseñado para un tipo de códigos, los \textacr{BCH}.

Las definiciones y los resultados comentados en esta sección seguirán lo descrito en \parencite[cap. 1, 3-5]{huffman_fundamentals_2003} y \parencite{podesta_introduccion_2006}.

\section{Códigos lineales}

Vamos a comenzar nuestro estudio con los códigos lineales, pues son los más sencillos de comprender. 
Consideremos el espacio vectorial de todas las \(n\)-tuplas sobre el cuerpo finito \(\mathbb F_q\), al que denotaremos en lo que sigue como \(\mathbb F_q^n\). 
A los elementos \((a_1, \dots, a_n)\) de \(\mathbb F_q^n\) los notaremos usualmente como \(a_1\!\cdots a_n\).

\begin{definition}
  Un \((n, M)\) \textit{código} \(\mathcal C\) sobre el cuerpo \(\mathbb F_q\) es un subconjunto de \(\mathbb F_q^n\) de tamaño \(M\). 
  Si no hay riesgo de confusión lo denotaremos simplemente por \(\mathcal C\). 
  A los elementos de \(\mathcal C\) los llamaremos \textit{palabras codificadas}, \textit{palabras código} o \textit{codewords} en inglés.
  A \(n\) se le llama \(longitud\) del código.
\end{definition}

Por ejemplo, un \((5,4)\) código sobre \(\mathbb F_2\) puede ser el formado por los siguientes elementos: \[
  10101,\qquad
  10010,\qquad
  01110,\qquad
  11111.
\]
Como se puede ver realmente un código es un objeto muy sencillo.
Concluimos que es necesario añadir más estructura a los códigos para que puedan ser de utilidad.
Esto motiva la siguiente definición.

\begin{definition}
  Decimos que un código \(\mathcal C\) es un código \textit{lineal de longitud \(n\) y dimensión \(k\)} —abreviado como \([n, k]\)-\textit{lineal}, o como \([n, k]_q\)~-\textit{lineal} en caso de querer informar del cuerpo base— si dicho código es un subespacio vectorial de \(\mathbb F_q^n\) de dimensión \(k\).
\end{definition}

\begin{remark}
  Un código lineal \(\mathcal C\) tiene \(q^k\) palabras código.
\end{remark}

Así, hemos pasado de trabajar con un objeto que no tiene estructura alguna a trabajar con espacios vectoriales, cuyas propiedades son ampliamente conocidas y disponemos de numerosas herramientas para tratarlos.
Por ejemplo, en ocasiones hablaremos de \emph{subcódigos} de un código \(\mathcal C\).
Si \(\mathcal C\) es un código lineal entonces el subcódigo será un subespacio vectorial del mismo.
En caso de que sea no lineal un subcódigo será simplemente un subconjunto de \(\mathcal C\).

Veamos en la siguiente definición otro ejemplo de las herramientas que nos proporciona trabajar con espacios vectoriales.

\begin{definition}
  Una \textit{matriz generadora} para un \([n, k]\) código \(\mathcal C\) es una matriz \(k \times n\) cuyas filas conforman una base de \(\mathcal C\)\footnote{Efectivamente la matriz generadora no es única: basta tomar la correspondiente a cualquier otra base del código —que no deja de ser un espacio vectorial— para obtener una distinta. Pero es más, podemos simplemente reordenar las filas de una matriz generadora y en esencia estaremos obteniendo otra distinta.}.
\end{definition}



\begin{definition}
  Para cada conjunto \(k\) de columnas independientes de una matriz generadora \(G\) el conjunto de coordenadas correspondiente se denomina \textit{conjunto de información} para un código \(\mathcal C\). 
  Las \(r = n - k\) coordenadas restantes se llaman \textit{conjunto redundante}, y el número \(r\), la \textit{redundancia} de \(\mathcal C\).
\end{definition}

Si las primeras \(k\) coordenadas de una matriz generadora \(G\) forman un conjunto de información entonces el código tiene una única matriz generadora de la forma \([I_k \mid A]\), donde \(I_k\) es la matriz identidad \(k \times k\) y \(A\) es una matriz \(k \times r\). 
Esta matriz generadora se dice que está en \textit{forma estándar}.
A partir de cualquier matriz generadora siempre es posible obtener una matriz en forma estándar realizando una permutación adecuada de las coordenadas.
%Esta matriz resultante no será una matriz generadora del código inicial, pero sí de un código equivalente.

%Como un código lineal es el subespacio de un espacio vectorial, es el núcleo de una transformación lineal. En particular, existe una matriz \(H\) de dimensiones \(r \times n\), llamada \textit{matriz de comprobación de paridad} para un \([n, k]\) código \(\mathcal C\) definida por \begin{equation}
%  \mathcal C = \left\{x \in \mathbb F_q^n : H \mathbf x^T = 0 %\right\}.
%\end{equation}

Como un código lineal \(\mathcal C\) es un subespacio de un espacio vectorial, podemos calcular el ortogonal a dicho subespacio, obteniendo lo que llamaremos el \textit{código dual} (\textit{euclídeo}, si usamos el producto escalar usual) y que denotaremos por \(\mathcal C^{\perp}\).

\begin{definition}
  El \textit{código dual} \(\mathcal C^{\perp}\) de un código \(\mathcal C\) viene dado por \[\mathcal C^{\perp} = \left\{x \in \mathbb F_q^n : x \cdot c = 0 \quad \forall c \in \mathcal C\right\},\]
  donde \((\cdot)\) representa el producto escalar usual.
\end{definition}

\begin{definition}
  Sea \(\mathcal C\) un \([n, k]\) código lineal. Una matriz \(H\) se dice que es \textit{matriz de paridad} si es una matriz generadora de \(\mathcal C^{\perp}\).
\end{definition}

\begin{proposition}
  \label{prop:cod-por-matriz-paridad}
  Sea \(H\) la matriz de paridad de un \([n, k]\) código lineal \(\mathcal C\). 
  Entonces, \[\mathcal C = \left\{x \in \mathbb F_q^n : xH^T = 0\right\} = \left\{x \in F_q^n : Hx^T = 0\right\}.\]
\end{proposition}

\begin{proof}
  Sea \(c \in \mathcal C\) una palabra código. 
  Sabemos que la podemos expresar como \(c = uG\), donde \(u \in \mathbb F_q^k\) y \(G\) es una matriz generadora de \(\mathcal C\). 
  Tenemos entonces que \(c\cramped{H^T} = uG\cramped{H^T}\) y como \(G\cramped{H^T} = 0\) —por ser H matriz generadora del subespacio ortogonal \(\mathcal C\)— se tiene que \[\mathcal C \subset S_H = \left\{x \in \mathbb F_q^n : Hx^T = 0\right\},\] que es el espacio solución de un sistema de \(n - k\) ecuaciones con \(n\) incógnitas y de rango \(n - k\). Como \(\dim(S_H) = n - (n - k) = k = \dim L\), concluimos que \[L = S_H = \left\{x \in \mathbb F_q^n : Hx^T = 0\right\}.\qedhere\]
\end{proof}

Este último resultado, junto a la definición previa, nos conducen al siguiente teorema. 

\begin{theorem}
  Si \(G = [I_k \mid A]\) es una matriz generadora para un \([n, k]\) código \(\mathcal C\) en forma estándar entonces \(H = [-A \mid I_{n-k}]\) es una matriz de paridad para \(\mathcal C\).
\end{theorem}

Como nota final sobre nomenclatura de códigos duales, apuntamos que un código se dice \textit{autoortogonal} cuando \(\mathcal C \subseteq \mathcal C^{\perp}\), y \textit{autodual} cuando \(\mathcal C = \mathcal C^{\perp}\).

\subsection{Codificación y decodificación}
\label{subsec:codificacion-descodificacion}

Codificar un mensaje consiste en escribirlo como palabra código de un código.
La forma estándar de codificar mensajes con códigos lineales es utilizando una matriz generadora.
Dado un mensaje \(\mathbf{m} \in \mathbb F_q^k\) podemos obtener la palabra código \(\mathbf{c}\) en \(\mathcal C\) realizando la operación \(\mathbf{c}= \mathbf{m}G\).
Vamos a verlo mejor con un ejemplo.

\begin{example}
  Sea \(\mathcal C\) \([3, 2]\) un código binario lineal y \(G\) la matriz generadora dada por 
  \[
    G = \begin{pmatrix}
      1 & 1 & 0 \\ 0 & 1 & 1
    \end{pmatrix} \in \mathcal M_{2 \times 3}(\mathbb F_2).
  \]
  Dado un mensaje \(\mathbf{m} = (x_1, x_2)\), se tiene que \[(x_1, x_2) \begin{pmatrix}
    1 & 1 & 0 \\ 0 & 1 & 1
  \end{pmatrix} = (x_1, x_1 + x_2, x_2),\] y por tanto esta matriz codifica de la forma \[00 \to 000, \quad 01 \to 011,\quad 10 \to 110,\quad 11 \to 101.\]
\end{example}

Observamos que una matriz generador \(G\) define una aplicación lineal de \(\mathbb F_q^k\) en \(\mathbb F_q^n\), de forma que el código obtenido es la imagen de dicha aplicación.
Podemos comprobar también que es posible codificar en los mismos códigos lineales utilizando distintas matrices generadores, lo que resultará en distintas palabras código para el mismo mensaje.
Veamos un ejemplo con el mismo código binario lineal que en el ejemplo anterior pero con distinta matriz generadora.

\begin{example}
  Sea \(\mathcal C\) un \([3, 2]\) código binario lineal y \(G\) la matriz generadora dada por 
  \[
    G = \begin{pmatrix}
      1 & 0 & 1 \\ 0 & 1 & 1
    \end{pmatrix} \in \mathcal M_{2 \times 3}(\mathbb F_2).
  \]
  Dado un mensaje \(\mathbf{m} = (x_1, x_2)\), se tiene que \[(x_1, x_2) \begin{pmatrix}
    1 & 0 & 1 \\ 0 & 1 & 1
  \end{pmatrix} = (x_1, x_2, x_1 + x_2),\] y por tanto esta matriz codifica de la forma \[00 \to 000, \quad 01 \to 011,\quad 10 \to 101,\quad 11 \to 110.\]
\end{example}

Observamos en este ejemplo que las primeras \(2\) coordenadas de cada palabra código son iguales a las del mensaje que las genera.
Pero en el código anterior también podemos encontrar el mensaje, lo que hay que fijarse en la primera y última coordenada.
Cuando un mensaje se encuentra incrustado íntegramente en la palabra código —aunque puede que desordenado— se dice que la codificación seguida es \textit{sistemática}.
En caso contrario, se dice que es \textit{no-sistemática}.
Veamos un ejemplo de codificación no-sistemática con el mismo código binario lineal de antes.

\begin{example}
  \label{ej:codificacion-no-sistematica}
  Sea \(\mathcal C\) un \([3, 2]\) código binario lineal y \(G\) la matriz generadora dada por 
  \[
    G = \begin{pmatrix}
      1 & 0 & 1 \\ 1 & 1 & 1
    \end{pmatrix} \in \mathcal M_{2 \times 3}(\mathbb F_2).
  \]
  Dado un mensaje \(\mathbf{m} = (x_1, x_2)\), se tiene que \[(x_1, x_2) \begin{pmatrix}
    1 & 0 & 1 \\ 1 & 1 & 1
  \end{pmatrix} = (x_1 + x_2, x_2, x_1 + x_2),\] y por tanto esta matriz codifica de la forma \[00 \to 000, \quad 01 \to 111,\quad 10 \to 101,\quad 11 \to 010.\]
  Comprobamos que los mensajes \(01\) y \(11\) no están contenidos en las palabras código correspondientes, \(111\) y \(010\), respectivamente, luego la codificación es no-sistemática.
\end{example}

Dada una palabra código \(\mathbf{c}\) si se desea obtener el mensaje \(\mathbf{m}\) a partir del que se obtuvo podemos realizar el procedimiento inverso a la codificación.
Para ello tenemos en cuenta que al codificar mediante una matriz generadora \(G\) de tamaño \(n \times k\) establecemos una correspondencia biyectiva entre mensajes y palabras código.
Existe por tanto una matriz \(K\) de tamaño \(k \times n\) llamada \textit{inversa por la derecha} tal que \(GK = I_k\).
Así, puesto que \(\mathbf{c} = \mathbf{m}G\) podemos obtener el mensaje original calculando \(\mathbf{c}K = \mathbf{m}GK = \mathbf{m}\).
Veamos un ejemplo de este proceso.

\begin{example}
  Sea \(\mathcal C\) un \([7, 3]\) código binario lineal y \(G\) la matriz generadora dada por 
  \[
    G = \left(\begin{array}{rrrrrrr}
      1 & 1 & 1 & 0 & 1 & 0 & 0 \\
      0 & 1 & 1 & 1 & 0 & 1 & 0 \\
      0 & 0 & 1 & 1 & 1 & 0 & 1
      \end{array}\right) \in \mathcal M_{3 \times 7}(\mathbb F_2).
  \]
  Esta matriz codifica el mensaje \(\mathbf{m} = (1, 0, 1)\) como:
  \[
    \mathbf{c} = (1, 0, 1)\left(\begin{array}{rrrrrrr}
      1 & 1 & 1 & 0 & 1 & 0 & 0 \\
      0 & 1 & 1 & 1 & 0 & 1 & 0 \\
      0 & 0 & 1 & 1 & 1 & 0 & 1
      \end{array}\right) = \left(1,\,1,\,0,\,1,\,0,\,0,\,1\right).
  \]
  Para realizar el procedimiento inverso buscamos una matriz \(K\) tal que \(GK = I_7\).
  Esta matriz viene dada por
  \[
    K = \left(\begin{array}{rrr}
      1 & 1 & 0 \\
      0 & 1 & 1 \\
      0 & 0 & 1 \\
      0 & 0 & 0 \\
      0 & 0 & 0 \\
      0 & 0 & 0 \\
      0 & 0 & 0
      \end{array}\right)
  \] y por tanto el mensaje original era
  \[
    \mathbf{m} = \left(1,\,1,\,0,\,1,\,0,\,0,\,1\right)\left(\begin{array}{rrr}
      1 & 1 & 0 \\
      0 & 1 & 1 \\
      0 & 0 & 1 \\
      0 & 0 & 0 \\
      0 & 0 & 0 \\
      0 & 0 & 0 \\
      0 & 0 & 0
      \end{array}\right) = (1, 0, 1).
  \]
  
\end{example}

El proceso de decodificación de los mensajes consiste en obtener una palabra código válida a partir de un mensaje recibido\footnote{Es importante llamar la atención sobre el hecho de que \textit{decodificar} no es el proceso inverso a \textit{codificar}. Codificar consiste en escribir un mensaje como palabra código y decodificar, en corregir los errores que se hayan podido producir en la transmisión de dicha palabra.}. 
Es una tarea mucho más complicada que los procesos comentados antes, pues como ya se ha mencionado hay que tener en cuenta las posibles interferencias que se hayan podido producir en la comunicación.
Existen numerosos métodos de decodificación, y en general, cada familia de códigos tendrá un sistema que se aproveche de sus propiedades para ofrecer mejores prestaciones.
Destacamos de entre todos ellos un sistema aplicable a los códigos lineales, el conocido como \emph{decodificación por síndromes}, pues en él se basará el algoritmo cuya descripción es el objetivo de este trabajo.
Este método se basa en la propiedad de la matriz de paridad de que para toda palabra código \(\mathbf{c} \in \mathcal C\) se tiene que \(H \mathbf{c} = 0\).
Someramente el método consiste en computar y almacenar previamente los resultados del producto de todos los posibles vectores error y cuando se recibe un mensaje \(\mathbf{y} = \mathbf{c} + \mathbf{e}\), donde \(\mathbf{e}\) representa el error que se ha producido en la transmisión, se calcula lo que se conoce como \emph{síndrome}, que es el producto
\[
  H \mathbf{y} = H(\mathbf{c} + \mathbf{e}) = H \mathbf{c} + H \mathbf{e} = H \mathbf{e}.
\]
Este síndrome obtenido se compara con los productos previamente calculados para determinar qué error \(\mathbf{e}\) se ha producido y el mensaje codificado se obtiene como \(\mathbf{c} = \mathbf{y} - \mathbf{e}\).
%Los métodos de decodificación de códigos lineales en general se escapan del alcance de este trabajo, pues su objetivo principal es la descripción de un algoritmo de decodificación para un tipo concreto de códigos que veremos más adelante.

\subsection{Distancias y pesos}

Códigos distintos poseen distintas propiedades, lo que implica que sus capacidades de corrección difieran.
En este apartado vamos a estudiar dos propiedades de los códigos muy relacionadas con esta idea.

\begin{definition}
  La \textit{distancia de Hamming} \(\operatorname{d}(\symbf{x}, \symbf{y})\) entre dos vectores \(\symbf{x}, \symbf{y} \in \mathbb F_q^n\) se define como el número de coordenadas en las que difieren \(\symbf{x}\) e \(\symbf{y}\).
\end{definition}

\begin{theorem}
  La función de distancia \(\operatorname{d}(\symbf{x}, \symbf{y})\) verifica las siguientes propiedades.
  \begin{enumerate}
    \item No negatividad: \(\operatorname{d}(\symbf{x}, \symbf{y}) \geq 0\) para todo \(\symbf{x}, \symbf{y}\in \mathbb F_q^n\).
    \item La distancia \(\operatorname{d}(\symbf{x}, \symbf{y}) = 0\) si y solo si \(\symbf{x} = \symbf{y}\).
    \item Simetría: \(\operatorname{d}(\symbf{x}, \symbf{y}) = \operatorname{d}(\symbf{y}, \symbf{x})\) para todo \(\symbf{x}, \symbf{y}\in \mathbb F_q^n\).
    \item Desigualdad triangular: \(\operatorname{d}(\symbf{x}, \symbf{z}) \leq \operatorname{d}(\symbf{x}, \symbf{y}) + \operatorname{d}(\symbf{y}, \symbf{z})\) para todo elemento \(\symbf{x}, \symbf{y}, \symbf{z}\in \mathbb F_q^n\).
  \end{enumerate}
\end{theorem}

\begin{proof}
  
\end{proof}

La \textit{distancia} (\textit{mínima}) de un código \(\mathcal C\) es la menor distancia posible entre dos palabras código distintas. 
Si la distancia mínima \(d\) de un \([n,k]\) código es conocida, nos referiremos a él como un \([n,k,d]\) código.
Este valor es importante pues nos ayuda a determinar la capacidad de corrección de errores del código \(\mathcal C\), como ilustra el siguiente teorema.

\begin{theorem}
  Sea \(\mathcal C\) un \([n, k, d]\) código. Entonces \(\mathcal C\) tiene capacidad de corrección de errores \[
    t = \left\lfloor \frac{d - 1}{2} \right\rfloor.
  \]
\end{theorem}

% MAYBE: hablar también de la capacidad de detección de errores? l = d - 1

Efectivamente, a mayor distancia mínima, mayor número de errores en el código se pueden corregir.
Otra medida interesante es el \textit{peso de Hamming}.

\begin{definition}
  El \textit{peso de Hamming} \(\operatorname{wt}(\symbf{x})\) de un vector \(\symbf{x}\) es el número de coordenadas distintas de cero de \(\symbf{x}\).
\end{definition}

El siguiente teorema nos ilustra la relación existente entre los conceptos de peso y distancia.

\begin{theorem}
  Si \(\symbf{x}, \symbf{y} \in \mathbb F_q^n\), entonces \(\operatorname{d}(\symbf{x}, \symbf{y}) = \operatorname{wt}(\symbf{x} - \symbf{y})\).
  Si \(\mathcal C\) es un código lineal, la distancia mínima es igual al peso mínimo de las palabras código de \(\mathcal C\) distintas de cero.
\end{theorem}

\begin{proof}
  
\end{proof}

Como consecuencia de este teorema —para códigos lineales— la distancia mínima también se llama \textit{peso mínimo} del código.

\begin{definition}
  Sea \(A_i(\mathcal C)\) —que abreviaremos \(A_i\)— el número de palabras código de peso \(i\) en \(\mathcal C\).
  Para cada \(0 \leq i \leq n\), la lista \(A_i\) se denomina \textit{distribución de peso} o \textit{espectro de peso} de \(\mathcal C\).
\end{definition}

\begin{example}
  Sea \(\mathcal C\) el código binario con matriz generadora
  \[
    G = \begin{pmatrix}
      1 & 1 & 0 & 0 & 0 & 0\\
      0 & 0 & 1 & 1 & 0 & 0 \\
      0 & 0 & 0 & 0 & 1 & 1
    \end{pmatrix}.
  \]
  Dado \((x_1, x_2, x_3)\), se tiene que \[(x_1, x_2, x_3) \begin{pmatrix}
    1 & 1 & 0 & 0 & 0 & 0\\
      0 & 0 & 1 & 1 & 0 & 0 \\
      0 & 0 & 0 & 0 & 1 & 1
  \end{pmatrix} = (x_1, x_1, x_2, x_2, x_3, x_3),\] y por tanto podemos obtener las palabras código de la forma 
  \[
    000 \to 000000, \!\quad 
    001 \to 000011,\!\quad 
    010 \to 001100,\!\quad 
    011 \to 001111,
  \]
  \[
    100 \to 110000, \!\quad 
    101 \to 110011,\!\quad 
    110 \to 111100,\!\quad 
    111 \to 111111.
  \]
  Luego la distribución de peso de \(\mathcal C\) es \(A_0 = A_6 = 1\) y \(A_2 = A_4 = 3\).
  Usualmente solo se listan los \(A_i\) que son distintos de cero.
\end{example}

\begin{theorem}
  Sea \(\mathcal C\) un \([n,k,d]\) código sobre \(\mathbb F_q\).
  Entonces, \begin{enumerate}
    \item \(A_0(\mathcal C) + A_1(\mathcal C) + \cdots + A_n(\mathcal C) = q^k\).
    \item \(A_0(\mathcal C) = 1\) y \(A_1(\mathcal C) = A_2(\mathcal C) = \cdots = A_{d-1}(\mathcal C) = 0\).
  \end{enumerate}
\end{theorem}

\begin{proof}
  TODO
\end{proof}

% Theorem 1.4.13, Corollary 1.4.14 (Huffman-Pless), p.12-13

\begin{theorem}
  Sea \(\mathcal C\) un código lineal con matriz de paridad \(H\). Si \(\symbf{c} \in \mathcal C\), las columnas de \(H\) que se corresponden con coordenadas no nulas de \(\symbf{c}\) son linealmente independientes.
  Recíprocamente, si entre \(w\) columnas de \(H\) existe una relación de dependencia lineal con coeficientes no nulos, entonces hay una palabra código en \(\mathcal C\) de peso \(w\) cuyas coordenadas no nulas se corresponden con dichas columnas.
\end{theorem}

\begin{proof}
  TODO
\end{proof}

\begin{corollary}
  \label{cor:peso-minimo-columnas-dependientes}
  Un código lineal tiene peso mínimo \(d\) si y solo si su matriz de paridad tiene un conjunto de \(d\) columnas linealmente dependientes pero no tiene un conjunto de \(d-1\) columnas linealmente dependientes.
\end{corollary}

\begin{proof}
  TODO
\end{proof}

\section{Ejemplos de códigos}

En esta sección vamos a describir someramente algunas familias de códigos relevantes: los códigos de repetición, los códigos de control de paridad y los códigos de Hamming.

\subsection{Códigos de repetición}

Los códigos de repetición son una de las familias de códigos más sencillas.
Dado un mensaje \(\mathbf{m} = (m_1, m_2, \dots, m_n) \in \mathbb F_q^n\) lo que hacemos para codificarlo es repetir cada elemento \(m_i\) de la tupla \(k\) veces: 
\[
  \mathbf{c} = (m_{11}, m_{12}, \dots, m_{1k}, m_{21}, m_{22}, \dots, m_{2k}, \dots, m_{n1}, m_{n2}, \dots, m_{nk}).
\]
A la hora de decodificar un mensaje cada bloque de \(k\) elementos se fija al valor del elemento que más se repita. 
Los más utilizados son los códigos de repetición binarios, es decir, los que se definen sobre \(\mathbb F_2\).
No son códigos lineales.

\subsection{Códigos de control de paridad}

Los \([n, n -1]\)-códigos lineales que tienen matriz de paridad \[
  H = \begin{pmatrix}
    1 & 1 & \dots & 1
  \end{pmatrix}
\] se llaman \textit{códigos de control de paridad} o \textit{códigos de peso par}.
Por la proposición \ref{prop:cod-por-matriz-paridad} las palabras código \(\mathbf{c}\) de este tipo de códigos han de cumplir que
\[
  \mathbf{c}H^T = c_1 + c_2 + \dots + c_n = 0,
\]
es decir, el número de \(1\) en la palabra código ha de ser par —de ahí el nombre—.
La codificación de mensajes se realiza entonces añadiendo un \textit{bit} de paridad al final del mensaje cuyo valor se fija para que el número de \(1\) en el mismo sea par.
Estos códigos tienen distancia \(2\) y pueden corregir un solo error.

\subsection{Códigos de Hamming}

Consideremos una matriz \(r \times (2^r - 1)\) cuyas columnas son los números \(1, 2, 3, \dots, 2^{r-1}\) escritos en binario. 
Dicha matriz es la matriz de paridad de un \([n=2^{r-1}, k=n-r]\) código binario.
A los códigos de esta forma los llamaremos códigos de Hamming de longitud \(n = 2^{r-1}\) y los denotamos por \(\mathcal H_r\) o \(\mathcal H_{2,r}\).

Como las columnas son distintas y no nulas, la distancia es al menos \(3\) por el corolario \ref{cor:peso-minimo-columnas-dependientes}.
Además, como las columnas correspondientes a los números \(1, 2, 3\) son linealmente independientes, la distancia mínima es 3 por el mismo corolario.
Podemos decir por tanto que los códigos de Hamming \(\mathcal H_r\) son \([2^{r-1}, 2^{r-1-r}, 3]\) códigos binarios.

Podemos generalizar esta definición y definir los códigos de Hamming \(\mathcal H_{q,r}\) sobre un cuerpo finito arbitrario \(\mathbb F_q\). 
Para \(r \geq 2\) un código \(\mathcal H_{q,r}\) tiene matriz de paridad \(H_{q,r}\), cuyas columnas están compuestas por un vector no nulo por cada uno de los subespacios de dimensión \(1\) de \(\mathbb F_q^r\).
Hay \((q^r-1)/(q-1)\) subespacios de dimensión \(1\), por lo que \(\mathcal H_{q,r}\) tiene longitud \(n = (q^r-1)/(q-1)\), dimensión \(n-r\) y redundancia \(r\).
Como todas las columnas son independientes unas de otras, \(\mathcal H_{q,r}\) tiene peso mínimo al menos 3.
Si sumamos dos vectores no nulos de dos subespacios unidimensionales distintos obtenemos un vector no nulo de un tercer subespacio unidimensional, por lo que \(\mathcal H_{q,r}\) tiene peso mínimo 3. 
Cuando \(q = 2\), \(\mathcal H_{2,r}\) es el código \(\mathcal H_r\).

% TODO
% - por qué los códigos de Hamming son importantes
% - en qué se utilizan los códigos de Hamming

\chapter{Códigos cíclicos}

En este capítulo vamos a estudiar los aspectos fundamentales de la familia de los códigos cíclicos, que representan la base del objeto de estudio de este trabajo.
Las fuentes de este capítulo han sido \parencite{huffman_fundamentals_2003}, \parencite{kelbert_information_2013} y \parencite{macwilliams_theory_1977}.

\begin{definition}
  Un código lineal \(\mathcal C\) de longitud \(n\) sobre \(\mathbb F_q\) es \textit{cíclico} si para cada vector \(\mathbf c = c_0\dots c_{n-2}c_{n-1}\) en \(\mathcal C\), el vector \(c_{n-1}c_0\dots c_{n-2}\) —obtenido a partir de \(\mathbf c\) desplazando cíclicamente las coordenadas, llevando \(i \mapsto i +1 \bmod n\)— también está en \(\mathcal C\).
\end{definition}

Al trabajar con códigos cíclicos pensamos en la posición de las coordenadas de forma cíclica, pues al llegar a \(n -1\) se comienza de nuevo en \(0\).
Al hablar de «coordenadas consecutivas» siempre tendremos en cuenta esta ciclicidad.
Representaremos las palabras código de los códigos cíclicos como polinomios, pues podemos definir de forma natural una biyección entre el vector \(\mathbf c = c_0c_1\dots c_{n-1}\) en \(\mathbb F_q\) y los polinomios de la forma \(c(x) = c_0 + c_1x + \dots c_{n-1}x^{n-1}\) en \(\mathbb F_q[x]\) de grado al menos \(n-1\).
Denotaremos a esta biyección por \(\mathfrak v\).
Obtenemos así un isomorfismo entre \(\mathbb F_q\)-espacios vectoriales.
Obsérvese que dado un polinomio \(c(x)\) descrito como antes, el polinomio \(xc(x) = c_{n-1}x^n + c_0x + c_1x^2 + \dots + c_{n-2}x^{n-1}\) equivale a representar la palabra código \(\mathbf c\) desplazada una posición a la derecha, siempre que \(x^n\) fuese igual a \(1\).

Formalmente, el hecho de que un código \(\mathcal C\) sea invariante bajo un desplazamiento cíclico implica que si \(c(x)\) está en \(\mathcal C\), también ha de estar \(xc(x)\), siempre que multipliquemos módulo \(x^n -1\). 
Esto nos sugiere que el contexto adecuado para estudiar códigos cíclicos es el anillo cociente \(\mathcal R_n = \mathbb F_q[x]/(x^n - 1)\).

Por tanto, bajo la correspondencia vectores-polinomios que hemos descrito antes, los códigos cíclicos son ideales de \(\mathcal R_n\), y los ideales de \(\mathcal R_n\) son códigos cíclicos.
En consecuencia, el estudio de los códigos cíclicos en \(\mathbb F_q^n\) es equivalente al estudio de los ideales en \(\mathcal R_n\), que va a depender de la factorización de \(x^n-1\) y por tanto lo vamos a abordar a continuación.

\section{Factorización de \texorpdfstring{\(x^n -1\)}{xn - 1}}

\label{sec:factorizacion-xn-1}

A la hora de factorizar \(x^n -1\) existen dos posibilidades, pues dicha factorización puede tener factores irreducibles repetidos o no.
Vamos a asumir que \(q\) y \(n\) son primos relativos y por tanto \(x^n - 1\) no tiene factores repetidos; en caso contrario el anillo cociente sería semisimple, lo que no nos aportaría nada: los ideales generados por los polinomios con factores repetidos no aumentan la distancia, pues es la misma que la del subcódigo sin factores repetidos.

Para factorizar \(x^n - 1\) sobre \(\mathbb F_q\) necesitamos considerar la extensión de cuerpos \(\mathbb F_{q^t}\) de \(\mathbb F_q\) que contenga todas sus raíces.
El cuerpo \(\mathbb F_{q^t}\) debe contener una enésima raíz primitiva de la unidad, que por el teorema \ref{th:Fq-ast-cilcico} sabemos que ocurre cuando \(n \mid (q^t - 1)\).
Definimos el \textit{orden} \(\operatorname{ord}_n(q)\) de \(q\) módulo \(n\) como el menor entero positivo \(a\) tal que \(q^{a} \equiv 1 \bmod n\).
Si \(t = \operatorname{ord}_n(q)\) entonces \(\mathbb F_{q^t}\) contiene una enésima raíz primitiva de la unidad \(\alpha\) pero no hay una extensión de cuerpos más pequeña de \(\mathbb F_q\) que la contenga.
Como todos los \(\alpha^{i}\) son distintos dos a dos para \(0 \leq i < n\) y \((\alpha^{i})^n = 1\), entonces \(\mathbb F_{q^t}\) contiene todas las raíces de \(x^n - 1\).
Por tanto, \(\mathbb F_{q^t}\) es lo que se conoce como \textit{cuerpo de descomposición} de \(x^n - 1\) sobre \(\mathbb F_q\).

Los factores irreducibles de \(x^n - 1\) sobre \(\mathbb F_q\) deben ser el producto de los distintos polinomios minimales de las enésimas raíces de la unidad en \(\mathbb F_{q^t}\).
Supongamos que \(\gamma\) es un elemento primitivo de \(\mathbb F_{q^t}\).
Entonces \(\alpha = \gamma^d\) es una enésima raíz primitiva de la unidad, donde \(d = (q^t - 1)/n\).
Las raíces del polinomio \(M_{a^s}(x)\) son \(\{\gamma^{ds}, \gamma^{dsq}, \gamma^{dsq^2}, \dots, \gamma^{dsq^{r-1}}\} = \{\alpha^s, \alpha^{sq}, \alpha^{sq^2}, \dots, \alpha^{sq^{r-1}}\}\), donde \(r\) es el menor entero positivo tal que \(dsq^r \equiv ds \bmod q^t - 1\) por el teorema \ref{th:pol-minimal-el-primitivo}.
Pero \(dsq^r \equiv ds \bmod q^t - 1\) si y solo si \(sq^r \equiv s \bmod n\).

Todo esto nos lleva a extender la definición de clases \(q\)-ciclotómicas que hemos introducido en la sección \ref{subsec:clases-ciclotomicas}.
Sea \(s\) un entero tal que \(0 \leq s < n\).
La \textit{clase \(q\)-ciclotómica de \(s\) módulo \(n\)} es el conjunto
\[
  C_s = \{s, sq, \dots, sq^{r-1}\} \bmod n, 
\]
donde \(r\) es el menor entero positivo tal que \(sq^r \equiv s \bmod n\).
Se deduce entonces que \(C_s\) es la órbita de la permutación \(i \mapsto iq \bmod n\) que contiene a \(s\).
Las distintas clases \(q\)-ciclotómicas módulo \(n\) dividen el conjunto de enteros \(\{0, 1, 2, \dots, n - 1\}\).
En la sección * estudiamos el caso particular en el que \(n = q^t - 1\).
Obsérvese que \(\operatorname{ord}_n(q)\) es el tamaño de la clase \(q\)-ciclotómica \(C_1\) módulo \(n\).

\begin{theorem}
  \label{th:pol-minimal-raiz-primitiva}
  Sea \(n\) un entero positivo primo relativo con \(q\) y sea \(t = \operatorname{ord}_n(q)\).
  Sea \(\alpha\) una raíz enésima primitiva de la unidad en \(\mathbb F_{q^t}\).
  Se verifican las siguientes afirmaciones.
  \begin{enumerate}
    \item Para cada entero \(s\) tal que \(0 \leq s < n\) el polinomio minimal de \(\alpha^s\) sobre \(\mathbb F_q\) es
    \[
      M_{\alpha^s}(x) = \prod_{i \in C_s}(x - \alpha^i),
    \]
    donde \(C_s\) es la clase \(q\)-ciclotómica de \(s\) módulo \(n\).
    \label{thi:pol-minimal-raiz-primitiva-producto}
    \item Los conjugados de \(\alpha^s\) son los elementos \(\alpha^i\) con \(i \in C_s\).
    \item Se tiene que
    \[
      x^n - 1 = \prod_s M_{\alpha^s}(x)
    \]
    es la factorización de \(x^n - 1\) en factores irreducibles sobre \(\mathbb F_q\), donde \(s\) varía en un conjunto de representantes de las clases \(q\)-ciclotómicas módulo \(n\).
  \end{enumerate}
\end{theorem}

\begin{proof}
  % TODO (en el libro no viene)
\end{proof}

\begin{theorem}
  El tamaño de cada clase \(q\)-ciclotómica es un divisor de \(\operatorname{ord}_n(q)\).
  Además, el tamaño de \(C_1\) es \(\operatorname{ord}_n(q)\).
\end{theorem}

\section{Construcción de códigos cíclicos}

Una vez factorizado \(x^n - 1\) vamos a ver que hay una correspondencia biyectiva entre sus polinomios divisores mónicos y los códigos cíclicos en \(\mathcal R_n\).
El siguiente teorema es el resultado fundamental de códigos cíclicos que nos va a permitir describirlos.

\begin{theorem}
  \label{th:corr-cod-div}
  Sea \(\mathcal C\) un ideal de \(\mathcal R_n\), es decir, un código cíclico de longitud \(n\). Entonces:
  \begin{enumerate}
    \item Existe un único polinomio mónico \(g(x)\) de grado mínimo en \(\mathcal C\).\label{thi:corr-codc-div:monico-minimo}
    \item El polinomio descrito en (\ref{thi:corr-codc-div:monico-minimo}) genera \(\mathcal C\), es decir, \(\mathcal C = \langle g(x)\rangle\).
    \item El polinomio descrito en (\ref{thi:corr-codc-div:monico-minimo}) verifica que \(g(x) \mid x^n -1\).\label{thi:corr-codc-div:div-xn-1}
  \end{enumerate}
  Sea \(k = n - \operatorname{gr} g(x)\) y sea \(g(x) = \sum_{i_0}^{n-k}g_ix^{i}\), donde \(g_{n-k} = 1\). Entonces:
  \begin{enumerate}[resume]
    %\item La dimensión de \(\mathcal C\) es \(k\) y \(\{g(x), xg(x), \dots, x^{k-1}g(x)\}\) es una base de \(\mathcal C\).
    %\item Cada elemento de \(\mathcal C\) se puede expresar como producto de \(g(x)\) por un polinomio \(f(x)\), donde \(f(x) = 0\) o bien \(\operatorname{gr} f(x) < k\).
    \item \label{thi:corr-codc-div:forma-elem} Se verifica que \[
      \mathcal C = \langle g(x) \rangle = \{f(x)g(x) : \operatorname{gr} f(x) < k\}.
    \]
    \item \label{thi:corr-codc-div:dim-ideal} El conjunto \(\{g(x), xg(x), \dots, x^{k-1}g(x)\}\) es una base de \(\mathcal C\) y \(\mathcal C\) tiene dimensión \(k\).
    \item \label{thi:corr-cod-div:mat-gen} La matriz \(G\) dada por \[
      G = \begin{bmatrix}
        g_0 & g_1 & g_2 & \dots & \dots & g_{n-k} & 0 & 0 & \dots & 0 \\
        0 & g_0 & g_1 & g_2 & \dots & \dots & g_{n-k} & 0 & \dots & 0 \\
        0 & 0 & g_0 & g_1 & g_2 & \dots & \dots & g_r & \ddots & \vdots \\
        \vdots & \vdots & \ddots & \ddots & \ddots & \ddots & & & \ddots & 0\\
        0 & 0 & \dots & 0 & g_0 & g_1 & g_2 & \dots & \dots & g_{n-k} 
      \end{bmatrix},
    \]
    donde cada fila es un desplazamiento cíclico de la fila previa, es una matriz generadora de \(\mathcal C\).
    \item \label{thi:pol-generador-prod-minimal} Si \(\alpha\) es una enésima raíz primitiva de la unidad en alguna extensión de cuerpos de \(\mathbb F_q\) entonces \[
      g(x) = \prod_s M_{\alpha^s}(x),
    \] siendo dicho producto sobre un subconjunto de representantes de las clases \(q\)-ciclotómicas módulo \(n\).
  \end{enumerate}
\end{theorem}

\begin{proof}
  Veamos la demostración apartado por apartado.
  \begin{enumerate}
    \item Supongamos que \(\mathcal C\) contiene dos polinomios mónicos distintos, \(g_1(x)\) y \(g_2(x)\), ambos de grado mínimo \(r\). 
    Entonces, \(g_1(x) - g_2(x)\) es un polinomio no nulo de grado menor que \(r\), lo que es absurdo. 
    Existe por tanto un único polinomio de grado mínimo \(r\) en \(\mathcal C\), como queríamos.
    \item Como \(g(x) \in \mathcal C\) y \(\mathcal C\) es un ideal, tenemos que \(\langle g(x)\rangle \subset \mathcal C\). 
    Por otra parte, dado \(p(x) \in \mathcal C\) el algoritmo de división nos da elementos \(q(x), r(x)\) tales que \(p(x) = q(x)g(x) + r(x)\), de forma que o bien \(r(x) = 0\) o bien \(\operatorname{gr} r(x) < \operatorname{gr} g(x)\). 
    Como podemos expresar \(r(x)\) de la forma \(r(x) = p(x) - q(x)g(x) \in \mathcal C\) y tiene grado menor que \(\operatorname{gr} g(x)\), al ser este último de grado mínimo necesariamente ha de darse que \(r(x) = 0\).
    Por tanto, \(p(x) = q(x)g(x) \in \langle g(x) \rangle\) y \(\mathcal C \subset \langle g(x) \rangle\).
    En consecuencia, \(\langle g(x) \rangle = \mathcal C\).
    \item Por el algoritmo de división, al dividir \(x^n - 1\) por \(g(x)\) tenemos que \(x^n - 1 = q(x)g(x) + r(x)\). De nuevo, o bien \(r(x) = 0\) o bien \(\operatorname{gr} r(x) < \operatorname{gr} g(x)\).
    Como en \(\mathcal R_n\) se tiene que \(x^n - 1 = 0 \in \mathcal C\), necesariamente \(r(x) \in \mathcal C\).
    Esto supone una contradicción, a menos que \(r(x) = 0\).
    En consecuencia, \(g(x) \mid x^n - 1\).
    \item El ideal generado por \(g(x)\) es \(\langle g(x) \rangle = \{f(x)g(x) : f(x) \in \mathcal R_n\}\).
    Queremos ver que podemos restringir los polinomios \(f(x)\) a aquellos que tengan grado menor que \(k\).
    Por (\ref{thi:corr-codc-div:div-xn-1}) sabemos que \(x^n-1 = h(x)g(x)\) para algún polinomio \(h(x)\) que tenga grado \(k = n - \operatorname{gr} g(x)\).
    Dividimos entonces \(f(x)\) por este polinomio \(h(x)\) y por el algoritmo de división obtenemos \(f(x) = q(x)h(x) + r(x)\), donde \(\operatorname{gr} r(x) < \operatorname{gr} h(x) = k\).
    Entonces, tenemos \begin{align*}
      f(x)g(x) &= q(x)h(x)g(x) + r(x)g(x)\\
               &= q(x)(x^n - 1) + r(x)g(x),
    \end{align*}
    luego \(f(x)g(x) = r(x)g(x)\), y puesto que antes ya hemos visto que \(\operatorname{gr} r(x) < k\), hemos obtenido lo que buscábamos.
    \item A partir de (\ref{thi:corr-codc-div:dim-ideal}) tenemos que el conjunto \[\{g(x), xg(x), \dots, x^{k-1}g(x)\}\] genera \(\mathcal C\), y como es linealmente independiente, forma una base de \(\mathcal C\).
    Esto demuestra también que la dimensión de \(\mathcal C\) es \(k\).
    \item La matriz \(G\) es matriz generadora de \(\mathcal C\) pues \[\{g(x), xg(x), \dots, x^{k-1}g(x)\}\] es una base de \(\mathcal C\).
    \item Se deduce del teorema \ref{th:pol-minimal-raiz-primitiva} y de (\ref{thi:corr-codc-div:div-xn-1}).\qedhere
  \end{enumerate}
\end{proof}

Este teorema nos proporciona una forma de obtener los códigos cíclicos de longitud \(n\) a partir de los divisores del polinomio \(x^n - 1\) así como describir una matriz generadora de dichos códigos a partir de ellos.
Vamos a ver a continuación que el polinomio mónico divisor de \(x^n - 1\) que genera a un código cíclico \(\mathcal C\) es único.

\begin{corollary}
  \label{cor:pol-gen-unico}
  Sea \(\mathcal C\) un código cíclico en \(\mathcal R_n\) distinto de cero.
  Son equivalentes:
  \begin{enumerate}
    \item El polinomio \(g(x)\) es el polinomio mónico de menor grado en \(\mathcal C\).
    \item Podemos expresar \(\mathcal C\) como \(\mathcal C = \langle g(x)\rangle\), \(g(x)\) es mónico y \(g(x) \mid (x^n -1)\).
  \end{enumerate}
\end{corollary}

\begin{proof}
  Que (1) implica (2) ya lo hemos probado en el teorema \ref{th:corr-cod-div}. 
  Veamos que partiendo de (2) obtenemos (1). 
  Sea \(g_1(x)\) el polinomio mónico de menor grado en \(\mathcal C\).
  Por el teorema \ref{th:corr-cod-div}, \(g_1(x) \mid g(x)\) en \(\mathbb F_q[x]\) y \(\mathcal C = \langle g_1(x)\rangle\).
  Como \(g_1(x) \in \mathcal C = \langle g(x) \rangle\), podemos expresarlo como \(g_1(x) = g(x)a(x) \bmod x^n - 1\), luego tenemos que \(g_1(x) = g(x)a(x) + (x^n - 1)b(x)\) en \(\mathbb F_q[x]\).
  Por otro lado, como \(g(x) \mid (x^n - 1)\), tenemos que \(g(x) \mid g(x)a(x) + (x^n-1)b(x)\), o lo que es lo mismo, que \(g(x) \mid g_1(x)\). 
  En consecuencia, como \(g_1(x)\) y \(g(x)\) son ambos mónicos y dividen el uno al otro en \(\mathbb F_q[x]\), son necesariamente iguales.
\end{proof}

A este polinomio \(g(x)\) lo llamamos \textit{polinomio generador} del código cíclico \(\mathcal C\).
Por el corolario anterior, este polinomio es tanto el polinomio mónico en \(\mathcal C\) de grado mínimo como el polinomio mónico que divide a \(x^n - 1\) y genera a \(\mathcal C\).
Existe por tanto una correspondencia biunívoca entre los códigos cíclicos distintos de cero y los divisores de \(x^n - 1\) distintos de él mismo.
Para extender dicha correspondencia entre todos los códigos cíclicos en \(\mathcal R_n\) y todos los divisores mónicos de \(x^n - 1\) definimos como polinomio generador del código cíclico \(\{\mathbf 0\}\) el polinomio \(x^n - 1\). 
Esta correspondencia biyectiva nos conduce al siguiente corolario.

\begin{corollary}
  El número de códigos cíclicos en \(\mathcal R_n\) es \(2^m\), donde \(m\) es el número de clases \(q\)-ciclotómicas módulo \(n\).
  %Es más, las dimensiones de los códigos cíclicos en \(\mathcal R_n\) son todas sumas de tamaños de las clases \(q\)-ciclotómicas módulo \(n\).
\end{corollary}

Ahora mismo todo este desarrollo puede parecer demasiado abstracto.
Vamos a ver un ejemplo exhaustivo para entender cómo podemos obtener los polinomios generadores de los códigos cíclicos de una longitud arbitraria y cómo éstos son generados a partir de ellos.

\begin{example}
  \label{ex:codigos-ciclicos-long-7}
  Vamos a describir todos los códigos cíclicos binarios de longitud 7.
  Para ello vamos a utilizar el código descrito en el anexo \ref{annex:sage-gen-idemp} tal y como mostramos en el listado siguiente.
  \begin{lstlisting}[gobble=4]
    sage: F = GF(2)
    sage: x = polygen(F)
    sage: (x^7 - 1).factor()
    > (x + 1) * (x^3 + x + 1) * (x^3 + x^2 + 1)
    sage: print(generadores(x^7 - 1))
    > [1, x + 1, x^3 + x + 1, x^3 + x^2 + 1, x^4 + x^3 + x^2 + 1, x^4 + x^2 + x + 1, x^6 + x^5 + x^4 + x^3 + x^2 + x + 1, x^7 + 1]
  \end{lstlisting}
  Así sobre \(\mathbb F_2\) podemos descomponer \(x^7 - 1\) como \[
    x^7 - 1 = (x + 1)(x^{3} + x + 1)(x^{3} + x^{2} + 1)
  \]
  y los 8 polinomios generadores, todos los divisores de \(x^7 - 1\), son: \begin{enumerate}
    \item \(1\)
    \item \((x + 1)\)
    \item \((x^{3} + x + 1)\)
    \item \((x^{3} + x^{2} + 1)\)
    \item \((x + 1)(x^{3} + x + 1) = x^4 + x^3 + x^2 + 1\)
    \item \((x + 1)(x^{3} + x^{2} + 1) = x^4 + x^2 + x + 1\)
    \item \((x^{3} + x + 1)(x^{3} + x^{2} + 1) = x^6 + x^5 + x^4 + x^3 + x^2 + x + 1\)
    \item \((x + 1)(x^{3} + x + 1)(x^{3} + x^{2} + 1) = x^7 - 1\)
  \end{enumerate}
  Vamos a ver qué códigos generan estos polinomios: \begin{enumerate}
    \item La dimensión del código es \(k = 7 - 0 = 7\), luego el código generado es un \([7, 7]\)-código lineal, que es evidentemente \(\mathbb F_2^7\). La matriz generadora que nos proporciona el teorema \ref{th:corr-cod-div}(\ref{thi:corr-cod-div:mat-gen}) es \[
      G = \left(\begin{array}{rrrrrrr}
        1 & 0 & 0 & 0 & 0 & 0 & 0 \\
        0 & 1 & 0 & 0 & 0 & 0 & 0 \\
        0 & 0 & 1 & 0 & 0 & 0 & 0 \\
        0 & 0 & 0 & 1 & 0 & 0 & 0 \\
        0 & 0 & 0 & 0 & 1 & 0 & 0 \\
        0 & 0 & 0 & 0 & 0 & 1 & 0 \\
        0 & 0 & 0 & 0 & 0 & 0 & 1
        \end{array}\right).
    \]
    \item La dimensión del código es \(k = 7 - 1 = 6\), luego el código generado es un \([7, 6]\)-código lineal.
    La matriz generadora que nos proporciona el teorema \ref{th:corr-cod-div}(\ref{thi:corr-cod-div:mat-gen}) es \[
      G = \left(\begin{array}{rrrrrrr}
        1 & 1 & 0 & 0 & 0 & 0 & 0 \\
        0 & 1 & 1 & 0 & 0 & 0 & 0 \\
        0 & 0 & 1 & 1 & 0 & 0 & 0 \\
        0 & 0 & 0 & 1 & 1 & 0 & 0 \\
        0 & 0 & 0 & 0 & 1 & 1 & 0 \\
        0 & 0 & 0 & 0 & 0 & 1 & 1
        \end{array}\right).
    \]
    Comprobamos que la matriz de paridad es \[
      H = \left(\begin{array}{rrrrrrr}
        1 & 1 & 1 & 1 & 1 & 1 & 1
        \end{array}\right)
    \]
    y por tanto el código obtenido es un código de control de paridad.
    \item La dimensión del código es \(k = 7 - 3 = 4\), luego el código generado es un \([7, 4]\)-código lineal.
    La matriz generadora que nos proporciona el teorema \ref{th:corr-cod-div}(\ref{thi:corr-cod-div:mat-gen}) es \[
      G = \left(\begin{array}{rrrrrrr}
        1 & 1 & 0 & 1 & 0 & 0 & 0 \\
        0 & 1 & 1 & 0 & 1 & 0 & 0 \\
        0 & 0 & 1 & 1 & 0 & 1 & 0 \\
        0 & 0 & 0 & 1 & 1 & 0 & 1
        \end{array}\right).
    \]
    La matriz de paridad en este caso es \[
      H = \left(\begin{array}{rrrrrrr}
        1 & 0 & 1 & 1 & 1 & 0 & 0 \\
        0 & 1 & 0 & 1 & 1 & 1 & 0 \\
        0 & 0 & 1 & 0 & 1 & 1 & 1
        \end{array}\right)
    \]
    y por tanto el código generado es un \(\mathcal H_3\) código de Hamming.
    \item La dimensión del código es \(k = 7 - 3 = 4\), luego el código generado es un \([7, 4]\)-código lineal.
    La matriz generadora que nos proporciona el teorema \ref{th:corr-cod-div}(\ref{thi:corr-cod-div:mat-gen}) es \[
      G = \left(\begin{array}{rrrrrrr}
        1 & 0 & 1 & 1 & 0 & 0 & 0 \\
        0 & 1 & 0 & 1 & 1 & 0 & 0 \\
        0 & 0 & 1 & 0 & 1 & 1 & 0 \\
        0 & 0 & 0 & 1 & 0 & 1 & 1
        \end{array}\right).
    \]
    La matriz de paridad en este caso es \[
      H = \left(\begin{array}{rrrrrrr}
        1 & 1 & 1 & 0 & 1 & 0 & 0 \\
        0 & 1 & 1 & 1 & 0 & 1 & 0 \\
        0 & 0 & 1 & 1 & 1 & 0 & 1
        \end{array}\right)
    \]
    y por tanto el código generado es un \(\mathcal H_3\) código de Hamming.
    \item La dimensión del código es \(k = 7 - 4 = 3\), luego el código generado es un \([7, 3]\)-código lineal.
    La matriz generadora que nos proporciona el teorema \ref{th:corr-cod-div}(\ref{thi:corr-cod-div:mat-gen}) es \[
      G = \left(\begin{array}{rrrrrrr}
        1 & 0 & 1 & 1 & 1 & 0 & 0 \\
        0 & 1 & 0 & 1 & 1 & 1 & 0 \\
        0 & 0 & 1 & 0 & 1 & 1 & 1
        \end{array}\right).
    \]
    La matriz de paridad en este caso es \[
      H = \left(\begin{array}{rrrrrrr}
        1 & 1 & 0 & 1 & 0 & 0 & 0 \\
        0 & 1 & 1 & 0 & 1 & 0 & 0 \\
        0 & 0 & 1 & 1 & 0 & 1 & 0 \\
        0 & 0 & 0 & 1 & 1 & 0 & 1
        \end{array}\right)
    \]
    y por tanto el código generado es un \(\mathcal H_4\) código de Hamming.
    \item La dimensión del código es \(k = 7 - 4 = 3\), luego el código generado es un \([7, 3]\)-código lineal.
    La matriz generadora que nos proporciona el teorema \ref{th:corr-cod-div}(\ref{thi:corr-cod-div:mat-gen}) es \[
      G = \left(\begin{array}{rrrrrrr}
        1 & 1 & 1 & 0 & 1 & 0 & 0 \\
        0 & 1 & 1 & 1 & 0 & 1 & 0 \\
        0 & 0 & 1 & 1 & 1 & 0 & 1
        \end{array}\right).
    \]
    La matriz de paridad en este caso es \[
      H = \left(\begin{array}{rrrrrrr}
        1 & 0 & 1 & 1 & 0 & 0 & 0 \\
        0 & 1 & 0 & 1 & 1 & 0 & 0 \\
        0 & 0 & 1 & 0 & 1 & 1 & 0 \\
        0 & 0 & 0 & 1 & 0 & 1 & 1
        \end{array}\right)
    \]
    y por tanto el código generado es un \(\mathcal H_4\) código de Hamming.
    \item La dimensión del código es \(k = 7 - 6 = 1\), luego el código generado es un \([7, 1]\)-código lineal.
    La matriz generadora que nos proporciona el teorema \ref{th:corr-cod-div}(\ref{thi:corr-cod-div:mat-gen}) es \[
      G = \left(\begin{array}{rrrrrrr}
        1 & 1 & 1 & 1 & 1 & 1 & 1
        \end{array}\right),
    \] por lo que concluimos que el código generado es el código de repetición de longitud \(7\).
    \item La dimensión del código es \(k = 7 - 7 = 0\), luego el código generado es \(\{\mathbf 0\}\).
  \end{enumerate}
\end{example}

Finalmente, el siguiente resultado nos muestra la relación entre dos polinomios generadores cuando un código es subcódigo de otro.

\begin{corollary}
  \label{cor:subcodigos-ciclicos}
  Sean \(\mathcal C_1\) y \(\mathcal C_2\) códigos cíclicos sobre \(\mathbb F_q\) con polinomios generadores \(g_1(x)\) y \(g_2(x)\), respectivamente.
  Entonces, \(\mathcal C_1 \subseteq \mathcal C_2 \) si y solo si \(g_2(x) \mid g_1(x)\).
\end{corollary}

\begin{proof}
  Recordamos que por el teorema \ref{th:corr-cod-div}(\ref{thi:corr-codc-div:forma-elem}) todos los elementos \(a(x)\) de los códigos cíclicos \(\mathcal C_1\) y \(\mathcal C_2 \in \mathcal R_n\) pueden expresarse como \(a(x) = g_i(x)f_i(x)\) —donde, o bien\(f_i(x) = 0\), o bien \(\deg(f_i(x)) < k = n - \deg(g_i(x))\)— para \(i = 1, 2\) respectivamente.
  Veamos ambas implicaciones por separado.
  \begin{enumerate}
    \item Comenzamos con que si \(g_2(x) | g_1(x)\) entonces \(\mathcal C_1 \subseteq \mathcal C_2\).
    Por hipótesis podemos expresar \(g_1(x) = r(x)g_2(x)\) para algún polinomio \(r(x)\).
    Así, todo elemento \(a(x)\) de \(\mathcal C_1\) puede expresarse como \(g_1(x)f_1(x) = r(x)g_2(x)f_1(x) = g_2(x)f_2(x)\) para algún \(f_2(x)\), por lo que si \(a(x) \in \mathcal C_1\), \(a(x) \in \mathcal C_2\).
    Por tanto, \(\mathcal C_1 \subseteq \mathcal C_2\).
    \item Vemos a continuación que si \(\mathcal C_1 \subseteq \mathcal C_2\) entonces \(g_2(x) | g_1(x)\).
    Vamos a usar un argumento similar al anterior.
    Como \(\mathcal C_1 \subseteq \mathcal C_2\) todo elemento de \(\mathcal C_1\) puede expresarse como \(g_1(x)f_1(x) = g_2(x)f_2(x)\) para ciertos \(f_1(x)\), \(f_2(x)\).
    Por tanto, \(g_1(x) = g_2(x)f_2(x)/f_1(x)\) y en consecuencia, \(g_2 | g_1(x)\), como queríamos.\qedhere
  \end{enumerate}
\end{proof}

\section{Codificación de códigos cíclicos}

%Los códigos cíclicos son más sencillos de decodificar que otros tipos de códigos debido a su estructura adicional.
Vamos a ver a continuación tres tipos de codificación de códigos cíclicos.
Consideraremos un código cíclico \(\mathcal C\) de longitud \(n\) sobre \(\mathbb F_q\) con polinomio generador \(g(x)\) de grado \(n - k\), por lo que \(\mathcal C\) tiene dimensión \(k\).

\paragraph{Codificación no-sistemática}

Esta forma de codificación está basada en la técnica natural de codificación que describimos en la sección \ref{subsec:codificacion-descodificacion}.
Sea \(G\) la matriz generadora obtenida a partir de los desplazamientos de \(g(x)\) descrita en el teorema \ref{th:corr-cod-div}.
Dado el mensaje \(\mathbf m \in \mathbb F_q^k\), lo codificamos como la palabra código \(\mathbf c = \mathbf mG\).
De igual forma, si \(m(x)\) y \(c(x)\) son los polinomios en \(\mathbb F_q[x]\) asociados a \(\mathbf{m}\) y \(\mathbf c\), entonces \(c(x) = m(x)g(x)\).

\paragraph{Codificación sistemática}

El polinomio \(m(x)\) asociado a un mensaje \(\mathbf m\) tendrá como mucho grado \(k -1\).
Por tanto, el polinomio \(n^{n-k}m(x)\) tendrá como mucho grado \(n - 1\) y sus primeros \( n - k\) coeficientes son nulos.
Por tanto, el mensaje está contenido en los coeficientes de \(x^{n-k}, x^{n-k+1}, \dots, x^{n-1}\).
Por el algoritmo de división tenemos que
\[x^{n-k}m(x) = g(x)a(x) + r(x), \qquad \text{donde } \operatorname{gr} r(x) < n - k \text{ o } r(x) = 0.\]
Sea \(c(x) = x^{n-k}m(x) - r(x)\).
Como \(c(x)\) es múltiplo de \(g(x)\), \(c(x) \in \mathcal C\).
El polinomio \(c(x)\) difiere de \(x^{n-k}m(x)\) en los coeficientes de \(1, x, \dots, x^{n-k-1}\) ya que \(\operatorname{gr} r(x) < n-k\).
Por tanto, \(c(x)\) contiene el mensaje \(\mathbf m\) en los coeficientes de los términos de grado al menos \(n - k\).

%\paragraph{Codificación sistemática usando el código dual}

\begin{example}
  Sea \(\mathcal C\) un código cíclico de longitud 15 con polinomio generador \(g(x) = (1 + x + x^4)(1 + x + x^2 + x^3 + x^4)\). Supongamos que queremos codificar el mensaje \(m(x) = 1 + x^2 + x^5\). Vamos a ver su codificación con los dos métodos descritos. Como la longitud de \(\mathcal C\) es \(15\) y el grado de su polinomio generador es \(8\), la dimensión del código es \(15 - 8 = 7\). Escribimos el mensaje \(m(x)\) en forma de vector: \(\mathbf{m}= (1, 0, 1, 0, 0, 1, 0)\). Una matriz generadora del código \(\mathcal C\) es: 
  \[
    G = \left(\begin{array}{rrrrrrrrrrrrrrr}
      1 & 0 & 0 & 0 & 1 & 0 & 1 & 1 & 1 & 0 & 0 & 0 & 0 & 0 & 0 \\
      0 & 1 & 0 & 0 & 0 & 1 & 0 & 1 & 1 & 1 & 0 & 0 & 0 & 0 & 0 \\
      0 & 0 & 1 & 0 & 0 & 0 & 1 & 0 & 1 & 1 & 1 & 0 & 0 & 0 & 0 \\
      0 & 0 & 0 & 1 & 0 & 0 & 0 & 1 & 0 & 1 & 1 & 1 & 0 & 0 & 0 \\
      0 & 0 & 0 & 0 & 1 & 0 & 0 & 0 & 1 & 0 & 1 & 1 & 1 & 0 & 0 \\
      0 & 0 & 0 & 0 & 0 & 1 & 0 & 0 & 0 & 1 & 0 & 1 & 1 & 1 & 0 \\
      0 & 0 & 0 & 0 & 0 & 0 & 1 & 0 & 0 & 0 & 1 & 0 & 1 & 1 & 1
      \end{array}\right).
  \]
  \begin{enumerate}
    \item Codificación no-sistemática. Simplemente multiplicamos \(\mathbf{m}\) por \(G\), obteniendo:
    \[
      \mathbf{c} = \mathbf{m}G = \left(1,\,0,\,1,\,0,\,1,\,1,\,0,\,1,\,0,\,0,\,1,\,1,\,1,\,1,\,0\right).
    \]
    \item Codificación sistemática. Calculamos el cociente de \(x^{n-k}m(x)\) por \(g(x)\) para obtener el resto \(r(x) = x^{6} + x + 1\). Entonces, la palabra código viene dada por \(c(x) = x^{n-k}m(x) - r(x) = x^{13} + x^{10} + x^{8} + x^{6} + x + 1\), que en forma de vector resulta \(\mathbf{c} = \left(1,\,1,\,0,\,0,\,0,\,0,\,1,\,0,\,1,\,0,\,1,\,0,\,0,\,1,\,0\right)\). Observamos que la codificación es efectivamente sistemática: nuestro mensaje \(m\) está contenido íntegramente en las últimas \(7\) coordenadas.
  \end{enumerate}
\end{example}

\section{Idempotentes y multiplicadores}

En esta sección vamos a estudiar otra forma alternativa de generar los códigos cíclicos en \(\mathcal R_n\).
Se basará en encontrar unos elementos concretos de \(\mathcal R_n\) que además podremos relacionar con los polinomios generadores que hemos definido hasta ahora.

Como ya vimos en la sección (cita) un elemento \(e\) de un anillo es idempotente si \(e^2 = e\).
Partiendo de la suposición de que \(\operatorname{mcd}(n, q) = 1\) afirmamos que el anillo \(\mathcal R_n\) es semisimple.
Esto implica, además de lo que ya comentamos, que cada ideal de \(\mathcal R_n\) tiene un único elemento idempotente que lo genera.
Este elemento se denomina \textit{idempotente generador} del código cíclico.
En el siguiente teorema probaremos este hecho y mostraremos además un método para determinar el idempotente generador de un código cíclico a partir de su polinomio generador.

\begin{theorem}
  \label{th:idempotente-unico-unidad}
  Sea \(\mathcal C\) un código cíclico en \(\mathcal R_n\). Entonces:
  \begin{enumerate}
    \item Existe un único idempotente \(e(x) \in \mathcal C\) tal que \(\mathcal C = \langle e(x)\rangle\).
    \item \label{th:idempotente-unico-unidad:unidad} Si \(e(x)\) es un idempotente no nulo en \(\mathcal C\), entonces \(\mathcal C = \langle e(x)\rangle\) si y solo si \(e(x)\) es una unidad de \(\mathcal C\).
  \end{enumerate}
\end{theorem}

\begin{proof}
  Si \(\mathcal C\) es el código cero, entonces el idempotente es el cero, con lo que (1) está claro y (2) no se aplica a este caso. 
  Veamos entonces la demostración por apartados suponiendo que \(\mathcal C\) es distinto de cero.
  \begin{enumerate}
    \item Supongamos primero que \(e(x)\) es una unidad en \(\mathcal C\). 
    Entonces, \(\langle e(x)\rangle \subset \mathcal C\), ya que \(\mathcal C\) es un ideal.
    Si \(c(x) \in \mathcal C\), entonces \(c(x)e(x) = c(x)\) en \(\mathcal C\).
    En consecuencia, \(\langle e(x)\rangle = \mathcal C\).
    Por otro lado, supongamos que \(e(x)\) es un idempotente distinto de cero y tal que \(\mathcal C = \langle e(x)\rangle\).
    Entonces, cada elemento \(c(x)\) lo podemos escribir como \(c(x) = f(x)e(x)\).
    Pero se tiene que \(c(x)e(x) = f(x)(e(x))^2 = f(x)e(x) = c(x)\), luego \(e(x)\) es la unidad de \(\mathcal C\).
    \item Tenemos que probar la existencia y la unicidad.
    Comenzamos con la existencia.
    Sea \(g(x)\) el polinomio generador dde \(\mathcal C\).
    Entonces, sabemos que \(g(x) \mid (x^n - 1)\) por el teorema \ref{th:corr-cod-div}.
    Tomemos \(h(x) = (x^n - 1)/g(x)\).
    Sabemos que \(\operatorname{mcd}(g(x), h(x)) = 1\) en \(\mathbb F_q[x]\), ya que \(x^n - 1\) tiene todas sus raíces distintas.
    En consecuencia, el algoritmo de Euclides nos proporciona los polinomios \(a(x),b(x) \in \mathbb F_q[x]\) tales que \(a(x)g(x) + b(x)h(x) = 1\).
    Llamemos \(e(x) \equiv a(x)g(x) \bmod x^n - 1\), que será el representante de dicha clase de equivalencia en \(\mathcal R_n\).
    Entonces, en \(\mathcal R_n\),
    \begin{align*}
      e(x)^2 &\equiv (a(x)g(x))(1 - b(x)h(x)) \bmod x^n - 1\\
        &\equiv a(x)g(x) - a(x)g(x)b(x)h(x) \bmod x^n - 1\\
        &\equiv a(x)g(x) - a(x)b(x)(x^n - 1) \bmod x^n - 1\\
        &\equiv a(x)g(x) \bmod x^n - 1\\
        &\equiv e(x) \bmod x^n - 1.
    \end{align*}
    Por tanto, este elemento \(e(x)\) es idempotente.
    Veamos ahora que si \(c(x) \in \mathcal C\), entonces \(c(x) = f(x)g(x)\), luego
    \begin{align*}
      c(x)e(x) &= f(x)g(x)(1 - b(x)h(x))\\
        &\equiv f(x)g(x) \bmod x^n - 1\\
        &\equiv c(x) \bmod x^n - 1,
    \end{align*}
    por lo que \(e(x)\) es una unidad en \(\mathcal C\).
    En consecuencia, podemos deducir la existencia a partir de (2).
    Veamos ahora la unicidad. Por (2), si tenemos dos elementos idempotentes \(e_1(x)\) y \(e_2(x)\) que generan \(\mathcal C\), ambos han de ser unidades, y en consecuencia se tiene que \(e_1(x) = e_1(x)e_2(x) = e_2(x)\), con lo que podemos deducir la unicidad.\qedhere
  \end{enumerate}
\end{proof}

Deducimos por tanto que un método para encontrar el idempotente generador \(e(x)\) de un código cíclico \(\mathcal C\) a partir del polinomio generador \(g(x)\) es resolver la ecuación \[1 = a(x)g(x) + b(x)h(x)\] para \(a(x)\) utilizando el algoritmo de Euclides, donde \(h(x) = (x^n - 1)/g(x)\).
Entonces, reduciendo \(a(x)g(x)\) módulo \(x^n - 1\) obtenemos el idempotente \(e(x)\) que buscamos.
Pero vamos a ver además esta relación a la inversa, es decir, que podemos obtener el polinomio generador \(g(x)\) a partir del idempotente \(e(x)\).

\begin{theorem}
  Sea \(\mathcal C\) un código cíclico sobre \(\mathbb F_q\) con idempotente generador \(e(x)\).
  Entonces, el polinomio generador de \(\mathcal C\) es \(g(x) = \operatorname{mcd}(e(x), x^n - 1)\), calculado en \(\mathbb F_q[x]\). 
\end{theorem}

\begin{proof}
  Sea \(d(x) = \operatorname{mcd}(e(x), x^n - 1)\) en \(\mathbb F_q[x]\) y sea \(g(x)\) el polinomio generador de \(\mathcal C\).
  Como \(d(x) \mid e(x)\), podemos expresarlo como \(e(x) = d(x)k(x)\) para algún \(k(x) \in \mathbb F_q[x]\).
  Por tanto cada elemento de \(\mathcal C = \langle e(x) \rangle\) es también múltiplo de \(d(x)\), por lo que \(\mathcal C \subset \langle d(x) \rangle\).
  Por el teorema \ref{th:corr-cod-div} tenemos que en \(\mathbb F_q[x]\), \(g(x) \mid (x^n -1)\) y que \(g(x) \mid e(x)\), ya que \(e(x) \in \mathcal C\).
  Luego, por la proposición \ref{prop:k-divisor-f-g} tenemos que \(g(x) \mid d(x)\) y en consecuencia \(d(x) \in \mathcal C\).
  Por tanto, \(\langle d(x) \rangle \subseteq \mathcal C\) y deducimos entonces que \(\mathcal C = \langle d(x) \rangle\).
  Como \(d(x)\) es divisor mónico de \(x^n - 1\) y genera a \(\mathcal C\), necesariamente \(d(x) = g(x)\) por el corolario \ref{cor:peso-minimo-columnas-dependientes}. 
\end{proof}

\begin{example}
  Continuando con el ejemplo \ref{ex:codigos-ciclicos-long-7} en el que describimos todos los códigos cíclicos binarios de longitud \(7\) vamos a indicar a continuación cuales son los idempotentes generadores de cada uno.
  Para ello vamos a utilizar el código descrito en el anexo \ref{annex:sage-gen-idemp} tal y como mostramos en el listado siguiente.
  \begin{lstlisting}[gobble=4]
    sage: F = GF(2)
    sage: x = polygen(F)
    sage: (x^7 - 1).factor()
    > (x + 1) * (x^3 + x + 1) * (x^3 + x^2 + 1)
    sage: print(generadores_idempotentes(x^7 - 1))
    > [(1, 1),
       (x + 1, x^6 + x^5 + x^4 + x^3 + x^2 + x),
       (x^3 + x + 1, x^4 + x^2 + x),
       (x^3 + x^2 + 1, x^6 + x^5 + x^3),
       (x^4 + x^3 + x^2 + 1, x^6 + x^5 + x^3 + 1),
       (x^4 + x^2 + x + 1, x^4 + x^2 + x + 1),
       (x^6 + x^5 + x^4 + x^3 + x^2 + x + 1, x^6 + x^5 + x^4 + x^3 + x^2 + x + 1),
       (x^7 + 1, 0)]
  \end{lstlisting}
  En la tabla \ref{tab:gen-idempotentes-7} mostramos de forma más clara cuáles son los generadores y cuáles los idempotentes correspondientes.
  \begin{table}[h]
    \centering
    \sffamily
    \begin{tabular}{lcll}
      \toprule
      \(i\) & dimensión & generador \(g_i(x)\) & idempotente \(e_i(x)\)\\
      \midrule
      \(0\) & \(0\) & \(x^7 + 1\) & \(0\)\\
      \(1\) & \(1\) & \(x^6 + x^5 + \dots + x + 1\) & \(x^6 + x^5 + \dots + x + 1\)\\
      \(2\) & \(3\) & \(x^4 + x^3 + x^2 + 1\) & \(x^6 + x^5 + x^3 + 1\)\\
      \(3\) & \(3\) & \(x^4 + x^2 + x +1 \) & \(x^4 + x^2 + x +1 \)\\
      \(4\) & \(4\) & \(x^3 + x + x +1 \) & \(x^4 + x^2 + x\)\\
      \(5\) & \(4\) & \(x^3 + x^2 +1 \) & \(x^6 + x^5 + x^3\)\\
      \(6\) & \(6\) & \(x + 1\) & \(x^6 + x^5 + \dots + x\)\\
      \(7\) & \(7\) & \(1\) & \(1\)\\
      \bottomrule
    \end{tabular}
    \caption{Polinomios generadores e idempotentes para los códigos cíclicos de longitud 7}
    \label{tab:gen-idempotentes-7}
  \end{table}
\end{example}

Puesto que los idempotentes generadores producen códigos cíclicos es de rigor preguntarse si a partir de los idempotentes podemos obtener una base de los códigos generados, tal y como ocurre con los polinomios generadores.
El siguiente teorema nos dice que sí, y además de la misma forma: a partir de los primeros \(k - 1\) desplazamientos cíclicos del idempotente generador.

\begin{theorem}
  Sea \(\mathcal C\) un \([n, k]\) código cíclico con idempotente generador \(e(x) = \sum_{i=0}^{n-1}e_ix^i\).
  Entonces, la matriz \(k \times n\)
  \[
    \begin{pmatrix*}
      e_0 & e_1 & e_2 & \dots & e_{n-2} & e_{n-1} \\
      e_{n-1} & e_0 & e_1 & \dots & e_{n-3} & e_{n-2} \\
       & & & \vdots & & \\
      e_{n-k+1} & e_{n-k+2} & e_{n-k+3} & \dots & e_{n-k-1} & e_{n-k}
    \end{pmatrix*}
  \] es una matriz generadora de \(\mathcal C\).
\end{theorem}

\begin{proof}
  Probar este resultado equivale a probar que el conjunto \(\{e(x), xe(x), \dots, x^{k-1}e(x)\}\) es una base de \(\mathcal C\).
  Entonces, solo hay que probar que si \(a(x) \in \mathbb F_q[x]\) tiene grado menor que \(k\), tal que \(a(x)e(x) = 0\), se tiene que \(a(x) = 0\).
  Sea \(g(x)\) el polinomio generador de \(\mathcal C\).
  Si \(a(x)e(x) = 0\), entonces \(0 = a(x)e(x)g(x) = a(x)g(x)\), tal que \(e(x)\) es la unidad de \(\mathcal C\) según el teorema \ref{th:idempotente-unico-unidad}, y por tanto, si \(a(x)\) no es cero estaríamos contradiciendo el teorema \ref{th:corr-cod-div}.
\end{proof}

El siguiente resultado, que nos informa sobre los polinomios generadores e idempotentes generadores de sumas e intersecciones de códigos cíclicos de la misma longitud, nos será útil un poco más adelante, pues veremos que .
Dados dos códigos cíclicos \(\mathcal C_1\) y \(\mathcal C_2\) de longitud \(n\) sobre \(\mathbb F_q\) definimos su suma como
\[
  \mathcal C_1 + \mathcal C_2 = \{\mathbf{c}_1 + c_2(x) : \mathbf{c}_1 \in \mathcal C_1 \text{ y } \mathbf{c}_2\}.
\]

\begin{theorem}
  \label{th:intersecciones-sumas-ciclicos}
  Sean \(\mathcal C_1\) y \(\mathcal C_2\) códigos cíclicos de longitud \(n\) sobre \(\mathbb F_q\) con polinomios generadores \(g_1(x)\) y \(g_2(x)\) e idempotentes generadores \(e_1(x)\) y \(e_2(x)\), respectivamente.
  Entonces \begin{enumerate}
    \item La intersección \(\mathcal C_1 \cap \mathcal C_2\) es también un código cíclico, con polinomio generador \(\operatorname{mcm}(g_1(x), g_2(x))\) e idempotente generador \(e_1(x)e_2(x)\).
    \item La suma \(\mathcal C_1 + \mathcal C_2\) es también un código cíclico, con polinomio generador \(\operatorname{mcd}(g_1(x), g_2(x))\) e idempotente generador \(e_1(x) + e_2(x) - e_1(x)e_2(x)\).
    \label{th:intersecciones-sumas-ciclicos:sumas}
  \end{enumerate}
\end{theorem}

\begin{proof}
  Veamos la demostración por apartados.
  \begin{enumerate}
    \item La intersección \(\mathcal C_1 \cap \mathcal C_2\) es un subcódigo y por tanto, por el corolario \ref{cor:subcodigos-ciclicos} es un código cíclico.
    Por el mismo corolario su polinomio generador debe ser divisible por \(g_1(x)\) y \(g_2(x)\), por lo que ha de ser divisible por el \(g(x) = \operatorname{mcm}(g_1(x), g_2(x))\).
    Así, \(g(x)\) es un polinomio generador de un código cíclico que está contenido tanto en \(\mathcal C_1\) como en \(\mathcal C_2\).
    Por tanto, \(g(x)\) ha de ser el generador de \(\mathcal C_1 \cap \mathcal C_2\), pues si no lo fuese, el código cíclico generador por \(g(x)\) ha de ser mayor que la intersección, lo que contradice la propia definición de intersección.
    Veamos ahora que el idempotente generador es \(e_1(x)e_2(x)\).
    Claramente \(e_1(x)e_2(x) \in \mathcal C_1 \cap \mathcal C_2\) y es idempotente, pues \((e_1(x)e_2(x))^2 = e_1(x)^2e_2(x)^2 = e_1(x)e_2(x)\).
    Si \(c(x) \in \mathcal C_1 \cap \mathcal C_2\) entonces \(e_1(x)e_2(x)c(x) = e_1(x)c(x) = c(x)\), pues por el teorema \ref{th:idempotente-unico-unidad}(\ref{th:idempotente-unico-unidad:unidad}) \(e_1\) y \(e_2\) son unidades de \(\mathcal C_1\) y \(\mathcal C_2\), respectivamente.
    El mismo teorema nos asegura entonces que \(e_1(x)e_2(x)\) es el idempotente generador que buscamos.
    \item Veamos primero que \(\mathcal C_1 + \mathcal C_2\) es un código cíclico.
    Sabemos que si \(c_1(x) \in \mathcal C_1\) y \(c_2(x) \in \mathcal C_2\) entonces \(xc_1(x) \in \mathcal C_1\) y \(xc_2(x) \in \mathcal C_2\).
    Dado un elemento \(c_1(x) + c_2(x) \in \mathcal C_1 + \mathcal C_2\) tenemos que \(x(c_1(x) + c_2(x)) = xc_1(x) + xc_2(x) \in \mathcal C_1 + \mathcal C_2\), por lo que \(\mathcal C_1 + \mathcal C_2\) es cíclico.
    A continuación, sea \(g(x) = \operatorname{mcd}(g_1(x), g_2(x))\).
    El algoritmo de Euclides nos proporciona \(a(x)\) y \(b(x) \in \mathbb F_q[x]\) tales que \(g(x) = g_1(x)a(x) + g_2(x)b(x)\).
    Por tanto, \(g(x) \in \mathcal C_1 + \mathcal C_2\).
    Como \(\mathcal C_1 + \mathcal C_2\) es cíclico, \(\langle g(x) \rangle \subseteq \mathcal C_1 + \mathcal C_2\).
    Por otro lado \(g(x) | g_1(x)\) luego por el corolario \ref{cor:subcodigos-ciclicos} \(\mathcal C_1 \subseteq \langle g(x) \rangle\).
    De la misma forma deducimos que \(\mathcal C_2 \subseteq \langle g(x) \rangle\) y por tanto \(\mathcal C_1 + \mathcal C_2 \subseteq \langle g(x) \rangle\).
    Así, \(\mathcal C_1 + \mathcal C_2 = \langle g(x) \rangle\).
    Se tiene que \(g(x) | (x^n - 1)\) puesto que \(g(x) | g_1(x)\).
    Además, como \(g(x)\) es mónico, se tiene por el corolario \ref{cor:pol-gen-unico} que \(g(x) = \operatorname{mcd}(g_1(x), g_2(x))\) es el polinomio generador de \(\mathcal C_1 + \mathcal C_2\).
    Veamos finalmente que dado \(c(x) = c_1(x) + c_2(x)\), con \(c_1 \in \mathcal C_1\) y \(c_2 \in \mathcal C_2\) se tiene que 
    \begin{align*}
      c(&x)(e_1(x) + e_2(x) - e_1(x)e_2(x))\\
      &= c_1(x) + c_1(x)e_2(x) - c_1(x)e_2(x) + c_2(x)e_1(x) + c_2(x) - c_2(x)e_1(x)\\
      &=  c_1(x) + c_2(x)\\
      &= c(x).
    \end{align*}
    Por tanto, por el teorema \ref{th:idempotente-unico-unidad} obtenemos que \(e_1(x) + e_2(x) - e_1(x)e_2(x) \in \mathcal C_1 + \mathcal C_2\) es el idempotente generador, como queríamos demostrar.\qedhere
  \end{enumerate}
\end{proof}

Estamos ya en disposición de describir los elementos que prometimos al comienzo de la sección.
Nos permitirán obtener todos los idempotentes en \(\mathcal R_n\), y en consecuencia, todos los códigos cíclicos en \(\mathcal R_n\).
Son los conocidos como \emph{idempotentes primitivos}.

Consideremos la descomposición en factores \(x^n - 1 = f_1(x)\cdots f_s(x)\), donde cada polinomio \(f_i(x)\) es irreducible sobre \(\mathbb F_q\) para \(1 \leq i \leq s\).
Sabemos que los factores \(f_i(x)\) son distintos, pues estamos en el supuesto de que \(x^n - 1\) tiene raíces distintas.
Sea \(\widehat{f_i}(x) = (x^n - 1)/f_i(x)\).
En el teorema \ref{th:idempotentes-ideales-minimales} a continuación vamos a ver que los ideales \(\langle \widehat{f_i}(x)\rangle\) de \(\mathcal R_n\) son los ideales minimales de \(\mathcal R_n\) y cómo podemos obtener \(\mathcal R_n\) a partir de ellos.
Al idempotente generador de \(\langle \widehat{f_i}(x)\rangle\) lo denotaremos por \(\widehat{e_i}(x)\).
Los elementos idempotentes \(\widehat{e_1}(x), \dots, \widehat{e_s}(x)\) son los \emph{idempotentes primitivos} de \(\mathcal R_n\).
El teorema \ref{th:idempotentes-ideales-minimales} que sigue nos muestra además la forma de obtener todos los idempotentes de \(\mathcal R_n\) a partir de los idempotentes primitivos.

\begin{theorem}
  \label{th:idempotentes-ideales-minimales}
  En \(\mathcal R_n\) se verifican las siguientes afirmaciones.
  \begin{enumerate}
    \item Los ideales \(\langle \widehat{f_i}(x)\rangle\) para cada \(1 \leq i \leq s\) son todos los ideales minimales de \(\mathcal R_n\).
    \item \(\mathcal R_n\) es el espacio vectorial suma directa de todos los \(\langle \widehat{f_i}(x)\rangle\) para \(1 \leq i \leq s\).
    \label{thi:idempotentes-ideales-minimales:suma-directa}
    \item Si \(i \neq j\) entonces \(\widehat{e_i}(x)\widehat{e_j}(x) = 0\) en \(\mathcal R_n\).
    \item \label{thi:idempotentes-ideales-minimales:cero}
    \item \label{thi:idempotentes-ideales-minimales:suma-idempotentes} La suma \(\sum_{i=1}^s \widehat{e_i}(x) = 1\) en \(\mathcal R_n\).
    \item \label{thi:idempotentes-ideales-minimales:unicos-idempotentes} Los únicos idempotentes en \(\langle \widehat{f_i}(x)\rangle\) son \(0\) y \(\widehat{e_i}(x)\).
    \item Si \(e(x)\) es un idempotente no nulo en \(\mathcal R_n\), entonces existe un subconjunto \(T\) de \(\{1, 2, \dots, s\}\) tal que \(e(x) = \sum_{i \in T}\widehat{e_i}(x)\) y \(\langle e(x) \rangle = \sum_{i \in T}\langle \widehat{f_i}(x)\rangle\).
  \end{enumerate}
\end{theorem}

\begin{proof}
  Veamos la demostración por apartados.
  \begin{enumerate}
    \item Veamos por reducción al absurdo que cada \(\langle \widehat{f_i}(x)\rangle\) es un ideal minimal de \(\mathcal R_n\).
    Supongamos que no es un ideal minimal.
    Entonces, por el corolario \ref{cor:subcodigos-ciclicos} existiría un polinomio generador \(g(x)\) de un ideal no trivial contenido en \(\langle \widehat{f_i}(x)\rangle\) tal que \(\widehat{f_i}(x) | g(x)\), con \(g(x) \neq \widehat{f_i}(x)\).
    Pero como \(f_i(x)\) es irreducible y \(g(x) | (x^n - 1)\), es imposible.
    Por tanto cada \(\langle \widehat{f_i}(x)\rangle\) es un ideal minimal de \(\mathcal R_n\).
    Veamos que estos son todos los ideales minimales de \(\mathcal R_n\).
    Sea \(\mathcal M = \langle m(x) \rangle\) un ideal minimal de \(\mathcal R_n\).
    Como el conjunto \(\{\widehat{f_i}(x) : 1 \leq i \leq s\}\) no tiene factores irreducibles de \(x^n - 1\) repetidos y cada uno de ellos divide a \(x^n - 1\) el \(\operatorname{mcd}(\widehat{f_1}(x), \dots, \widehat{f_s}(x)) = 1\).
    Por tanto, aplicando el algoritmo de Euclides inductivamente obtenemos
    \begin{equation}
      1 = \sum_{i = 1}^s a_i(x)\widehat{f_i}(x)
      \label{eq:1-suma-ideales-minimales}
    \end{equation}
    para ciertos \(a_i(x) \in \mathbb F_q[x]\).
    Así, como 
    \[
      0 \neq m(x) = m(x) \cdot 1 = \sum_{i = 1}^s m(x)a_i(x)\widehat{f_i}(x)
    \]
    existe un \(i\) tal que \(m(x)a_i(x)\widehat{f_i}(x) \neq 0\).
    Por tanto, \(\mathcal M \cap \langle \widehat{f_i}(x) \rangle \neq \{0\}\), pues \(m(x)a_i(x)\widehat{f_i}(x) \in \mathcal M \cap \langle \widehat{f_i}(x) \rangle\).
    Pero entonces \(\mathcal M = \langle \widehat{f_i}(x) \rangle\) pues tanto \(\mathcal M\) como \(\langle \widehat{f_i}(x) \rangle\) son minimales.
    Por tanto todos los ideales minimales son de la forma \(\langle \widehat{f_i}(x) \rangle\), como queríamos.
    \item Por (\ref{eq:1-suma-ideales-minimales}) concluimos que el \(1\) está en la suma de los ideales \(\langle \widehat{f_i}(x) \rangle\), que es en sí mismo un ideal de \(\mathcal R_n\).
    Por tanto, por la proposición \ref{prop:ideal-unidad}, \(\mathcal R_n\) es la suma de los ideales \(\langle \widehat{f_i}(x) \rangle\).
    Para probar que es una suma directa tenemos que comprobar que los ideales son disjuntos, es decir, \(\langle \widehat{f_i}(x) \rangle \cap \sum_{j\neq i} \langle \widehat{f_j}(x) \rangle = \{0\}\) para \(1 \leq i \leq s\).
    Como \(f_i(x) | \widehat{f_j}(x)\) para \(j \neq i\), \(f_j(x) \not| \widehat{f_j}(x)\) y los factores irreducibles de \(x^n - 1\) son todos distintos, concluimos que
    \[
      f_i(x) = \operatorname{mcd}\{\widehat{f_j}(x) : 1 \leq j \leq s, j \neq i\}.
    \]
    Utilizando inducción sobre el teorema \ref{th:intersecciones-sumas-ciclicos}(\ref{th:intersecciones-sumas-ciclicos:sumas}) concluimos que \(\langle \widehat{f_i}(x) \rangle = \sum_{j\neq i}\langle \widehat{f_j}(x) \rangle\).
    Por tanto, 
    \begin{align*}
      \langle \widehat{f_i}(x) \rangle \cap \sum_{j\neq i}\langle \widehat{f_j}(x) \rangle 
       &= \langle \widehat{f_i}(x) \rangle \cap \langle f_i(x) \rangle \\
       &= \langle\operatorname{mcm}(\widehat{f_i}(x), f_i(x))\rangle \\ 
       &= \langle x^n - 1\rangle \\
       &= \{0\},
    \end{align*}
    por lo que los \(\langle \widehat{f_i}(x) \rangle\) son disjuntos y la suma es directa, como queríamos ver.
    \item Si \(i \neq j\), \(\widehat{e_i}(x)\widehat{e_j}(x) \in \langle \widehat{f_i}(x) \rangle \cap \langle \widehat{f_j}(x) \rangle = \{0\}\) por (\ref{thi:idempotentes-ideales-minimales:suma-directa}), luego \(\widehat{e_i}(x)\widehat{e_j}(x) = 0\) como queríamos.
    \item Usando (\ref{thi:idempotentes-ideales-minimales:cero}) y aplicando inducción al teorema \ref{th:intersecciones-sumas-ciclicos}(\ref{th:intersecciones-sumas-ciclicos:sumas}) obtenemos que \(\sum_{i=1}^s \widehat{e_i}(x)\) es el idempotente generador de \(\sum_{i=1}^s \langle \widehat{f_i}(x) \rangle = \mathcal R_n\), por (\ref{thi:idempotentes-ideales-minimales:suma-directa}).
    Luego el idempotente generador de \(\mathcal R_n\) es \(1\).
    \item Si \(e(x)\) es un idempotente no nulo en \(\langle \widehat{f_i}(x) \rangle\) entonces \(\langle e(x) \rangle\) es un ideal contenido en \(\langle \widehat{f_i}(x) \rangle\).
    Por minimalidad, dado que \(e(x)\) es distinto de cero, \(\langle \widehat{f_i}(x) \rangle = \langle e(x)\rangle\), lo que por el teorema \ref{th:idempotente-unico-unidad} implica que \(e(x) = \widehat{e_i}(x)\) ya que ambos son la unidad de \(\langle \widehat{f_i}(x) \rangle\).
    \item Notemos que \(e(x)\widehat{e_i}(x)\) es un idempotente en  \(\langle \widehat{f_i}(x) \rangle\).
    Por tanto, por (\ref{thi:idempotentes-ideales-minimales:unicos-idempotentes}), \(e(x)\widehat{e_i}(x)\) es, o bien \(0\) o bien \(\widehat{e_i}(x)\).
    Sea \(T = \{i : e(x)\widehat{e_i}(x) \neq 0\}\).
    Entonces, por (\ref{thi:idempotentes-ideales-minimales:suma-idempotentes}), \(e(x) = e(x) \cdot 1 = e(x)\sum_{i=1}^s \widehat{e_i}(x) = \sum_{i=1}^s e(x)\widehat{e_i}(x) = \sum_{i \in T}\widehat{e_i}(x)\).
    De hecho, \(\langle e(x)\rangle = \langle \sum_{i \in T}\widehat{e_i}(x)\) = \(\sum_{i \in T}\langle\widehat{e_i}(x)\rangle\) aplicando por inducción el teorema \ref{th:intersecciones-sumas-ciclicos}(\ref{th:intersecciones-sumas-ciclicos:sumas}).\qedhere
  \end{enumerate}
\end{proof}

El siguiente teorema nos muestra que los ideales minimales descritos en el teorema \ref{th:idempotentes-ideales-minimales} son extensiones de cuerpos de \(\mathbb F_q\).

\begin{theorem}
  Sea \(\mathcal M\) un ideal minimal de \(\mathcal R_n\).
  Entonces \(\mathcal M\) es una extensión de cuerpos de \(\mathbb F_q\).
\end{theorem}

\begin{proof}
  Basta con probar que cada elemento distinto de cero en \(\mathcal M\) tiene inverso para el producto.
  Sea \(a(x) \in \mathcal M\) distinto de cero.
  Entonces \(\langle a(x) \rangle\) es un ideal de \(\mathcal R_n\) distinto de cero contenido en \(\mathcal M\), y por tanto, \(\langle a(x) \rangle = \mathcal M\).
  Por tanto, si \(e(x)\) es la unidad de \(\mathcal M\) existe un elemento \(b(x) \in \mathcal R_n\) tal que \(a(x)b(x) = e(x)\).
  Sea ahora \(c(x) = b(x)e(x) \in \mathcal M\), pues \(e(x) \in \mathcal M\).
  Por tanto, \(a(x)c(x) = e(x)^2 = e(x)\), con lo que \(a(x)\) tiene inverso, como queríamos.
\end{proof}

A continuación vamos a describir una permutación que lleva idempotentes de \(\mathcal R_n\) en idempotentes de \(\mathcal R_n\).
Sea \(a\) un entero tal que \(\operatorname{mcd}(a, n) = 1\).
La función \(\mu_a\) definida sobre \(\{0, 1, \dots, n -1\}\) por \(i\mu_a \equiv ia \bmod n\) es una permutación de las posiciones de coordenadas \(\{0, 1, \dots, n - 1\}\) de un código cíclico de longitud \(n\) y se denomina \textit{multiplicador}.
Dado que los códigos cíclicos de longitud \(n\) se representan como ideales de \(\mathcal R_n\), para \(a > 0\) es conveniente interpretar que \(\mu_a\) actúa sobre \(\mathcal R_n\) como
\begin{equation}
  \label{eq:multiplier-rn}
  f(x)\mu_a \equiv f(x^a) \bmod x^n - 1.
\end{equation}

Esta ecuación es consistente con la definición original de \(\mu_a\) pues \(x^i\mu_a = x^{ia} = x^{ia + jn}\) en \(\mathcal R_n\) para un entero \(j\) tal que \(0 \leq ia + jn\), pues \(x^n = 1\) en \(\mathcal R_n\).
En otras palabras, \(x^i\mu_a = x^{ia \bmod n}\).
Si \(a < 0\) podemos dar significado a \(f(x^a)\) en \(\mathcal R_n\) definiendo \(x^{i}\mu_a = x^{ia \bmod n}\), donde \(0 \leq ia \bmod n < n\).
Con esta interpretación la ecuación (\ref{eq:multiplier-rn}) es consistente con la definición original de \(\mu_a\) cuando \(a < 0\).

\section{Ceros y conjuntos característicos}

En esta sección vamos a ver que podemos caracterizar los códigos cíclicos \(\mathcal R_n\) de otra forma: a partir de los ceros del polinomio \(x^n - 1\), es decir, a partir de ciertas raíces enésimas de la unidad.
Esta caracterización cobrará especial relevancia cuando estudiemos los conocidos como códigos \textacr{BCH}.

Como vimos en la sección \ref{sec:factorizacion-xn-1}, si \(t = \operatorname{ord}_n(q)\) entonces \(\mathbb F_{q^t}\) es un cuerpo de descomposición de \(x^n - 1\).
Por tanto, \(\mathbb F_{q^t}\) contiene una enésima raíz primitiva de la unidad \(\alpha\), y \(x^n - 1 = \prod_{i=0}^{n-1}(x - \alpha^{i})\) es la factorización de \(x^n - 1\) sobre \(\mathbb F_{q^t}\).
De hecho, \(x^n - 1 = \prod_s M_{\alpha^s}(x)\) es la factorización de \(x^n - 1\) en factores irreducibles sobre \(\mathbb F_q\), donde \(s\) varía en un conjunto de representantes de las clases \(q\)-ciclotómicas módulo \(n\).

Sea \(\mathcal C\) un código cíclico en \(\mathcal R_n\) con polinomio generador \(g(x)\).
Por los teoremas \ref{th:pol-minimal-raiz-primitiva}(\ref{thi:pol-minimal-raiz-primitiva-producto}) y \ref{th:corr-cod-div}(\ref{thi:pol-generador-prod-minimal}), podemos expresar el polinomio generador como \(g(x) = \prod_{s}M_{\alpha^s}(x) = \prod_s\prod_{i \in C_s}(x - \alpha^{i})\), donde \(s\) de nuevo varía en un conjunto \(C_s\) de representantes de las clases \(q\)-ciclotómicas módulo \(n\).
Sea \(T = \bigcup_s C_s\) la unión de estas clases \(q\)-ciclotómicas.
Las raíces de la unidad \(\mathcal Z = \{\alpha^{i} : i \in T\}\) se denominan los \textit{ceros} del código cíclico \(\mathcal C\) y los elementos \(\{\alpha^{i} : i \notin T\}\), los \textit{elementos no nulos} de \(\mathcal C\).
El conjunto \(T\) se denomina \textit{conjunto característico} de \(\mathcal C\).

\begin{theorem}
  \label{th:cicl-cto-caracteristico}
  Sea \(\alpha\) una raíz primitiva de la unidad en una extensión de cuerpos de \(\mathbb F_q\) y sea \(\mathcal C\) un código cíclico de longitud \(n\) sobre \(\mathbb F_q\) con conjunto característico \(T\) y polinomio generador \(g(x)\).
  Se verifica que: \begin{enumerate}
    %\item El polinomio generador se puede expresar como \(g(x) = \prod_{i \in T}(x - \alpha^i)\).
    \item Una palabra código \(c(x) \in \mathcal R_n\) está en \(\mathcal C\) si y solo si \(c(\alpha^i) = 0\) para todo \(i \in T\).
    \item La dimensión de \(\mathcal C\) es \(n - |T|\).
  \end{enumerate}
\end{theorem}

\begin{proof}
  Veamos la demostración por apartados.
  \begin{enumerate}
    \item Se deduce directamente del teorema \ref{th:corr-cod-div}, pues \(c(x)\) será un múltiplo del polinomio generador \(g(x)\) de \(\mathcal C\), que por \ref{th:corr-cod-div}(\ref{thi:pol-generador-prod-minimal}) verifica que \(g(\alpha^i) = 0\) para todo \(i \in T\).
    \item Se de deduce del teorema \ref{th:corr-cod-div}, pues \(|T|\) es el grado de \(g(x)\).\qedhere
  \end{enumerate}
\end{proof}

Es importante observar que \(T\), y por ello tanto el conjunto de ceros como el de elementos distintos de cero, determinan por completo el polinomio generador \(g(x)\).

\begin{example}
  Continuando con el ejemplo \ref{ex:codigos-ciclicos-long-7} en el que describimos todos los códigos cíclicos binarios de longitud \(7\) vamos a indicar a continuación cuales son los conjuntos característicos, tomando como \(\alpha = \zeta_7^3\).
  Para ello vamos a utilizar el código descrito en el anexo \ref{annex:sage-gen-idemp} tal y como mostramos en el listado siguiente.
  \begin{lstlisting}[gobble=4]
    sage: F = GF(2)
    sage: x = polygen(F)
    sage: ctos_caracteristicos(x^7 - 1)
    > [(1, [], z3),
       (x + 1, [0], z3),
       (x^3 + x + 1, [1, 2, 4], z3),
       (x^3 + x^2 + 1, [3, 5, 6], z3),
       (x^4 + x^3 + x^2 + 1, [0, 1, 2, 4], z3),
       (x^4 + x^2 + x + 1, [0, 3, 5, 6], z3),
       (x^6 + x^5 + x^4 + x^3 + x^2 + x + 1, [1, 2, 3, 4, 5, 6], z3),
       (x^7 + 1, [0, 1, 2, 3, 4, 5, 6], z3)]
  \end{lstlisting}
  En la tabla siguiente mostramos de forma más clara cuáles son los generadores e idempotentes correspondientes para cada código.
  \begin{table}[h]
    \centering
    \sffamily
    \begin{tabular}{ccll}
      \toprule
      \(i\) & dimensión & generador \(g_i(x)\) & conjunto \(T\)\\
      \midrule
      \(0\) & \(0\) & \(x^7 + 1\) & \(\{0, 1, 2, 3, 4, 5, 6\}\)\\
      \(1\) & \(1\) & \(x^6 + x^5 + \dots + x + 1\) & \(\{1, 2, 3, 4, 5, 6\}\)\\
      \(2\) & \(3\) & \(x^4 + x^3 + x^2 + 1\) & \(\{0, 3, 5, 6\}\)\\
      \(3\) & \(3\) & \(x^4 + x^2 + x + 1\) & \(\{0, 1, 2, 4\}\)\\
      \(4\) & \(4\) & \(x^3 + x + 1\) & \(\{3, 5, 6\}\)\\
      \(5\) & \(4\) & \(x^3 + x^2 + 1\) & \(\{1, 2, 4\}\)\\
      \(6\) & \(6\) & \(x + 1\) & \(\{0\}\)\\
      \(7\) & \(7\) & \(1\) & \(\emptyset\)\\
      \bottomrule
    \end{tabular}
    \caption{Polinomios generadores y conjuntos característicos para los códigos cíclicos de longitud 7}
  \end{table}
\end{example}



\chapter{Códigos BCH}

% 4.5 Mininum distance of cyclic codes (intro cota BCH)

En esta sección vamos a estudiar los códigos \textacr{BCH}, un tipo de códigos cíclicos que permiten ser diseñados con una capacidad de corrección concreta.
Como ya sabemos, para cualquier tipo de código es importante determinar la distancia mínima si queremos determinar su capacidad de corrección de errores.
A este respecto es útil disponer de cotas en la distancia mínima, especialmente cotas inferiores, pues son las que maximizan la capacidad de corrección.
Existen varias cotas conocidas para la distancia mínima de un código cíclico, pero nos vamos a centrar en la llamada \textit{cota de Bose-Ray-Chaudhuri-Hocquenghem}, usualmente abreviada como \textit{cota \textacr{BCH}}.
Esta cota es esencial para comprender la definición de los códigos \textacr{BCH} que estudiamos en esta sección.
La cota \textacr{BCH} va a depender de los ceros del código, concretamente en la posibilidad de encontrar cadenas de ceros «consecutivos».
La fuente principal de este capítulo ha sido \parencite{huffman_fundamentals_2003}.

\section{Construcción de códigos BCH}

En lo que sigue vamos a considerar un código cíclico \(\mathcal C\)  de longitud \(n\) sobre \(\mathbb F_q\) y \(\alpha\) una enésima raíz primitiva de la unidad en \(\mathbb F_{q^t}\), donde \(t = \operatorname{ord}_n(q)\).
Recordemos que \(T\) es un conjunto característico de \(\mathcal C\) siempre y cuando los ceros de \(\mathcal C\) sean \(\{\alpha^{i} : i \in T\}\).
Por tanto \(T\) ha de ser una unión de clases \(q\)-ciclotómicas módulo \(n\).
Decimos que \(T\) contiene un conjunto de \(s\) \textit{elementos consecutivos} si existe un conjunto \(\{b, b + 1, \dots, b + s - 1\}\) de \(s\) enteros consecutivos tal que
\[
  \{b, b + 1, \dots, b + s - 1\} \bmod n = S \subseteq T.
\]

Antes de proceder con la cota \textacr{BCH} vamos a enunciar un lema —que será utilizado en la demostración de dicha cota— sobre el determinante de una matriz de Vandermonde.
Sean \(\alpha_1, \dots, \alpha_s\) elementos de un cuerpo \(\mathbb F\).
La matriz de tamaño \(s \times s\) dada por \(V = (v_{i, j})\), donde \(v_{i,j} = \alpha_{j}^{i-1}\) se denomina \textit{matriz de Vandermonde}.
Observamos que la transpuesta de una matriz de Vandermonde es otra matriz de Vandermonde.

\begin{lemma}
  \label{lem:vandermonde}
  El determinante de una matriz de Vandermonde \(V\) viene dado por \(\operatorname{det}V = \prod_{1 \leq i < j \leq s}(\alpha_j - \alpha_i)\).
  En particular, \(V\) es no singular si los elementos \(\alpha_1, \dots, \alpha_s\) son todos diferentes dos a dos.
\end{lemma}

Estamos ya en condiciones de presentar y demostrar el teorema de la cota \textacr{BCH}.

\begin{theorem}[cota bch]
  Sea \(\mathcal C\) un código cíclico de longitud \(n\) sobre \(\mathbb F_q\) con conjunto característico \(T\).
  Supongamos que \(\mathcal C\) tiene peso mínimo \(d\).
  Asumamos que \(T\) contiene \(\delta - 1\) elementos consecutivos para algún entero \(\delta\).
  Entonces, \(d \geq \delta\).
\end{theorem}

\begin{proof}
  Asumimos que el código \(\mathcal C\) tiene ceros que incluyen
  \[
    \alpha^b, \alpha^{b+1}, \dots, \alpha^{b + \delta - 2}.
  \]
  Sea \(c(x)\) una palabra código de \(\mathcal C\) de peso \(w\) de la forma
  \[
    c(c) = \sum_{j = 1}^w c_{i_j}x^{i_j}.
  \]
  Vamos a proceder por reducción al absurdo.
  Supongamos que \(w \leq \delta\).
  Como \(c(\alpha^i) = 0\) para \(b \leq i \leq b + \delta - 2\), \(M \mathbf{u}^T = \mathbf{0}\), donde
  \[
    M = \begin{pmatrix}
      \alpha^{i_1b} & \alpha^{i_2b} & \dots & \alpha^{i_wb}\\
      \alpha^{i_1(b+1)} & \alpha^{i_2(b+1)} & \dots & \alpha^{i_w(b+1)}\\
        & & \vdots & \\
      \alpha^{i_1(b+w-1)} & \alpha^{i_2(b+w-1)} & \dots & \alpha^{i_w(b+w-1)}\\
    \end{pmatrix}
  \]
  y \(\mathbf{u} = c_{i_1}c_{i_2}\dotsc_{i_w}\).
  Como \(\mathbf{u} \neq \mathbf{0}\) la matriz \(M\) es singular, y por tanto \(\det M = 0\).
  Pero \(\det M = \alpha^{(i_1 + i_2 + \dots + i_w)b}\det V\), donde \(V\) es la matriz de Vandermonde
  \[
    V = \begin{pmatrix}
      1 & 1 & \dots & 1 \\
      \alpha^{i_1} & \alpha^{i_2} & \dots & \alpha^{i_w} \\
       &  & \vdots & \\
      \alpha^{i_1(w - 1)} & \alpha^{i_2(w - 1)} & \dots & \alpha^{i_w(w - 1)} \\
    \end{pmatrix}.
  \]
  Como los \(\alpha^{i_j}\) son todos distintos dos a dos, \(\det V \neq 0\) por el lema \ref{lem:vandermonde}, lo que contradice que \(\det M = 0\).
\end{proof}

Los códigos \textacr{BCH} son códigos cíclicos diseñados para aprovechar la cota \textacr{BCH}.
Nos gustaría poder construir un código cíclico \(\mathcal C\) de longitud \(n\) sobre \(\mathbb F_q\) que tenga a la vez un peso mínimo grande y una dimensión grande.
Tener un peso mínimo grande, basándonos en la cota \textacr{BCH}, se puede conseguir escogiendo un conjunto característico para \(\mathcal C\) que tenga un gran número de elementos consecutivos.

Como la dimensión de \(\mathcal C\) es \(n - |T|\) por el teorema \ref{th:cicl-cto-caracteristico}, nos gustaría que \(|T|\) fuese tan pequeño como sea posible.
Por tanto, si quisiésemos que \(\mathcal C\) tenga distancia mínima de al menos \(\delta\), podemos escoger un conjunto característico tan pequeño como sea posible que sea una unión de clases \(q\)-ciclotómicas con \(\delta - 1\) elementos consecutivos.

Sea \(\delta\) un entero tal que \(2 \leq \delta \leq n\). Un \textit{código \textacr{BCH}} \(\mathcal C\) sobre \(\mathbb F_q\) de longitud \(n\) y \textit{distancia mínima prevista} \(\delta\) es un código cíclico con conjunto característico
\begin{equation}
  \label{eq:bch-conjunto-caracteristico}
  T = C_b \cup C_{b+1} \cup \dots \cup C_{b + \delta - 2},
\end{equation}
donde \(C_i\) es la clase \(q\)-ciclotómica módulo \(n\) que contiene a \(i\).
Por la cota \textacr{BCH} este código tiene distancia mínima prevista al menos \(\delta\).

\begin{theorem}
  Un código \textacr{BCH} de distancia mínima prevista \(\delta\) tiene peso mínimo de al menos \(\delta\).
\end{theorem}

\begin{proof}
  El conjunto característico \ref{eq:bch-conjunto-caracteristico} tiene al menos \(\delta - 1\) elementos.
  El resultado se deduce de la cota \textacr{BCH}.
\end{proof}

Al variar el valor de \(b\) obtenemos distintos códigos con distancias mínimas y dimensiones diferentes.
Cuando \(b = 1\) el código \(\mathcal C\) se dice que es un código \textacr{BCH} \textit{en sentido estricto}.
Como con cualquier código cíclico, si \(n = q^t - 1\) entonces \(\mathcal C\) es un código \textacr{BCH} \textit{primitivo}.
En la sección siguiente vamos a estudiar un algoritmo de decodificación que permite aprovechar las ventajas de los códigos \textacr{BCH}.

\section{Códigos Reed-Solomon}

Vamos a describir brevemente los códigos Reed-Solomon, que abreviaremos como códigos RS, pues aludiremos a ellos cuando hablemos de códigos cíclicos sesgados.
Son una subfamilia de los códigos BCH que acabamos de definir.

\begin{definition}
  Un código RS sobre \(\mathbb F_q\) es un código BCH de longitud \(n = q - 1\).
\end{definition}

Veamos un par de propiedades importantes.

\begin{theorem}
  Sea \(\mathcal C\) un código RS sobre \(\mathbb F_q\) de longitud \(n = q - 1\) y distancia mínima prevista \(\delta\).
  Entonces:
  \begin{enumerate}
    \item El código \(\mathcal C\) tiene conjunto característico \(T = \{b, b + 1, \dots, b + \delta - 2\}\) para algún entero \(b\).
    \item El código \(\mathcal C\) tiene distancia mínima \(d = \delta\) y dimensión \(k = n - d + 1\).
  \end{enumerate}
\end{theorem}

\begin{proof}
  %Comenzamos observando que \(\operatorname{ord}_n(q) = 1\), por lo que todos los factores irreducibles de \(x^n - 1\) tienen grado \(1\) y las clases \(q\)-ciclotómicas módulo \(n\) tienen tamaño \(1\).
  Veamos la demostración por apartados.
  \begin{enumerate}
    \item Como es un código BCH de distancia mínima prevista \(\delta\) su conjutno característico \(T\) tiene que tener tamaño \(\delta - 1\) y en consecuencia es \(T = \{b, b + 1, \dots, b + \delta - 2\}\).
    \item La distancia es obvia, pues es un código BCH y la dimensión proviene del teorema \ref{th:cicl-cto-caracteristico}, pues \(n - |T| = n - \delta + 1\).
  \end{enumerate}
\end{proof}


\section{Algoritmo de Peterson-Gorenstein-Zierler}

El algoritmo de Peterson-Gorenstein-Zierler —de ahora en adelante, algoritmo \textacr{PGZ}— es un algoritmo de decodificación de códigos \textacr{BCH} que permite corregir hasta \(t = \lfloor (\delta - 1)/2 \rfloor\) errores.
Fue desarrollado originalmente en 1960 por Peterson \parencite{peterson_encoding_1960} para decodificar códigos \textacr{BCH} binarios, pero fue generalizado poco después por Gorenstein y Zierler para códigos no binarios \parencite{gorenstein_class_1961}.

Como en cualquier otro método de decodificación el objetivo es obtener el mensaje original \(c(x)\) a partir de un mensaje recibido \(y(x)\), para lo que hay que hallar primero los errores \(e(x)\) que se han producido en la transmisión, de forma que \(c(x) = y(x) - e(x)\).
El vector de errores ha de tener peso \(v \leq t\), ya que no podemos corregir más errores de los que el código permite.
Vamos a considerar que los errores se han producido en coordenadas desconocidas \(k_1, k_2, \dots, k_v\), de forma que el vector de errores lo podemos expresar como
\[
  e(x) = e_{k_1}x^{k_1} + e_{k_2}x^{k_2} + \dots + e_{k_v}x^{k_v}.
\]
Como nuestro objetivo es determinar \(e(x)\) tenemos que hallar: \begin{itemize}
  \item las \textit{coordenadas de error} \(k_j\)\,;
  \item las \textit{magnitudes de error} \(e_{k_j}\).
\end{itemize}

Vamos a estudiar a continuación el desarrollo teórico y la justificación del funcionamiento del algoritmo para después dar una versión del mismo esquematizada en pseudocódigo.
El comienzo de este método es similar al del descrito en la sección \ref{subsec:codificacion-descodificacion}, solo que en lugar de utilizar la propiedad de la matriz de paridad allí descrita utilizaremos la propiedad análoga de que, por el teorema \ref{th:cicl-cto-caracteristico}, un elemento \(c(x) \in \mathcal C\) si y solo si \(c(\alpha^i) = 0\) para todo \(i \in T\).
En nuestro caso particular, dado que \(t = \lfloor (\delta - 1)/2 \rfloor\) y \(T\) contiene a \(\{1, 2, \dots, \delta - 1\}\), se tiene que
\[
  y(\alpha^i) = c(\alpha^i) + e(\alpha^i) = e(\alpha^i)
\]
para todo \(1 \leq i \leq 2t\).
Estas ecuaciones van a ser fundamentales para encontrar el error \(e(x)\).
En este caso llamaremos \textit{síndrome} \(s_i\) de \(y(x)\) al elemento de \(\mathbb F_{q}^m\) dado por \(s_i = y(\alpha^i)\).
El primer paso del algoritmo es encontrar los síndromes para todo \(1 \leq i \leq 2t\).
Estos síndromes nos conducen a un sistema de ecuaciones en el que se encuentran las coordenadas de error \(k_j\) y las magnitudes de error \(e_{k_j}\).
Desarrollando lo anterior podemos expresar los síndromes como
\begin{equation}
  \label{eq:sindromes}
  s_i = y(\alpha^i) = \sum_{j = 1}^v e_{k_j}(\alpha^i)^{k_j}
   = \sum_{j = 1}^v e_{k_j}(\alpha^{k_j})^i
\end{equation}
para todo \(1 \leq i \leq 2t\).
A fin de simplificar la notación, para \(1 \leq j \leq v\) definimos: \begin{itemize}[label={—}, noitemsep, leftmargin=*]
  \item \(E_j = e_{k_j}\), que llamaremos \textit{magnitud de error en la coordenada} \(k_j\), y
  \item \(X_j = \alpha^{k_j}\), que llamaremos \textit{número de coordenada de error correspondiente a la coordenada} \(k_j\).
\end{itemize}
Observamos que al conocer \(X_j\) conocemos de forma unívoca la coordenada de error \(k_j\), ya que si \(\alpha^i = \alpha^k\) para \(i\) y \(k\) entre \(0\) y \(n-1\), entonces \(i = k\).
Con la notación que hemos descrito la igualdad (\ref{eq:sindromes}) la podemos escribir como \begin{equation}
  \label{eq:sindromes-alt}
  S_i = \sum_{j = 1}^v E_jX_{j}^i, \quad\text{para } 1 \leq i \leq 2t,
\end{equation}
lo que nos conduce al sistema de ecuaciones: \begin{equation}
  \begin{cases}
    S_1 = E_1X_1 + E_2X_2 + \dots + E_vX_v,\\
    S_2 = E_1X_1^2 + E_2X_2^2 + \dots + E_vX_v^2,\\
    S_3 = E_1X_1^3 + E_2X_2^3 + \dots + E_vX_v^3,\\
    \quad\;\vdots\\
    S_{2t} = E_1X_1^{2t} + E_2X_2^{2t} + \dots + E_vX_v^{2t}.
  \end{cases}
  \label{eq:sindromes-alt-sistema}
\end{equation}
De este sistema desconocemos tanto los valores de los \(X_j\) como los de los \(E_j\), pero es que además no es lineal para los \(X_j\).
Como no podemos resolverlo directamente vamos a tratar de encontrar otra forma con la que calcular los valores \(X_j\) y utilizarlos para resolver el sistema lineal que forman los \(E_j\).
Para ello vamos a buscar un sistema lineal que dependa de otras variables \(\sigma_1, \dots, \sigma_v\) que nos conduzca a los valores \(X_j\).
Definimos el \textit{polinomio localizador de errores} \(\sigma(x)\) como
\[
  \sigma(x) = (1 - xX_1)(1 - xX_2) \dots (1 - xX_v) = 1 + \sum_{i=1}^v \sigma_ix^i.
\]
Como vemos inmediatamente por su definición, las raíces de \(\sigma(x)\) son los inversos de los números de coordenadas de error.
Por tanto,
\[
  \sigma(X_j^{-1}) = 1 + \sigma_1X_j^{-1} + \sigma_2X_j^{-2} + \dots + \sigma_vX_j^{-v} = 0
\]
para \(1 \leq j \leq v\).
Si multiplicamos a ambos lados de la expresión por \(E_jX_j^{i+v}\) obtenemos
\[
  E_jX_j^{i+v} + \sigma_1E_jX_j^{i+v-1} + \dots + \sigma_vE_jX_j^{i} = 0
\]
para todo \(i\).
Si sumamos para todo \(j\) en \(1 \leq j \leq v\) tenemos
\[
  \sum_{j=1}^v E_jX_j^{i+v} + \sigma_1\sum_{j=1}^v E_jX_j^{i+v-1} + \dots + \sigma_v \sum_{j=1}^v E_jX_j^i = 0.
\]
Lo que hemos obtenido en estas sumas son los síndromes descritos en (\ref{eq:sindromes-alt}), ya que \(1 \leq i \) y \(i+v \leq 2t\).
Como \(v \leq t\) la expresión anterior se convierte en
\[
  S_{i+v} + \sigma_1S_{i+v-1} + \sigma_2S_{i+v-2} + \dots + \sigma_vS_i = 0,
\]
que equivale a
  \[
    \sigma_1S_{i+v-1} + \sigma_2S_{i+v-2} + \dots + \sigma_vS_i = -S_{i+v},
  \]
para todo \(1 \leq i \leq v\).
Por tanto podemos encontrar los \(\sigma_k\) si resolvemos el sistema de ecuaciones dado por:
\[
  \begin{pmatrix}
    S_1 & S_2 & S_3 & \dots & S_{v-1} & S_v \\
    S_1 & S_2 & S_3 & \dots & S_{v-1} & S_v \\
    S_1 & S_2 & S_3 & \dots & S_{v-1} & S_v \\
     & & & \vdots & & \\
    S_1 & S_2 & S_3 & \dots & S_{v-1} & S_v \\
  \end{pmatrix} \begin{pmatrix}
    \sigma_v \\
    \sigma_{v-1} \\
    \sigma_{v-2} \\
    \vdots\\
    \sigma_1
  \end{pmatrix} = \begin{pmatrix}
    -S_{v+1}\\
    -S_{v+2}\\
    -S_{v+3}\\
    \vdots\\
    -S_{2v}
  \end{pmatrix}.
\]
La dificultad de este paso es que desconocemos el valor de \(v\) (el número de errores), por lo que vamos a realizar un procedimiento iterativo.
Suponemos que nuestro número de errores es \(\mu = t\), que es el máximo que podemos corregir.
Tenemos que quedarnos con el menor valor de \(v\) que sea posible.
Para ello tenemos en cuenta que la matriz
\[
  M_{\mu} = \begin{pmatrix}
    S_1 & S_2 & \dots & S_{\mu} \\
    S_2 & S_3 & \dots & S_{\mu + 1}\\
     & & \vdots & \\
    S_{\mu} & S_{\mu+1} & \dots & S_{2\mu - 1}
  \end{pmatrix}
\] será no singular si \(\mu = v\) y singular si \(\mu > v\) \parencite[Lema 5.4.2]{huffman_fundamentals_2003}.
% MAYBE: copiar demostración de esto????
Así, si \(M_{\mu}\) es singular reducimos el valor de \(\mu\) en 1, \(\mu = \mu - 1\) y probamos de nuevo si \(M_{\mu}\) es singular.
Repetimos hasta encontrar una matriz que no sea singular.
Ese valor \(\mu\) será el número de errores \(v\).
Conocido el tamaño podemos resolver el sistema y obtener los valores \(\sigma_k\).
Ahora solo tenemos que deshacer el camino que hemos recorrido hasta ahora.
Conocidos los \(\sigma_k\) podemos determinar \(\sigma(x)\) y con él, sus raíces, utilizando el procedimiento que queramos, usualmente, calculando reiteradamente \(\sigma(\alpha^i)\) para \(0 \leq i < n\) hasta encontrarlas.
Como ya dijimos, si las invertimos hallaremos los valores de \(X_j\), y con ellos ya podemos resolver el sistema (\ref{eq:sindromes-alt-sistema}), obteniendo así los valores de los \(E_j\).
Conocidos todos los valores de \(X_j\) y \(E_j\) podemos obtener los de \(k_j\) y \(e_{k_j}\), con los que podemos determinar el vector de error \(e(x)\).
Ya solo queda restar \(y(x) - e(x)\) para obtener el mensaje original.

En resumen, el algoritmo consiste en: \begin{enumerate}
  \item Determinar los síndromes del mensaje recibido.
  \item Encontrar el polinomio localizador.
  \item Hallar las raíces del polinomio localizador e invertirlas para obtener las coordenadas de error \(k_j\).
  \item Utilizar estos inversos para resolver el sistema de ecuaciones formado por los síndromes, obteniendo así las magnitudes de error \(e_{k_j}\).
  \item Hallar el vector de error \(e(x)\) y restárselo al mensaje \(y(x)\).
\end{enumerate}

Hemos expresado en el algoritmo \ref{alg:pgz-cc} el algoritmo PGZ en pseudocódigo siguiendo este esquema.
A partir de él se ha realizado una implementación en el sistema Sage que puede consultarse en los archivos enlazadas en el anexo \ref{annex:pgz-sage}.
Veamos un par de ejemplos que utilizan esta implementación.

\begin{example}
  Sea \(\mathcal C\) un \([15, 7]\) código BCH en sentido estricto de distancia designada \(\delta = 5\).
  Supongamos que recibimos el mensaje \(y(x) = x^{10} + x^9 + x^6 + x^5 + x + 1\).
  Vamos a corregir los errores que puedan haberse producido en la transmisión.
  \begin{lstlisting}[gobble=4]
    sage: C = codes.BCHCode(GF(2), 15, 5, offset=1)
    sage: x = polygen(GF(2))
    sage: y = 1 + x +x^5 + x^6 + x^9 + x^10
    sage: DEBUG = true
    sage: PGZ(C, y)
    > polinomio generador: x^8 + x^7 + x^6 + x^4 + 1
      raíz primitiva: z4
      síndromes: [z4^2, z4 + 1, z4^3 + z4^2 + z4, z4^2 + 1]
      tamaño de m_mu: 2
      matriz m_mu: 
      [            z4^2           z4 + 1]
      [          z4 + 1 z4^3 + z4^2 + z4]
      vector b_mu: (z4^3 + z4^2 + z4, z4^2 + 1)
      matriz de soluciones de m_mu*S = b_mu: (z4^3 + 1, z4^2)
      polinomio localizador sigma(x): (z4^3 + 1)*x^2 + z4^2*x + 1
      raíces de sigma(x): [(z4^2 + z4, 1), (z4^3 + z4^2 + z4, 1)]
      X_j: [z4^2 + z4 + 1, z4 + 1]
      k_j: [10, 4]
      error e: x^10 + x^4
    > x^9 + x^6 + x^5 + x^4 + x + 1
  \end{lstlisting}
  El mensaje corregido es por tanto \(m(x) = x^9 + x^6 + x^5 + x^4 + x + 1\).
\end{example}

\begin{example}
  Sea \(\mathcal C\) un \([15, 5]\) código BCH en sentido estricto de distancia designada \(\delta = 5\).
  Supongamos que recibimos el mensaje \(y(x) = x^{12} + x^9 + x^7 + x^5 + x^4 + x + 1\).
  Vamos a corregir los errores que puedan haberse producido en la transmisión.
  \begin{lstlisting}[gobble=4, breaklines=false, basicstyle=\small\ttfamily]
    sage: C = codes.BCHCode(GF(2), 15, 7, offset=1)
    sage: x = polygen(GF(2))
    sage: y = 1 + x +x^4 + x^5 + x^7 + x^9 + x^12
    sage: DEBUG = true
    sage: PGZ(C, y)
    > polinomio generador: x^10 + x^8 + x^5 + x^4 + x^2 + x + 1
      raíz primitiva: z4
      síndromes: [z4^3, z4^3 + z4^2, z4^3, z4^3 + z4^2 + z4 + 1, 0, z4^3 + z4^2]
      tamaño de m_mu: 3
      matriz m_mu: 
      [                z4^3          z4^3 + z4^2                 z4^3]
      [         z4^3 + z4^2                 z4^3 z4^3 + z4^2 + z4 + 1]
      [                z4^3 z4^3 + z4^2 + z4 + 1                    0]
      vector b_mu: (z4^3 + z4^2 + z4 + 1, 0, z4^3 + z4^2)
      matriz de soluciones de m_mu*S = b_mu: (z4^3 + z4^2, z4^2 + 1, z4^3)
      polinomio localizador sigma(x): (z4^3 + z4^2)*x^3 + (z4^2 + 1)*x^2 + z4^3*x + 1
      raíces de sigma(x): [(1, 1), (z4 + 1, 1), (z4^2 + z4, 1)]
      X_j: [1, z4^3 + z4^2 + z4, z4^2 + z4 + 1]
      k_j: [0, 11, 10]
      error e: x^11 + x^10 + 1
    > x^12 + x^11 + x^10 + x^9 + x^7 + x^5 + x^4 + x
  \end{lstlisting}
  El mensaje corregido es por tanto \(m(x) = x^{12} + x^{11} + x^{10} + x^9 + x^7 + x^5 + x^4 + x\).
\end{example}

\begin{Ualgorithm}[htbp]
  \DontPrintSemicolon
  \KwIn{el código \(\mathcal C\), el mensaje recibido \(y(x)\)}
  \KwOut{el mensaje decodificado \(c(x)\)}
  \(\delta \longleftarrow\) distancia designada de \(\mathcal C\)\;
  \(t \longleftarrow \lfloor(\delta - 1)/2\rfloor\)\;
  \(g \longleftarrow\) polinomio generador de \(\mathcal C\)\;
  \(\alpha \longleftarrow\) raíz primitiva del cuerpo de descomposición usado para generar el conjunto característico de \(\mathcal C\)\;
  \tcp{Paso 1: calcular síndromes}
  \For{\(1 \leq i \leq 2t\)}{
      $S_i \longleftarrow y(\alpha^i)$\;
  }
  \tcp{Paso 2: hallar polinomio localizador}
  \(\mu \longleftarrow t\)\;
  \(M_{\mu} \longleftarrow (S_{i + j - 1})_{i, j}\footnotemark\,, 1 \leq i, j \leq \mu \)\;
  \While{\(M_{\mu}\) es no singular}{
  \(\mu \longleftarrow \mu - 1\)\;
  \(M_{\mu} \longleftarrow (S_{i + j - 1})_{i, j}\,, 1 \leq i, j \leq \mu \)\;
  }
  \(v \longleftarrow \mu\)\;
  \(\sigma \longleftarrow (\sigma_{v - i + 1})_i, 1 \leq i \leq v\)\;
  \(b_{\mu} \longleftarrow (-S_{v + i})_i, 1 \leq i \leq v\)\;
  \(\sigma_k \longleftarrow\) soluciones del sistema \(M_{\mu}\sigma = b_{\mu}\)\;
  \(\sigma(x) \longleftarrow 1 + \sum_{i=1}^{v} \sigma_i x^i\)\;
  \tcp{Paso 3: obtener las coordenadas de error}
  \(r_k \longleftarrow\) raíces de \(\sigma(x)\)\;
  \(X_j \longleftarrow r_j^{-1}\) \;
  \(k_j \longleftarrow \log_{\alpha}(X_j)\) \;
  \tcp{Paso 4: obtener las magnitudes de error}
  \(M_{S} \longleftarrow (X_{j}^{i})_{i, j}\,, 1 \leq i, j \leq v \)\;
  \(E \longleftarrow (E_i)_i, 1 \leq i \leq v\)\;
  \(b_{S} \longleftarrow (S_{i})_i, 1 \leq i \leq v\)\;
  \(E_k \longleftarrow\) soluciones del sistema \(M_{S}E = b_{S}\)\;
  \tcp{Paso 5: calcular el mensaje original}
  \(e(x) \longleftarrow \sum_{j=1}^v E_ix^{k_i}\)\;
  \(c(x) = y(x) - e(x)\)\;
  \caption{Peterson-Gorenstein-Zierler para códigos cíclicos.}
  \label{alg:pgz-cc}
\end{Ualgorithm}
\footnotetext{Aquí estamos describiendo una matriz por sus entradas, las filas varían en \(i\), y las columnas, en \(j\).}
\chapter{Anillos de polinomios de Ore}

En esta sección vamos a hablar sobre los anillos de polinomios de Ore, que serán la base de los códigos cíclicos sesgados.
% TODO: nota histórica sobre polinomios Ore
[Introducir nota histórica]
Primero vamos a dar la definición general, sin detenernos a justificar su construcción, pues acto seguido vamos a centrarnos en el caso que nos va a ocupar cuando trabajemos con códigos cíclicos sesgados.
Las definiciones y el desarrollo teórico seguidos en esta sección proceden de \parencite{jacobson_finite-dimensional_1996}, \parencite{ore_theory_1933} y \parencite{gomez-torrecillas_factoring_2020}.

\begin{definition}
  Sea \(R\) un anillo, \(\sigma\) un endomorfismo de \(R\) y \(\delta\) una \(\sigma\)-\textit{derivación} de \(R\), es decir, \(\delta\) es un homomorfismo de grupos abelianos tal que para \(a, b \in R\) se verifica que
  \[
    \delta(ab) = (\sigma a)(\delta b) + (\delta a)b.
  \]
  Entonces, el anillo \(R[t; \sigma, \delta]\) de los polinomios en \(R[t]\) de la forma
  \[
    a_0 + a_1t + \dots + a_nt^n,
  \]
  donde \(a_i \in R\), con la igualdad y suma usuales, y en el que la multiplicación verifica la relación 
  \[
  ta = (\sigma a)t + \delta a, \qquad a \in R,
  \]
  se conoce como \textit{anillo de polinomios de Ore} o \textit{anillos de polinomios sesgados}.
\end{definition}

Para comprobar que \(R[t; \sigma, \delta]\) es un anillo tendríamos que ver que efectivamente con las operaciones que hemos dado se verifican todas las propiedades de los anillos.
Puesto que hemos usado la suma usual de los polinomios, bastaría probar que se verifica la propiedad asociativa para la multiplicación que hemos definido.
No vamos a entrar en detalle, pues no es el objetivo de este trabajo el estudio de los anillos de polinomios de Ore en general.

Trabajar con códigos cíclicos sesgados requiere del estudio del anillo \(\mathbb F_q[x; \sigma]\), con \(\sigma\) un automorfismo.
Por tanto, nos vamos a centrar los anillos de polinomios de Ore en los que \(R = \mathbb F_q\) —cuerpo finito de \(q\) elementos—, hemos llamado \(x\) a \(t\), \(\sigma\) es un automorfismo y \(\delta = 0\).
En estos anillos la multiplicación verifica la relación 
\[
  xa = (\sigma a)x, \qquad a \in R.
\]
Es este caso particular el que sí vamos a estudiar en profundidad, justificando que, como ya hemos adelantado, se trata de un anillo.

Vamos a ver por inducción que, como podemos intuir, \(x^n a = (\sigma^n a)x^n\). 
Estudiado el caso base anterior y supuesto que se verifica la igualdad para \(n - 1\), para \(n\) tenemos que
\[
  x^{n}a = xx^{n - 1}a = x(\sigma^{n-1} a)x^{n-1} = \sigma(\sigma^{n-1} a)x^{n-1}x = (\sigma^{n}a)x^{n}.
\]
Ahora definimos
\[
  (ax^n)(bx^m) = a(\sigma^n b)x^{n+m},
\]
con lo que, junto a la propiedad distributiva podemos definir el producto de polinomios en \(x\) como
\[
  \textstyle(\sum a_nx^n)(\sum b_mx^m) = \sum(a_nx^n)(b_mx^m).
\]

Para comprobar que \(\mathbb F_q[x; \sigma]\) es un anillo, como ya hemos comentado en el caso general, necesitamos comprobar que se verifica la propiedad asociativa para la multiplicación.
Comprobar esta afirmación directamente es tedioso, pero en \parencite[p. 2-3]{jacobson_finite-dimensional_1996} puede consultarse una demostración utilizando una representación matricial de los elementos.

A continuación vemos que, partiendo de que \(\mathbb F_q\) es en particular un anillo de división, \(\mathbb F_q[x; \sigma]\) es un dominio de integridad no conmutativo.
Dado un polinomio \(f(x) = a_0 + a_1x + \dots + a_nx^n\) con \(a_n \neq 0\) y definimos \(\deg(f(x)) = n\) y \(\deg(0) = -\infty\).
Si consideramos otro polinomio \(g(x) = b_0 + b_1x + \dots + b_mx^m\) con \(b_m\) entonces \(f(x)g(x) = \dots + a_n(\sigma^n b_m)x^{n+m}\) y \(a_n(\sigma^nb_m) \neq 0\) y así,
\[
  \deg(f(x)g(x)) = \deg(f(x)) + \deg(g(x)).
\]
Por tanto, \(\mathbb F_q[x; \sigma]\) no tiene divisores de cero distintos del cero, por lo que es un dominio de integridad no conmutativo, como habíamos afirmado.

Podemos definir algoritmos de división en \(\mathbb F_q[x; \sigma]\) tanto a la izquierda como a la derecha (cita)  —descritos en los algoritmos \ref{alg:ore-fq-division-izquierda} y \ref{alg:ore-fq-division-derecha}—, de forma que para cada \(f(x), g(x) \in \mathbb F_q[x; \sigma]\) —con \(g(x) \neq 0\)— existen elementos \(q(x), r(x)\) únicos, con \(\deg(r) < \deg(g)\) tales que al dividir por la izquierda obtenemos
\[
  f(x) = q(x)g(x) + r(x),
\]
y al dividir por la derecha, 
\[
  f(x) = g(x)q(x) + r(x),
\]
Cuando dividimos por la izquierda (respectivamente por la derecha) el polinomio \(g(x)\) se le llama \textit{cociente por la izquierda} (\textit{derecha}) y a \(r(x)\), \textit{resto por la izquierda} (\textit{derecha}).
Los denotaremos por \(g(x) = \operatorname{coi}(f(x), g(x))\) o \(\operatorname{cod}(f(x), g(x))\) y \(r(x) = \operatorname{rei}(f(x), g(x))\) o \(\operatorname{red}(f(x), g(x))\).

\begin{Ualgorithm}[h]
  \DontPrintSemicolon
  \KwIn{polinomios \(f, g \in \mathbb F_q[x; \sigma]\) con \(g \neq 0\)}
  \KwOut{polinomios \(q, r \in \mathbb F_q[x; \sigma]\) tales que \(f = qg + r\), y \(\deg(r) < \deg(g)\)}
  \(q \longleftarrow 0\)\;
  \(r \longleftarrow f\)\;
  \While{\(\deg(g) \leq \deg(r)\)}{
    \(a \longleftarrow \lc (r) \sigma^{\deg (r) - \deg (g)}(\lc (g)^{-1})\)\;
    \(q \longleftarrow q + ax^{\deg (r) - \deg (g)}\)\;
    \(r \longleftarrow r - ax^{\deg (r) - \deg (g)}g\)
    }
    \caption{División por la izquierda en \(\mathbb F_q[x; \sigma]\)}
  \label{alg:ore-fq-division-izquierda}
\end{Ualgorithm}

\begin{Ualgorithm}[h]
  \DontPrintSemicolon
  \KwIn{polinomios \(f, g \in \mathbb F_q[x; \sigma]\) con \(g \neq 0\)}
  \KwOut{polinomios \(q, r \in \mathbb F_q[x; \sigma]\) tales que \(f = gq + r\), y \(\deg(r) < \deg(g)\)}
  \(q \longleftarrow 0\)\;
  \(r \longleftarrow f\)\;
  \While{\(\deg(g) \leq \deg(r)\)}{
    \(a \longleftarrow \sigma^{-\deg(g)}(\lc(g)^{-1}\lc(r))\)\;
    \(q \longleftarrow q + ax^{\deg (r) - \deg (g)}\)\;
    \(r \longleftarrow r - gax^{\deg (r) - \deg (g)}\)
    }
    \caption{División por la derecha en \(\mathbb F_q[x; \sigma]\)
  }
  \label{alg:ore-fq-division-derecha}
\end{Ualgorithm}

La existencia de algoritmos de algoritmos de división a izquierda y a derecha implica que \(F_q[x, \sigma]\) es un dominio de ideales principales a izquierda y a derecha, es decir, que es lo que llamamos un dominio de ideales principales a secas.

A partir de ahora, para ser más concisos con la notación vamos a llamar \(R = F_q[x, \sigma]\) y cuando no sea necesario hacer referencia a la variable \(x\), a un polinomio \(f(x)\) lo denotaremos simplemente por \(f\).
Los ideales biláteros de \(R\) serán de la forma \(I = Rf = f^{*}R\) y para todo \(g \in R\) existirán \(g', \tilde{g} \in R\) tales que \(fg = g'f\) y \(gf^{*} = f^{*}\tilde{g}\).
Los elementos \(f\) tales que para todo \(g \in R\) existen \(g'\) y \(\tilde{g}\) tales que \(fg = g'f\) y \(gf = f\tilde{g}\) se llaman elementos \emph{biláteros} y además \(Rf = fR\) es un ideal.

\begin{theorem}
  \label{th:anillos-ore-centro}
  Sea \(R = F_q[x, \sigma]\). Se verifican las siguientes afirmaciones.
  \begin{enumerate}
    \item Los elementos biláteros de \(R\) son de la forma \(ac(t)x^n\), donde \(a \in \mathbb F_q\), \(n = 0, 1, \dots\) y \(c(t) \in \operatorname{Cent}(R)\), el centro de \(R\).
    \item Supongamos ahora que \(\sigma\) tiene orden \(n\), de forma que \(\sigma^n = \operatorname{Id}\).
    El centro de \(R\) es el conjunto de los polinomios de la forma
    \[
      \gamma_0 + \gamma_1x^{n} + \gamma_2x^{2n} + \dots + \gamma_sx^{sn},
    \]
    donde \(\gamma_i \in \mathbb F_q\).
  \end{enumerate}
\end{theorem}

Dados \(g, f \in R\) supongamos que \(Rg \subseteq Rf\) con \(Rg \neq 0\).
Entonces \(g = hf\), por lo que decimos que \(f\) es un \emph{divisor por la derecha} de \(g\) y lo notaremos por \(f \mid_{d} g\).
Equivalentemente, podemos decir que \(g\) es un \emph{múltiplo por la izquierda} de \(f\).
Observemos que de igual forma, si \(f \mid_{d} g\) entonces \(Rg \subseteq Rf\).

Tenemos que \(Rf = Rg \neq 0\) si y solo si \(f \mid_d g\) y \(g \mid_d f\).
Así, \(g = hf\) y \(f = lg\), por lo que \(g = hlg\).
Por tanto, \(hl = lh = 1\) por lo que \(h\) y \(l\) son unidades de \(R\).
Se dice entonces que \(f\) y \(g\) son \emph{asociados por la izquierda} en el sentido de que \(g = uf\), siendo \(u\) una unidad de \(R\).

Se tiene que \(Rf + Rg = Rh\).
Entonces \(h \mid_d f\) y \(h \mid_d g\).
De hecho si \(l \mid_d f\) y \(l \mid_d f\) entonces \(Rf \subset Rl\) y \(Rg \subset Rl\), por lo que \(Rh \subset Rl\) y \(l \mid_d h\).
Por tanto \(h\) es un \emph{máximo común divisor por la derecha} de \(f\) y \(g\) y lo notamos como \(h = (f, g)_d\).
Dos máximo común divisor por la derecha de los mismos dos elementos son asociados por la izquierda.

Se puede comprobar que \(R\) satisface la condición de Ore por la izquierda \parencite[ver][p. 4]{jacobson_finite-dimensional_1996}, por lo que si \(f \neq 0\) y \(g \neq 0\) se tiene que \(Rf \cap Rg \neq 0\).
Tenemos por tanto que \(Rf \cap Rg Rh\) para algún \(h\) por lo que \(m = g'f = f'g\).
De hecho si \(f \mid_d l\) y \(g \mid_d l\) entonces \(Rl \subset Rf \cap Rg = Rh\), por lo que \(h \mid_d l\).
Por tanto \(h\) es un \emph{mínimo común múltiplo por la izquierda} y lo notamos por \(h = [f, g]_i\).
De nuevo, dos mínimo común múltiplo por la izquierda de los mismos dos elementos son asociados por la izquierda.

Puede definirse una versión del algoritmo extendido de Euclides en este contexto (ver el algoritmo \ref{alg:ore-fq-euclides}), que nos permite calcular tanto el máximo común divisor como el mínimo común múltiplo.

\begin{Ualgorithm}[h]
  \DontPrintSemicolon
  \KwIn{polinomios \(f, g \in \mathbb F_q[x; \sigma]\) con \(f \neq 0\), \(g \neq 0\)}
  \KwOut{un número \(n \in \mathbb N\), polinomios \(u_i, v_i, q_i, f_i \in \mathbb F_q[x; \sigma]\) tales que \(f_i = u_if + v_ig\), \(q_i = \operatorname{coi}(f_{i-1}, f_i)\), para \(1 \leq i \leq n + 1\) y \(f_n = (f, g)_d\), \(u_nf = -v_ng = [f, g]_i\).}
  \(u_0 \longleftarrow v_1 = 1\)\;
  \(u_1 \longleftarrow v_0 = 1\)\;
  \(f_0 \longleftarrow f\)\;
  \(f_1 \longleftarrow g\)\;
  \(i \longleftarrow 1\)\;
  \While{\(f_i \neq 0\)}{
    \(q_i \longleftarrow \operatorname{coi}(f_{i-1}, f_i)\)\;
    \(u_{i+1} \longleftarrow u_{i-1} - q_iu_i\)\;
    \(v_{i+1} \longleftarrow v_{i-1} - q_iv_i\)\;
    \(f_{i+1} \longleftarrow f_{i-1} - q_if_i\)\;
    \(n \longleftarrow i\)\;
    }
    \caption{Algoritmo extendido de Euclides por la izquierda en \(\mathbb F_q[x; \sigma]\)
  }
  \label{alg:ore-fq-euclides}
\end{Ualgorithm}
  
Como \(R\) es un dominio de ideales principales es posible descomponer cada polinomio \(f \in R\) en un producto de factores irreducibles.
Pero esta factorización no será única.

Decimos que dos polinomios \(f, g \in R\) distintos de cero son \emph{similares por la izquierda}, que notamos \(f \sim_i g\) si existe un polinomio \(h \in R\) tal que 
\[
  (h, g)_d = 1 \quad\text{y}\quad f = [g, h]_ih^{-1}.
\]
La condición \((h, g)_d = 1\) equivale a que existan \(a\) y \(b \in R\) tales que
\[
  1 = ah + bg
\]
y \(f =  [g, h]_ih^{-1}\) equivale a que 
\[
  l = h'g = fh,
\]
donde \((h', f)_i = 1\).
Por tanto tenemos un \(h'\) tal que \((h', f)_i = 1\) y \(g = h^{'-1}[h', f]_d\).
Por tanto si \(f\) es similar por la izquierda a \(g\) entonces \(g\) es similar por la derecha a \(f\), por lo que escribiremos simplemente que \(f \sim g\).
Es posible comprobar que la \emph{similitud} es una relación de equivalencia \parencite[ver][p. 11]{jacobson_finite-dimensional_1996}.

% MAYBE: Esto se deriva de que... (módulos etc) (consultar tal).

\begin{theorem}
  Si \(f = p_1 \dots p_r\) y \(f = q_1 \dots q_t\) son factorizaciones de \(f \in R\) como producto de irreducibles entonces \(r= t\) y salvo una posible reordenación, \(q_i \sim p_i\).
\end{theorem}

\begin{proof}
  Puede consultarse una generalización de la demostración en \parencite[Teorema 1.2.9]{jacobson_finite-dimensional_1996}.
\end{proof}

El problema de comprobar si dos polinomios \(f, g \in R\) verifican que \(f \sim g\).

% TODO: completar con la dificultad

Definimos la \emph{norma} \(i\)\emph{-ésima} de un elemento \(\gamma \in \mathbb F_q\) como
\[
  N_i(\gamma) = (\sigma^{i-1}\gamma)\dots (\sigma \gamma)\gamma \quad\text{para } i > 0 \quad\text{y } N_0(\gamma) = 1.
\]

\begin{proposition}
  \label{prop:norma-divisor}
  Si \(f(x) = \sum_0^n a_ix^{n-i} \in \mathbb F_q[x; \sigma]\) y \(\gamma \in \mathbb F_q\) entonces \((x - \gamma) \mid_d f(x)\) si y solo si \(\sum_0^n a_iN_{n-i}(\gamma) = 0\).
\end{proposition}

\begin{proof}
  Tenemos la identidad
\end{proof}

También se dan las siguientes identidades, que nos serán útiles cuando estudiemos los códigos cíclicos sesgados en el capítulo siguiente.
Dados \(\alpha, \beta, \gamma \in \mathbb F_q\) tales que \(\beta = \alpha^{-1}\sigma(\alpha)\) se tiene que
\begin{align}
  N_i(\sigma^k(\gamma)) &= \sigma^k(N_i(\gamma)),\nonumber\\
  N_i(\sigma^k(\beta)) &= \sigma^k(\alpha)^{-1}\sigma^{k+1}(\alpha).
  \label{eq:norma-beta}
\end{align}
\chapter{Códigos cíclicos sesgados}

En este capítulo definiremos los códigos cíclicos sesgados sobre un cuerpo finito \(\mathbb F_q\).
Consideremos el anillo de polinomios de Ore \(R = \mathbb F_q[x, \sigma]\).
Supondremos que el orden del automorfismo \(\sigma\) es \(n\).
Entonces el polinomio \(x^n - 1\) es central en \(R\), por lo que podemos definir el anillo cociente \(\mathcal R = \mathbb F_q[x, \sigma]/(x^n - 1)\).
\(\mathcal R\) es isomorfo a \(\mathbb F_q^n\) mediante la aplicación de coordenadas \(\mathfrak v : \mathcal R \to \mathbb F_q^n\).

\begin{definition}
  Un \emph{código cíclico sesgado} sobre \(\mathbb F_q\) es un subespacio vectorial \(\mathcal C \subseteq \mathbb F_q^n\) tal que \(\mathfrak v^{-1}(\mathcal C)\) es un ideal por la izquierda de \(\mathcal R\).
  Equivalentemente, es un subespacio vectorial \(\mathcal C \subseteq \mathbb F_q^n\) tal que si \((a_0, \dots, a_{n-2}, a_{n-1}) \in \mathcal C\) entonces \((\sigma(a_{n-1}), \sigma(a_0), \dots, \sigma(a_{n-2})) \in \mathcal C\).
\end{definition}

Sabemos que todo ideal por la izquierda de \(\mathcal R\) es principal, por lo que todo código cíclico sesgado está generado por un polinomio en \(R\).
De forma análoga a como ocurría con los códigos cíclicos, este generador es un divisor por la derecha de \(x^n - 1\), por lo que nos interesa conocer de nuevo la descomposición de \(x^n - 1\) en factores, esta vez sobre \(R\).

No existe un algoritmo de factorización completo para los polinomios de Ore, por lo que vamos a utilizar a continuación un método concreto para \(x^n - 1\), descrito en \parencite{gomez-torrecillas_new_2016}.
Por el teorema de la base normal podemos tomar un elemento \(\alpha \in \mathbb F_q\) tal que \(\{\alpha, \sigma(\alpha), \dots, \sigma^{n-1}(\alpha)\}\) sea una base de  \(\mathbb F_q\) como \(\mathbb F_q^{\sigma}\)-espacio vectorial.
% TODO: explicar todo esto un poquito
Fijamos en lo que sigue \(\beta = \alpha^{-1}\sigma(\alpha)\).

\begin{lemma}
  \label{lem:pol-t-beta}
  Para cada subconjunto \(\{t_1, t_2, \dots, t_m\} \subseteq \{0, 1, \dots, n - 1\}\) el polinomio 
  \[
    g = \left[x - \sigma^{t_1}(\beta), x - \sigma^{t_{2}}(\beta), \dots, x - \sigma^{t_m}(\beta)\right]_{i}
  \]
  tiene grado \(m\).
  Por tanto, si \(x - \sigma^s(\beta) \mid_d g\) entonces \(s \in T\).
\end{lemma}

\begin{proof}
  TODO: referenciar.
\end{proof}

\begin{corollary}
  Se tiene que
  \[
  x^n - 1 = \left[x - \beta, x - \sigma(\beta), \dots, x - \sigma^{n-1}(\beta)\right]_i
  \]
\end{corollary}

\begin{proof}
  Como consecuencia del lema \ref{lem:pol-t-beta} se tiene que 
  \[
    \left[x - \beta, x - \sigma(\beta), \dots, x - \sigma^{n-1}(\beta)\right]_i
  \]
  tiene grado \(n\).
  Pero además, por (\ref{eq:norma-beta}) se tiene que
  \[
  N_n(\sigma^k(\beta)) = \sigma^k(\alpha)^{-1}\sigma^{k+n}(\alpha) = \sigma^k(\alpha^{-1}\alpha) = 1
  \]
  y por tanto 
  \[
    -1N_0(\sigma^k(\beta)) + 1N_n(\sigma^k(\beta)) = -1 + 1 = 0,
  \]
  por lo que por la proposición \ref{prop:norma-divisor} cada \(x - \sigma^k(\beta)\) divide a \(x^n -1\) por la derecha para todo \(0 \leq k \leq n -1\).
  Estas dos afirmaciones nos conducen a que 
  \[
  x^n - 1 = \left[x - \beta, x - \sigma(\beta), \dots, x - \sigma^{n-1}(\beta)\right]_i,
  \]
  como queríamos. 
\end{proof}

Este corolario nos permite afirmar que dados \(\{t_1, \dots, t_k\} \subset \{0, 1, \dots, n - 1\}\) el polinomio \(g = [x - \sigma^{t_1}(\beta), \dots, x - \sigma^{t_k}(\beta)]_i\) genera un ideal por la izquierda \(\mathcal Rg\) tal que \(\mathfrak v(\mathcal Rg)\) es un código cíclico sesgado de dimensión \(n - k\).

% MAYBE: antes de proceder a definir los códigos que vamos a usar en el algoritmo...
Antes de proceder a definir el tipo de códigos cíclicos sesgados que vamos a utilizar es necesario que comentemos algunos resultados que necesitaremos más adelante.
Comenzamos indicando que llamaremos \(\beta\)-raíces a los elementos del conjunto \(\{\beta, \sigma(\beta), \dots, \sigma^{n-1}(\beta)\}\).
Por la proposición \ref{prop:norma-divisor} y (\ref{eq:norma-beta}) se tiene que, dado un polinomio \(f = \sum_{i=0}^{n-1}a_ix^i \in \mathcal R\),
\[
  x - \sigma^j(\beta) \mid_d f \iff \sum_{i=0}^{n-1}a_iN_i(\sigma^j(\beta)) = 0 \iff \sum_{i=0}^{n-1}a_i\sigma^{i+j}(\alpha) = 0.
\]
Sea \(N\) la matriz formada por las normas de las \(\beta\)-raíces:
\[
  N = \begin{pmatrix}
    N_0(\beta) & N_0(\sigma(\beta)) & \cdots & N_0(\sigma^{n-1}(\beta))\\
    N_1(\beta) & N_1(\sigma(\beta)) & \cdots & N_1(\sigma^{n-1}(\beta))\\
    \vdots & \vdots & \ddots & \vdots\\
    N_{n-1}(\beta) & N_{n-1}(\sigma(\beta)) & \cdots & N_{n-1}(\sigma^{n-1}(\beta))\\
  \end{pmatrix}.
\]
Las componentes de \(\mathfrak v(f)N = (a_1, \dots, a_{n-1})N\) son las evaluaciones por la derecha de \(f\) en el conjunto de las \(\beta\)-raíces, es decir, el vector compuesto por los restos por la izquierda obtenidos al dividir \(f\) por los polinomios \(x - \sigma^i(\beta)\) para \(i = 0, \dots, n - 1\).
Por tanto, tenemos que el diagrama es conmutativo
\begin{center}
  \begin{tikzcd}[column sep=large, row sep=large]
    \mathcal R \arrow[d, "\mathfrak v"] \arrow[rd, bend left, "ev"] & \\
    L^n \arrow[r, "\cdot N"] & L^n
  \end{tikzcd}
\end{center}
en el que \(ev\) representa una aplicación que lleva cada polinomio \(f\) en la \(n\)-tupla formada por los restos por la izquierda de dividir \(f\) por \(x - \sigma^i(\beta)\), para \(i = 0, \dots, n - 1\).
Es posible probar que la matriz \(N\) es no singular \parencite[ver][Lema 2.1]{gomez-torrecillas_petersongorensteinzierler_2018}, por lo que es un cambio de base de \(L^n\).

Dado un polinomio \(f\) llamamos \emph{conjunto de}\(\beta\)\emph{-raíces} del polinomio \(f\) al conjunto formado por las \(\beta\)-raíces \(\gamma\) que cumplen \(x - \gamma \mid_d f\), es decir, a aquellas correspondientes a las coordenadas nulas de \((a_0, \dots, a_{n-1})N\).
Decimos que un divisor por la derecha no constante \(f \mid_d x^n - 1\) \(\beta\)\emph{-descompone totalmente} si existen \(\{t_1, \dots, t_m\} \subseteq \{0, 1, \dots, n-1\}\) tales que
\[
  f = \left[x - \sigma^{t_1}(\beta), \dots, x - \sigma^{t_m}(\beta)\right]_{i}.
\]
Sabemos por el lema \ref{lem:pol-t-beta} que \(\deg f = m\), el cardinal del conjunto de las \(\beta\)-raíces de \(f\).

\begin{lemma}
  Sea \(f = \sum_{i=0}^m a_ix^i \in \mathcal R\) con \(a_m \neq 0\) y
  \[
    M_f = \begin{pmatrix}
      a_0 & a_1 & \cdots & a_m & 0 & \cdots & 0 \\
      0 & \sigma(a_0) & \cdots & \sigma(a_{m-1}) & \sigma(a_m) & \cdots & 0 \\
       &  & \ddots &  &  & \ddots & 0 \\
      0 & \cdots & 0 & \sigma^{n-m-1}(a_0) & \cdots & \cdots & \sigma^{n-m-1}(a_m) \\
    \end{pmatrix}_{(n-m) \times n}.
  \]
  Entonces las filas de \(M_f\) son la base de \(\mathfrak v(\mathcal Rf)\) como un \(L\)-espacio vectorial.
  Es más, \(f\) \(\beta\)-descompone totalmente si y solo si 
  \[
    \operatorname{mepf}(M_jN) = \left( \begin{array}{@{}c@{}}
      \varepsilon_{i_1}\\\hline
      \vdots\\\hline
      \varepsilon_{i_{m-n}}
    \end{array}\right)
  \]
  para algunos \(0 \leq i_1 < \dots < i_{n-m} \leq n -1\), donde \(\operatorname{mepf}\) denota una matriz escalonada por filas y \(\varepsilon_i\) es un vector canónico de longitud \(n\).
\end{lemma}
%TODO: consultar lo del vector canónico
\begin{proof}
  Una \(\mathbb F_q\)-base de \(\mathcal Rf\) es \(\{f, xf, \dots, x^{n-m-1}f\}\) cuyas coordenadas corresponden precisamente a las filas de \(M_f\).
  Tenemos entonces que \(f = \left[x - \sigma^{t_1}(\beta), \dots, x - \sigma^{t_m}(\beta)\right]_i\) si y solo si cada múltiplo por la izquierda de \(f\) es también múltiplo por la izquierda de \(x - \sigma^{t_i}(\beta)\) para \(1 \leq i \leq m\), si y solo si las \(t_i\)-ésimas columnas de \(M_fN\) son cero para \(i = 1, \dots, m\).
  Como \(M_fN\) tiene \(n - m\) filas, rango \(n - m\) y \(n - m\) columnas distintas de cero, el resultado se deduce fácilmente.
\end{proof}
\begin{lemma}
  Sean \(f, g \in \mathcal R\) polinomios que pueden \(\beta\)-descomponerse totalmente.
  Entonces \((f, g)_d\) y \([f, g]_i\) también pueden \(\beta\)-descomponerse totalmente.
\end{lemma}

\begin{proof}
  Como \(f\) y \(g\) pueden \(\beta\)-descomponerse totalmente existen subconjuntos \(T_1, T_2 \subseteq \{0, \dots, n - 1\}\) tales que 
  \[
    f = \left[\{x - \sigma^i(\beta)\}_{i \in T_1}\right]_i \quad \text{y} \quad g = \left[\{x - \sigma^i(\beta)\}_{i \in T_2}\right]_i.
  \]
  Se deduce rápidamente entonces que
  \[
    [f, g]_i = \left[\{x - \sigma^i(\beta)\}_{i \in T_1 \cup T_2}\right]_i.
  \]
  Por otro lado es evidente que 
  \[
    \left[\{x - \sigma^i(\beta)\}_{i \in T_1 \cap T_2}\right]_i \mid_d (f, g)_d,
  \]
  pero como \(\deg(f) + \deg(g) = \deg((f, g)_d) + \deg([f, g]_i)\) por el lema \ref{lem:pol-t-beta} se tiene la igualdad.
\end{proof}

Estamos ya en disposición de definir la clase de códigos que vamos a utilizar.

\begin{definition}
  Bajo las condiciones y notación de este capítulo, un \emph{código RS (Reed-Solomon) sesgado} de distancia mínima prevista \(\delta\) es un código cíclico \(\mathcal C\) tal que \(\mathfrak v^{-1}(\mathcal C)\) está generado por 
  \[
    \left[x - \sigma^r(\beta), x - \sigma^{r+1}(\beta), \dots, x - \sigma^{r+\delta-2}(\beta)\right]_i
  \] 
  para algún \(r \geq 0\).
\end{definition}

\begin{theorem}
  Un código RS sesgado de distancia mínima prevista \(\delta\) tiene distancia \(\delta\).
\end{theorem}

\begin{proof}
  (TODO: citar)
\end{proof}
\chapter{Algoritmo de Peterson-Gorenstein-Zierler para códigos cíclicos sesgados}

En este capítulo nos adentramos finalmente en el algoritmo que es el objeto de nuestro estudio.

\begin{Ualgorithm}[htbp]
  \DontPrintSemicolon
  \KwIn{el código \(\mathcal C\), el mensaje recibido \(y = (y_0, \dots, y_{n-1}) \in \mathbb F_q^n\) con no más de \(t\) errores}
  \KwOut{el error \(e = (e_0, \dots, e_{n-1})\) tal que \(y - e \in \mathcal C\)}
  \tcp{Paso 1: calcular síndromes}
  \For{\(0 \leq i \leq 2t - 1\)}{
      $s_i \longleftarrow \sum_{j=0}^{n-1}y_jN_j(\sigma^i(\beta))$\;
  }
  \If{\(s_i = 0\) para todo \(0 \leq i \leq 2t - 1\)}{\Return{\(0\)}}
  \tcp{Paso 2: hallar polinomio localizador y las coordenadas de error}
  \(S^t \longleftarrow \left(\sigma^{-j}(s_{i+j})\sigma^i(\alpha)\right)_{0 \leq i \leq t, 0 \leq j \leq t -1}\)\;
  Calcular
  \[
    \operatorname{mepc}(S^t) = \left( \begin{array}{@{}c|c@{}}
      I_{\mu} & \multirow{3}{*}{\(0_{(t+ 1)\times (t - \mu)}\)} \\\cline{1-1}
      a_0 \cdots a_{\mu -1 } & \\\cline{1-1}
      H' &
    \end{array}\right)
  \]\;
  \(\rho = (\rho_0, \dots, \rho_{\mu}) \longleftarrow (-a_0, \dots, -a_{\mu-1}, 1)\) y \(\rho N \longleftarrow (\rho_0, \dots, \rho_{\mu}, 0, \dots, 0)N\)\;
  \(\{k_1, \dots, k_v\} \longleftarrow \) coordenadas igual a cero de \(\rho N\)\;
  \If{\(\mu \neq v\)}{
    Calcular \[M_{\rho} \longleftarrow \begin{pmatrix}
      \rho_0 & \rho_1 & \dots & \rho_{\mu} & 0 & \dots & 0\\
      0 & \sigma(\rho_0) & \dots & \sigma(\rho_{\mu - 1}) & \sigma(\rho_{\mu}) & \dots & 0\\
       & & \ddots & & & \ddots & \\
      0 & \dots & 0 & \sigma^{n - \mu - 1}(\rho_0) & \dots & \dots & \sigma^{n - \mu - 1}(\rho_{\mu})
    \end{pmatrix}_{(n - \mu) \times n}\]\;
    \(N_{\rho} \longleftarrow M_{\rho}N\)\;
    \(H_{\rho} \longleftarrow \operatorname{mepf}(N_{\rho})\)\;
    \(H' \longleftarrow\) la matriz obtenida al eliminar las filas de \(H_{\rho}\) distintas de \(\varepsilon_i\) para algún \(i\)\;
    \(\{k_1, \dots, k_v\} \longleftarrow\) las coordenadas de las columnas igual a cero de \(H'\)\;
  }
  \caption{Peterson-Gorenstein-Zierler para códigos cíclicos sesgados (I).}
\end{Ualgorithm}

\begin{Ualgorithm}[htbp]
  \DontPrintSemicolon
  \setcounter{AlgoLine}{17}
  \tcp{Paso 3: resolver el sistema de los síndromes, obteniendo las magnitudes de error}
  Encontrar \((x_1, \dots, x_v)\) tal que \((x_1, \dots, x_v)(\Sigma^{v-1})^T = (\alpha s_0, \sigma(\alpha)s_1, \dots, \sigma^{v-1}(\alpha)s_{v-1})\)\;
  \tcp{Paso 4: construir el error y devolverlo}
  \Return{\((e_0, \dots, e_{n-1})\) con \(e_i = x_i\) para \(i \in \{k_1, \dots, k_v\}\), cero en otro caso}
  \caption{Peterson-Gorenstein-Zierler para códigos cíclicos sesgados (II).}
  \label{alg:pgz-skwcc-2}
\end{Ualgorithm}

\appendix
\chapter{Funciones en SageMath usadas en los ejemplos}
\label{annex:sage-gen-idemp}

En este anexo se describen algunas de las funciones utilizadas durante los ejemplos a lo largo del trabajo.
De nuevo, el código puede encontrarse en
\begin{center}
  \url{https://github.com/jmml97/tfg/tree/master/code}.
\end{center}

\begin{description}[font=\ttfamily, style=nextline]
  \item[generators(poly)] Devuelve una lista de polinomios generadores de códigos cíclicos de longitud el grado de \texttt{poly}.
  
  \textsc{Argumentos}
  \begin{description}[font=\normalfont\ttfamily]
    \item[poly] Un polinomio de la forma \(x^n - 1\)
  \end{description}
  
  \textsc{Salida}
  \begin{itemize}
    \item La lista de tupla polinomio generador e idempotente generador
  \end{itemize}

  \item[defining\_sets(poly)] Devuelve una lista de tuplas consistentes en un polinomio generador de un código cíclico de longitud el grado de \texttt{poly}, un conjunto característico y una raíz primitiva.
  
  \textsc{Argumentos}
  \begin{description}[font=\normalfont\ttfamily]
    \item[poly] Un polinomio de la forma \(x^n - 1\)
  \end{description}

  \textsc{Salida}
  \begin{itemize}
    \item La lista de polinomios generadores
  \end{itemize}

  \item[generator\_and\_idempotents(poly)] Devuelve una lista de tuplas consistentes en un polinomio generador de un código cíclico de longitud el grado de \texttt{poly} y el idempotente generador correspondiente.
  
  \textsc{Argumentos}
  \begin{description}[font=\normalfont\ttfamily]
    \item[poly] Un polinomio de la forma \(x^n - 1\)
  \end{description}

  \textsc{Salida}
  \begin{itemize}
    \item La lista de tupla polinomio generador e idempotente generador
  \end{itemize}

  \item[mult(iterable)] Devuelve el producto de todos los elementos de \texttt{iterable}.
  
  \textsc{Argumentos}
  \begin{description}[font=\normalfont\ttfamily]
    \item[iterable] Un objeto iterable
  \end{description}

  \textsc{Salida}
  \begin{itemize}
    \item El producto de todos los elementos de \texttt{iterable}
  \end{itemize}

  \item[powerset(iterable)] Devuelve todas las posibles combinaciones de elementos de \texttt{iterable}.
  
  \textsc{Argumentos}
  \begin{description}[font=\normalfont\ttfamily]
    \item[iterable] Un objeto iterable
  \end{description}

  \textsc{Salida}
  \begin{itemize}
    \item Todas las posibles combinaciones de elementos de \texttt{iterable}
  \end{itemize}

  \textsc{Ejemplos}
  \begin{lstlisting}[gobble=4]
    sage: powerset([1,2,3])
    > () (1,) (2,) (3,) (1,2) (1,3) (2,3) (1,2,3)
  \end{lstlisting}

\end{description}
\chapter[Implementación en SageMath del algoritmo PGZ]{Implementación en SageMath del algoritmo de Peterson-Gorenstein-Zierler}
\label{annex:pgz-sage}

Se han desarrollado implementaciones en SageMath del algoritmo de Peterson-Gorenstein-Zierler, tanto en su versión para códigos BCH como para códigos RS sesgados.

Dichas implementaciones aprovechan la estructura de códigos que ya tiene implementada SageMath.
Así, para la versión de códigos BCH se ha implementado un decodificador para códigos BCH, \texttt{BCHPGZDecoder}, que hereda de la clase \texttt{Decoder} de SageMath.
Por otro lado, para la versión de códigos cíclicos ha sido necesario implementar primero la clase \texttt{SkewCyclicCode}, que hereda de la clase \texttt{AbstractLinearCode} de SageMath, y que implementa de forma sencilla los aspectos básicos de códigos cíclicos sesgados, utilizando para ello la implementación existente de anillos de polinomios de Ore de SageMath.
Una vez diseñada esta clase que permite trabajar con códigos cíclicos sesgados se ha implementado una clase \texttt{SkewRSCode} para manejar códigos RS sesgados y un decodificador para este tipo de códigos, \texttt{SkewRSPGZDecoder}.

Su uso es muy sencillo.
Con la orden \texttt{load()} de SageMath pueden cargarse los archivos proporcionados, que incluyen todas las clases descritas antes.

\begin{lstlisting}[gobble=2]
  sage: load(pgz.sage)
  sage: load(pgz-sesgados.sage)
\end{lstlisting}

En este anexo describimos la documentación de las clases y funciones desarrolladas.
El código puede encontrarse en
\begin{center}
  \url{https://github.com/jmml97/tfg/tree/master/code}
\end{center}

\section{Decodificador basado en el algoritmo PGZ para códigos BCH}

\begin{description}[leftmargin=1em, font=\normalfont\ttfamily, style=nextline]
  \item[class BCHPGZDecoder(self, code)]
  
  \emph{Hereda de:} \texttt{Decoder}

  Construye un decodificador para códigos BCH basado en el algoritmo de Peterson-Gorenstein-Zierler para códigos BCH.

  \textsc{Argumentos}
  \begin{description}[font=\normalfont\ttfamily]
    \item[code] Código asociado a este decodificador
  \end{description}

  \textsc{Ejemplos}
  \begin{lstlisting}[gobble=4]
    sage: F = GF(2)
    sage: C = codes.BCHCode(F, 15, 5, offset=1); C
    > [15, 7] BCH Code over GF(2) with designed distance 5
    sage: D = BCHPGZDecoder(C); D
    > Peterson-Gorenstein-Zierler algorithm based decoder for [15, 7] BCH Code over GF(2) with designed distance 5
  \end{lstlisting}

  \begin{description}[font=\ttfamily, style=nextline]
    \item[decode\_to\_code(self, word)] Corrige los errores de \texttt{word} y devuelve una palabra código del código asociado a \texttt{self}.
    
    \textsc{Argumentos}
    \begin{description}[font=\normalfont\ttfamily]
      \item[word] Mensaje recibido que se quiere decodificar. 
      Puede representarse en forma vectorial o polinómica.
    \end{description}
    
    \textsc{Ejemplos}
    \begin{lstlisting}[gobble=6]
      sage: F = GF(2)
      sage: C = codes.BCHCode(F, 15, 5, offset=1)
      sage: D = BCHPGZDecoder(C)
      sage: x = polygen(F)
      sage: y = 1 + x +x^5 + x^6 + x^9 + x^10
      sage: D.decode_to_code(y)
      > (1, 1, 0, 0, 1, 1, 1, 0, 0, 1, 0, 0, 0, 0, 0)
      sage: y = vector(F, (1, 1, 0, 0, 0, 1, 1, 0, 0, 1, 1, 0, 0, 0, 0))
      sage: D.decode_to_code(y)
      > (1, 1, 0, 0, 1, 1, 1, 0, 0, 1, 0, 0, 0, 0, 0)
    \end{lstlisting}

    \item[correction\_capability(self)] Devuelve la capacidad de corrección de errores del decodificador \texttt{self}.
    
    \textsc{Ejemplos}
    \begin{lstlisting}[gobble=6]
      sage: F = GF(2)
      sage: C = codes.BCHCode(F, 15, 5, offset=1)
      sage: D = BCHPGZDecoder(C)
      sage: D.correction_capability()
      > 2
    \end{lstlisting}
  \end{description}
\end{description}

\section{Clase para códigos cíclicos sesgados}

Esta clase \emph{esqueleto} sirve como modelo para la implementación de los códigos RS sesgados.

\begin{description}[leftmargin=1em, font=\normalfont\ttfamily, style=nextline]
  \item[class SkewCyclicCode(self, generator\_pol=None)]
  
  \emph{Hereda de:} \texttt{AbstractLinearCode}

  Representación de un código cíclico sesgado como un código lineal.

  \textsc{Argumentos}
  \begin{description}[font=\normalfont\ttfamily]
    \item[generator\_pol] Polinomio generador utilizado para construir el código
  \end{description}

  \textsc{Ejemplos}
  \begin{lstlisting}[gobble=4]
    sage: F.<a> = GF(2^12)
    sage: Frob = F.frobenius_endomorphism()
    sage: sigma = Frob^10
    sage: S.<x> = SkewPolynomialRing(F, sigma)
    sage: g = left_lcm([x - a^1023, x - a^3327, x - a^3903, x - a^4047])
    sage: C = SkewCyclicCode(g); C
    > [6, 2] Skew Cyclic Code on Skew Polynomial Ring in x over Finite Field in a of size 2^12 twisted by a |--> a^(2^10)
  \end{lstlisting}

  \begin{description}[font=\ttfamily, style=nextline]
    \item[generator\_polynomial()] Devuelve un polinomio generador del código.
    
    \textsc{Ejemplos}
    \begin{lstlisting}[gobble=6]
      sage: F.<a> = GF(2^12)
      sage: Frob = F.frobenius_endomorphism()
      sage: sigma = Frob^10
      sage: S.<x> = SkewPolynomialRing(F, sigma)
      sage: g = left_lcm([x - a^1023, x - a^3327, x - a^3903, x - a^4047])
      sage: C = SkewCyclicCode(g)
      sage: C.generator_polynomial()
      > x^4 + (a^11 + a^10 + a^9 + a^8 + a^7 + a^5 + a^4 + a^2)*x^3 + (a^4 + a^2 + a)*x^2 + (a^11 + a^10 + a^9 + a^8 + a^6 + a^3)*x + a^11 + a^8 + a^7 + a^6 + a^2 + a
    \end{lstlisting}
    \item[polynomial\_ring()] Devuelve el anillo de polinomios sobre el que está definido el código.
     
    \textsc{Ejemplos}
    \begin{lstlisting}[gobble=6]
      sage: F.<a> = GF(2^12)
      sage: Frob = F.frobenius_endomorphism()
      sage: sigma = Frob^10
      sage: S.<x> = SkewPolynomialRing(F, sigma)
      sage: g = left_lcm([x - a^1023, x - a^3327, x - a^3903, x - a^4047])
      sage: C = SkewCyclicCode(g)
      sage: C.polynomial_ring()
      > Skew Polynomial Ring in x over Finite Field in a of size 2^12 twisted by a |--> a^(2^10)
    \end{lstlisting} 
    \item[primitive\_root()] Devuelve una raíz primitiva del cuerpo sobre el que está definido el código.
     
    \textsc{Ejemplos}
    \begin{lstlisting}[gobble=6]
      sage: F.<a> = GF(2^12)
      sage: Frob = F.frobenius_endomorphism()
      sage: sigma = Frob^10
      sage: S.<x> = SkewPolynomialRing(F, sigma)
      sage: g = left_lcm([x - a^1023, x - a^3327, x - a^3903, x - a^4047])
      sage: C = SkewCyclicCode(g)
      sage: C.primitive_root()
      > a
    \end{lstlisting}
    \item[ring\_automorphism()] Devuelve el automorfismo usado en el anillo de polinomios de Ore sobre el que está definido el código.
     
    \textsc{Ejemplos}
    \begin{lstlisting}[gobble=6]
      sage: F.<a> = GF(2^12)
      sage: Frob = F.frobenius_endomorphism()
      sage: sigma = Frob^10
      sage: S.<x> = SkewPolynomialRing(F, sigma)
      sage: g = left_lcm([x - a^1023, x - a^3327, x - a^3903, x - a^4047])
      sage: C = SkewCyclicCode(g)
      sage: C.ring_automorphism()
      > Frobenius endomorphism a |--> a^(2^10) on Finite Field in a of size 2^12
    \end{lstlisting} 
  \end{description}
\end{description}

\section{Codificadores para códigos cíclicos sesgados}

Las siguientes clases son codificadores para los códigos cíclicos sesgados.
Uno de ellos codifica vetores en palabras código y el otro, polinomios en palabras código.
La clase \texttt{SkewCyclicVectorEncoder} está indicada como clase codificadora por defecto y por tanto puede utilizarse directamente con el método \texttt{encode()} del código.

\begin{description}[leftmargin=1em, font=\normalfont\ttfamily, style=nextline]
  \item[class SkewCyclicVectorEncoder(self, code)]
  
  \emph{Hereda de:} \texttt{Encoder}

  Codificador que codifica vectores en palabras código.
  Sea \(\mathcal C\) un código cíclico sesgado sobre un cuerpo finito \(\mathbb F\) y \(g\) un polinomio generador suyo.
  Sea \(m = (m_1, m_2, \dots, m_k)\) un vector en \(\mathbb F^k\).
  Para codificar \(m\) este codificador realiza el producto \(mM(g)\), donde \(M(g)\) es una matriz generadora de \(g\).  

  \textsc{Argumentos}
  \begin{description}[font=\normalfont\ttfamily]
    \item[code] Código asociado a este codificador
  \end{description}

  \textsc{Ejemplos}
  \begin{lstlisting}[gobble=4]
    sage: F.<a> = GF(2^12)
    sage: Frob = F.frobenius_endomorphism()
    sage: sigma = Frob^10
    sage: S.<x> = SkewPolynomialRing(F, sigma)
    sage: g = left_lcm([x - a^1023, x - a^3327, x - a^3903, x - a^4047])
    sage: C = SkewCyclicCode(g)
    sage: E = SkewCyclicVectorEncoder(C); E
    > Vector-style encoder for [6, 2] Skew Cyclic Code on Skew Polynomial Ring in x over Finite Field in a of size 2^12 twisted by a |--> a^(2^10)
    sage: E.encode(vector(F, [a, 1]))
    > (a^9 + a^8 + a^6 + a^5 + a^2 + a + 1, a^9 + a^6 + a^5 + a^4 + a^3 + 1, a^11 + a^9 + a^8 + a^2 + a + 1, a^11 + a^10 + a^8 + a^6 + a^4 + a^3 + a^2 + 1, a^11 + a^8 + a^7 + a^6 + a^5 + a^4 + a^3 + a^2 + a, 1)
  \end{lstlisting}

  El siguiente ejemplo usa el codificador directamente desde el código, ya que está fijado como codificador por defecto.

  \begin{lstlisting}[gobble=4]
    sage: C.encode(vector(F, [a, 1]))
    > (a^9 + a^8 + a^6 + a^5 + a^2 + a + 1, a^9 + a^6 + a^5 + a^4 + a^3 + 1, a^11 + a^9 + a^8 + a^2 + a + 1, a^11 + a^10 + a^8 + a^6 + a^4 + a^3 + a^2 + 1, a^11 + a^8 + a^7 + a^6 + a^5 + a^4 + a^3 + a^2 + a, 1)
  \end{lstlisting}

  \begin{description}[font=\ttfamily, style=nextline]
    \item[generator\_matrix()] Devuelve una matriz generadora del código sobre el que está construido el codificador.
    
    \textsc{Ejemplos}
    \begin{lstlisting}[gobble=6]
      sage: F.<a> = GF(2^12)
      sage: Frob = F.frobenius_endomorphism()
      sage: sigma = Frob^10
      sage: S.<x> = SkewPolynomialRing(F, sigma)
      sage: g = left_lcm([x - a^1023, x - a^3327, x - a^3903, x - a^4047])
      sage: C = SkewCyclicCode(g)
      sage: E = SkewCyclicVectorEncoder(C); E
      sage: E.generator_matrix()
      > [               a^11 + a^8 + a^7 + a^6 + a^2 + a             a^11 + a^10 + a^9 + a^8 + a^6 + a^3                                   a^4 + a^2 + a a^11 + a^10 + a^9 + a^8 + a^7 + a^5 + a^4 + a^2                                               1                                               0]
      [                                              0                                 a^11 + a^10 + a            a^11 + a^9 + a^8 + a^5 + a^3 + a + 1           a^9 + a^7 + a^6 + a^4 + a^3 + a^2 + a  a^11 + a^8 + a^7 + a^6 + a^5 + a^4 + a^3 + a^2                                               1]
    \end{lstlisting}
  \end{description}

  \item[class SkewCyclicPolynomialEncoder(self, code)]

  \emph{Hereda de:} \texttt{Encoder}

  Codificador que codifica polinomios en palabras código.
  Sea \(\mathcal C\) un código cíclico sesgado sobre un cuerpo finito \(\mathbb F\) y \(g\) un polinomio generador suyo.
  Dado cualquier polinomio \(p \in \mathbb F[x]\) calculando el producto \(c = p \cdot g\) y devolviendo el vector de coeficientes correspondiente.

  \textsc{Argumentos}
  \begin{description}[font=\normalfont\ttfamily]
    \item[code] Código asociado a este codificador
  \end{description}

  \textsc{Ejemplos}
  \begin{lstlisting}[gobble=4]
    sage: F.<a> = GF(2^12)
    sage: Frob = F.frobenius_endomorphism()
    sage: sigma = Frob^10
    sage: S.<x> = SkewPolynomialRing(F, sigma)
    sage: g = left_lcm([x - a^1023, x - a^3327, x - a^3903, x - a^4047])
    sage: C = SkewCyclicCode(g)
    sage: E = SkewCyclicPolynomialEncoder(C); E
    > Polynomial-style encoder for [6, 2] Skew Cyclic Code on Skew Polynomial Ring in x over Finite Field in a of size 2^12 twisted by a |--> a^(2^10)
  \end{lstlisting}

  \begin{description}[font=\ttfamily, style=nextline]
    \item[message\_space(self)] Devuelve el espacio de mensajes del codificador, que es el anillo de polinomios sobre el que está definido el código asociado.
    
    \textsc{Ejemplos}
    \begin{lstlisting}[gobble=6]
      sage: F.<a> = GF(2^12)
      sage: Frob = F.frobenius_endomorphism()
      sage: sigma = Frob^10
      sage: S.<x> = SkewPolynomialRing(F, sigma)
      sage: g = left_lcm([x - a^1023, x - a^3327, x - a^3903, x - a^4047])
      sage: C = SkewCyclicCode(g)
      sage: E = SkewCyclicPolynomialEncoder(C)
      sage: E.message_space()
      > Skew Polynomial Ring in x over Finite Field in a of size 2^12 twisted by a |--> a^(2^10)
    \end{lstlisting}

    \item[encode(self, p)] Transforma \texttt{p} en un elemento del código asociado a \texttt{self}.
    
    \textsc{Argumentos}
    \begin{description}[font=\normalfont\ttfamily]
      \item[p] Un polinomio del espacio de mensajes de \texttt{self}.
    \end{description}

    \textsc{Salida}
    \begin{itemize}
      \item Una palabra código del código asociado a \texttt{self}
    \end{itemize}
    
    \textsc{Ejemplos}
    \begin{lstlisting}[gobble=6]
      sage: F.<a> = GF(2^12)
      sage: Frob = F.frobenius_endomorphism()
      sage: sigma = Frob^10
      sage: S.<x> = SkewPolynomialRing(F, sigma)
      sage: g = left_lcm([x - a^1023, x - a^3327, x - a^3903, x - a^4047])
      sage: C = SkewCyclicCode(g)
      sage: E = SkewCyclicPolynomialEncoder(C)
      sage: E.encode(x + a)
      > (a^9 + a^8 + a^6 + a^5 + a^2 + a + 1, a^9 + a^6 + a^5 + a^4 + a^3 + 1, a^11 + a^9 + a^8 + a^2 + a + 1, a^11 + a^10 + a^8 + a^6 + a^4 + a^3 + a^2 + 1, a^11 + a^8 + a^7 + a^6 + a^5 + a^4 + a^3 + a^2 + a, 1)
    \end{lstlisting}

    \item[unencode\_nocheck(self, c)] Devuelve el mensaje correspondiente a \texttt{c}.
    No comprueba si \texttt{c} pertenece al código asociado a \texttt{self}.
    
    \textsc{Argumentos}
    \begin{description}[font=\normalfont\ttfamily]
      \item[c] Un vector de la misma longitud que el código asociado a \texttt{self}.
    \end{description}

    \textsc{Salida}
    \begin{itemize}
      \item Un polinomio del espacio de mensajes de \texttt{self}.
    \end{itemize}
    
    \textsc{Ejemplos}
    \begin{lstlisting}[gobble=6]
      sage: F.<a> = GF(2^12)
      sage: Frob = F.frobenius_endomorphism()
      sage: sigma = Frob^10
      sage: S.<x> = SkewPolynomialRing(F, sigma)
      sage: g = left_lcm([x - a^1023, x - a^3327, x - a^3903, x - a^4047])
      sage: C = SkewCyclicCode(g)
      sage: E = SkewCyclicPolynomialEncoder(C)
      sage: E.unencode_nocheck(vector(F, (a^9 + a^8 + a^6 + a^5 + a^2 + a + 1, a^9 + a^6 + a^5 + a^4 + a^3 + 1, a^11 + a^9 + a^8 + a^2 + a + 1, a^11 + a^10 + a^8 + a^6 + a^4 + a^3 + a^2 + 1, a^11 + a^8 + a^7 + a^6 + a^5 + a^4 + a^3 + a^2 + a, 1)))
      > x + a
    \end{lstlisting}
  \end{description}
\end{description}


\section{Clase para códigos RS sesgados}

\begin{description}[leftmargin=1em, font=\normalfont\ttfamily, style=nextline]
  \item[class SkewRSCode(self, generator\_pol=None, b\_roots=None)]
  
  \emph{Hereda de:} \texttt{SkewCyclicCode}

  Representación de un código RS sesgado.
  Puede construirse de dos formas equivalentes, o bien mediante un polinomio generador o bien mediante las raíces del mismo.
  En cualquier caso el polinomio generador ha de ser un divisor de \(x^n - 1\), donde \(n\) es el orden del automorfismo del anillo de polinomios de Ore subyacente.

  \textsc{Argumentos}
  \begin{description}[font=\normalfont\ttfamily]
    \item[generator\_pol] Polinomio generador utilizado para construir el código
    \item[b\_roots] \(\beta\)-raíces utilizadas para construir un polinomio generador del código 
  \end{description}

  \textsc{Ejemplos}
  \begin{lstlisting}[gobble=4]
    sage: F.<a> = GF(2^12)
    sage: Frob = F.frobenius_endomorphism()
    sage: sigma = Frob^10
    sage: S.<x> = SkewPolynomialRing(F, sigma)
    sage: g = left_lcm([x - a^1023, x - a^3327, x - a^3903, x - a^4047])
    sage: C = SkewRSCode(generator_pol=g); C
    > [6, 2] Skew RS Code on Skew Polynomial Ring in x over Finite Field in a of size 2^12 twisted by a |--> a^(2^10)
    sage: C = SkewRSCode(b_roots=[x - a^1023, x - a^3327, x - a^3903, x - a^4047]); C
    > [6, 2] Skew RS Code on Skew Polynomial Ring in x over Finite Field in a of size 2^12 twisted by a |--> a^(2^10)
  \end{lstlisting}

  \begin{description}[font=\ttfamily, style=nextline]
    \item[designed\_distance(self)] Devuelve la distancia mínima prevista del código.
    
    \textsc{Ejemplos}
    \begin{lstlisting}[gobble=6]
      sage: F.<a> = GF(2^12)
      sage: Frob = F.frobenius_endomorphism()
      sage: sigma = Frob^10
      sage: S.<x> = SkewPolynomialRing(F, sigma)
      sage: g = left_lcm([x - a^1023, x - a^3327, x - a^3903, x - a^4047])
      sage: C = SkewRSCode(generator_pol=g)
      sage: C.designed_distance()
      > 5
    \end{lstlisting}
  \end{description}
\end{description}


\section{Decodificador basado en el algoritmo PGZ para códigos RS sesgados}

\begin{description}[leftmargin=1em, font=\normalfont\ttfamily, style=nextline]
  \item[class SkewRSPGZDecoder(self, code)]
  
  \emph{Hereda de:} \texttt{Decoder}

  Construye un decodificador para códigos RS sesgados basado en el algoritmo de Peterson-Gorenstein-Zierler para códigos RS sesgados.

  \textsc{Argumentos}
  \begin{description}[font=\normalfont\ttfamily]
    \item[code] Código asociado a este decodificador
  \end{description}

  \textsc{Ejemplos}
  \begin{lstlisting}[gobble=4]
    sage: F.<a> = GF(2^12)
    sage: Frob = F.frobenius_endomorphism()
    sage: sigma = Frob^10
    sage: S.<x> = SkewPolynomialRing(F, sigma)
    sage: g = left_lcm([x - a^1023, x - a^3327, x - a^3903, x - a^4047])
    sage: C = SkewRSCode(generator_pol=g)
    sage: D = SkewRSPGZDecoder(C); D
    > Peterson-Gorenstein-Zierler algorithm based decoder for [6, 2] Skew RS Code on Skew Polynomial Ring in x over Finite Field in a of size 2^12 twisted by a |--> a^(2^10)
  \end{lstlisting}

  \begin{description}[font=\ttfamily, style=nextline]
    \item[decode\_to\_code(self, word)] Corrige los errores de \texttt{word} y devuelve una palabra código del código asociado a \texttt{self}.
    
    \textsc{Ejemplos}
    \begin{lstlisting}[gobble=6]
      sage: y = x^5 + a^3953*x^4 + a^671*x^3 + a^2604*x^2 + a^1596*x + a^3699
      sage: D.decode_to_code(y)
      > (a^9 + a^8 + a^6 + a^5 + a^2 + a + 1, a^9 + a^6 + a^5 + a^4 + a^3 + 1, a^11 + a^9 + a^8 + a^2 + a + 1, a^11 + a^10 + a^8 + a^6 + a^4 + a^3 + a^2 + 1, a^11 + a^8 + a^7 + a^6 + a^5 + a^4 + a^3 + a^2 + a, 1)
      sage: C.encode(x + a, "SkewCyclicPolynomialEncoder") == D.decode_to_code(y)
      > True
    \end{lstlisting}
  
  \end{description}
\end{description}

\section{Funciones auxiliares}

Para el desarrollo de las clases anteriores fueron necesarias varias funciones auxiliares que realizan tareas que se repiten a lo largo de todo el código.
Las describimos a continuación.

\begin{description}[font=\ttfamily, style=nextline]
  \item[\_to\_complete\_list(poly, length)] Devuelve una lista de longitud exactamente \texttt{length} correspondiente a los coeficientes del polinomio \texttt{poly}.
  Si es necesario, se completa con ceros. 
  
  \textsc{Argumentos}
  \begin{description}[font=\normalfont\ttfamily]
    \item[poly] Un polinomio
    \item[length] Un entero 
  \end{description}

  \textsc{Salida}
  \begin{itemize}
    \item La lista de los coeficientes
  \end{itemize}
  
  \textsc{Ejemplos}
  \begin{lstlisting}[gobble=4]
    sage: F.<a> = GF(2^12)
    sage: Frob = F.frobenius_endomorphism()
    sage: sigma = Frob^10
    sage: S.<x> = SkewPolynomialRing(F, sigma)
    sage: _to_complete_list(x + a, 15)
    > [a, 1, 0, 0, 0, 0, 0, 0, 0, 0, 0, 0, 0, 0, 0]
  \end{lstlisting}

  \item[left\_lcm(pols)] Calcula el mínimo común múltiplo por la izquierda de todos los polinomios de la lista \texttt{pols}.
  
  \textsc{Argumentos}
  \begin{description}[font=\normalfont\ttfamily]
    \item[pols] Lista de polinomios
  \end{description}

  \textsc{Salida}
  \begin{itemize}
    \item El mínimo común múltiplo de todos los polinomios en \texttt{pols}
  \end{itemize}
  
  \textsc{Ejemplos}
  \begin{lstlisting}[gobble=4]
    sage: F.<a> = GF(2^12)
    sage: Frob = F.frobenius_endomorphism()
    sage: sigma = Frob^10
    sage: S.<x> = SkewPolynomialRing(F, sigma)
    sage: g = left_lcm([x - a^1023, x - a^3327, x - a^3903, x - a^4047])
    > x^4 + (a^11 + a^10 + a^9 + a^8 + a^7 + a^5 + a^4 + a^2)*x^3 + (a^4 + a^2 + a)*x^2 + (a^11 + a^10 + a^9 + a^8 + a^6 + a^3)*x + a^11 + a^8 + a^7 + a^6 + a^2 + a
  \end{lstlisting}

  \item[norm(i, gamma, sigma)] Calcula la \texttt{i}-ésima norma de \texttt{gamma} con el automorfismo \texttt{sigma}.
  
  Recordemos que definimos la \emph{norma} \(i\)\emph{-ésima} de un elemento \(\gamma \in \mathbb F_q\) como \(N_i(\gamma) = \sigma(N_{i-1}(\gamma))(\gamma) = \sigma^{i-1}(\gamma)\dots \sigma(\gamma)\gamma\) para \(i > 0\) y \(N_0(\gamma) = 1\).
  
  \textsc{Argumentos}
  \begin{description}[font=\normalfont\ttfamily]
    \item[i] El orden de la norma
    \item[gamma] El elemento al que se le quiere calcular la norma
    \item[sigma] El automorfismo usado para calcular la norma  
  \end{description}

  \textsc{Salida}
  \begin{itemize}
    \item La \texttt{i}-ésima norma de \texttt{gamma} con el automorfismo \texttt{sigma}
  \end{itemize}
  
  \textsc{Ejemplos}
  \begin{lstlisting}[gobble=4]
    sage: F.<a> = GF(2^12)
    sage: Frob = F.frobenius_endomorphism()
    sage: sigma = Frob^10
    sage: S.<x> = SkewPolynomialRing(F, sigma)
    sage: norm(3, F(1 + a), sigma)
    > a^11 + a^10 + a^8 + a^3 + a^2 + a
  \end{lstlisting}
\end{description}

%-------------------------------------------------------------------------------
%	BIBLIOGRAFÍA
%-------------------------------------------------------------------------------

\newpage
\printbibliography

\pagestyle{empty}

\hfill

\vfill


\pdfbookmark[0]{Colofón}{colofon}
\section*{Colofón}
Este trabajo comenzó a escribirse en Granada en octubre de 2019 y fue terminado en Rincón de la Victoria en junio de 2020, mientras el mundo se encontraba inmerso en la pandemia de la covid-19.

Este trabajo ha sido compuesto utilizando \LaTeX\ con el estilo del paquete \texttt{classicthesis}, desarrollado por André Miede e Ivo Pletikosić e inspirado en el del libro de Robert Bringhurst, «\emph{The Elements of Typographic Style}».
Puede obtenerse una copia de dicho paquete en
\begin{center}
\url{https://bitbucket.org/amiede/classicthesis/}
\end{center}
Las tipografías utilizadas han sido \emph{EBGaramond} para el cuerpo, \emph{Garamond Math} para las matemáticas, \emph{Open Sans} para las leyendas y \emph{Go Mono} para el código.
\bigskip

\end{document}
