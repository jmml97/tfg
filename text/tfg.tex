\documentclass[
  a4paper,
  12pt,
  spanish,
  dvipsnames,
  footinclude,
  headinclude,
]{scrbook}

%-------------------------------------------------------------------------------
%	PAQUETES
%-------------------------------------------------------------------------------

% Idioma

\usepackage[es-noindentfirst, es-lcroman, es-tabla]{babel}

% Matemáticas

\usepackage{amsmath, amsthm, amssymb}
\usepackage{mathtools}
\usepackage{commath}

% Enlaces y colores

\usepackage{hyperref}
\usepackage{xcolor}
\hypersetup{
  colorlinks=true,
  citecolor=,
}

% Otros elementos de página

\usepackage{enumitem}
\usepackage[labelfont=sf, textfont=sf]{caption}

% Tikz

\usepackage{tikz}
\usetikzlibrary{babel}
\usepackage{float}

% Código

\usepackage{listings}
\lstset{
	basicstyle=\footnotesize\ttfamily,%
	breaklines=true,%
	captionpos=b,                    % sets the caption-position to bottom
  tabsize=2,	                   % sets default tabsize to 2 spaces
  frame=lines,
  numbers=left,
  xleftmargin=18pt,
  stepnumber=1,
  aboveskip=12pt,
  showstringspaces=false,
}
\renewcommand{\lstlistingname}{Listado}

% Citas de texto en línea/bloque

\usepackage[autostyle]{csquotes}

% Bibliografía

\usepackage[sorting=none, style=apa, isbn=true]{biblatex}
\addbibresource{bibliografia.bib}

% Lorem ipsum

\usepackage{blindtext}

%-------------------------------------------------------------------------------
%	ENTORNOS MATEMÁTICOS
%-------------------------------------------------------------------------------

\newtheoremstyle{theorem-style}  % Nombre del estilo
{\topsep}                                  % Espacio por encima
{\topsep}                                  % Espacio por debajo
{\itshape}                                  % Fuente del cuerpo
{0pt}                                  % Identación
{\scshape}                      % Fuente para la cabecera
{.}                                 % Puntuación tras la cabecera
{5pt plus 1pt minus 1pt}                              % Espacio tras la cabecera
{{\thmname{#1}\thmnumber{ #2}}\thmnote{ (#3)}}  % Especificación de la cabecera
\theoremstyle{theorem-style}
\newtheorem{theorem}{Teorema}[section]
\newtheorem{corollary}[theorem]{Corolario}
\newtheorem{lemma}[theorem]{Lema}
\newtheorem{proposition}[theorem]{Proposición}
\newtheorem{question}{Pregunta}
\newtheorem{conjecture}[theorem]{Conjetura}
\newtheoremstyle{definition-style}  % Nombre del estilo
{\topsep}                                  % Espacio por encima
{\topsep}                                  % Espacio por debajo
{}                                  % Fuente del cuerpo
{0pt}                                  % Identación
{\scshape}                      % Fuente para la cabecera
{.}                                 % Puntuación tras la cabecera
{5pt plus 1pt minus 1pt}                              % Espacio tras la cabecera
{{\thmname{#1}\thmnumber{ #2}}\thmnote{ (#3)}}  % Especificación de la cabecera
\theoremstyle{definition-style}
\newtheorem{definition}[theorem]{Definición}
\newtheorem{example}[theorem]{Ejemplo}
\newtheorem{notation}[theorem]{Notación}
\newtheorem{exercise}[theorem]{Ejercicio}

%-------------------------------------------------------------------------------
%	NOTSOCLASSICTHESIS & FUENTES
%-------------------------------------------------------------------------------

\usepackage[
  drafting=true,
  %tocaligned=true,
  dottedtoc=true,
  parts,
  floatperchapter,
  pdfspacing,
  beramono=false,
  palatino=false,
]{notsoclassicthesis}

\setmainfont[WordSpace=1.3]{EBGaramond}
\setsansfont{Source Sans Pro}
\renewfontface\chapterNumber[Scale=7, Color=000000]{EBGaramond}
\setmonofont[Scale=0.8]{Vera Mono}

\usepackage[math-style=TeX]{unicode-math}
\setmathfont{Garamond Math}[StylisticSet={3}]

%-------------------------------------------------------------------------------
%	TÍTULO
%-------------------------------------------------------------------------------

\title{Algoritmo de Peterson-Gorenstein-Zierler para códigos cíclicos sesgados}
\renewcommand{\myVersion}{1.0}

\author{José María Martín Luque}

\date{\normalsize\today}

% Portada

\usepackage{eso-pic}
\newcommand\BackgroundPic{%
	\put(0,0){%
		\parbox[b][\paperheight]{\paperwidth}{%
			\vfill
			\centering
      % Indicar la imagen de fondo en el siguiente comando
			\includegraphics[width=\paperwidth,height=\paperheight,%
			keepaspectratio]{assets/portada-ugr-color}%
			\vfill
    }
  }
}

%-------------------------------------------------------------------------------
%	CONTENIDO
%-------------------------------------------------------------------------------

\begin{document}

%\maketitle

\begin{titlepage}
  \AddToShipoutPicture*{\BackgroundPic}
  \phantomsection
  \pdfbookmark[1]{Title}{title}

  % Para que el título esté centrado en la página.
  % Los valores numéricos deberán elegirse de acuerdo con el diseño de
  % página (sobre todo si se cambia la opción BCOR o DIV).
  \begin{addmargin}[2.05cm]{0cm}
  \begin{flushleft}
    \Large
    \hfill\vfil

    \vfill\vfill

    \textsf{Facultad de Ciencias}\\[.5em]
    \textsf{E.T.S. de Ingenierías Informática y de Telecomunicación} \\
    \vfill\vfill

    \textsf{\huge Doble Grado en Ingeniería Informática y Matemáticas} \vfill


    %\textsc{trabajo de fin de grado}
    \spacedlowsmallcaps{Trabajo de Fin de Grado}

    \begingroup
    \Huge{Algoritmo de Peterson-Gorenstein-Zierler para códigos cíclicos sesgados} \\ \bigskip
    \endgroup

    [\,\today\ a las \thistime\ -- \myVersion\,]

    \vfill\vfill\vfill\vfill

    \textsf{\normalsize{Presentado por}}\\[6pt]
    {\huge\textrm{José María Martín Luque}}

    \vspace*{0.7cm}

    \textsf{\normalsize{Tutorizado por}}\\[6pt]
    {\huge\textrm{Gabriel Navarro Garulo}}

    \vfill
    \textsf{Curso académico 2019--2020}
  \end{flushleft}
  \end{addmargin}

\end{titlepage}

\newpage

{\hypersetup{hidelinks}
\tableofcontents
}

\newpage

\chapter*{Resumen}

\blindtext[2]

\chapter*{Summary}

\blindtext[6]

\chapter*{Introducción}

\chapter*{Introducción}
 
El artículo de Claude Shannon \textit{Mathematical Theory of Cryptography} \parencite{shannon_mathematical_1945} y su ampliación posterior, \textit{Mathematical Theory of Communication} \parencite{shannon_mathematical_1948} dieron a luz a dos disciplinas hoy plenamente establecidas, la teoría de información y la teoría de códigos.
El objetivo principal de estas disciplinas es el de establecer mecanismos de comunicación que sean \emph{eficientes} y \emph{fiables} en ambientes posiblemente hostiles.
La eficiencia requiere que la transmisión de la información no necesite de demasiados recursos, sean materiales o temporales.
Por otro lado, la fiabilidad requiere que el mensaje recibido en una comunicación sea lo más parecido posible, dentro de unos márgenes de tolerancia, al mensaje original.
La teoría de la información se encarga del estudio tanto de la representación de la información como de la capacidad que tienen los sistemas para transmitir y procesar la información. 
Por otra parte, la teoría de códigos se basa en los resultados de la teoría de la información para el diseño y desarrollo de modelos de transmisión de información mediante herramientas algebraicas.

A pesar de que digamos que Shannon fue el padre de estas disciplinas, el problema de codificar la información no surge, ni mucho menos, entonces.
De hecho, el filósofo inglés Francis Bacon ya afirmó en el año 1623 que únicamente son necesarios dos símbolos para codificar toda la comunicación.
\blockquote[{\cite[30]{dyson_catedral_2015}}]{La transposición de dos letras en cinco emplazamientos bastará para dar 32 diferencias [y] por este arte se abre un camino por el que un hombre puede expresar y señalar las intenciones de su mente, a un lugar situado a cualquier distancia, mediante objetos ... capaces solo de una doble diferencia.}
Efectivamente, hoy en día la codificación binaria forma parte de prácticamente todos los aspectos nuestras vidas.

Sin embargo, el problema más relevante —y el que nos ocupa— es el de codificar la información para que se produzca una correcta transmisión y recepción de los mensajes a través de un canal.
Como parte de su teoría Shannon introdujo lo que posteriormente se conocería como \emph{modelo de comunicación} de Shannon, un esquema simplificado de cómo se produce la transmisión de información entre emisor y receptor a través de un canal.
\begin{figure}
  \includegraphics[width=\textwidth]{assets/shannon-communication-model.pdf}
  \caption*{Modelo de comunicación de Shannon}
\end{figure}
Al enviar la información a través del canal es posible que, debido al ruido que pueda haber en el mismo, se produzcan interferencias que alteren el mensaje enviado.
Por supuesto recibir un mensaje alterado no es deseable, pues es posible que la alteración sea tal que sea ilegible.
La teoría de códigos trata por tanto de diseñar estructuras algebraicas que permitan la corrección de errores por parte del receptor del mensaje.
Estas estructuras son los códigos, y son el objeto matemático sobre el que orbita todo este trabajo.


\section*{Enfoque}

El título de este trabajo es «Algoritmo de Peterson-Gorenstein-Zierler para códigos cíclicos sesgados», y como tal, expresa el objetivo último del mismo: estudiar y exponer este algoritmo.
En consecuencia, dirigiremos nuestra atención a los conceptos y herramientas necesarios para ello.
En cuanto a los fundamentos matemáticos necesarios, hablaremos de cuerpos finitos, anillos de polinomios sobre cuerpos finitos y anillos de polinomios de Ore sobre cuerpos finitos.
La teoría de códigos se centrará en la exposición de los códigos lineales y de los códigos cíclicos, así como de las familias de códigos concretas sobre las que trabajarán los algoritmos que vamos a describir.
Para comprender mejor el citado algoritmo estudiaremos también la versión original, de mitad del siglo pasado, para los conocidos como códigos BCH.
A lo largo del trabajo ilustraremos algunos ejemplos con el sistema algebraico computacional \href{https://www.sagemath.org/}{Sage}.

\section*{Motivación}

Los códigos cíclicos son muy utilizados porque son muy sencillos de implementar en sistemas digitales utilizando lo que se conoce como registros de desplazamiento.
El uso de anillos de polinomios de Ore nos permite obtener una mayor cantidad de códigos cíclicos.
El algoritmo Peterson-Gorenstein-Zierler para códigos BCH es interesante porque es el que, en general, tiene un menor coste computacional para un número de errores pequeño.
La adaptación para códigos cíclicos sesgados ofrece.

\section*{Objetivos}

Los objetivos de este trabajo son los siguientes. \begin{itemize}
  \item Estudio de las nociones básicas sobre Teoría de Códigos lineales.
  \item Estudio de las extensiones de Ore y de sus cocientes.
  \item Exponer el algoritmo de Peterson-Gorenstein-Zierler para códigos cíclicos sesgados.
  \item Implementación de sistemas de decodificación en Python usando Sage.
\end{itemize}

\section*{Esquema}

En los tres primeros capítulos introduciremos todos los fundamentos matemáticos necesarios, así como las definiciones y propiedades fundamentales de los códigos lineales y de los códigos cíclicos.
El capítulo cuarto ahondaremos en la familia de los códigos BCH y presentaremos el algoritmo original de Peterson-Gorenstein-Zierler para esta familia de códigos.
En el capítulo quinto explicamos las estructuras matemáticas que constituyen la base de los códigos cíclicos sesgados, los anillos de polinomios de Ore, para ya introducirlos propiamente en el capítulo sexto.
Finalmente, en el capítulo séptimo describimos el funcionamiento del algoritmo de Peterson-Gorenstein-Zierler para códigos cíclicos sesgados.





\part{Parte de prueba}

\chapter{Preliminares}

En este capítulo se detallan algunos conceptos básicos de Álgebra que son necesarios para la comprensión más adelante de la teoría de códigos. Según \parencite{cohn_algebra_1982} y \parencite{cohn_algebra_1989} y el libro de Jacobson.

\section{Monoides y grupos}

\begin{definition}
  Un \textit{monoide} es un conjunto \(S\) con un elemento \(e\) y una aplicación \(\mu: S^2 \to S\) tal que si \(\mu(x, y)\) es el resultado de aplicar \(\mu\) a la pareja de elementos \(x, y \in S\), se verifican: \begin{enumerate}
    \item \(\mu(x, \mu(y, z)) = \mu(\mu(x, y), z)\) para todo \(x, y, z \in S\).
    \item \(\mu(e, x) = \mu(x, e) = x\) para todo \(x \in S\).
  \end{enumerate}
\end{definition}

Obsérvese que, por definición, un monoide siempre tiene al menos un elemento. A la aplicación \(\mu\) que actúa sobre parejas de elementos se le llama \textit{operación binaria} y al elemento \(e\), elemento neutro de \(\mu\).

Un \textit{grupo} es un monoide en el que todo elemento tiene inverso.
Un grupo que además verifica la propiedad conmutativa es un \textit{grupo abeliano}.

\begin{definition}
  Un \textit{grupo} \(G\) es un conjunto junto a una operación binaria \(xy\) definida sobre él que verifica las siguientes propiedades.
  \begin{enumerate}
    \item Propiedad asociativa: \((xy)z = x(yz)\) para todo \(x, y, z \in G\).
    \item Existencia de elemento neutro \(1\): \(1x = x1 = x\) para todo \(x \in G\).
    \item Existencia de elemento inverso: cada elemento \(x\) tiene un inverso \(x^{-1}\) tal que \(xx^{-1} = x^{-1}x = 1\). 
  \end{enumerate}
\end{definition}

\section{Anillos}

\begin{definition}
  Un \textit{anillo} es un conjunto junto a dos operaciones: la suma \((+)\) y la multiplicación \((\cdot)\), que verifican las siguientes propiedades.
  \begin{itemize}[itemsep=0pt]
    \item Propiedad asociativa:
    \[(x+y)+z = x + (y+z), \qquad (xy)z = x(yz).\]
    \item Propiedad conmutativa para la suma:
    \[x + y = y + x.\]
    \item Existencia de elemento neutro:
    \[x + 0 = x, \qquad x1 = x.\]
    \item Existencia de elemento inverso para la suma:
    \[x + (-x) = 0.\]
    \item Propiedad distributiva para la multiplicación sobre la suma:
    \[x(y+z) = xy + xz.\]
  \end{itemize}

  Si un anillo verifica la propiedad conmutativa para la multiplicación, es decir, \(xy = yx\), se dice que es un \textit{anillo conmutativo}.
\end{definition}

Dos anillos son \textit{isomorfos} si hay un \textit{isomorfismo} entre ellos, es decir, existe una biyección que preserva todas las operaciones.
Decimos que los anillos isomorfos son idénticos, pues intrínsecamente son iguales.

En  cualquier anillo \(R\) se verifica que \(0x = x0 = 0\) para todo \(x \in R\), ya que \(x0 = x(0+0) = x0 + x0\), de donde concluimos que \(x0 = 0\) e igualmente, \(0x = 0\).

El \textit{anillo trivial} es aquel que solo tiene un elemento. 
Necesariamente entonces \(1 = 0\), pero este es el único caso en el que ocurre. 
Supongamos que \(1 = 0\). 
Entonces, para cada elemento \(x\) del anillo se tiene que \(x = x1 = x0 = 0\), luego tiene un único elemento.

% MAYBE: abreviar na = a + a + a (Cohn 137 [147])

Un elemento \(a\) de un anillo se dice que es \textit{invertible} si existe un elemento \(a'\) en el anillo tal que \(aa' = a'a = 1\).
A este elemento, que es único lo llamamos \textit{elemento inverso} de \(a\) y lo denotamos por \(a^{-1}\).
% TODO: por qué es único (anillo está formado por monoide multiplicativo de R)
El elemento \(0\) no puede tener inverso porque ya hemos visto que siempre que se multiplique por él se obtiene el \(0\).
Los anillos en los que todo elemento distinto de 0 es invertible se llaman \textit{anillos de división}.

% Dados dos anillos \(R\) y \(S\), decimos que \(S\) es un \textit{subanillo} de \(R\) si \(S\) está contenido en \(R\) y forma un anillo con las mismas operaciones que \(R\) (y con los mismos elemento neutro e identidad).

% CITEME: definciiones de Niederreiter (13)

\begin{definition}[Subanillo]
  Un subconjunto \(S\) de un anillo \(R\) se denomina \textit{subanillo} si \(S\) es cerrado bajo las operaciones de suma y producto de \(R\) y forma un anillo con estas operaciones.
\end{definition}

\begin{definition}[Ideal]
  Un subconjunto \(J\) de un anillo \(R\) se denomina \textit{ideal} si \(J\) es un subanillo de \(R\) y para todo \(a \in J\) y \(r \in R\) se verifica que \(ar \in J\) y \(ra \in J\).
\end{definition}

\begin{example}\hfill
  \begin{itemize}
    \item Sea \(R\) el cuerpo \(\mathbb Q\) de números racionales. Entonces, el conjunto \(\mathbb Z\) de los enteros es un subanillo de \(\mathbb Q\) pero no es un ideal porque, por ejemplo, \(1 \in \mathbb Z, 1/2 \in \mathbb Q\), pero \(1/2 \cdot 1 = 1/2 \notin \mathbb Z\). 
    \item Sea \(R\) un anillo conmutativo, \(a \in R\) y sea \(J = \{ra : r \in R\}\). Entonces, \(J\) es un ideal.
    % TODO: algún ejemplillo más?
  \end{itemize}
\end{example}

\begin{definition}
  Sea \(R\) un anillo conmutativo. Un ideal \(J\) de \(R\) se dice que es \textit{principal} si existe un elemento \(a \in R\) tal que \(J = (a) = \{ra : r \in R\}\).
\end{definition}

% MAYBE: si el anillo no es conmutativo, ideales por la izquierda y por la derecha.

Decimos que un ideal \(J\) es \textit{minimal} si no existe ningún otro ideal entre \(\{0\}\) y \(J\).

Dado un elemento del anillo distinto de cero, podemos clasificarlo en dos tipos. 
Sea \(a \neq 0\). 
Si existe \(b \neq 0\) tal que \(ab\) o \(ba\) es cero, entonces \(a\) es un elemento \textit{divisor de cero}, y en caso contrario, un elemento \textit{regular}.

Un anillo no trivial sin divisores de cero se dice que es \textit{entero}, un anillo entero conmutativo se denomina \textit{dominio de integridad}.

Una propiedad importante de los elementos regulares es la ley de cancelación.
\begin{proposition}
  Si \(c\) es un elemento regular de un anillo \(R\) entonces para cada \(a, b \in R\), tales que \(ca = cb\) o bien \(ac = bc\), se tiene que \(a = b\).
\end{proposition}

\begin{definition}
  Sea \(R\) un anillo. La \textit{característica} del anillo es el menor natural \(n\) tal que \(n1 = 0\).
  Si no existe tal número, la característica del anillo es \(0\).  
\end{definition}

\begin{definition}
  Un elemento \(e\) de un anillo tal que \(e^2 = e\) se dice que es \textit{idempotente}.
\end{definition}

\section{Cuerpos finitos}

\begin{definition}
  Un \textit{cuerpo} es un anillo de división conmutativo.
  Se dice que un cuerpo es \textit{finito} si tiene un número finito de elementos, al que llamamos \textit{orden} del cuerpo.
\end{definition}

% MAYBE: todo dominio de integridad finito es un cuerpo (Niederreiter, 12 [21])

Sea \(F\) un cuerpo. Un subconjunto \(K\) de \(F\) que es por sí mismo un cuerpo bajo las operaciones de \(F\) se denomina \textit{subcuerpo} de \(F\).
También podemos decir que \(F\) es una \textit{extensión} de \(K\).

Las extensiones de cuerpos se suelen notar de la forma \(F/K\).
Podemos decir que \(F\) es implícitamente un espacio vectorial sobre \(K\).
Esto es, los elementos de \(F\) pueden ser vistos como vectores sobre el cuerpo de escalares \(K\), con las operaciones de suma \(\alpha + \beta\), para \(\alpha, \beta \in F\) y multiplicación por escalares \(a\alpha\), para \(a \in K\) y \(\alpha \in F\), dadas por las propias operaciones de suma y multiplicación en \(E\).

De hecho todas las nociones que hemos definido para anillos (característica, ...) son válidas para los cuerpos, pues un cuerpo no deja de ser un anillo.

\begin{theorem}
  % Cohn Vol 2, 63 [74]
  \label{th:cuerpo-subcuerpo-primo-caracteristica}
  Todo cuerpo \(F\) tiene al menos un subcuerpo \(P\), el subcuerpo primo de \(F\) que está contenido en cada subcuerpo de \(F\).
  O bien \(F\) tiene característica 0 y \(P \cong \mathbb Q\) o bien \(F\) tiene característica \(p\), un número primo, y entonces \(P \cong \mathbb F_p\).
\end{theorem}


% CITEME ? Cohn Algebra Vol 2

Mención especial merecen los cuerpos finitos.

Un cuerpo con un número finito de elementos se denomina \textit{cuerpo finito} o \textit{cuerpo de Galois}, por su descubridor.

Sea \(V\) un espacio vectorial \(n\)-dimensional sobre \(\mathbb F_p\), el cuerpo de \(p\) elementos.
Si \(u_1, \dots, u_n\) es una base de \(V\), entonces cada elemento de \(V\) se escribe de forma única en la forma \(\sum \alpha_iu_i\), donde \(\alpha_i \in \mathbb F_p\).
Como cada coeficiente puede tener hasta \(p\) valores distintos, obtenemos un total de \(p^n\) elementos.

\begin{lemma}
  Un espacio vectorial \(n\)-dimensional sobre \(\mathbb F_p\) tiene \(p^n\) elementos.
\end{lemma}

Todo cuerpo finito \(F\) tiene claramente característica \(p\) en virtud del teorema \ref{th:cuerpo-subcuerpo-primo-caracteristica} y su subcuerpo primo es \(\mathbb F_p\).

\subsection{Anillos de polinomios sobre cuerpos finitos}

Para cualquier anillo \(R\) podemos definir un anillo de polinomios en \(x\) con coeficientes en \(R\).

Trabajaremos con anillos de polinomios en cuerpos finitos.

Denotamos el anillo de le los polinomios con coeficientes en \(\mathbb F_q\) por \(\mathbb F_q[x]\).
Es un anillo conmutativo con las operaciones habituales de suma y multiplicación de polinomios.
De hecho, es un dominio de integridad.

Un polinomio en \(\mathbb F_q[x]\) viene dado por \(f(x) = \sum_{i=0}^n a_ix^i\), donde \(a_i\) son los coeficientes del término de grado \(i\) y pertenecen a \(\mathbb F_q\).

El grado de un polinomio es el mayor grado de cualquier término con coeficiente distinto de cero.

\begin{proposition}
  Grado de sumas y productos
  % TODO: (Cohn, 149 [159]) (Ejercicio 158 de ECC)
\end{proposition}

\begin{proposition}
  \label{prop:raices-factores-pol-Fq}
  % TODO: Ejercicio 159 de ECC
\end{proposition}

El coeficiente del término de mayor grado se denomina \textit{coeficiente líder}.

Un polinomio es \textit{mónico} si su coeficiente líder es 1.
Sean \(f(x)\) y \(g(x)\) polinomios en \(\mathbb F_q[x]\).
Decimos que \(f(x)\) \textit{divide a} \(g(x)\), denotado por \(f(x) | g(x)\), si existe un polinomio \(h(x) \in \mathbb F_q[x]\) tal que \(g(x) = f(x)h(x)\).
El polinomio \(f(x)\) se llama \textit{divisor} o \textit{factor} de \(g(x)\).
El \textit{máximo común divisor} de \(f(x)\) y \(g(x)\), siendo al menos uno de ellos distinto de cero, es el polinomio mónico de \(\mathbb F_q[x]\) de mayor grado que divida tanto a \(f(x)\) como a \(g(x)\).
Lo denotamos por \(\operatorname{mcd}(f(x), g(x))\).
Dos polinomios son \textit{primos relativos} si su máximo común divisor es 1.

% TODO: polinomios irreducibles

% Algoritmo de división

\begin{theorem}[Algoritmo de división]
  \label{th:algoritmo-division}
  Sean \(f(x)\) y \(g(x)\) polinomios de \(\mathbb F_q[x]\), siendo \(g(x)\) distinto de cero.
  \begin{enumerate}
    \item \label{thi:algoritmo-division-division} Existen polinomios únicos, \(h(x)\), \(r(x) \in \mathbb F_q[x]\) tales que \[
      f(x) = g(x)h(x) + r(x), \quad \text{donde } \operatorname{gr} r(x) < \operatorname{gr} g(x) \text{ o } r(x) = 0. 
    \]
    \item \label{thi:algoritmo-division-mcd} Si \(f(x) = g(x)h(x) + r(x)\), entonces \[\operatorname{mcd}(f(x), g(x)) = \operatorname{mcd}(g(x), r(x)).\]
  \end{enumerate}
\end{theorem}

% TODO: algoritmo de euclides

Podemos utilizar el algoritmo de división del apartado \ref{thi:algoritmo-division-division} del teorema \ref{th:algoritmo-division} junto al apartado \ref{thi:algoritmo-division-mcd} de ese mismo teorema para hallar el máximo común divisor de dos polinomios.
Este procedimiento se conoce como \textit{algoritmo de Euclides} y es muy parecido a su homólogo para números enteros.

\begin{theorem}[Algoritmo de Euclides]
  Sean \(f(x)\) y \(g(x)\) dos polinomios en \(\mathbb F_q[x]\) con \(g(x)\) distinto de cero.
  \begin{enumerate}
    \item Realiza los siguientes pasos hasta que \(r_n(x) = 0\) para algún \(n\):
    \begin{align*}
      f(x) &= g(x)h_1(x) + r_1(x), \qquad \text{donde } \operatorname{gr} r_1(x) < \operatorname{gr} g(x),\\
      g(x) &= r_1(x)h_2(x) + r_2(x), \qquad \text{donde } \operatorname{gr} r_2(x) < \operatorname{gr} r_1(x),\\
      r_1(x) &= r_2(x)h_3(x) + r_3(x), \qquad \text{donde } \operatorname{gr} r_2(x) < \operatorname{gr} r_3(x),\\
        &\,\vdots \\
      r_{n-3}(x) &= r_{n-2}(x)h_{n-1}(x) + r_{n-1}(x), \ \text{donde } \operatorname{gr} r_{n-1}(x) < \operatorname{gr} r_{n-2}(x),\\
      r_{n-2}(x) &= r_{n-1}(x)h_n(x) + r_n(x), \qquad \text{donde } r_n(x) = 0.
    \end{align*} 
    Entonces, \(\operatorname{mcd}(f(x), g(x)) = cr_{n-1}(x)\), donde \(c \in \mathbb F_q\) se escoge para que \(cr_{n-1}(x)\) sea mónico.
    \item Existen polinomios \(a(x), b(x) \in \mathbb F_q[x]\) tales que 
    \[
      a(x)f(x) + b(x)g(x) = \operatorname{mcd}(f(x), g(x)).
    \]
  \end{enumerate}
\end{theorem}

La secuencia de pasos descrita termina porque en cada paso el grado del resto se reduce al menos en 1.

A continuación vamos a comentar un resultado que nos será útil en el futuro.

\begin{proposition}
  Sean \(f(x)\) y \(g(x)\) polinomios en \(\mathbb F_q[x]\).
  \begin{enumerate}
    \item Si \(k(x)\) es un divisor de \(f(x)\) y de \(g(x)\), entonces \(k(x)\) es divisor de \(a(x)f(x) + b(x)g(x)\) para todo \(a(x), b(x) \in \mathbb F_q[x]\).
    \item Si \(k(x)\) es un divisor de \(f(x)\) y de \(g(x)\) entonces \(k(x)\) es divisor de \(\operatorname{mcd}(f(x), g(x))\).
  \end{enumerate}
\end{proposition}

\begin{proof}
  % TODO
\end{proof}

% 3.3 Primitive elements

\subsection{Elementos primitivos}

El conjunto \(\mathbb F_q^*\) --de los elementos de \(\mathbb F_q\) distintos de cero-- es un grupo.

\begin{theorem}
  \label{th:Fq-ast-cilcico}
  Se verifican las siguientes afirmaciones.
  \begin{enumerate}
    \item El grupo \(\mathbb F_q^*\) es cíclico de orden \(q - 1\) con la multiplicación de \(\mathbb F_q\).
    \item Si \(\gamma\) es un generador de este grupo cíclico entonces
    \[
      \mathbb F_q = \{0, 1 = \gamma^0, \gamma, \gamma^2, \dots, \gamma^{q-2}\},
    \] y \(\gamma^i = 1\) si y solo si \((q-1) \mid i\).
  \end{enumerate}
\end{theorem}

\begin{proof}
  % TODO
\end{proof}

Cada generador \(\gamma\) de \(\mathbb F_q^*\) se llama \textit{elemento primitivo} de \(\mathbb F_q\).
Cuando los elementos distintos de cero de un cuerpo finito se expresan como potencias de \(\gamma\) podemos multiplicar de forma sencilla teniendo en cuenta que \(\gamma^i\gamma^j = \gamma^{i+j} = \gamma^s\), donde \(0 \leq s \leq q-2\) e \(i + j \equiv s \bmod q - 1\).

\begin{theorem}
  \label{th:el-Fq-raices-pol}
  Los elementos de \(\mathbb F_q\) son las raíces del polinomio \(x^q - x\).
\end{theorem}

\begin{proof}
  Sea \(\gamma\) un elemento primitivo de \(\mathbb F_q\).
  Entonces, \(\gamma^{q-1} = 1\) por definición.
  Por tanto, \((\gamma^i)^{q-1} = 1\) para todo \(i\) tal que \(0 \leq i \leq q - 2\).
  En consecuencia, los elementos de \(\mathbb F_q^*\) son las raíces de \(x^{q-1}-1 \in \mathbb F_p[x]\) y en consecuencia, de \(x^q - x\).
  Como \(0\) es raíz de \(x^q - x\), por la proposición \ref{prop:raices-factores-pol-Fq} sabemos que los elementos de \(\mathbb F_q\) son las raíces de \(x^q - x\), como queríamos.
\end{proof}

Un elemento \(\xi \in \mathbb F_q\) es una raíz enésima de la unidad si \(\xi^n = 1\), y es una raíz enésima primitiva de la unidad si además \(\xi^s \neq 1\) para todo s tal que \(0 < s < n\).
Un elemento primitivo \(\gamma\) de \(\mathbb F_q\) es por tanto una raíz \((q-1)\)-ésima de la unidad.
Se deduce del teorema \ref{th:Fq-ast-cilcico} que el cuerpo \(\mathbb F_q\) contiene una raíz enésima primitiva de la unidad si y solo si \(n \mid (q - 1)\), en cuyo caso \(\gamma^{(q-1)/n}\) es dicha raíz.

% 3.4 Constructing finite fields

\subsection{Construcción de cuerpos finitos}

Un polinomio no constante \(f(x) \in \mathbb F_q[x]\) es \textit{irreducible sobre} \(\mathbb F_q\) si no es posible factorizarlo como producto de dos polinomios de \(\mathbb F_q[x]\) de grado menor.

\begin{theorem}
  Sea \(f(x)\) un polinomio no constante. Entonces, 
  \[
    f(x) = p_1(x)^{a_1}p_2(x)^{a_2}\dots p_k(x)^{a_k},
  \]
  donde cada \(p_i(x)\) es irreducible, los polinomios \(p_i(x)\) son únicos salvo producto por escalares y los elementos \(a_i\) son únicos.
\end{theorem}

Esto nos dice que \(\mathbb F_q[x]\) es lo que se conoce como \textit{dominio de factorización única}.
Pero es además un dominio de ideales principales.

\begin{proof}
  % TODO: demostración de que F_q[x] es un dominio de ideales principales.
  % Ejercicio 153 (p. 102 [121]), F_q[x] es un anillo conmutativo con unidad y un dominio de integridad
  % Ejercicio 166 (p. 106 [125]), cada ideal de F_q[x] es un ideal principal
\end{proof}

Para construir un cuerpo de característica \(p\) comenzamos con un polinomio \(f(x) \in \mathbb F_q[x]\) que es irreducible sobre \(\mathbb F_q\) y que tiene grado \(m\).
Usando el algoritmo de euclides podemos demostrar que el anillo cociente dado por \(\mathbb F_p[x]/(f(x))\) es un cuerpo, y de hecho, un cuerpo finito con \(q = p^m\) elementos.

\begin{proof}
  % TODO: demostración de que F_p[x]/(f(x)) es un cuerpo de q = p^m elementos
  % Ejercicio 167 (p. 107 [126])
\end{proof}

Cada elemento de dicho anillo cociente es una clase lateral de la forma \(g(x) + (f(x))\), donde \(g(x)\) es único y tiene grado ocmo mucho \(m-1\).

Escribiremos cada clase lateral como un vector en \(\mathbb F_p^m\) siguiendo la correspondencia:
\[
  g_{m-1}x^{m-1} + g_{m-2}x^{m-2}+ \dots + g_1x + g_0 + (f(x)) \iff g_{m-1}g_{m-2}\dots g_1g_0.
\]

Esta notación de vectorial nos permite realizar la suma en el cuerpo utilizando la suma habitual de los vectores.
Multiplicar es una tarea a priori más complicada.
Para multiplicar \(g_1(x) + (f(x))\) por \(g_2(x) + (f(x))\) primero utilizamos al algoritmo de división para escribir
\[
  g_1(x)g_2(x) = f(x)h(x) + r(x),
\]
donde como sabemos o bien \(\operatorname{gr} r(x) \leq m -1\) o bien \(r(x) = 0\).
Puesto que estamos en el anillo cociente \(\mathbb F_p[x]/(f(x))\) nos queda
\[
  (g_1(x) + (f(x)))(g_2(x) + (f(x))) = r(x) + (f(x)).
\]
Esta notación es engorrosa, por lo que habitualmente operaremos en \(\alpha\) en vez de en \(x\) suponiendo que \(f(\alpha) = 0\).
Así, \(g_1(\alpha)g_2(\alpha) = r(\alpha)\).
% TODO: correspondencia vectorial de los polinomios en alpha (3.4)
En consecuencia, multiplicamos los polinomios en \(\alpha\) de la forma habitual y utilizamos la ecuación \(f(\alpha) = 0\) para reducir las potencias de \(\alpha\) de grado mayor a \(m-1\) a polinomios en \(\alpha\) de grado menor que \(m\).

El conjunto \(\{0\alpha^{m-1} + 0\alpha^{m-2} + \dots + 0\alpha + a_0 \mid a_0 \in \mathbb F_p\} = \{a_0 \mid a_0 \in \mathbb F_p\}\) es el subcuerpo primo de \(\mathbb F_q\).

Decimos que obtenemos \(\mathbb F_q\) a partir de \(\mathbb F_p\) yuxtaponiendo una raíz \(\alpha\) de \(f(x)\) a \(\mathbb F_p\).
Esta raíz viene dada formalmente por \(\alpha = x + (f(x))\) en el anillo cociente \(\mathbb F_p[x]/(f(x))\).
Por tanto, ya hemos visto antes que \(g(x) + (f(x)) = g(\alpha)\) y \(f(\alpha) = f(x + (f(x))) = f(x) + (f(x)) = 0 + (f(x))\).

Un polinomio irreducible sobre \(\mathbb F_p\) de grado \(m\) es \textit{primitivo} si tiene una raíz que es un elemento primitivo de \(\mathbb F_q = \mathbb F_{p^m}\).
Puede probarse que existen polinomios irreducibles de cualquier grado.
% TODO: probar? referencia prueba? mejorar afirmación?

\begin{theorem}
  Para cualquier primo \(p\) y cualquier entero positivo \(m\), existe un cuerpo finito, único salvo isomorfismos, con \(q = p^m\) elementos.
\end{theorem}

% 3.7 Cyclotomic cosets and minimal polynomials

\subsection{Clases ciclotómicas y polinomios minimales}

Sea \(\mathbb F_{q^t}/\mathbb F\) una extensión de cuerpos.
Por el teorema \ref{th:el-Fq-raices-pol} cada elemento de \(\mathbb F_{q^t}\) es raíz del polinomio \(x^{q^t} - x\).
Existe por tanto un polinomio mónico \(M_{\alpha}\) en \(\mathbb F-q[x]\) de grado mínimo que tiene a \(\alpha\) como raíz.
Este polinomio se conoce como \textit{polinomio minimal} de \(\alpha\) sobre \(\mathbb F_q\).
En el siguiente teorema vamos a estudiar algunas propiedades de los polinomios minimales.

\begin{theorem}
  Sea \(\mathbb F_{q^t}/\mathbb F\) una extensión de cuerpos y sea \(\alpha\) un elemento de \(\mathbb F_{q^t}\) cuyo polinomio minimal es \(M_{\alpha} \in \mathbb F_q[x]\).
  Se verifica:
  \begin{enumerate}
    \item El polinomio \(M_{\alpha}(x)\) es irreducible sobre \(\mathbb F_q\).
    \item Si \(g(x)\) es cualqueir polinomio en \(\mathbb F_q[x]\) tal que \(g(\alpha) = 0\) entonces \(M_{\alpha}(x) \mid g(x)\).
    \item El polinomio \(M_{\alpha}(x)\) es único.
  \end{enumerate}
\end{theorem}

\begin{proof}
  % TODO
\end{proof}

Si partimos de \(f(x)\), un polinomio irreducible sobre \(\mathbb F_q\) de grado \(r\), podemos considerar la extensión generada por una de las raíces de \(f(x)\) y obtendremos el cuerpo \(\mathbb F_{q^r}\).
De hecho, el siguiente teorema afirma que en ese caso todas las raíces de \(f(x)\) estarán en \(\mathbb F_{q^r}\).

\begin{theorem}
  Sea \(f(x)\) un polinomio irreducible mónico sobre \(\mathbb F_q\) de grado \(r\).
  Entonces:
  \begin{enumerate}
    \item Todas las raíces de \(f(x)\) están en \(\mathbb F_{q^r}\) y en cualquier extensión de cuerpos de \(\mathbb F_q\) generada por una de sus raíces.
    \item Podemos expresar \(f(x)\) como \(f(x) = \prod_{i=1}^r (x - \alpha_i)\), donde \(\alpha_i \in \mathbb F_{q^r}\) para \(1 \leq i \leq r\).
    \item El polinomio \(f(x)\) divide a \(x^{q^r} - x\).
  \end{enumerate}
\end{theorem}

\begin{proof}
  % TODO (ECC 113 [132] theorem 3.7.2)
\end{proof}

En particular este teorema se verifica para los polinomios minimales \(M_{\alpha}(x)\) sobre \(\mathbb F_q\), pues son mónicos irreducibles.

\begin{theorem}
  \label{th:prop-pol-minimal}
  Sea \(\mathbb F_{q^t}/\mathbb F_q\) una extensión de cuerpos y sea \(\alpha\) un elemento de \(\mathbb F_{q^t}\) con polinomio minimal \(M_{\alpha}\) en \(\mathbb F_q[x]\).
  Se verifican las siguientes afirmaciones.
  \begin{enumerate}
    \item El polinomio \(M_{\alpha}(x)\) divide a \(x^{q^t} - x\).
    \item El polinomio \(M_{\alpha}(x)\) tiene raíces distintas dos a dos, todas en \(\mathbb F_{q^t}\).
    \item El grado de \(M_{\alpha}(x)\) divide a \(t\).
    \item Podemos expresar \(x^{q^t}- x = \prod_{\alpha}M_{\alpha}(x)\), donde \(\alpha\) varía entre los elementos de un subconjunto de \(\mathbb F_{q^t}\) que enumera los polinomios minimales de todos los elementos de \(\mathbb F_{q^t}\) una sola vez.
    \item Podemos expresar \(x^{q^t}- x = \prod_{f}f(x)\), donde \(f\) varía entre todos los mónicos irreducibles sobre \(\mathbb F_q\) cuyo grado divide a \(t\).
  \end{enumerate}
\end{theorem}

\begin{proof}
  % TODO (ECC teorema 3.7.3, p 113 [132])
\end{proof}

Dos elementos de \(\mathbb F_{q^t}\) que tienen el mismo polinomio minimal en \(\mathbb F_q[x]\) se llaman \textit{conjugados sobre} \(\mathbb F_q\).
Es importante encontrar todos los conjugados de \(\alpha \in \mathbb F_q\), es decir, todas las raíces de \(M_{\alpha}(x)\).
Sabemos por el teorema \ref{th:prop-pol-minimal} que las raíces de \(M_{\alpha}(x)\) son todas distintas dos a dos y que se encuentran en el cuerpo \(\mathbb F_{q^t}\).
Podemos encontrar estas raíces con ayuda del siguiente teorema.

\begin{theorem}
  Sea \(f(x)\) un polinomio en \(\mathbb F_q[x]\) y sea \(\alpha\) una raíz de \(f(x)\) en una extensión \(\mathbb F_{q^t}/\mathbb F_q\).
  Entonces se verifican las siguientes afirmaciones.
  \begin{enumerate}
    \item Evaluando el polinomio obtenemos que \(f(x^q) = f(x)^q\).
    \item El elemento \(\alpha^q\) es también una raíz de \(f(x)\) en \(\mathbb F_q\).
  \end{enumerate}
\end{theorem}

\begin{proof}
  % TODO (ECC teorema 3.7.4, p114 [133])
\end{proof}

Si aplicamos este teorema de forma consecutiva podremos obtener todas las raíces de \(M_{\alpha}(x)\), que serán de la forma \(\alpha, \alpha^q, \alpha^{q^2},\)\,etc.; secuencia que terminará tras \(r\) términos, cuando \(\alpha^{q^r} = \alpha\).

Supongamos ahora que \(\gamma\) es un elemento primitivo de \(\mathbb F_{q^t}\).
Entonces sabemos que \(\alpha = \gamma^s\) para algún \(s\).
Por tanto, \(\alpha^{q^r} = \alpha\) si y solo si \(\gamma^{sq^r - s} = 1\).
Por el teorema \ref{th:Fq-ast-cilcico} se tiene que \(sq^r \equiv s \bmod q^t - 1\).
Basándonos en esta idea podemos definir la \textit{clase q-ciclotómica de s módulo} \(q^t - 1\) como el conjunto
\[
  C_s = \{s, sq, \dots, sq^{r-1}\} \bmod q^t - 1, 
\]
donde \(r\) es el menor entero positivo tal que \(sq^r \equiv s \bmod q^t - 1\).
Los conjuntos \(C_s\) dividen el conjunto de enteros \(\{0, 1, 2, \dots, q^t - 2\}\) en conjuntos disjuntos.

% TODO: justificación de este teorema

\begin{theorem}
  Si \(\gamma\) es un elemento primitivo de \(\mathbb F_{q^t}\) entonces el polinomio minimal de \(\gamma^s\) sobre \(\mathbb F_q\) es
  \[
    M_{\gamma^s}(x) = \prod_{i \in C_s}(x - \gamma^i).
  \]
\end{theorem}

\begin{proof}
  % TODO ECC theorem 3.7.6 p.115 [133]
\end{proof}

\section{Automorfismos}

\chapter{Fundamentos de teoría de códigos}

La teoría de códigos tal y cual. Las definiciones aquí ... según lo descrito en \parencite[1-48]{huffman-pless-2003} y \parencite{podesta-2006}.

\section{Códigos lineales}

Vamos a comenzar nuestro estudio con los códigos lineales, pues son los más sencillos de comprender. Consideremos el espacio vectorial de todas las \(n\)-tuplas sobre el cuerpo finito \(\mathbb F_q\), al que denotaremos en lo que sigue como \(\mathbb F_q^n\). Los elementos \((a_1, \dots, a_n)\) de \(\mathbb F_q^n\) los notaremos usualmente como \(a_1\!\cdots a_n\).

\begin{definition}
  Un \((n, M)\) \textit{código} \(\mathcal C\) sobre el cuerpo \(\mathbb F_q\) es un subconjunto de \(\mathbb F_q^n\) de tamaño \(M\). A los elementos de \(\mathcal C\) los llamaremos \textit{palabras codificadas} —o \textit{codewords} en inglés—.
\end{definition}

Es necesario añadir más estructura a los códigos para que sean de utilidad.

\begin{definition}
  Decimos que un código \(\mathcal C\) es un código \textit{lineal de longitud \(n\) y rango \(k\)} —abreviado como \([n, k]\) \textit{lineal}— si dicho código es un subespacio vectorial de \(\mathbb F_q^n\) de dimensión \(k\).
\end{definition}

Un código lineal \(\mathcal C\) tiene \(q^k\) palabras codificadas.

\begin{definition}
  Una \textit{matriz generadora} para un \([n, k]\) código \(\mathcal C\) es una matriz \(k \times n\) cuyas filas conforman una base de \(\mathcal C\).
\end{definition}

Veamos un ejemplo de matriz generadora. Consideremos la matriz \(G = \begin{psmallmatrix}
  1 & 1 & 0 \\ 0 & 1 & 1
\end{psmallmatrix} \in \mathcal M_{2 \times 3}(\mathbb F_2)\). Dicha matriz genera un \([3, 2]\) código binario, pues dado \((x_1, x_2)\), se tiene que \[(x_1, x_2) \begin{pmatrix}
  1 & 1 & 0 \\ 0 & 1 & 1
\end{pmatrix} = (x_1, x_1 + x_2, x_2),\] y por tanto este código codifica de la forma \[00 \to 000, \quad 01 \to 011,\quad 10 \to 110,\quad 11 \to 101.\]

\begin{definition}
  Para cada conjunto \(k\) de columnas independientes de una matriz generadora \(G\) el conjunto de coordenadas correspondiente se denomina \textit{conjunto de información} para un código \(\mathcal C\). Las \(r = n - k\) coordenadas restantes se llaman \textit{conjunto redundante}, y el número \(r\), la \textit{redundancia} de \(\mathcal C\).
\end{definition}

Si las primeras \(k\) coordenadas forman un conjunto de información el código tiene una única matriz generadora de la forma \([I_k \mid A]\), donde \(I_k\) es la matriz identidad \(k \times k\) y \(A\) es una matriz \(k \times r\). Esta matriz generadora se dice que está en \textit{forma estándar}.

%Como un código lineal es el subespacio de un espacio vectorial, es el núcleo de una transformación lineal. En particular, existe una matriz \(H\) de dimensiones \(r \times n\), llamada \textit{matriz de comprobación de paridad} para un \([n, k]\) código \(\mathcal C\) definida por \begin{equation}
%  \mathcal C = \left\{x \in \mathbb F_q^n : H \boldsymbol x^T = 0 %\right\}.
%\end{equation}

Como un código lineal \(\mathcal C\) es un subespacio de un espacio vectorial, podemos calcular el ortogonal a dicho subespacio, obteniendo lo que llamaremos el \textit{código dual} \(\mathcal C^{\perp}\). 

\begin{definition}
  Sea \(\mathcal C\) un \([n, k]\) código lineal. Una matriz \(H\) se dice que es \textit{matriz de paridad} si es una matriz generadora de \(\mathcal C^{\perp}\).
\end{definition}

\begin{proposition}
  Sea \(H\) la matriz de paridad de un \([n, k]\) código lineal \(\mathcal C\). Entonces, \[\mathcal C = \{x \in \mathbb F_q^n : xH^T = 0\} = \{x \in F_q^n : Hx^T = 0\}.\]
\end{proposition}

\begin{proof}
  Si \(c \in \mathcal C\) entonces \(c = uG\), donde \(u \in \mathbb F_q^k\) y \(G\) es la matriz generadora de \(\mathcal C\). Tenemos entonces que \(c\cramped{H^T} = uG\cramped{H^T}\) y como \(G\cramped{H^T} = 0\) —por ser H matriz generadora del subespacio ortogonal \(\mathcal C\)— se tiene que \(\mathcal C \subset S_H = \{x \in \mathbb F_q^n : Hx^T = 0\}\), que espacio solución de un sistema de \(n - k\) ecuaciones con \(n\) incógnitas y de rango \(n - k\). Como \(\dim(S_H) = n - (n - k) = k = \dim L\), tenemos que \(L = \{x \in \mathbb F_q^n : Hx^T = 0\}\).
\end{proof}





%-------------------------------------------------------------------------------
%	BIBLIOGRAFÍA
%-------------------------------------------------------------------------------

\newpage
\printbibliography

\end{document}
