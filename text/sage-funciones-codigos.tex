\chapter{Funciones en SageMath usadas en los ejemplos}
\label{annex:sage-gen-idemp}

En este anexo se describen algunas de las funciones utilizadas durante los ejemplos a lo largo del trabajo.
De nuevo, el código puede encontrarse en
\begin{center}
  \url{https://github.com/jmml97/tfg/tree/master/code}.
\end{center}

\begin{description}[font=\ttfamily, style=nextline]
  \item[generators(poly)] Devuelve una lista de polinomios generadores de códigos cíclicos de longitud el grado de \texttt{poly}.
  
  \textsc{Argumentos}
  \begin{description}[font=\normalfont\ttfamily]
    \item[poly] Un polinomio de la forma \(x^n - 1\)
  \end{description}
  
  \textsc{Salida}
  \begin{itemize}
    \item La lista de tupla polinomio generador e idempotente generador
  \end{itemize}

  \item[defining\_sets(poly)] Devuelve una lista de tuplas consistentes en un polinomio generador de un código cíclico de longitud el grado de \texttt{poly}, un conjunto característico y una raíz primitiva.
  
  \textsc{Argumentos}
  \begin{description}[font=\normalfont\ttfamily]
    \item[poly] Un polinomio de la forma \(x^n - 1\)
  \end{description}

  \textsc{Salida}
  \begin{itemize}
    \item La lista de polinomios generadores
  \end{itemize}

  \item[generator\_and\_idempotents(poly)] Devuelve una lista de tuplas consistentes en un polinomio generador de un código cíclico de longitud el grado de \texttt{poly} y el idempotente generador correspondiente.
  
  \textsc{Argumentos}
  \begin{description}[font=\normalfont\ttfamily]
    \item[poly] Un polinomio de la forma \(x^n - 1\)
  \end{description}

  \textsc{Salida}
  \begin{itemize}
    \item La lista de tupla polinomio generador e idempotente generador
  \end{itemize}

  \item[mult(iterable)] Devuelve el producto de todos los elementos de \texttt{iterable}.
  
  \textsc{Argumentos}
  \begin{description}[font=\normalfont\ttfamily]
    \item[iterable] Un objeto iterable
  \end{description}

  \textsc{Salida}
  \begin{itemize}
    \item El producto de todos los elementos de \texttt{iterable}
  \end{itemize}

  \item[powerset(iterable)] Devuelve todas las posibles combinaciones de elementos de \texttt{iterable}.
  
  \textsc{Argumentos}
  \begin{description}[font=\normalfont\ttfamily]
    \item[iterable] Un objeto iterable
  \end{description}

  \textsc{Salida}
  \begin{itemize}
    \item Todas las posibles combinaciones de elementos de \texttt{iterable}
  \end{itemize}

  \textsc{Ejemplos}
  \begin{lstlisting}[gobble=4]
    sage: powerset([1,2,3])
    > () (1,) (2,) (3,) (1,2) (1,3) (2,3) (1,2,3)
  \end{lstlisting}

\end{description}