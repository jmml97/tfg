\chapter[Implementación en Sage del algoritmo PGZ]{Implementación en Sage del algoritmo de Peterson-Gorenstein-Zierler}
\label{annex:pgz-sage}

Se han desarrollado implementaciones en sage del algoritmo de Peterson-Gorenstein-Zierler, tanto en su versión para códigos BCH como para códigos RS sesgados.

Dichas implementaciones aprovechan la estructura de códigos que ya tiene implementada Sage.
Así, para la versión de códigos BCH se ha implementado un decodificador para códigos BCH, \texttt{BCHPGZDecoder}, que hereda de la clase \texttt{Decoder} de Sage.
Por otro lado, para la versión de códigos cíclicos ha sido necesario implementar primero la clase \texttt{SkewCyclicCode}, que hereda de la clase \texttt{AbstractLinearCode} de Sage, y que implementa de forma sencilla los aspectos básicos de códigos cíclicos sesgados, utilizando para ello la implementación existente de anillos de polinomios de Ore de Sage.
Una vez diseñada esta clase que permite trabajar con códigos cíclicos sesgados se ha implementado una clase \texttt{SkewRSCode} para manejar códigos RS sesgados y un decodificador para este tipo de códigos, \texttt{SkewRSPGZDecoder}.

Su uso es muy sencillo.
Con la orden \texttt{load()} de Sage pueden cargarse los archivos proporcionados, que incluyen todas las clases descritas antes.

\begin{lstlisting}[gobble=2]
  sage: load(pgz.sage)
  sage: load(pgz-sesgados.sage)
\end{lstlisting}

\section{Decodificador basado en el algoritmo PGZ para códigos BCH}

El decodificador \texttt{BCHPGZDecoder} debe utilizarse sobre un código BCH de Sage, de la clase \texttt{BCHCode}.

\begin{lstlisting}[gobble=2]
  sage: C = codes.BCHCode(GF(2), 15, 5, offset=1); C
  > [15, 7] BCH Code over GF(2) with designed distance 5
  sage: D = BCHPGZDecoder(C); D
  > Peterson-Gorenstein-Zierler algorithm based decoder for [15, 7] BCH Code over GF(2) with designed distance 5
\end{lstlisting}




%\lstinputlisting[language=Python, caption=Implementación del algoritmo de Peterson-Gorenstein-Zierler en Sage]{../code/pgz.sage}