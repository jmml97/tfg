\chapter{Anillos de polinomios de Ore}

En esta sección vamos a hablar sobre los anillos de polinomios de Ore, que serán la base de los códigos cíclicos sesgados.
% TODO: nota histórica sobre polinomios Ore
[Introducir nota histórica]
Primero vamos a dar la definición general, sin detenernos a justificar su construcción, pues acto seguido vamos a centrarnos en el caso que nos va a ocupar cuando trabajemos con códigos cíclicos sesgados.
Las definiciones y el desarrollo teórico seguidos en esta sección proceden de \parencite{jacobson_finite-dimensional_1996}, \parencite{ore_theory_1933} y [añadir: Factoring Ore Polynomials over Fields].

\begin{definition}
  Sea \(R\) un anillo, \(\sigma\) un endomorfismo de \(R\) y \(\delta\) una \(\sigma\)-\textit{derivación} de \(R\), es decir, \(\delta\) es un homomorfismo de grupos abelianos tal que para \(a, b \in R\) se verifica que
  \[
    \delta(ab) = (\sigma a)(\delta b) + (\delta a)b.
  \]
  Entonces, el anillo \(R[t; \sigma, \delta]\) de los polinomios en \(R[t]\) de la forma
  \[
    a_0 + a_1t + \dots + a_nt^n,
  \]
  donde \(a_i \in R\), con la igualdad y suma usuales, y en el que la multiplicación verifica la relación 
  \[
  ta = (\sigma a)t + \delta a, \qquad a \in R,
  \]
  se conoce como \textit{anillo de polinomios de Ore} o \textit{anillos de polinomios torcidos}.
\end{definition}

Para comprobar que \(R[t; \sigma, \delta]\) es un anillo tendríamos que ver que efectivamente con las operaciones que hemos dado se verifican todas las propiedades de los anillos.
Puesto que hemos usado la suma usual de los polinomios, bastaría probar que se verifica la propiedad asociativa para la multiplicación que hemos definido.
No vamos a entrar en detalle, pues no es el objetivo de este trabajo el estudio de los anillos de polinomios de Ore en general.

Trabajar con códigos cíclicos sesgados requiere del estudio del anillo \(\mathbb F_q[x, \sigma]\), con \(\sigma\) un automorfismo.
Por tanto, nos vamos a centrar los anillos de polinomios de Ore en los que \(R = \mathbb F_q\) —cuerpo finito de \(q\) elementos—, hemos llamado \(x\) a \(t\), \(\sigma\) es un automorfismo y \(\delta = 0\).
En estos anillos la multiplicación verifica la relación 
\[
  xa = (\sigma a)x, \qquad a \in R.
\]
Es este caso particular el que sí vamos a estudiar en profundidad, justificando que, como ya hemos adelantado, se trata de un anillo.

Vamos a ver por inducción que, como podemos intuir, \(x^n a = (\sigma^n a)x^n\). 
Estudiado el caso base anterior y supuesto que se verifica la igualdad para \(n - 1\), para \(n\) tenemos que
\[
  x^{n}a = xx^{n - 1}a = x(\sigma^{n-1} a)x^{n-1} = \sigma(\sigma^{n-1} a)x^{n-1}x = (\sigma^{n}a)x^{n}.
\]
Ahora definimos
\[
  (ax^n)(bx^m) = a(\sigma^n b)x^{n+m},
\]
con lo que, junto a la propiedad distributiva podemos definir el producto de polinomios en \(x\) como
\[
  \textstyle(\sum a_nx^n)(\sum b_mx^m) = \sum(a_nx^n)(b_mx^m).
\]

Para comprobar que \(\mathbb F_q[x, \sigma]\) es un anillo, como ya hemos comentado en el caso general, necesitamos comprobar que se verifica la propiedad asociativa para la multiplicación.
Comprobar esta afirmación directamente es tedioso, pero en \parencite[p. 2-3]{jacobson_finite-dimensional_1996} puede consultarse una demostración utilizando una representación matricial de los elementos.

A continuación vemos que, partiendo de que \(\mathbb F_q\) es en particular un anillo de división, \(\mathbb F_q[x, \sigma]\) es un dominio de integridad no conmutativo.
Dado un polinomio \(f(x) = a_0 + a_1x + \dots + a_nx^n\) con \(a_n \neq 0\) y definimos \(\deg(f(x)) = n\) y \(\deg(0) = -\infty\).
Si consideramos otro polinomio \(g(x) = b_0 + b_1x + \dots + b_mx^m\) con \(b_m\) entonces \(f(x)g(x) = \dots + a_n(\sigma^n b_m)x^{n+m}\) y \(a_n(\sigma^nb_m) \neq 0\) y así,
\[
  \deg(f(x)g(x)) = \deg(f(x)) + \deg(g(x)).
\]
Por tanto, \(\mathbb F_q[x, \sigma]\) no tiene divisores de cero distintos del cero, por lo que es un dominio de integridad no conmutativo, como habíamos afirmado.

Podemos definir algoritmos de división en \(\mathbb F_q[x, \sigma]\) tanto a la izquierda como a la derecha (cita)  —descritos en los algoritmos \ref{alg:ore-fq-division-izquierda} y \ref{alg:ore-fq-division-derecha}—, de forma que para cada \(f(x), g(x) \in \mathbb F_q[x, \sigma]\) —con \(g(x) \neq 0\)— existen elementos \(q(x), r(x)\) únicos, con \(\deg(r) < \deg(g)\) tales que al dividir por la izquierda obtenemos
\[
  f(x) = q(x)g(x) + r(x),
\]
y al dividir por la derecha, 
\[
  f(x) = g(x)q(x) + r(x),
\]
Cuando dividimos por la izquierda (respectivamente por la derecha) el polinomio \(g(x)\) se le llama \textit{cociente por la izquierda} (\textit{derecha}) y a \(r(x)\), \textit{resto por la izquierda} (\textit{derecha}).
Los denotaremos por \(g(x) = \operatorname{coi}(f(x), g(x))\) o \(\operatorname{cod}(f(x), g(x))\) y \(r(x) = \operatorname{rei}(f(x), g(x))\) o \(\operatorname{red}(f(x), g(x))\).

\begin{Ualgorithm}[h]
  \DontPrintSemicolon
  \KwIn{polinomios \(f, g \in \mathbb F_q[x, \sigma]\) con \(g \neq 0\)}
  \KwOut{polinomios \(q, r \in \mathbb F_q[x, \sigma]\) tales que \(f = qg + r\), y \(\deg(r) < \deg(g)\)}
  \(q \longleftarrow 0\)\;
  \(r \longleftarrow f\)\;
  \While{\(\deg(g) \leq \deg(r)\)}{
    \(a \longleftarrow \lc (r) \sigma^{\deg (r) - \deg (g)}(\lc (g)^{-1})\)\;
    \(q \longleftarrow q + ax^{\deg (r) - \deg (g)}\)\;
    \(r \longleftarrow r - ax^{\deg (r) - \deg (g)}g\)
    }
    \caption{División por la izquierda en \(\mathbb F_q[x, \sigma]\)}
  \label{alg:ore-fq-division-izquierda}
\end{Ualgorithm}

\begin{Ualgorithm}[h]
  \DontPrintSemicolon
  \KwIn{polinomios \(f, g \in \mathbb F_q[x, \sigma]\) con \(g \neq 0\)}
  \KwOut{polinomios \(q, r \in \mathbb F_q[x, \sigma]\) tales que \(f = gq + r\), y \(\deg(r) < \deg(g)\)}
  \(q \longleftarrow 0\)\;
  \(r \longleftarrow f\)\;
  \While{\(\deg(g) \leq \deg(r)\)}{
    \(a \longleftarrow \sigma^{-\deg(g)}(\lc(g)^{-1}\lc(r))\)\;
    \(q \longleftarrow q + ax^{\deg (r) - \deg (g)}\)\;
    \(r \longleftarrow r - gax^{\deg (r) - \deg (g)}\)
    }
    \caption{División por la derecha en \(\mathbb F_q[x, \sigma]\)
  }
  \label{alg:ore-fq-division-derecha}
\end{Ualgorithm}

La existencia de algoritmos de algoritmos de división a izquierda y a derecha implica que \(F_q[x, \sigma]\) es un dominio de ideales principales a izquierda y a derecha, es decir, que es lo que llamamos un dominio de ideales principales a secas.

A partir de ahora, para ser más concisos con la notación vamos a llamar \(R = F_q[x, \sigma]\) y cuando no sea necesario hacer referencia a la variable \(x\), a un polinomio \(f(x)\) lo denotaremos simplemente por \(f\).
Los ideales biláteros de \(R\) serán de la forma \(I = Rf = f^{*}R\) y para todo \(g \in R\) existirán \(g', \tilde{g} \in R\) tales que \(fg = g'f\) y \(gf^{*} = f^{*}\tilde{g}\).
Los elementos \(f\) tales que para todo \(g \in R\) existen \(g'\) y \(\tilde{g}\) tales que \(fg = g'f\) y \(gf = f\tilde{g}\) se llaman elementos \emph{biláteros} y además \(Rf = fR\) es un ideal.

\begin{theorem}
  \label{th:anillos-ore-centro}
  Sea \(R = F_q[x, \sigma]\). Se verifican las siguientes afirmaciones.
  \begin{enumerate}
    \item Los elementos biláteros de \(R\) son de la forma \(ac(t)x^n\), donde \(a \in \mathbb F_q\), \(n = 0, 1, \dots\) y \(c(t) \in \operatorname{Cent}(R)\), el centro de \(R\).
    \item Supongamos ahora que \(\sigma\) tiene orden \(n\), de forma que \(\sigma^n = \operatorname{Id}\).
    El centro de \(R\) es el conjunto de los polinomios de la forma
    \[
      \gamma_0 + \gamma_1x^{n} + \gamma_2x^{2n} + \dots + \gamma_sx^{sn},
    \]
    donde \(\gamma_i \in \mathbb F_q\).
  \end{enumerate}
\end{theorem}

Dados \(g, f \in R\) supongamos que \(Rg \subseteq Rf\) con \(Rg \neq 0\).
Entonces \(g = hf\), por lo que decimos que \(f\) es un \emph{divisor por la derecha} de \(g\) y lo notaremos por \(f \mid_{d} g\).
Equivalentemente, podemos decir que \(g\) es un \emph{múltiplo por la izquierda} de \(f\).
Observemos que de igual forma, si \(f \mid_{d} g\) entonces \(Rg \subseteq Rf\).

Tenemos que \(Rf = Rg \neq 0\) si y solo si \(f \mid_d g\) y \(g \mid_d f\).
Así, \(g = hf\) y \(f = lg\), por lo que \(g = hlg\).
Por tanto, \(hl = lh = 1\) por lo que \(h\) y \(l\) son unidades de \(R\).
Se dice entonces que \(f\) y \(g\) son \emph{asociados por la izquierda} en el sentido de que \(g = uf\), siendo \(u\) una unidad de \(R\).

Se tiene que \(Rf + Rg = Rh\).
Entonces \(h \mid_d f\) y \(h \mid_d g\).
De hecho si \(l \mid_d f\) y \(l \mid_d f\) entonces \(Rf \subset Rl\) y \(Rg \subset Rl\), por lo que \(Rh \subset Rl\) y \(l \mid_d h\).
Por tanto \(h\) es un \emph{máximo común divisor por la derecha} de \(f\) y \(g\) y lo notamos como \(h = (f, g)_d\).
Dos máximo común divisor por la derecha de los mismos dos elementos son asociados por la izquierda.

Se puede comprobar que \(R\) satisface la condición de Ore por la izquierda \parencite[ver][p. 4]{jacobson_finite-dimensional_1996}, por lo que si \(f \neq 0\) y \(g \neq 0\) se tiene que \(Rf \cap Rg \neq 0\).
Tenemos por tanto que \(Rf \cap Rg Rh\) para algún \(h\) por lo que \(m = g'f = f'g\).
De hecho si \(f \mid_d l\) y \(g \mid_d l\) entonces \(Rl \subset Rf \cap Rg = Rh\), por lo que \(h \mid_d l\).
Por tanto \(h\) es un \emph{mínimo común múltiplo por la izquierda} y lo notamos por \(h = [f, g]_i\).
De nuevo, dos mínimo común múltiplo por la izquierda de los mismos dos elementos son asociados por la izquierda.

Puede definirse una versión del algoritmo extendido de Euclides en este contexto (ver el algoritmo \ref{alg:ore-fq-euclides}), que nos permite calcular tanto el máximo común divisor como el mínimo común múltiplo.

\begin{Ualgorithm}[h]
  \DontPrintSemicolon
  \KwIn{polinomios \(f, g \in \mathbb F_q[x, \sigma]\) con \(f \neq 0\), \(g \neq 0\)}
  \KwOut{un número \(n \in \mathbb N\), polinomios \(u_i, v_i, q_i, f_i \in \mathbb F_q[x, \sigma]\) tales que \(f_i = u_if + v_ig\), \(q_i = \operatorname{coi}(f_{i-1}, f_i)\), para \(1 \leq i \leq n + 1\) y \(f_n = (f, g)_d\), \(u_nf = -v_ng = [f, g]_i\).}
  \(u_0 \longleftarrow v_1 = 1\)\;
  \(u_1 \longleftarrow v_0 = 1\)\;
  \(f_0 \longleftarrow f\)\;
  \(f_1 \longleftarrow g\)\;
  \(i \longleftarrow 1\)\;
  \While{\(f_i \neq 0\)}{
    \(q_i \longleftarrow \operatorname{coi}(f_{i-1}, f_i)\)\;
    \(u_{i+1} \longleftarrow u_{i-1} - q_iu_i\)\;
    \(v_{i+1} \longleftarrow v_{i-1} - q_iv_i\)\;
    \(f_{i+1} \longleftarrow f_{i-1} - q_if_i\)\;
    \(n \longleftarrow i\)\;
    }
    \caption{Algoritmo extendido de Euclides por la izquierda en \(\mathbb F_q[x, \sigma]\)
  }
  \label{alg:ore-fq-euclides}
\end{Ualgorithm}
  
Como \(R\) es un dominio de ideales principales es posible descomponer cada polinomio \(f \in R\) en un producto de factores irreducibles.
Pero esta factorización no será única.

Decimos que dos polinomios \(f, g \in R\) distintos de cero son \emph{similares por la izquierda}, que notamos \(f \sim_i g\) si existe un polinomio \(h \in R\) tal que 
\[
  (h, g)_d = 1 \quad\text{y}\quad f = [g, h]_ih^{-1}.
\]
La condición \((h, g)_d = 1\) equivale a que existan \(a\) y \(b \in R\) tales que
\[
  1 = ah + bg
\]
y \(f =  [g, h]_ih^{-1}\) equivale a que 
\[
  l = h'g = fh,
\]
donde \((h', f)_i = 1\).
Por tanto tenemos un \(h'\) tal que \((h', f)_i = 1\) y \(g = h^{'-1}[h', f]_d\).
Por tanto si \(f\) es similar por la izquierda a \(g\) entonces \(g\) es similar por la derecha a \(f\), por lo que escribiremos simplemente que \(f \sim g\).
Es posible comprobar que la \emph{similitud} es una relación de equivalencia \parencite[ver][p. 11]{jacobson_finite-dimensional_1996}.

% MAYBE: Esto se deriva de que... (módulos etc) (consultar tal).

\begin{theorem}
  Si \(f = p_1 \dots p_r\) y \(f = q_1 \dots q_t\) son factorizaciones de \(f \in R\) como producto de irreducibles entonces \(r= t\) y salvo una posible reordenación, \(q_i \sim p_i\).
\end{theorem}

\begin{proof}
  Puede consultarse una generalización de la demostración en \parencite[Teorema 1.2.9]{jacobson_finite-dimensional_1996}.
\end{proof}

El problema de comprobar si dos polinomios \(f, g \in R\) verifican que \(f \sim g\).

% TODO: completar con la dificultad

Definimos la \emph{norma} \(i\)\emph{-ésima} de un elemento \(\gamma \in \mathbb F_q\) como
\[
  N_i(\gamma) = (\sigma^{i-1}\gamma)\dots (\sigma \gamma)\gamma \quad\text{para } i > 0 \quad\text{y } N_0(\gamma) = 1.
\]

\begin{proposition}
  \label{prop:norma-divisor}
  Si \(f(x) = \sum_0^n a_ix^{n-i} \in \mathbb F_q[x, \sigma]\) y \(\gamma \in \mathbb F_q\) entonces \((x - \gamma) \mid_d f(x)\) si y solo si \(\sum_0^n a_iN_{n-i}(\gamma) = 0\).
\end{proposition}

\begin{proof}
  Tenemos la identidad
\end{proof}

También se dan las siguientes identidades, que nos serán útiles cuando estudiemos los códigos cíclicos sesgados en el capítulo siguiente.
Dados \(\alpha, \beta, \gamma \in \mathbb F_q\) tales que \(\beta = \alpha^{-1}\sigma(\alpha)\) se tiene que
\begin{align}
  N_i(\sigma^k(\gamma)) &= \sigma^k(N_i(\gamma)),\nonumber\\
  N_i(\sigma^k(\beta)) &= \sigma^k(\alpha)^{-1}\sigma^{k+1}(\alpha).
  \label{eq:norma-beta}
\end{align}