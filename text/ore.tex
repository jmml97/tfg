\chapter{Anillos de polinomios de Ore}

En esta sección vamos a hablar sobre los anillos de polinomios de Ore, que serán la base de los códigos cíclicos sesgados.
% TODO: nota histórica sobre polinomios Ore
[Introducir nota histórica]
Primero vamos a dar la definición general, sin detenernos a justificar su construcción, pues acto seguido vamos a centrarnos en el caso que nos va a ocupar cuando trabajemos con códigos cíclicos sesgados.
Las definiciones y el desarrollo teórico seguidos en esta sección proceden de (citas).
% CITEME: libro Jacobson, artículo original de Ore

\begin{definition}
  Sea \(R\) un anillo, \(\sigma\) un endomorfismo de \(R\) y \(\delta\) una \(\sigma\)-\textit{derivación} de \(R\), es decir, \(\delta\) es un homomorfismo de grupos abelianos tal que para \(a, b \in R\) se verifica que
  \[
    \delta(ab) = (\sigma a)(\delta b) + (\delta a)b.
  \]
  Entonces, el anillo \(R[t; \sigma, \delta]\) de los polinomios en \(R[t]\) de la forma
  \[
    a_0 + a_1t + \dots + a_nt^n,
  \]
  donde \(a_i \in R\), con la igualdad y suma usuales, y en el que la multiplicación verifica la relación 
  \[
  ta = (\sigma a)t + \delta a, \qquad a \in R,
  \]
  se conoce como \textit{anillo de polinomios de Ore} o \textit{anillos de polinomios torcidos}.
\end{definition}

Para comprobar que \(R[t; \sigma, \delta]\) es un anillo tendríamos que comprobar que efectivamente se verifican las propiedades de los anillos.
Puesto que hemos usado la suma usual de los polinomios, bastaría probar que se verifica la propiedad asociativa para la multiplicación que hemos definido.

El estudio de los códigos cíclicos sesgados se hará sobre el anillo \(\mathbb F_q[x, \sigma]\), con \(\sigma\) un automorfismo.
Es decir, nos limitaremos al estudio de los anillos de polinomios de Ore en los que \(R = \mathbb F_q\) —cuerpo finito de \(q\) elementos—, hemos llamado \(x\) a \(t\), \(\sigma\) es un automorfismo y \(\delta = 0\).
Por tanto, vamos a centrarnos en comprobar que \(\mathbb F_q[x, \sigma]\) es un anillo, que como hemos comentado, se reduce a comprobar que se verifica la propiedad asociativa para la multiplicación, operación que en este caso verifica la relación 
\[
  xa = (\sigma a)x, \qquad a \in R.
\]
Vamos a ver inductivamente que, como podemos intuir, \(x^n a = (\sigma^n a)x^n\). 
Estudiado el caso base anterior y supuesto que se verifica la igualdad para \(n - 1\), para \(n\) tenemos que
\[
  x^{n}a = xx^{n - 1}a = x(\sigma^{n-1} a)x^{n-1} = \sigma(\sigma^{n-1} a)x^{n-1}x = (\sigma^{n}a)x^{n}.
\]
Ahora definimos
\[
  (ax^n)(bx^m) = a(\sigma^n b)x^{n+m},
\]
con lo que, junto a la propiedad distributiva podemos definir el producto de polinomios en \(x\) como
\[
  \textstyle(\sum a_nx^n)(\sum b_mx^m) = \sum(a_nx^n)(b_mx^m).
\]