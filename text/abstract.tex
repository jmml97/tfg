\chapter*{Resumen}

Un código es una estructura algebraica con la que es posible transmitir información de forma que el receptor pueda corregir los errores que se hayan podido producir durante la transmisión.
El objetivo principal de este trabajo es presentar, estudiar e implementar el algoritmo de Peterson-Gorenstein-Zierler para códigos cíclicos sesgados.
Esta clase de códigos es relevante por la gran cantidad de códigos que pueden diseñarse, de forma que es posible encontrar algunos con propiedades interesantes.
El algoritmo mencionado proporciona un método de decodificación eficiente para ellos.
Para abordar nuestro estudio es necesaria una ingente cantidad de conocimiento previo, cuya exposición ocupa la mayor parte de este trabajo.
Nos centramos primero en los fundamentos necesarios de álgebra —teoría de anillos y cuerpos finitos— y de teoría de códigos —códigos lineales y códigos cíclicos—.
Estudiamos además la versión original del algoritmo \textacr{PGZ} para ayudarnos en la comprensión del algoritmo para códigos cíclicos sesgados.
Posteriormente abordamos los anillos de polinomios de Ore, que constituyen la base de los códigos cíclicos sesgados.
Una vez explicada esta clase de códigos, estaremos en disposición de estudiar e implementar el algoritmo prometido: Peterson-Gorenstein-Zierler para códigos cíclicos sesgados.

\paragraph{Palabras clave}
\begin{itemize*}[label=,itemsep=4em,itemjoin=\hspace{2em}]
  \item teoría de códigos
  \item códigos cíclicos
  \item polinomios torcidos
  \item SageMath
  \item Peterson-Gorenstein-Zierler
\end{itemize*}

\chapter*{Summary}

\begin{otherlanguage}{english}

The main objetive of this project is to present, study and implement the Peterson-Gorenstein-Zierler algorithm for skew cyclic codes.
But, in order to do so, we need a seizable amount of knowledge, and that is precisely what we will cover in the first chapters of this project.

\paragraph{Chapter 1} This chapter will set the foundations of the mathematic knowledge that is needed to understand the latter parts of the project. 
Skew cyclic codes require knowledge of two concepts to be understood: Ore polynomial rings and cyclic codes.
Then, Ore polynomial rings are dependant on the theory of rings and field automorphisms, and cyclic codes are dependant on the theory of finite fields and rings.
We will tackle in this chapter all the concepts and relevant results of these areas.

\paragraph{Chapter 2} In this chapter, we will introduce the concept of a code and the most studied class of codes: linear codes.
We will also explain that linear codes over a finite field are a vector subspace, and therefore they are a mathematical structure that is well known and easy to study.
We will also use this chapter to introduce some concepts that we will refer to throughout the project and that will be fundamental to the further development of coding theory.
To that extent, we will present the definition of a generator matrix: a base of the vector subspace that a linear code ultimately is.
As we shall see, this matrix is the basis of the most common procedure of message encoding.
Another important concept we will introduce is the distance of a code, that is, the minimum number of coordinates that differentiate one codeword —an element of a code— from another.
Finally, some simple families of codes will be presented, such as repetition codes or Hamming codes.

\paragraph{Chapter 3} In this chapter, we will shift our focus to the class of cyclic codes. 
As the name may suggest, these codes have the property that cyclic shifts of codewords in a code are also codewords of said code.
We will see that codewords of these kinds of codes can be mapped to certain polynomials, and cyclic codes are simply the ideals of a polynomial ring quotient over the ideal generated by the polynomial \(x^n - 1\) for some \(n\).
That is why we will dedicate the first part of this chapter to the study of the factorization of this polynomial on polynomial rings over finite fields.
Once we know how to do this, we can properly describe all cyclic codes of any length.
We will also show two methods for encoding messages in cyclic codes.
Subsequently, an alternative way of describing cyclic codes is explained, using what we will call generating idempotents: elements whose product by themselves results in the same element.
Finally, we will explain the concept of zeroes of cyclic codes, a notion we will see is fundamental in the definition of \textacr{BCH} codes in next chapter.

\paragraph{Chapter 4} In this chapter, we will describe the family of \textacr{BCH} codes and the original version of the Peterson-Gorenstein-Zierler algorithm for \textacr{BCH} codes.
We will also take the opportunity to briefly introduce the family of \textacr{RS} codes, as we will refer to them in subsequent chapters.
To properly introduce \textacr{BCH} codes, we will first explain the \textacr{BCH} bound, a result that links the concept of zeroes of a cyclic code and the minimum distance of said code.
We will see that \textacr{BCH} codes are defined to take advantage of the \textacr{BCH} bound.
This means that it is possible to design \textacr{BCH} codes with any error correction capability, although their length will vary accordingly.
To finish this chapter we will take on the Peterson-Gorenstein-Zierler algorithm for \textacr{BCH} codes: we will explain the decoding procedure and prove that it succesfully corrects errors.

\paragraph{Chapter 5} In this chapter, Ore polynomial rings will be discussed.
Since in coding theory we will be working with finite fields we will limit our study to Ore polynomial rings over finite fields.
We will introduce the main concepts as well as algorithms to calculate division from left or right, as well as an extented Euclid algorithm.
We will also elaborate on the fact that factorization of this kind of polynomials is not unique in the common sense, but we will introduce a concept that is similar in practice.

\paragraph{Chapter 6} In this chapter, we will describe the class of skew cyclic codes.
Throughout the chapter we will set the basis for the introduction of skew \textacr{RS} codes at the end.
This is the family of codes that we will be using with the algorithm that will be explained in the following chapter.

\paragraph{Chapter 7} Finally, in this chapter, we will introduce the algorithm that is the main objective of this project: the Peterson-Gorenstein-Zierler algorithm for skew cyclic codes.
As we did with the analogous algorithm for \textacr{BCH} codes, we will explain the decoding procedure as well as prove that it succesfully corrects errors.

As part of this project we have developed various classes for SageMath.
\begin{itemize}
  \item A decoder class for the \textacr{BCH} codes implementation of SageMath that uses the Peterson-Gorenstein-Zierler algorithm described here.
  \item A skeleton class for skew cyclic codes. We have not developed methods for this class but rather left a framework for other classes to be inherited by this one.
  \item A skew \textacr{RS} code class that implements the definition we study in this project, as well as a simple encoder class for them.
  \item Finally, a decoder class for the skew \textacr{RS} codes that uses the Peterson-Gorenstein-Zierler algorithm for skew cyclic codes that is the main goal of this project.
\end{itemize}

These classes allow us to work with the structures in SageMath that are already implemented.
The documentation for these classes can be found in annex \ref{annex:pgz-sage}.
Throughout the project we will present some examples for the concepts we explain, using SageMath either with the pre-existing classes or the classes we created.
We have also used some other helpful functions that we have described in annex \ref{annex:sage-gen-idemp}.

\paragraph{Keywords}
\begin{itemize*}[label=,itemsep=4em,itemjoin=\hspace{2em}]
  \item coding theory
  \item cyclic codes 
  \item skew polynomials
  \item SageMath
  \item Peterson-Gorenstein-Zierler
\end{itemize*}

\end{otherlanguage}