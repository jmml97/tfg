\documentclass[crop]{standalone}

\usepackage{tikz}
\usetikzlibrary{shapes,arrows}

\usepackage{fontspec}
\setmainfont[WordSpace=1.3]{EBGaramond}
\setsansfont{Source Sans Pro}
\setmonofont[Scale=0.8]{Vera Mono}

\usepackage[math-style=TeX, bold-style=ISO]{unicode-math}
\setmathfont{Garamond Math}[StylisticSet={3}]

\begin{document}

\sffamily

\tikzstyle{block} = [draw, rectangle, 
    minimum height=6em, minimum width=6em, text width=6em, align=center]

% The block diagram code is probably more verbose than necessary
\begin{tikzpicture}[auto, node distance=2cm,>=latex']
    % We start by placing the blocks
    
    %\node [input, name=input] {};
    %\node [sum, right of=input] (sum) {};
    \node [block] (fuente) {Fuente de la información};
    \node [block, right of=fuente, node distance=9em] (transmisor) {Transmisor};
    \node [block, right of=transmisor, node distance=12em] (canal) {Canal};
    \node [block, right of=canal, node distance=12em] (receptor) {Receptor};
    \node [block, right of=receptor, node distance=9em] (destino) {Destino};
    \node [block, below of=canal, node distance=9em] (ruido) {Fuente de ruido};

    %\node [block, right of=controller, pin={[pinstyle]above:Disturbances}, node distance=3cm] (system) {System};

    % We draw an edge between the controller and system block to 
    % calculate the coordinate u. We need it to place the measurement block. 
    \draw [->] (fuente) -- node[below, yshift=-4em, name=mensaje] {Mensaje} (transmisor);
    \draw [->] (transmisor) -- node[below, name=señal] {Señal} (canal);
    \draw [->] (canal) -- node[below, name=señalr, text width=4em, align=center] {Señal recibida} (receptor);
    \draw [->] (receptor) -- node[below, yshift=-4em, name=mensajer] {Mensaje} (destino);
    \draw [->] (ruido) -- (canal);
    
    %\node [output, right of=system] (output) {};
    %\node [block, below of=u] (measurements) {Measurements};

    % Once the nodes are placed, connecting them is easy. 
    %\draw [draw,->] (input) -- node {$r$} (sum);
    %\draw [->] (sum) -- node {$e$} (controller);
    %\draw [->] (system) -- node [name=y] {$y$}(output);
    %\draw [->] (y) |- (measurements);
    %\draw [->] (measurements) -| node[pos=0.99] {$-$} node [near end] {$y_m$} (sum);
\end{tikzpicture}

\end{document}