
\chapter{Códigos cíclicos}

En este capítulo vamos a estudiar los aspectos fundamentales de la familia de los códigos cíclicos, que representan la base del objeto de estudio de este trabajo.
Las fuentes de este capítulo han sido \parencite{huffman_fundamentals_2003}, \parencite{kelbert_information_2013} y \parencite{macwilliams_theory_1977}.

\begin{definition}
  Un código lineal \(\mathcal C\) de longitud \(n\) sobre \(\mathbb F_q\) es \textit{cíclico} si para cada vector \(\mathbf c = c_0\dots c_{n-2}c_{n-1}\) en \(\mathcal C\), el vector \(c_{n-1}c_0\dots c_{n-2}\) —obtenido a partir de \(\mathbf c\) desplazando cíclicamente las coordenadas, llevando \(i \mapsto i +1 \bmod n\)— también está en \(\mathcal C\).
\end{definition}

Al trabajar con códigos cíclicos pensamos en la posición de las coordenadas de forma cíclica, pues al llegar a \(n -1\) se comienza de nuevo en \(0\).
Al hablar de «coordenadas consecutivas» siempre tendremos en cuenta esta ciclicidad.
Representaremos las palabras código de los códigos cíclicos como polinomios, pues podemos definir de forma natural una biyección entre el vector \(\mathbf c = c_0c_1\dots c_{n-1}\) en \(\mathbb F_q\) y los polinomios de la forma \(c(x) = c_0 + c_1x + \dots c_{n-1}x^{n-1}\) en \(\mathbb F_q[x]\) de grado al menos \(n-1\). 
Obtenemos así un isomorfismo entre \(\mathbb F_q\)-espacios vectoriales.
Obsérvese que dado un polinomio \(c(x)\) descrito como antes, el polinomio \(xc(x) = c_{n-1}x^n + c_0x + c_1x^2 + \dots + c_{n-2}x^{n-1}\) equivale a representar la palabra código \(\mathbf c\) desplazada una posición a la derecha, siempre que \(x^n\) fuese igual a \(1\).

Formalmente, el hecho de que un código \(\mathcal C\) sea invariante bajo un desplazamiento cíclico implica que si \(c(x)\) está en \(\mathcal C\), también ha de estar \(xc(x)\), siempre que multipliquemos módulo \(x^n -1\). 
Esto nos sugiere que el contexto adecuado para estudiar códigos cíclicos es el anillo cociente \(\mathcal R_n = \mathbb F_q[x]/(x^n - 1)\).

Por tanto, bajo la correspondencia vectores-polinomios que hemos descrito antes, los códigos cíclicos son ideales de \(\mathcal R_n\), y los ideales de \(\mathcal R_n\) son códigos cíclicos.
En consecuencia, el estudio de los códigos cíclicos en \(\mathbb F_q^n\) es equivalente al estudio de los ideales en \(\mathcal R_n\), que va a depender de la factorización de \(x^n-1\) y por tanto lo vamos a abordar a continuación.

\section{Factorización de \texorpdfstring{\(x^n -1\)}{xn - 1}}

\label{sec:factorizacion-xn-1}

A la hora de factorizar \(x^n -1\) existen dos posibilidades, pues dicha factorización puede tener factores irreducibles repetidos o no.
Vamos a asumir que \(q\) y \(n\) son primos relativos y por tanto \(x^n - 1\) no tiene factores repetidos; en caso contrario el anillo cociente sería semisimple, lo que no nos aportaría nada: los ideales generados por los polinomios con factores repetidos no aumentan la distancia, pues es la misma que la del subcódigo sin factores repetidos.

Para factorizar \(x^n - 1\) sobre \(\mathbb F_q\) necesitamos considerar la extensión de cuerpos \(\mathbb F_{q^t}\) de \(\mathbb F_q\) que contenga todas sus raíces.
El cuerpo \(\mathbb F_{q^t}\) debe contener una enésima raíz primitiva de la unidad, que por el teorema \ref{th:Fq-ast-cilcico} sabemos que ocurre cuando \(n \mid (q^t - 1)\).
Definimos el \textit{orden} \(\operatorname{ord}_n(q)\) de \(q\) módulo \(n\) como el menor entero positivo \(a\) tal que \(q^{a} \equiv 1 \bmod n\).
Si \(t = \operatorname{ord}_n(q)\) entonces \(\mathbb F_{q^t}\) contiene una enésima raíz primitiva de la unidad \(\alpha\) pero no hay una extensión de cuerpos más pequeña de \(\mathbb F_q\) que la contenga.
Como todos los \(\alpha^{i}\) son distintos dos a dos para \(0 \leq i < n\) y \((\alpha^{i})^n = 1\), entonces \(\mathbb F_{q^t}\) contiene todas las raíces de \(x^n - 1\).
Por tanto, \(\mathbb F_{q^t}\) es lo que se conoce como \textit{cuerpo de descomposición} de \(x^n - 1\) sobre \(\mathbb F_q\).

Los factores irreducibles de \(x^n - 1\) sobre \(\mathbb F_q\) deben ser el producto de los distintos polinomios minimales de las enésimas raíces de la unidad en \(\mathbb F_{q^t}\).
Supongamos que \(\gamma\) es un elemento primitivo de \(\mathbb F_{q^t}\).
Entonces \(\alpha = \gamma^d\) es una enésima raíz primitiva de la unidad, donde \(d = (q^t - 1)/n\).
Las raíces del polinomio \(M_{a^s}(x)\) son \(\{\gamma^{ds}, \gamma^{dsq}, \gamma^{dsq^2}, \dots, \gamma^{dsq^{r-1}}\} = \{\alpha^s, \alpha^{sq}, \alpha^{sq^2}, \dots, \alpha^{sq^{r-1}}\}\), donde \(r\) es el menor entero positivo tal que \(dsq^r \equiv ds \bmod q^t - 1\) por el teorema \ref{th:pol-minimal-el-primitivo}.
Pero \(dsq^r \equiv ds \bmod q^t - 1\) si y solo si \(sq^r \equiv s \bmod n\).

Todo esto nos lleva a extender la definición de clases \(q\)-ciclotómicas que hemos introducido en la sección \ref{subsec:clases-ciclotomicas}.
Sea \(s\) un entero tal que \(0 \leq s < n\).
La \textit{clase \(q\)-ciclotómica de \(s\) módulo \(n\)} es el conjunto
\[
  C_s = \{s, sq, \dots, sq^{r-1}\} \bmod n, 
\]
donde \(r\) es el menor entero positivo tal que \(sq^r \equiv s \bmod n\).
Se deduce entonces que \(C_s\) es la órbita de la permutación \(i \mapsto iq \bmod n\) que contiene a \(s\).
Las distintas clases \(q\)-ciclotómicas módulo \(n\) dividen el conjunto de enteros \(\{0, 1, 2, \dots, n - 1\}\).
En la sección * estudiamos el caso particular en el que \(n = q^t - 1\).
Obsérvese que \(\operatorname{ord}_n(q)\) es el tamaño de la clase \(q\)-ciclotómica \(C_1\) módulo \(n\).

\begin{theorem}
  \label{th:pol-minimal-raiz-primitiva}
  Sea \(n\) un entero positivo primo relativo con \(q\) y sea \(t = \operatorname{ord}_n(q)\).
  Sea \(\alpha\) una raíz enésima primitiva de la unidad en \(\mathbb F_{q^t}\).
  Se verifican las siguientes afirmaciones.
  \begin{enumerate}
    \item Para cada entero \(s\) tal que \(0 \leq s < n\) el polinomio minimal de \(\alpha^s\) sobre \(\mathbb F_q\) es
    \[
      M_{\alpha^s}(x) = \prod_{i \in C_s}(x - \alpha^i),
    \]
    donde \(C_s\) es la clase \(q\)-ciclotómica de \(s\) módulo \(n\).
    \label{thi:pol-minimal-raiz-primitiva-producto}
    \item Los conjugados de \(\alpha^s\) son los elementos \(\alpha^i\) con \(i \in C_s\).
    \item Se tiene que
    \[
      x^n - 1 = \prod_s M_{\alpha^s}(x)
    \]
    es la factorización de \(x^n - 1\) en factores irreducibles sobre \(\mathbb F_q\), donde \(s\) varía en un conjunto de representantes de las clases \(q\)-ciclotómicas módulo \(n\).
  \end{enumerate}
\end{theorem}

\begin{proof}
  % TODO (en el libro no viene)
\end{proof}

\begin{theorem}
  El tamaño de cada clase \(q\)-ciclotómica es un divisor de \(\operatorname{ord}_n(q)\).
  Además, el tamaño de \(C_1\) es \(\operatorname{ord}_n(q)\).
\end{theorem}

\section{Construcción de códigos cíclicos}

Una vez factorizado \(x^n - 1\) vamos a ver que hay una correspondencia biyectiva entre sus polinomios divisores mónicos y los códigos cíclicos en \(\mathcal R_n\).
El siguiente teorema es el resultado fundamental de códigos cíclicos que nos va a permitir describirlos.

\begin{theorem}
  \label{th:corr-cod-div}
  Sea \(\mathcal C\) un ideal de \(\mathcal R_n\), es decir, un código cíclico de longitud \(n\). Entonces:
  \begin{enumerate}
    \item Existe un único polinomio mónico \(g(x)\) de grado mínimo en \(\mathcal C\).\label{thi:corr-codc-div:monico-minimo}
    \item El polinomio descrito en (\ref{thi:corr-codc-div:monico-minimo}) genera \(\mathcal C\), es decir, \(\mathcal C = \langle g(x)\rangle\).
    \item El polinomio descrito en (\ref{thi:corr-codc-div:monico-minimo}) verifica que \(g(x) \mid x^n -1\).\label{thi:corr-codc-div:div-xn-1}
  \end{enumerate}
  Sea \(k = n - \operatorname{gr} g(x)\) y sea \(g(x) = \sum_{i_0}^{n-k}g_ix^{i}\), donde \(g_{n-k} = 1\). Entonces:
  \begin{enumerate}[resume]
    %\item La dimensión de \(\mathcal C\) es \(k\) y \(\{g(x), xg(x), \dots, x^{k-1}g(x)\}\) es una base de \(\mathcal C\).
    %\item Cada elemento de \(\mathcal C\) se puede expresar como producto de \(g(x)\) por un polinomio \(f(x)\), donde \(f(x) = 0\) o bien \(\operatorname{gr} f(x) < k\).
    \item \label{thi:corr-codc-div:forma-elem} Se verifica que \[
      \mathcal C = \langle g(x) \rangle = \{f(x)g(x) : \operatorname{gr} f(x) < k\}.
    \]
    \item \label{thi:corr-codc-div:dim-ideal} El conjunto \(\{g(x), xg(x), \dots, x^{k-1}g(x)\}\) es una base de \(\mathcal C\) y \(\mathcal C\) tiene dimensión \(k\).
    \item \label{thi:corr-cod-div:mat-gen} La matriz \(G\) dada por \[
      G = \begin{bmatrix}
        g_0 & g_1 & g_2 & \dots & \dots & g_{n-k} & 0 & 0 & \dots & 0 \\
        0 & g_0 & g_1 & g_2 & \dots & \dots & g_{n-k} & 0 & \dots & 0 \\
        0 & 0 & g_0 & g_1 & g_2 & \dots & \dots & g_r & \ddots & \vdots \\
        \vdots & \vdots & \ddots & \ddots & \ddots & \ddots & & & \ddots & 0\\
        0 & 0 & \dots & 0 & g_0 & g_1 & g_2 & \dots & \dots & g_{n-k} 
      \end{bmatrix},
    \]
    donde cada fila es un desplazamiento cíclico de la fila previa, es una matriz generadora de \(\mathcal C\).
    \item \label{thi:pol-generador-prod-minimal} Si \(\alpha\) es una enésima raíz primitiva de la unidad en alguna extensión de cuerpos de \(\mathbb F_q\) entonces \[
      g(x) = \prod_s M_{\alpha^s}(x),
    \] siendo dicho producto sobre un subconjunto de representantes de las clases \(q\)-ciclotómicas módulo \(n\).
  \end{enumerate}
\end{theorem}

\begin{proof}
  Veamos la demostración apartado por apartado.
  \begin{enumerate}
    \item Supongamos que \(\mathcal C\) contiene dos polinomios mónicos distintos, \(g_1(x)\) y \(g_2(x)\), ambos de grado mínimo \(r\). 
    Entonces, \(g_1(x) - g_2(x)\) es un polinomio no nulo de grado menor que \(r\), lo que es absurdo. 
    Existe por tanto un único polinomio de grado mínimo \(r\) en \(\mathcal C\), como queríamos.
    \item Como \(g(x) \in \mathcal C\) y \(\mathcal C\) es un ideal, tenemos que \(\langle g(x)\rangle \subset \mathcal C\). 
    Por otra parte, dado \(p(x) \in \mathcal C\) el algoritmo de división nos da elementos \(q(x), r(x)\) tales que \(p(x) = q(x)g(x) + r(x)\), de forma que o bien \(r(x) = 0\) o bien \(\operatorname{gr} r(x) < \operatorname{gr} g(x)\). 
    Como podemos expresar \(r(x)\) de la forma \(r(x) = p(x) - q(x)g(x) \in \mathcal C\) y tiene grado menor que \(\operatorname{gr} g(x)\), al ser este último de grado mínimo necesariamente ha de darse que \(r(x) = 0\).
    Por tanto, \(p(x) = q(x)g(x) \in \langle g(x) \rangle\) y \(\mathcal C \subset \langle g(x) \rangle\).
    En consecuencia, \(\langle g(x) \rangle = \mathcal C\).
    \item Por el algoritmo de división, al dividir \(x^n - 1\) por \(g(x)\) tenemos que \(x^n - 1 = q(x)g(x) + r(x)\). De nuevo, o bien \(r(x) = 0\) o bien \(\operatorname{gr} r(x) < \operatorname{gr} g(x)\).
    Como en \(\mathcal R_n\) se tiene que \(x^n - 1 = 0 \in \mathcal C\), necesariamente \(r(x) \in \mathcal C\).
    Esto supone una contradicción, a menos que \(r(x) = 0\).
    En consecuencia, \(g(x) \mid x^n - 1\).
    \item El ideal generado por \(g(x)\) es \(\langle g(x) \rangle = \{f(x)g(x) : f(x) \in \mathcal R_n\}\).
    Queremos ver que podemos restringir los polinomios \(f(x)\) a aquellos que tengan grado menor que \(k\).
    Por (\ref{thi:corr-codc-div:div-xn-1}) sabemos que \(x^n-1 = h(x)g(x)\) para algún polinomio \(h(x)\) que tenga grado \(k = n - \operatorname{gr} g(x)\).
    Dividimos entonces \(f(x)\) por este polinomio \(h(x)\) y por el algoritmo de división obtenemos \(f(x) = q(x)h(x) + r(x)\), donde \(\operatorname{gr} r(x) < \operatorname{gr} h(x) = k\).
    Entonces, tenemos \begin{align*}
      f(x)g(x) &= q(x)h(x)g(x) + r(x)g(x)\\
               &= q(x)(x^n - 1) + r(x)g(x),
    \end{align*}
    luego \(f(x)g(x) = r(x)g(x)\), y puesto que antes ya hemos visto que \(\operatorname{gr} r(x) < k\), hemos obtenido lo que buscábamos.
    \item A partir de (\ref{thi:corr-codc-div:dim-ideal}) tenemos que el conjunto \[\{g(x), xg(x), \dots, x^{k-1}g(x)\}\] genera \(\mathcal C\), y como es linealmente independiente, forma una base de \(\mathcal C\).
    Esto demuestra también que la dimensión de \(\mathcal C\) es \(k\).
    \item La matriz \(G\) es matriz generadora de \(\mathcal C\) pues \[\{g(x), xg(x), \dots, x^{k-1}g(x)\}\] es una base de \(\mathcal C\).
    \item Se deduce del teorema \ref{th:pol-minimal-raiz-primitiva} y de (\ref{thi:corr-codc-div:div-xn-1}).\qedhere
  \end{enumerate}
\end{proof}

Este teorema nos proporciona una forma de obtener los códigos cíclicos de longitud \(n\) a partir de los divisores del polinomio \(x^n - 1\) así como describir una matriz generadora de dichos códigos a partir de ellos.
Vamos a ver a continuación que el polinomio mónico divisor de \(x^n - 1\) que genera a un código cíclico \(\mathcal C\) es único.

\begin{corollary}
  \label{cor:pol-gen-unico}
  Sea \(\mathcal C\) un código cíclico en \(\mathcal R_n\) distinto de cero.
  Son equivalentes:
  \begin{enumerate}
    \item El polinomio \(g(x)\) es el polinomio mónico de menor grado en \(\mathcal C\).
    \item Podemos expresar \(\mathcal C\) como \(\mathcal C = \langle g(x)\rangle\), \(g(x)\) es mónico y \(g(x) \mid (x^n -1)\).
  \end{enumerate}
\end{corollary}

\begin{proof}
  Que (1) implica (2) ya lo hemos probado en el teorema \ref{th:corr-cod-div}. 
  Veamos que partiendo de (2) obtenemos (1). 
  Sea \(g_1(x)\) el polinomio mónico de menor grado en \(\mathcal C\).
  Por el teorema \ref{th:corr-cod-div}, \(g_1(x) \mid g(x)\) en \(\mathbb F_q[x]\) y \(\mathcal C = \langle g_1(x)\rangle\).
  Como \(g_1(x) \in \mathcal C = \langle g(x) \rangle\), podemos expresarlo como \(g_1(x) = g(x)a(x) \bmod x^n - 1\), luego tenemos que \(g_1(x) = g(x)a(x) + (x^n - 1)b(x)\) en \(\mathbb F_q[x]\).
  Por otro lado, como \(g(x) \mid (x^n - 1)\), tenemos que \(g(x) \mid g(x)a(x) + (x^n-1)b(x)\), o lo que es lo mismo, que \(g(x) \mid g_1(x)\). 
  En consecuencia, como \(g_1(x)\) y \(g(x)\) son ambos mónicos y dividen el uno al otro en \(\mathbb F_q[x]\), son necesariamente iguales.
\end{proof}

A este polinomio \(g(x)\) lo llamamos \textit{polinomio generador} del código cíclico \(\mathcal C\).
Por el corolario anterior, este polinomio es tanto el polinomio mónico en \(\mathcal C\) de grado mínimo como el polinomio mónico que divide a \(x^n - 1\) y genera a \(\mathcal C\).
Existe por tanto una correspondencia biunívoca entre los códigos cíclicos distintos de cero y los divisores de \(x^n - 1\) distintos de él mismo.
Para extender dicha correspondencia entre todos los códigos cíclicos en \(\mathcal R_n\) y todos los divisores mónicos de \(x^n - 1\) definimos como polinomio generador del código cíclico \(\{\mathbf 0\}\) el polinomio \(x^n - 1\). 
Esta correspondencia biyectiva nos conduce al siguiente corolario.

\begin{corollary}
  El número de códigos cíclicos en \(\mathcal R_n\) es \(2^m\), donde \(m\) es el número de clases \(q\)-ciclotómicas módulo \(n\).
  %Es más, las dimensiones de los códigos cíclicos en \(\mathcal R_n\) son todas sumas de tamaños de las clases \(q\)-ciclotómicas módulo \(n\).
\end{corollary}

Ahora mismo todo este desarrollo puede parecer demasiado abstracto.
Vamos a ver un ejemplo exhaustivo para entender cómo podemos obtener los polinomios generadores de los códigos cíclicos de una longitud arbitraria y cómo éstos son generados a partir de ellos.

\begin{example}
  \label{ex:codigos-ciclicos-long-7}
  Vamos a describir todos los códigos cíclicos binarios de longitud 7.
  Para ello vamos a utilizar el código descrito en el anexo \ref{annex:sage-gen-idemp} tal y como mostramos en el listado siguiente.
  \begin{lstlisting}[gobble=4]
    sage: F = GF(2)
    sage: x = polygen(F)
    sage: (x^7 - 1).factor()
    > (x + 1) * (x^3 + x + 1) * (x^3 + x^2 + 1)
    sage: print(generadores(x^7 - 1))
    > [1, x + 1, x^3 + x + 1, x^3 + x^2 + 1, x^4 + x^3 + x^2 + 1, x^4 + x^2 + x + 1, x^6 + x^5 + x^4 + x^3 + x^2 + x + 1, x^7 + 1]
  \end{lstlisting}
  Así sobre \(\mathbb F_2\) podemos descomponer \(x^7 - 1\) como \[
    x^7 - 1 = (x + 1)(x^{3} + x + 1)(x^{3} + x^{2} + 1)
  \]
  y los 8 polinomios generadores, todos los divisores de \(x^7 - 1\), son: \begin{enumerate}
    \item \(1\)
    \item \((x + 1)\)
    \item \((x^{3} + x + 1)\)
    \item \((x^{3} + x^{2} + 1)\)
    \item \((x + 1)(x^{3} + x + 1) = x^4 + x^3 + x^2 + 1\)
    \item \((x + 1)(x^{3} + x^{2} + 1) = x^4 + x^2 + x + 1\)
    \item \((x^{3} + x + 1)(x^{3} + x^{2} + 1) = x^6 + x^5 + x^4 + x^3 + x^2 + x + 1\)
    \item \((x + 1)(x^{3} + x + 1)(x^{3} + x^{2} + 1) = x^7 - 1\)
  \end{enumerate}
  Vamos a ver qué códigos generan estos polinomios: \begin{enumerate}
    \item La dimensión del código es \(k = 7 - 0 = 7\), luego el código generado es un \([7, 7]\)-código lineal, que es evidentemente \(\mathbb F_2^7\). La matriz generadora que nos proporciona el teorema \ref{th:corr-cod-div}(\ref{thi:corr-cod-div:mat-gen}) es \[
      G = \left(\begin{array}{rrrrrrr}
        1 & 0 & 0 & 0 & 0 & 0 & 0 \\
        0 & 1 & 0 & 0 & 0 & 0 & 0 \\
        0 & 0 & 1 & 0 & 0 & 0 & 0 \\
        0 & 0 & 0 & 1 & 0 & 0 & 0 \\
        0 & 0 & 0 & 0 & 1 & 0 & 0 \\
        0 & 0 & 0 & 0 & 0 & 1 & 0 \\
        0 & 0 & 0 & 0 & 0 & 0 & 1
        \end{array}\right).
    \]
    \item La dimensión del código es \(k = 7 - 1 = 6\), luego el código generado es un \([7, 6]\)-código lineal.
    La matriz generadora que nos proporciona el teorema \ref{th:corr-cod-div}(\ref{thi:corr-cod-div:mat-gen}) es \[
      G = \left(\begin{array}{rrrrrrr}
        1 & 1 & 0 & 0 & 0 & 0 & 0 \\
        0 & 1 & 1 & 0 & 0 & 0 & 0 \\
        0 & 0 & 1 & 1 & 0 & 0 & 0 \\
        0 & 0 & 0 & 1 & 1 & 0 & 0 \\
        0 & 0 & 0 & 0 & 1 & 1 & 0 \\
        0 & 0 & 0 & 0 & 0 & 1 & 1
        \end{array}\right).
    \]
    Comprobamos que la matriz de paridad es \[
      H = \left(\begin{array}{rrrrrrr}
        1 & 1 & 1 & 1 & 1 & 1 & 1
        \end{array}\right)
    \]
    y por tanto el código obtenido es un código de control de paridad.
    \item La dimensión del código es \(k = 7 - 3 = 4\), luego el código generado es un \([7, 4]\)-código lineal.
    La matriz generadora que nos proporciona el teorema \ref{th:corr-cod-div}(\ref{thi:corr-cod-div:mat-gen}) es \[
      G = \left(\begin{array}{rrrrrrr}
        1 & 1 & 0 & 1 & 0 & 0 & 0 \\
        0 & 1 & 1 & 0 & 1 & 0 & 0 \\
        0 & 0 & 1 & 1 & 0 & 1 & 0 \\
        0 & 0 & 0 & 1 & 1 & 0 & 1
        \end{array}\right).
    \]
    La matriz de paridad en este caso es \[
      H = \left(\begin{array}{rrrrrrr}
        1 & 0 & 1 & 1 & 1 & 0 & 0 \\
        0 & 1 & 0 & 1 & 1 & 1 & 0 \\
        0 & 0 & 1 & 0 & 1 & 1 & 1
        \end{array}\right)
    \]
    y por tanto el código generado es un \(\mathcal H_3\) código de Hamming.
    \item La dimensión del código es \(k = 7 - 3 = 4\), luego el código generado es un \([7, 4]\)-código lineal.
    La matriz generadora que nos proporciona el teorema \ref{th:corr-cod-div}(\ref{thi:corr-cod-div:mat-gen}) es \[
      G = \left(\begin{array}{rrrrrrr}
        1 & 0 & 1 & 1 & 0 & 0 & 0 \\
        0 & 1 & 0 & 1 & 1 & 0 & 0 \\
        0 & 0 & 1 & 0 & 1 & 1 & 0 \\
        0 & 0 & 0 & 1 & 0 & 1 & 1
        \end{array}\right).
    \]
    La matriz de paridad en este caso es \[
      H = \left(\begin{array}{rrrrrrr}
        1 & 1 & 1 & 0 & 1 & 0 & 0 \\
        0 & 1 & 1 & 1 & 0 & 1 & 0 \\
        0 & 0 & 1 & 1 & 1 & 0 & 1
        \end{array}\right)
    \]
    y por tanto el código generado es un \(\mathcal H_3\) código de Hamming.
    \item La dimensión del código es \(k = 7 - 4 = 3\), luego el código generado es un \([7, 3]\)-código lineal.
    La matriz generadora que nos proporciona el teorema \ref{th:corr-cod-div}(\ref{thi:corr-cod-div:mat-gen}) es \[
      G = \left(\begin{array}{rrrrrrr}
        1 & 0 & 1 & 1 & 1 & 0 & 0 \\
        0 & 1 & 0 & 1 & 1 & 1 & 0 \\
        0 & 0 & 1 & 0 & 1 & 1 & 1
        \end{array}\right).
    \]
    La matriz de paridad en este caso es \[
      H = \left(\begin{array}{rrrrrrr}
        1 & 1 & 0 & 1 & 0 & 0 & 0 \\
        0 & 1 & 1 & 0 & 1 & 0 & 0 \\
        0 & 0 & 1 & 1 & 0 & 1 & 0 \\
        0 & 0 & 0 & 1 & 1 & 0 & 1
        \end{array}\right)
    \]
    y por tanto el código generado es un \(\mathcal H_4\) código de Hamming.
    \item La dimensión del código es \(k = 7 - 4 = 3\), luego el código generado es un \([7, 3]\)-código lineal.
    La matriz generadora que nos proporciona el teorema \ref{th:corr-cod-div}(\ref{thi:corr-cod-div:mat-gen}) es \[
      G = \left(\begin{array}{rrrrrrr}
        1 & 1 & 1 & 0 & 1 & 0 & 0 \\
        0 & 1 & 1 & 1 & 0 & 1 & 0 \\
        0 & 0 & 1 & 1 & 1 & 0 & 1
        \end{array}\right).
    \]
    La matriz de paridad en este caso es \[
      H = \left(\begin{array}{rrrrrrr}
        1 & 0 & 1 & 1 & 0 & 0 & 0 \\
        0 & 1 & 0 & 1 & 1 & 0 & 0 \\
        0 & 0 & 1 & 0 & 1 & 1 & 0 \\
        0 & 0 & 0 & 1 & 0 & 1 & 1
        \end{array}\right)
    \]
    y por tanto el código generado es un \(\mathcal H_4\) código de Hamming.
    \item La dimensión del código es \(k = 7 - 6 = 1\), luego el código generado es un \([7, 1]\)-código lineal.
    La matriz generadora que nos proporciona el teorema \ref{th:corr-cod-div}(\ref{thi:corr-cod-div:mat-gen}) es \[
      G = \left(\begin{array}{rrrrrrr}
        1 & 1 & 1 & 1 & 1 & 1 & 1
        \end{array}\right),
    \] por lo que concluimos que el código generado es el código de repetición de longitud \(7\).
    \item La dimensión del código es \(k = 7 - 7 = 0\), luego el código generado es \(\{\mathbf 0\}\).
  \end{enumerate}
\end{example}

Finalmente, el siguiente resultado nos muestra la relación entre dos polinomios generadores cuando un código es subcódigo de otro.

\begin{corollary}
  \label{cor:subcodigos-ciclicos}
  Sean \(\mathcal C_1\) y \(\mathcal C_2\) códigos cíclicos sobre \(\mathbb F_q\) con polinomios generadores \(g_1(x)\) y \(g_2(x)\), respectivamente.
  Entonces, \(\mathcal C_1 \subseteq \mathcal C_2 \) si y solo si \(g_2(x) \mid g_1(x)\).
\end{corollary}

\begin{proof}
  Recordamos que por el teorema \ref{th:corr-cod-div}(\ref{thi:corr-codc-div:forma-elem}) todos los elementos \(a(x)\) de los códigos cíclicos \(\mathcal C_1\) y \(\mathcal C_2 \in \mathcal R_n\) pueden expresarse como \(a(x) = g_i(x)f_i(x)\) —donde, o bien\(f_i(x) = 0\), o bien \(\deg(f_i(x)) < k = n - \deg(g_i(x))\)— para \(i = 1, 2\) respectivamente.
  Veamos ambas implicaciones por separado.
  \begin{enumerate}
    \item Comenzamos con que si \(g_2(x) | g_1(x)\) entonces \(\mathcal C_1 \subseteq \mathcal C_2\).
    Por hipótesis podemos expresar \(g_1(x) = r(x)g_2(x)\) para algún polinomio \(r(x)\).
    Así, todo elemento \(a(x)\) de \(\mathcal C_1\) puede expresarse como \(g_1(x)f_1(x) = r(x)g_2(x)f_1(x) = g_2(x)f_2(x)\) para algún \(f_2(x)\), por lo que si \(a(x) \in \mathcal C_1\), \(a(x) \in \mathcal C_2\).
    Por tanto, \(\mathcal C_1 \subseteq \mathcal C_2\).
    \item Vemos a continuación que si \(\mathcal C_1 \subseteq \mathcal C_2\) entonces \(g_2(x) | g_1(x)\).
    Vamos a usar un argumento similar al anterior.
    Como \(\mathcal C_1 \subseteq \mathcal C_2\) todo elemento de \(\mathcal C_1\) puede expresarse como \(g_1(x)f_1(x) = g_2(x)f_2(x)\) para ciertos \(f_1(x)\), \(f_2(x)\).
    Por tanto, \(g_1(x) = g_2(x)f_2(x)/f_1(x)\) y en consecuencia, \(g_2 | g_1(x)\), como queríamos.\qedhere
  \end{enumerate}
\end{proof}

% MAYBE: resultados sobre duales de códigos cíclicos (4.2.6, 4.2.7)

\section{Codificación de códigos cíclicos}

%Los códigos cíclicos son más sencillos de decodificar que otros tipos de códigos debido a su estructura adicional.
Vamos a ver a continuación tres tipos de codificación de códigos cíclicos.
Consideraremos un código cíclico \(\mathcal C\) de longitud \(n\) sobre \(\mathbb F_q\) con polinomio generador \(g(x)\) de grado \(n - k\), por lo que \(\mathcal C\) tiene dimensión \(k\).

\paragraph{Codificación no-sistemática}

Esta forma de codificación está basada en la técnica natural de codificación que describimos en la sección \ref{subsec:codificacion-descodificacion}.
Sea \(G\) la matriz generadora obtenida a partir de los desplazamientos de \(g(x)\) descrita en el teorema \ref{th:corr-cod-div}.
Dado el mensaje \(\mathbf m \in \mathbb F_q^k\), lo codificamos como la palabra código \(\mathbf c = \mathbf mG\).
De igual forma, si \(m(x)\) y \(c(x)\) son los polinomios en \(\mathbb F_q[x]\) asociados a \(\mathbf{m}\) y \(\mathbf c\), entonces \(c(x) = m(x)g(x)\).

\paragraph{Codificación sistemática}

El polinomio \(m(x)\) asociado a un mensaje \(\mathbf m\) tendrá como mucho grado \(k -1\).
Por tanto, el polinomio \(n^{n-k}m(x)\) tendrá como mucho grado \(n - 1\) y sus primeros \( n - k\) coeficientes son nulos.
Por tanto, el mensaje está contenido en los coeficientes de \(x^{n-k}, x^{n-k+1}, \dots, x^{n-1}\).
Por el algoritmo de división tenemos que
\[x^{n-k}m(x) = g(x)a(x) + r(x), \qquad \text{donde } \operatorname{gr} r(x) < n - k \text{ o } r(x) = 0.\]
Sea \(c(x) = x^{n-k}m(x) - r(x)\).
Como \(c(x)\) es múltiplo de \(g(x)\), \(c(x) \in \mathcal C\).
El polinomio \(c(x)\) difiere de \(x^{n-k}m(x)\) en los coeficientes de \(1, x, \dots, x^{n-k-1}\) ya que \(\operatorname{gr} r(x) < n-k\).
Por tanto, \(c(x)\) contiene el mensaje \(\mathbf m\) en los coeficientes de los términos de grado al menos \(n - k\).

%\paragraph{Codificación sistemática usando el código dual}

\begin{example}
  Sea \(\mathcal C\) un código cíclico de longitud 15 con polinomio generador \(g(x) = (1 + x + x^4)(1 + x + x^2 + x^3 + x^4)\). Supongamos que queremos codificar el mensaje \(m(x) = 1 + x^2 + x^5\). Vamos a ver su codificación con los dos métodos descritos. Como la longitud de \(\mathcal C\) es \(15\) y el grado de su polinomio generador es \(8\), la dimensión del código es \(15 - 8 = 7\). Escribimos el mensaje \(m(x)\) en forma de vector: \(\mathbf{m}= (1, 0, 1, 0, 0, 1, 0)\). Una matriz generadora del código \(\mathcal C\) es: 
  \[
    G = \left(\begin{array}{rrrrrrrrrrrrrrr}
      1 & 0 & 0 & 0 & 1 & 0 & 1 & 1 & 1 & 0 & 0 & 0 & 0 & 0 & 0 \\
      0 & 1 & 0 & 0 & 0 & 1 & 0 & 1 & 1 & 1 & 0 & 0 & 0 & 0 & 0 \\
      0 & 0 & 1 & 0 & 0 & 0 & 1 & 0 & 1 & 1 & 1 & 0 & 0 & 0 & 0 \\
      0 & 0 & 0 & 1 & 0 & 0 & 0 & 1 & 0 & 1 & 1 & 1 & 0 & 0 & 0 \\
      0 & 0 & 0 & 0 & 1 & 0 & 0 & 0 & 1 & 0 & 1 & 1 & 1 & 0 & 0 \\
      0 & 0 & 0 & 0 & 0 & 1 & 0 & 0 & 0 & 1 & 0 & 1 & 1 & 1 & 0 \\
      0 & 0 & 0 & 0 & 0 & 0 & 1 & 0 & 0 & 0 & 1 & 0 & 1 & 1 & 1
      \end{array}\right).
  \]
  \begin{enumerate}
    \item Codificación no-sistemática. Simplemente multiplicamos \(\mathbf{m}\) por \(G\), obteniendo:
    \[
      \mathbf{c} = \mathbf{m}G = \left(1,\,0,\,1,\,0,\,1,\,1,\,0,\,1,\,0,\,0,\,1,\,1,\,1,\,1,\,0\right).
    \]
    \item Codificación sistemática. Calculamos el cociente de \(x^{n-k}m(x)\) por \(g(x)\) para obtener el resto \(r(x) = x^{6} + x + 1\). Entonces, la palabra código viene dada por \(c(x) = x^{n-k}m(x) - r(x) = x^{13} + x^{10} + x^{8} + x^{6} + x + 1\), que en forma de vector resulta \(\mathbf{c} = \left(1,\,1,\,0,\,0,\,0,\,0,\,1,\,0,\,1,\,0,\,1,\,0,\,0,\,1,\,0\right)\). Observamos que la codificación es efectivamente sistemática: nuestro mensaje \(m\) está contenido íntegramente en las últimas \(7\) coordenadas.
  \end{enumerate}
\end{example}

\section{Idempotentes y multiplicadores}

En esta sección vamos a estudiar otra forma alternativa de generar los códigos cíclicos en \(\mathcal R_n\).
Se basará en encontrar unos elementos concretos de \(\mathcal R_n\) que además podremos relacionar con los polinomios generadores que hemos definido hasta ahora.

Como ya vimos en la sección (cita) un elemento \(e\) de un anillo es idempotente si \(e^2 = e\).
Partiendo de la suposición de que \(\operatorname{mcd}(n, q) = 1\) afirmamos que el anillo \(\mathcal R_n\) es semisimple.
Esto implica, además de lo que ya comentamos, que cada ideal de \(\mathcal R_n\) tiene un único elemento idempotente que lo genera.
Este elemento se denomina \textit{idempotente generador} del código cíclico.
En el siguiente teorema probaremos este hecho y mostraremos además un método para determinar el idempotente generador de un código cíclico a partir de su polinomio generador.

\begin{theorem}
  \label{th:idempotente-unico-unidad}
  Sea \(\mathcal C\) un código cíclico en \(\mathcal R_n\). Entonces:
  \begin{enumerate}
    \item Existe un único idempotente \(e(x) \in \mathcal C\) tal que \(\mathcal C = \langle e(x)\rangle\).
    \item \label{th:idempotente-unico-unidad:unidad} Si \(e(x)\) es un idempotente no nulo en \(\mathcal C\), entonces \(\mathcal C = \langle e(x)\rangle\) si y solo si \(e(x)\) es una unidad de \(\mathcal C\).
  \end{enumerate}
\end{theorem}

\begin{proof}
  Si \(\mathcal C\) es el código cero, entonces el idempotente es el cero, con lo que (1) está claro y (2) no se aplica a este caso. 
  Veamos entonces la demostración por apartados suponiendo que \(\mathcal C\) es distinto de cero.
  \begin{enumerate}
    \item Supongamos primero que \(e(x)\) es una unidad en \(\mathcal C\). 
    Entonces, \(\langle e(x)\rangle \subset \mathcal C\), ya que \(\mathcal C\) es un ideal.
    Si \(c(x) \in \mathcal C\), entonces \(c(x)e(x) = c(x)\) en \(\mathcal C\).
    En consecuencia, \(\langle e(x)\rangle = \mathcal C\).
    Por otro lado, supongamos que \(e(x)\) es un idempotente distinto de cero y tal que \(\mathcal C = \langle e(x)\rangle\).
    Entonces, cada elemento \(c(x)\) lo podemos escribir como \(c(x) = f(x)e(x)\).
    Pero se tiene que \(c(x)e(x) = f(x)(e(x))^2 = f(x)e(x) = c(x)\), luego \(e(x)\) es la unidad de \(\mathcal C\).
    \item Tenemos que probar la existencia y la unicidad.
    Comenzamos con la existencia.
    Sea \(g(x)\) el polinomio generador dde \(\mathcal C\).
    Entonces, sabemos que \(g(x) \mid (x^n - 1)\) por el teorema \ref{th:corr-cod-div}.
    Tomemos \(h(x) = (x^n - 1)/g(x)\).
    Sabemos que \(\operatorname{mcd}(g(x), h(x)) = 1\) en \(\mathbb F_q[x]\), ya que \(x^n - 1\) tiene todas sus raíces distintas.
    En consecuencia, el algoritmo de Euclides nos proporciona los polinomios \(a(x),b(x) \in \mathbb F_q[x]\) tales que \(a(x)g(x) + b(x)h(x) = 1\).
    Llamemos \(e(x) \equiv a(x)g(x) \bmod x^n - 1\), que será el representante de dicha clase de equivalencia en \(\mathcal R_n\).
    Entonces, en \(\mathcal R_n\),
    \begin{align*}
      e(x)^2 &\equiv (a(x)g(x))(1 - b(x)h(x)) \bmod x^n - 1\\
        &\equiv a(x)g(x) - a(x)g(x)b(x)h(x) \bmod x^n - 1\\
        &\equiv a(x)g(x) - a(x)b(x)(x^n - 1) \bmod x^n - 1\\
        &\equiv a(x)g(x) \bmod x^n - 1\\
        &\equiv e(x) \bmod x^n - 1.
    \end{align*}
    Por tanto, este elemento \(e(x)\) es idempotente.
    Veamos ahora que si \(c(x) \in \mathcal C\), entonces \(c(x) = f(x)g(x)\), luego
    \begin{align*}
      c(x)e(x) &= f(x)g(x)(1 - b(x)h(x))\\
        &\equiv f(x)g(x) \bmod x^n - 1\\
        &\equiv c(x) \bmod x^n - 1,
    \end{align*}
    por lo que \(e(x)\) es una unidad en \(\mathcal C\).
    En consecuencia, podemos deducir la existencia a partir de (2).
    Veamos ahora la unicidad. Por (2), si tenemos dos elementos idempotentes \(e_1(x)\) y \(e_2(x)\) que generan \(\mathcal C\), ambos han de ser unidades, y en consecuencia se tiene que \(e_1(x) = e_1(x)e_2(x) = e_2(x)\), con lo que podemos deducir la unicidad.\qedhere
  \end{enumerate}
\end{proof}

Deducimos por tanto que un método para encontrar el idempotente generador \(e(x)\) de un código cíclico \(\mathcal C\) a partir del polinomio generador \(g(x)\) es resolver la ecuación \[1 = a(x)g(x) + b(x)h(x)\] para \(a(x)\) utilizando el algoritmo de Euclides, donde \(h(x) = (x^n - 1)/g(x)\).
Entonces, reduciendo \(a(x)g(x)\) módulo \(x^n - 1\) obtenemos el idempotente \(e(x)\) que buscamos.
Pero vamos a ver además esta relación a la inversa, es decir, que podemos obtener el polinomio generador \(g(x)\) a partir del idempotente \(e(x)\).

\begin{theorem}
  Sea \(\mathcal C\) un código cíclico sobre \(\mathbb F_q\) con idempotente generador \(e(x)\).
  Entonces, el polinomio generador de \(\mathcal C\) es \(g(x) = \operatorname{mcd}(e(x), x^n - 1)\), calculado en \(\mathbb F_q[x]\). 
\end{theorem}

\begin{proof}
  Sea \(d(x) = \operatorname{mcd}(e(x), x^n - 1)\) en \(\mathbb F_q[x]\) y sea \(g(x)\) el polinomio generador de \(\mathcal C\).
  Como \(d(x) \mid e(x)\), podemos expresarlo como \(e(x) = d(x)k(x)\) para algún \(k(x) \in \mathbb F_q[x]\).
  Por tanto cada elemento de \(\mathcal C = \langle e(x) \rangle\) es también múltiplo de \(d(x)\), por lo que \(\mathcal C \subset \langle d(x) \rangle\).
  Por el teorema \ref{th:corr-cod-div} tenemos que en \(\mathbb F_q[x]\), \(g(x) \mid (x^n -1)\) y que \(g(x) \mid e(x)\), ya que \(e(x) \in \mathcal C\).
  Luego, por la proposición \ref{prop:k-divisor-f-g} tenemos que \(g(x) \mid d(x)\) y en consecuencia \(d(x) \in \mathcal C\).
  Por tanto, \(\langle d(x) \rangle \subseteq \mathcal C\) y deducimos entonces que \(\mathcal C = \langle d(x) \rangle\).
  Como \(d(x)\) es divisor mónico de \(x^n - 1\) y genera a \(\mathcal C\), necesariamente \(d(x) = g(x)\) por el corolario \ref{cor:peso-minimo-columnas-dependientes}. 
\end{proof}

\begin{example}
  Continuando con el ejemplo \ref{ex:codigos-ciclicos-long-7} en el que describimos todos los códigos cíclicos binarios de longitud \(7\) vamos a indicar a continuación cuales son los idempotentes generadores de cada uno.
  Para ello vamos a utilizar el código descrito en el anexo \ref{annex:sage-gen-idemp} tal y como mostramos en el listado siguiente.
  \begin{lstlisting}[gobble=4]
    sage: F = GF(2)
    sage: x = polygen(F)
    sage: (x^7 - 1).factor()
    > (x + 1) * (x^3 + x + 1) * (x^3 + x^2 + 1)
    sage: print(generadores_idempotentes(x^7 - 1))
    > [(1, 1),
       (x + 1, x^6 + x^5 + x^4 + x^3 + x^2 + x),
       (x^3 + x + 1, x^4 + x^2 + x),
       (x^3 + x^2 + 1, x^6 + x^5 + x^3),
       (x^4 + x^3 + x^2 + 1, x^6 + x^5 + x^3 + 1),
       (x^4 + x^2 + x + 1, x^4 + x^2 + x + 1),
       (x^6 + x^5 + x^4 + x^3 + x^2 + x + 1, x^6 + x^5 + x^4 + x^3 + x^2 + x + 1),
       (x^7 + 1, 0)]
  \end{lstlisting}
  En la tabla \ref{tab:gen-idempotentes-7} mostramos de forma más clara cuáles son los generadores y cuáles los idempotentes correspondientes.
  \begin{table}[h]
    \centering
    \sffamily
    \begin{tabular}{lcll}
      \toprule
      \(i\) & dimensión & generador \(g_i(x)\) & idempotente \(e_i(x)\)\\
      \midrule
      \(0\) & \(0\) & \(x^7 + 1\) & \(0\)\\
      \(1\) & \(1\) & \(x^6 + x^5 + \dots + x + 1\) & \(x^6 + x^5 + \dots + x + 1\)\\
      \(2\) & \(3\) & \(x^4 + x^3 + x^2 + 1\) & \(x^6 + x^5 + x^3 + 1\)\\
      \(3\) & \(3\) & \(x^4 + x^2 + x +1 \) & \(x^4 + x^2 + x +1 \)\\
      \(4\) & \(4\) & \(x^3 + x + x +1 \) & \(x^4 + x^2 + x\)\\
      \(5\) & \(4\) & \(x^3 + x^2 +1 \) & \(x^6 + x^5 + x^3\)\\
      \(6\) & \(6\) & \(x + 1\) & \(x^6 + x^5 + \dots + x\)\\
      \(7\) & \(7\) & \(1\) & \(1\)\\
      \bottomrule
    \end{tabular}
    \caption{Polinomios generadores e idempotentes para los códigos cíclicos de longitud 7}
    \label{tab:gen-idempotentes-7}
  \end{table}
\end{example}

Puesto que los idempotentes generadores producen códigos cíclicos es de rigor preguntarse si a partir de los idempotentes podemos obtener una base de los códigos generados, tal y como ocurre con los polinomios generadores.
El siguiente teorema nos dice que sí, y además de la misma forma: a partir de los primeros \(k - 1\) desplazamientos cíclicos del idempotente generador.

\begin{theorem}
  Sea \(\mathcal C\) un \([n, k]\) código cíclico con idempotente generador \(e(x) = \sum_{i=0}^{n-1}e_ix^i\).
  Entonces, la matriz \(k \times n\)
  \[
    \begin{pmatrix*}
      e_0 & e_1 & e_2 & \dots & e_{n-2} & e_{n-1} \\
      e_{n-1} & e_0 & e_1 & \dots & e_{n-3} & e_{n-2} \\
       & & & \vdots & & \\
      e_{n-k+1} & e_{n-k+2} & e_{n-k+3} & \dots & e_{n-k-1} & e_{n-k}
    \end{pmatrix*}
  \] es una matriz generadora de \(\mathcal C\).
\end{theorem}

\begin{proof}
  Probar este resultado equivale a probar que el conjunto \(\{e(x), xe(x), \dots, x^{k-1}e(x)\}\) es una base de \(\mathcal C\).
  Entonces, solo hay que probar que si \(a(x) \in \mathbb F_q[x]\) tiene grado menor que \(k\), tal que \(a(x)e(x) = 0\), se tiene que \(a(x) = 0\).
  Sea \(g(x)\) el polinomio generador de \(\mathcal C\).
  Si \(a(x)e(x) = 0\), entonces \(0 = a(x)e(x)g(x) = a(x)g(x)\), tal que \(e(x)\) es la unidad de \(\mathcal C\) según el teorema \ref{th:idempotente-unico-unidad}, y por tanto, si \(a(x)\) no es cero estaríamos contradiciendo el teorema \ref{th:corr-cod-div}.
\end{proof}

El siguiente resultado, que nos informa sobre los polinomios generadores e idempotentes generadores de sumas e intersecciones de códigos cíclicos de la misma longitud, nos será útil un poco más adelante, pues veremos que .
Dados dos códigos cíclicos \(\mathcal C_1\) y \(\mathcal C_2\) de longitud \(n\) sobre \(\mathbb F_q\) definimos su suma como
\[
  \mathcal C_1 + \mathcal C_2 = \{\mathbf{c}_1 + c_2(x) : \mathbf{c}_1 \in \mathcal C_1 \text{ y } \mathbf{c}_2\}.
\]

\begin{theorem}
  \label{th:intersecciones-sumas-ciclicos}
  Sean \(\mathcal C_1\) y \(\mathcal C_2\) códigos cíclicos de longitud \(n\) sobre \(\mathbb F_q\) con polinomios generadores \(g_1(x)\) y \(g_2(x)\) e idempotentes generadores \(e_1(x)\) y \(e_2(x)\), respectivamente.
  Entonces \begin{enumerate}
    \item La intersección \(\mathcal C_1 \cap \mathcal C_2\) es también un código cíclico, con polinomio generador \(\operatorname{mcm}(g_1(x), g_2(x))\) e idempotente generador \(e_1(x)e_2(x)\).
    \item La suma \(\mathcal C_1 + \mathcal C_2\) es también un código cíclico, con polinomio generador \(\operatorname{mcd}(g_1(x), g_2(x))\) e idempotente generador \(e_1(x) + e_2(x) - e_1(x)e_2(x)\).
    \label{th:intersecciones-sumas-ciclicos:sumas}
  \end{enumerate}
\end{theorem}

\begin{proof}
  Veamos la demostración por apartados.
  \begin{enumerate}
    \item La intersección \(\mathcal C_1 \cap \mathcal C_2\) es un subcódigo y por tanto, por el corolario \ref{cor:subcodigos-ciclicos} es un código cíclico.
    Por el mismo corolario su polinomio generador debe ser divisible por \(g_1(x)\) y \(g_2(x)\), por lo que ha de ser divisible por el \(g(x) = \operatorname{mcm}(g_1(x), g_2(x))\).
    Así, \(g(x)\) es un polinomio generador de un código cíclico que está contenido tanto en \(\mathcal C_1\) como en \(\mathcal C_2\).
    Por tanto, \(g(x)\) ha de ser el generador de \(\mathcal C_1 \cap \mathcal C_2\), pues si no lo fuese, el código cíclico generador por \(g(x)\) ha de ser mayor que la intersección, lo que contradice la propia definición de intersección.
    Veamos ahora que el idempotente generador es \(e_1(x)e_2(x)\).
    Claramente \(e_1(x)e_2(x) \in \mathcal C_1 \cap \mathcal C_2\) y es idempotente, pues \((e_1(x)e_2(x))^2 = e_1(x)^2e_2(x)^2 = e_1(x)e_2(x)\).
    Si \(c(x) \in \mathcal C_1 \cap \mathcal C_2\) entonces \(e_1(x)e_2(x)c(x) = e_1(x)c(x) = c(x)\), pues por el teorema \ref{th:idempotente-unico-unidad}(\ref{th:idempotente-unico-unidad:unidad}) \(e_1\) y \(e_2\) son unidades de \(\mathcal C_1\) y \(\mathcal C_2\), respectivamente.
    El mismo teorema nos asegura entonces que \(e_1(x)e_2(x)\) es el idempotente generador que buscamos.
    \item Veamos primero que \(\mathcal C_1 + \mathcal C_2\) es un código cíclico.
    Sabemos que si \(c_1(x) \in \mathcal C_1\) y \(c_2(x) \in \mathcal C_2\) entonces \(xc_1(x) \in \mathcal C_1\) y \(xc_2(x) \in \mathcal C_2\).
    Dado un elemento \(c_1(x) + c_2(x) \in \mathcal C_1 + \mathcal C_2\) tenemos que \(x(c_1(x) + c_2(x)) = xc_1(x) + xc_2(x) \in \mathcal C_1 + \mathcal C_2\), por lo que \(\mathcal C_1 + \mathcal C_2\) es cíclico.
    A continuación, sea \(g(x) = \operatorname{mcd}(g_1(x), g_2(x))\).
    El algoritmo de Euclides nos proporciona \(a(x)\) y \(b(x) \in \mathbb F_q[x]\) tales que \(g(x) = g_1(x)a(x) + g_2(x)b(x)\).
    Por tanto, \(g(x) \in \mathcal C_1 + \mathcal C_2\).
    Como \(\mathcal C_1 + \mathcal C_2\) es cíclico, \(\langle g(x) \rangle \subseteq \mathcal C_1 + \mathcal C_2\).
    Por otro lado \(g(x) | g_1(x)\) luego por el corolario \ref{cor:subcodigos-ciclicos} \(\mathcal C_1 \subseteq \langle g(x) \rangle\).
    De la misma forma deducimos que \(\mathcal C_2 \subseteq \langle g(x) \rangle\) y por tanto \(\mathcal C_1 + \mathcal C_2 \subseteq \langle g(x) \rangle\).
    Así, \(\mathcal C_1 + \mathcal C_2 = \langle g(x) \rangle\).
    Se tiene que \(g(x) | (x^n - 1)\) puesto que \(g(x) | g_1(x)\).
    Además, como \(g(x)\) es mónico, se tiene por el corolario \ref{cor:pol-gen-unico} que \(g(x) = \operatorname{mcd}(g_1(x), g_2(x))\) es el polinomio generador de \(\mathcal C_1 + \mathcal C_2\).
    Veamos finalmente que dado \(c(x) = c_1(x) + c_2(x)\), con \(c_1 \in \mathcal C_1\) y \(c_2 \in \mathcal C_2\) se tiene que 
    \begin{align*}
      c(&x)(e_1(x) + e_2(x) - e_1(x)e_2(x))\\
      &= c_1(x) + c_1(x)e_2(x) - c_1(x)e_2(x) + c_2(x)e_1(x) + c_2(x) - c_2(x)e_1(x)\\
      &=  c_1(x) + c_2(x)\\
      &= c(x).
    \end{align*}
    Por tanto, por el teorema \ref{th:idempotente-unico-unidad} obtenemos que \(e_1(x) + e_2(x) - e_1(x)e_2(x) \in \mathcal C_1 + \mathcal C_2\) es el idempotente generador, como queríamos demostrar.\qedhere
  \end{enumerate}
\end{proof}

Estamos ya en disposición de describir los elementos que prometimos al comienzo de la sección.
Nos permitirán obtener todos los idempotentes en \(\mathcal R_n\), y en consecuencia, todos los códigos cíclicos en \(\mathcal R_n\).
Son los conocidos como \emph{idempotentes primitivos}.

Consideremos la descomposición en factores \(x^n - 1 = f_1(x)\cdots f_s(x)\), donde cada polinomio \(f_i(x)\) es irreducible sobre \(\mathbb F_q\) para \(1 \leq i \leq s\).
Sabemos que los factores \(f_i(x)\) son distintos, pues estamos en el supuesto de que \(x^n - 1\) tiene raíces distintas.
Sea \(\widehat{f_i}(x) = (x^n - 1)/f_i(x)\).
En el teorema \ref{th:idempotentes-ideales-minimales} a continuación vamos a ver que los ideales \(\langle \widehat{f_i}(x)\rangle\) de \(\mathcal R_n\) son los ideales minimales de \(\mathcal R_n\) y cómo podemos obtener \(\mathcal R_n\) a partir de ellos.
Al idempotente generador de \(\langle \widehat{f_i}(x)\rangle\) lo denotaremos por \(\widehat{e_i}(x)\).
Los elementos idempotentes \(\widehat{e_1}(x), \dots, \widehat{e_s}(x)\) son los \emph{idempotentes primitivos} de \(\mathcal R_n\).
El teorema \ref{th:idempotentes-ideales-minimales} que sigue nos muestra además la forma de obtener todos los idempotentes de \(\mathcal R_n\) a partir de los idempotentes primitivos.

\begin{theorem}
  \label{th:idempotentes-ideales-minimales}
  En \(\mathcal R_n\) se verifican las siguientes afirmaciones.
  \begin{enumerate}
    \item Los ideales \(\langle \widehat{f_i}(x)\rangle\) para cada \(1 \leq i \leq s\) son todos los ideales minimales de \(\mathcal R_n\).
    \item \(\mathcal R_n\) es el espacio vectorial suma directa de todos los \(\langle \widehat{f_i}(x)\rangle\) para \(1 \leq i \leq s\).
    \label{thi:idempotentes-ideales-minimales:suma-directa}
    \item Si \(i \neq j\) entonces \(\widehat{e_i}(x)\widehat{e_j}(x) = 0\) en \(\mathcal R_n\).
    \item \label{thi:idempotentes-ideales-minimales:cero}
    \item \label{thi:idempotentes-ideales-minimales:suma-idempotentes} La suma \(\sum_{i=1}^s \widehat{e_i}(x) = 1\) en \(\mathcal R_n\).
    \item \label{thi:idempotentes-ideales-minimales:unicos-idempotentes} Los únicos idempotentes en \(\langle \widehat{f_i}(x)\rangle\) son \(0\) y \(\widehat{e_i}(x)\).
    \item Si \(e(x)\) es un idempotente no nulo en \(\mathcal R_n\), entonces existe un subconjunto \(T\) de \(\{1, 2, \dots, s\}\) tal que \(e(x) = \sum_{i \in T}\widehat{e_i}(x)\) y \(\langle e(x) \rangle = \sum_{i \in T}\langle \widehat{f_i}(x)\rangle\).
  \end{enumerate}
\end{theorem}

\begin{proof}
  Veamos la demostración por apartados.
  \begin{enumerate}
    \item Veamos por reducción al absurdo que cada \(\langle \widehat{f_i}(x)\rangle\) es un ideal minimal de \(\mathcal R_n\).
    Supongamos que no es un ideal minimal.
    Entonces, por el corolario \ref{cor:subcodigos-ciclicos} existiría un polinomio generador \(g(x)\) de un ideal no trivial contenido en \(\langle \widehat{f_i}(x)\rangle\) tal que \(\widehat{f_i}(x) | g(x)\), con \(g(x) \neq \widehat{f_i}(x)\).
    Pero como \(f_i(x)\) es irreducible y \(g(x) | (x^n - 1)\), es imposible.
    Por tanto cada \(\langle \widehat{f_i}(x)\rangle\) es un ideal minimal de \(\mathcal R_n\).
    Veamos que estos son todos los ideales minimales de \(\mathcal R_n\).
    Sea \(\mathcal M = \langle m(x) \rangle\) un ideal minimal de \(\mathcal R_n\).
    Como el conjunto \(\{\widehat{f_i}(x) : 1 \leq i \leq s\}\) no tiene factores irreducibles de \(x^n - 1\) repetidos y cada uno de ellos divide a \(x^n - 1\) el \(\operatorname{mcd}(\widehat{f_1}(x), \dots, \widehat{f_s}(x)) = 1\).
    Por tanto, aplicando el algoritmo de Euclides inductivamente obtenemos
    \begin{equation}
      1 = \sum_{i = 1}^s a_i(x)\widehat{f_i}(x)
      \label{eq:1-suma-ideales-minimales}
    \end{equation}
    para ciertos \(a_i(x) \in \mathbb F_q[x]\).
    Así, como 
    \[
      0 \neq m(x) = m(x) \cdot 1 = \sum_{i = 1}^s m(x)a_i(x)\widehat{f_i}(x)
    \]
    existe un \(i\) tal que \(m(x)a_i(x)\widehat{f_i}(x) \neq 0\).
    Por tanto, \(\mathcal M \cap \langle \widehat{f_i}(x) \rangle \neq \{0\}\), pues \(m(x)a_i(x)\widehat{f_i}(x) \in \mathcal M \cap \langle \widehat{f_i}(x) \rangle\).
    Pero entonces \(\mathcal M = \langle \widehat{f_i}(x) \rangle\) pues tanto \(\mathcal M\) como \(\langle \widehat{f_i}(x) \rangle\) son minimales.
    Por tanto todos los ideales minimales son de la forma \(\langle \widehat{f_i}(x) \rangle\), como queríamos.
    \item Por (\ref{eq:1-suma-ideales-minimales}) concluimos que el \(1\) está en la suma de los ideales \(\langle \widehat{f_i}(x) \rangle\), que es en sí mismo un ideal de \(\mathcal R_n\).
    Por tanto, por la proposición \ref{prop:ideal-unidad}, \(\mathcal R_n\) es la suma de los ideales \(\langle \widehat{f_i}(x) \rangle\).
    Para probar que es una suma directa tenemos que comprobar que los ideales son disjuntos, es decir, \(\langle \widehat{f_i}(x) \rangle \cap \sum_{j\neq i} \langle \widehat{f_j}(x) \rangle = \{0\}\) para \(1 \leq i \leq s\).
    Como \(f_i(x) | \widehat{f_j}(x)\) para \(j \neq i\), \(f_j(x) \not| \widehat{f_j}(x)\) y los factores irreducibles de \(x^n - 1\) son todos distintos, concluimos que
    \[
      f_i(x) = \operatorname{mcd}\{\widehat{f_j}(x) : 1 \leq j \leq s, j \neq i\}.
    \]
    Utilizando inducción sobre el teorema \ref{th:intersecciones-sumas-ciclicos}(\ref{th:intersecciones-sumas-ciclicos:sumas}) concluimos que \(\langle \widehat{f_i}(x) \rangle = \sum_{j\neq i}\langle \widehat{f_j}(x) \rangle\).
    Por tanto, 
    \begin{align*}
      \langle \widehat{f_i}(x) \rangle \cap \sum_{j\neq i}\langle \widehat{f_j}(x) \rangle 
       &= \langle \widehat{f_i}(x) \rangle \cap \langle f_i(x) \rangle \\
       &= \langle\operatorname{mcm}(\widehat{f_i}(x), f_i(x))\rangle \\ 
       &= \langle x^n - 1\rangle \\
       &= \{0\},
    \end{align*}
    por lo que los \(\langle \widehat{f_i}(x) \rangle\) son disjuntos y la suma es directa, como queríamos ver.
    \item Si \(i \neq j\), \(\widehat{e_i}(x)\widehat{e_j}(x) \in \langle \widehat{f_i}(x) \rangle \cap \langle \widehat{f_j}(x) \rangle = \{0\}\) por (\ref{thi:idempotentes-ideales-minimales:suma-directa}), luego \(\widehat{e_i}(x)\widehat{e_j}(x) = 0\) como queríamos.
    \item Usando (\ref{thi:idempotentes-ideales-minimales:cero}) y aplicando inducción al teorema \ref{th:intersecciones-sumas-ciclicos}(\ref{th:intersecciones-sumas-ciclicos:sumas}) obtenemos que \(\sum_{i=1}^s \widehat{e_i}(x)\) es el idempotente generador de \(\sum_{i=1}^s \langle \widehat{f_i}(x) \rangle = \mathcal R_n\), por (\ref{thi:idempotentes-ideales-minimales:suma-directa}).
    Luego el idempotente generador de \(\mathcal R_n\) es \(1\).
    \item Si \(e(x)\) es un idempotente no nulo en \(\langle \widehat{f_i}(x) \rangle\) entonces \(\langle e(x) \rangle\) es un ideal contenido en \(\langle \widehat{f_i}(x) \rangle\).
    Por minimalidad, dado que \(e(x)\) es distinto de cero, \(\langle \widehat{f_i}(x) \rangle = \langle e(x)\rangle\), lo que por el teorema \ref{th:idempotente-unico-unidad} implica que \(e(x) = \widehat{e_i}(x)\) ya que ambos son la unidad de \(\langle \widehat{f_i}(x) \rangle\).
    \item Notemos que \(e(x)\widehat{e_i}(x)\) es un idempotente en  \(\langle \widehat{f_i}(x) \rangle\).
    Por tanto, por (\ref{thi:idempotentes-ideales-minimales:unicos-idempotentes}), \(e(x)\widehat{e_i}(x)\) es, o bien \(0\) o bien \(\widehat{e_i}(x)\).
    Sea \(T = \{i : e(x)\widehat{e_i}(x) \neq 0\}\).
    Entonces, por (\ref{thi:idempotentes-ideales-minimales:suma-idempotentes}), \(e(x) = e(x) \cdot 1 = e(x)\sum_{i=1}^s \widehat{e_i}(x) = \sum_{i=1}^s e(x)\widehat{e_i}(x) = \sum_{i \in T}\widehat{e_i}(x)\).
    De hecho, \(\langle e(x)\rangle = \langle \sum_{i \in T}\widehat{e_i}(x)\) = \(\sum_{i \in T}\langle\widehat{e_i}(x)\rangle\) aplicando por inducción el teorema \ref{th:intersecciones-sumas-ciclicos}(\ref{th:intersecciones-sumas-ciclicos:sumas}).\qedhere
  \end{enumerate}
\end{proof}

El siguiente teorema nos muestra que los ideales minimales descritos en el teorema \ref{th:idempotentes-ideales-minimales} son extensiones de cuerpos de \(\mathbb F_q\).

\begin{theorem}
  Sea \(\mathcal M\) un ideal minimal de \(\mathcal R_n\).
  Entonces \(\mathcal M\) es una extensión de cuerpos de \(\mathbb F_q\).
\end{theorem}

\begin{proof}
  Basta con probar que cada elemento distinto de cero en \(\mathcal M\) tiene inverso para el producto.
  Sea \(a(x) \in \mathcal M\) distinto de cero.
  Entonces \(\langle a(x) \rangle\) es un ideal de \(\mathcal R_n\) distinto de cero contenido en \(\mathcal M\), y por tanto, \(\langle a(x) \rangle = \mathcal M\).
  Por tanto, si \(e(x)\) es la unidad de \(\mathcal M\) existe un elemento \(b(x) \in \mathcal R_n\) tal que \(a(x)b(x) = e(x)\).
  Sea ahora \(c(x) = b(x)e(x) \in \mathcal M\), pues \(e(x) \in \mathcal M\).
  Por tanto, \(a(x)c(x) = e(x)^2 = e(x)\), con lo que \(a(x)\) tiene inverso, como queríamos.
\end{proof}

A continuación vamos a describir una permutación que lleva idempotentes de \(\mathcal R_n\) en idempotentes de \(\mathcal R_n\).
Sra \(a\) un entero tal que \(\operatorname{mcd}(a, n) = 1\).
La función \(\mu_a\) definida sobre \(\{0, 1, \dots, n -1\}\) por \(i\mu_a \equiv ia \bmod n\) es una permutación de las posiciones de coordenadas \(\{0, 1, \dots, n - 1\}\) de un código cíclico de longitud \(n\) y se denomina \textit{multiplicador}.
Dado que los códigos cíclicos de longitud \(n\) se representan como ideales de \(\mathcal R_n\), para \(a > 0\) es conveniente interpretar que \(\mu_a\) actúa sobre \(\mathcal R_n\) como
\begin{equation}
  \label{eq:multiplier-rn}
  f(x)\mu_a \equiv f(x^a) \bmod x^n - 1.
\end{equation}

Esta ecuación es consistente con la definición original de \(\mu_a\) pues \(x^i\mu_a = x^{ia} = x^{ia + jn}\) en \(\mathcal R_n\) para un entero \(j\) tal que \(0 \leq ia + jn\), pues \(x^n = 1\) en \(\mathcal R_n\).
En otras palabras, \(x^i\mu_a = x^{ia \bmod n}\).
Si \(a < 0\) podemos dar significado a \(f(x^a)\) en \(\mathcal R_n\) definiendo \(x^{i}\mu_a = x^{ia \bmod n}\), donde \(0 \leq ia \bmod n < n\).
Con esta interpretación la ecuación (\ref{eq:multiplier-rn}) es consistente con la definición original de \(\mu_a\) cuando \(a < 0\).

% MAYBE: Más sobre multiplicadores (4.3)

\section{Ceros y conjuntos característicos}

En esta sección vamos a ver que podemos caracterizar los códigos cíclicos \(\mathcal R_n\) de otra forma: a partir de los ceros del polinomio \(x^n - 1\), es decir, a partir de ciertas raíces enésimas de la unidad.
Esta caracterización cobrará especial relevancia cuando estudiemos los conocidos como códigos \textacr{BCH}.

Como vimos en la sección \ref{sec:factorizacion-xn-1}, si \(t = \operatorname{ord}_n(q)\) entonces \(\mathbb F_{q^t}\) es un cuerpo de descomposición de \(x^n - 1\).
Por tanto, \(\mathbb F_{q^t}\) contiene una enésima raíz primitiva de la unidad \(\alpha\), y \(x^n - 1 = \prod_{i=0}^{n-1}(x - \alpha^{i})\) es la factorización de \(x^n - 1\) sobre \(\mathbb F_{q^t}\).
De hecho, \(x^n - 1 = \prod_s M_{\alpha^s}(x)\) es la factorización de \(x^n - 1\) en factores irreducibles sobre \(\mathbb F_q\), donde \(s\) varía en un conjunto de representantes de las clases \(q\)-ciclotómicas módulo \(n\).

Sea \(\mathcal C\) un código cíclico en \(\mathcal R_n\) con polinomio generador \(g(x)\).
Por los teoremas \ref{th:pol-minimal-raiz-primitiva}(\ref{thi:pol-minimal-raiz-primitiva-producto}) y \ref{th:corr-cod-div}(\ref{thi:pol-generador-prod-minimal}), podemos expresar el polinomio generador como \(g(x) = \prod_{s}M_{\alpha^s}(x) = \prod_s\prod_{i \in C_s}(x - \alpha^{i})\), donde \(s\) de nuevo varía en un conjunto \(C_s\) de representantes de las clases \(q\)-ciclotómicas módulo \(n\).
Sea \(T = \bigcup_s C_s\) la unión de estas clases \(q\)-ciclotómicas.
Las raíces de la unidad \(\mathcal Z = \{\alpha^{i} \mid i \in T\}\) se denominan los \textit{ceros} del código cíclico \(\mathcal C\) y los elementos \(\{\alpha^{i} \mid i \notin T\}\), los \textit{elementos no nulos} de \(\mathcal C\).
El conjunto \(T\) se denomina \textit{conjunto característico} de \(\mathcal C\).

\begin{theorem}
  \label{th:cicl-cto-caracteristico}
  Sea \(\alpha\) una raíz primitiva de la unidad en una extensión de cuerpos de \(\mathbb F_q\) y sea \(\mathcal C\) un código cíclico de longitud \(n\) sobre \(\mathbb F_q\) con conjunto característico \(T\) y polinomio generador \(g(x)\).
  Se verifica que: \begin{enumerate}
    %\item El polinomio generador se puede expresar como \(g(x) = \prod_{i \in T}(x - \alpha^i)\).
    \item Una palabra código \(c(x) \in \mathcal R_n\) está en \(\mathcal C\) si y solo si \(c(\alpha^i) = 0\) para todo \(i \in T\).
    \item La dimensión de \(\mathcal C\) es \(n - |T|\).
  \end{enumerate}
\end{theorem}

\begin{proof}
  Veamos la demostración por apartados.
  \begin{enumerate}
    \item Se deduce directamente del teorema \ref{th:corr-cod-div}, pues \(c(x)\) será un múltiplo del polinomio generador \(g(x)\) de \(\mathcal C\), que por \ref{th:corr-cod-div}(\ref{thi:pol-generador-prod-minimal}) verifica que \(g(\alpha^i) = 0\) para todo \(i \in T\).
    \item Se de deduce del teorema \ref{th:corr-cod-div}, pues \(|T|\) es el grado de \(g(x)\).\qedhere
  \end{enumerate}
\end{proof}

Es importante observar que \(T\), y por ello tanto el conjunto de ceros como el de elementos distintos de cero, determinan por completo el polinomio generador \(g(x)\).

\begin{example}
  Continuando con el ejemplo \ref{ex:codigos-ciclicos-long-7} en el que describimos todos los códigos cíclicos binarios de longitud \(7\) vamos a indicar a continuación cuales son los conjuntos característicos, tomando como \(\alpha = \zeta_7^3\).
  Para ello vamos a utilizar el código descrito en el anexo \ref{annex:sage-gen-idemp} tal y como mostramos en el listado siguiente.
  \begin{lstlisting}[gobble=4]
    sage: F = GF(2)
    sage: x = polygen(F)
    sage: ctos_caracteristicos(x^7 - 1)
    > [(1, [], z3),
       (x + 1, [0], z3),
       (x^3 + x + 1, [1, 2, 4], z3),
       (x^3 + x^2 + 1, [3, 5, 6], z3),
       (x^4 + x^3 + x^2 + 1, [0, 1, 2, 4], z3),
       (x^4 + x^2 + x + 1, [0, 3, 5, 6], z3),
       (x^6 + x^5 + x^4 + x^3 + x^2 + x + 1, [1, 2, 3, 4, 5, 6], z3),
       (x^7 + 1, [0, 1, 2, 3, 4, 5, 6], z3)]
  \end{lstlisting}
  En la tabla siguiente mostramos de forma más clara cuáles son los generadores e idempotentes correspondientes para cada código.
  \begin{table}[h]
    \centering
    \sffamily
    \begin{tabular}{ccll}
      \toprule
      \(i\) & dimensión & generador \(g_i(x)\) & conjunto \(T\)\\
      \midrule
      \(0\) & \(0\) & \(x^7 + 1\) & \(\{0, 1, 2, 3, 4, 5, 6\}\)\\
      \(1\) & \(1\) & \(x^6 + x^5 + \dots + x + 1\) & \(\{1, 2, 3, 4, 5, 6\}\)\\
      \(2\) & \(3\) & \(x^4 + x^3 + x^2 + 1\) & \(\{0, 3, 5, 6\}\)\\
      \(3\) & \(3\) & \(x^4 + x^2 + x + 1\) & \(\{0, 1, 2, 4\}\)\\
      \(4\) & \(4\) & \(x^3 + x + 1\) & \(\{3, 5, 6\}\)\\
      \(5\) & \(4\) & \(x^3 + x^2 + 1\) & \(\{1, 2, 4\}\)\\
      \(6\) & \(6\) & \(x + 1\) & \(\{0\}\)\\
      \(7\) & \(7\) & \(1\) & \(\emptyset\)\\
      \bottomrule
    \end{tabular}
    \caption{Polinomios generadores y conjuntos característicos para los códigos cíclicos de longitud 7}
  \end{table}
\end{example}
