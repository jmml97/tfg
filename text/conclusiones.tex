\pdfbookmark[1]{Conclusiones}{Conclusiones}
\addcontentsline{toc}{chapter}{\tocEntry{Conclusiones}}
\chapter*{Conclusiones}

Como comentamos en la introducción, este trabajo tenía como objetivo el estudio de la teoría de códigos lineales, de las extensiones de Ore y del algoritmo de Peterson-Gorenstein-Zierler, así como su implementación en SageMath.

Podemos afirmar con satisfacción que estos objetivos se han cumplido, pues a lo largo de los siete capítulos de los que consta este trabajo hemos estudiado todo lo que nos habíamos propuesto, además de haber repasado las nociones básicas de Álgebra necesarias para su correcta comprensión.
Adicionalmente, se ha realizado una implementación en SageMath de los códigos \textacr{RS} sesgados y del mencionado algoritmo.

Como posible vía futura del trabajo podría plantearse contribuir al proyecto SageMath las clases realizadas, así como completar la implementación de los códigos cíclicos sesgados en general.