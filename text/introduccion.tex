\chapter*{Introducción}

El padre de la teoría de la información Claude Shannon. El libro \textit{Mathematical Theory of Cryptography} (1945) y su ampliación posterior, \textit{Mathematical Theory of Communication} (1948)

Francis Bacon ya afirmó en el año 1623 que únicamente son necesarios dos símbolos para codificar toda la comunicación.

\blockquote[{\cite[30]{dyson_catedral_2015}}]{La transposición de dos letras en cinco emplazamientos bastará para dar 32 diferencias [y] por este arte se abre un camino por el que un hombre puede expresar y señalar las intenciones de su mente, a un lugar situado a cualquier distancia, mediante objetos ... capaces solo de una doble diferencia.}

Bla bla bla