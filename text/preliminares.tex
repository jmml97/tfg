\chapter{Preliminares}

El objetivo de este capítulo es el de presentar todas las herramientas matemáticas que vamos a necesitar para el desarrollo de la teoría de códigos objeto de este trabajo.
Dicha teoría se sustenta fundamentalmente en la teoría de espacios vectoriales, cuerpos finitos, de anillos y de polinomios.
Los resultados presentados son ampliamente conocidos por lo que en general no se aportarán demostraciones de los mismos, que pueden encontrarse con facilidad en libros especializados.
Las fuentes principales de este capítulo son \parencite[cap. 3 y 6]{cohn_algebra_1982}, \parencite[cap. 3]{cohn_algebra_1989} y \parencite[cap. 2]{lidl_introduction_1986}.

% \section{Monoides y grupos}

% \begin{definition}
%   Un \textit{monoide} es un conjunto \(S\) con un elemento \(e\) y una aplicación \(\mu: S^2 \to S\) tal que si \(\mu(x, y)\) es el resultado de aplicar \(\mu\) a la pareja de elementos \(x, y \in S\), se verifican: \begin{enumerate}
%     \item \(\mu(x, \mu(y, z)) = \mu(\mu(x, y), z)\) para todo \(x, y, z \in S\).
%     \item \(\mu(e, x) = \mu(x, e) = x\) para todo \(x \in S\).
%   \end{enumerate}
% \end{definition}

% Obsérvese que, por definición, un monoide siempre tiene al menos un elemento. A la aplicación \(\mu\) que actúa sobre parejas de elementos se le llama \textit{operación binaria} y al elemento \(e\), elemento neutro de \(\mu\).

% Un \textit{grupo} es un monoide en el que todo elemento tiene inverso.
% Un grupo que además verifica la propiedad conmutativa es un \textit{grupo abeliano}.

% \begin{definition}
%   Un \textit{grupo} \(G\) es un conjunto junto a una operación binaria \(xy\) definida sobre él que verifica las siguientes propiedades.
%   \begin{enumerate}
%     \item Propiedad asociativa: \((xy)z = x(yz)\) para todo \(x, y, z \in G\).
%     \item Existencia de elemento neutro \(1\): \(1x = x1 = x\) para todo \(x \in G\).
%     \item Existencia de elemento inverso: cada elemento \(x\) tiene un inverso \(x^{-1}\) tal que \(xx^{-1} = x^{-1}x = 1\). 
%   \end{enumerate}
% \end{definition}

\section{Anillos}

Por anillo entendemos un conjunto \(R\) junto a dos operaciones binarias: \(x + y\), llamada \textit{suma}, y \(xy\), llamada \textit{producto}, tales que:
\begin{enumerate}
  \item \(R\) es un grupo abeliano con la suma.
  \item \(R\) es un monoide con el producto.
  \item La suma y el producto están relacionadas mediante la propiedad distributiva:
  \[
    (x+y)z = xz + yz, \qquad x(y+z) = xy + xz.
  \]
\end{enumerate}
El elemento neutro para la suma se llama \textit{cero} y se escribe \(0\), mientras que el elemento neutro para el producto se llama \textit{uno} o \textit{la unidad} y se escribe \(1\).
El inverso de la suma para \(x\) se denota \(-x\).

A continuación vamos a dar una definición completa de anillo, sin depender de remisiones a las definiciones de grupo y monoide.

\begin{definition}
  Un \textit{anillo} es un conjunto junto a dos operaciones: la suma \((+)\) y la multiplicación \((\cdot)\), que verifican las siguientes propiedades.
  \begin{itemize}[itemsep=0pt]
    \item Propiedad asociativa:
    \[(x+y)+z = x + (y+z), \qquad (xy)z = x(yz).\]
    \item Propiedad conmutativa para la suma:
    \[x + y = y + x.\]
    \item Existencia de elemento neutro:
    \[x + 0 = x, \qquad x1 = x.\]
    \item Existencia de elemento inverso para la suma:
    \[x + (-x) = 0.\]
    \item Propiedad distributiva para la multiplicación sobre la suma:
    \[x(y+z) = xy + xz.\]
  \end{itemize}

  Si un anillo verifica la propiedad conmutativa para la multiplicación, es decir, \(xy = yx\), se dice que es un \textit{anillo conmutativo}.
\end{definition}

Comprobamos que en  cualquier anillo \(R\) se verifica que \(0x = x0 = 0\) para todo \(x \in R\), ya que \(x0 = x(0+0) = x0 + x0\), de donde concluimos que \(x0 = 0\) e igualmente, \(0x = 0\).
Cuando un anillo \(R\) tiene solo un elemento es necesario que \(1 = 0\).
Un anillo de este tipo se denomina \textit{anillo trivial}.
Podemos ver que este es el único caso en el que se da la igualdad \(1 = 0\).
Supongamos que en cualquier otro anillo se da que \(1 = 0\). 
Entonces para cada elemento \(x\) del anillo se tiene que \(x = x1 = x0 = 0\), luego tiene un único elemento.

Dos anillos son \textit{isomorfos} si hay un \textit{isomorfismo} entre ellos, es decir, existe una biyección que preserva todas las operaciones.
% MAYBE: abreviar na = a + a + a (Cohn 137 [147])
Un elemento \(a\) de un anillo se dice que es \textit{invertible} si existe un elemento \(a'\) en el anillo tal que \(aa' = a'a = 1\).
A este elemento, que es único, lo llamamos \textit{elemento inverso} de \(a\) y lo denotamos por \(a^{-1}\).
El elemento \(0\) no puede tener inverso porque ya hemos visto que siempre que se multiplique por él se obtiene el \(0\).
Los anillos en los que todo elemento distinto de 0 es invertible se llaman \textit{anillos de división}.

% Dados dos anillos \(R\) y \(S\), decimos que \(S\) es un \textit{subanillo} de \(R\) si \(S\) está contenido en \(R\) y forma un anillo con las mismas operaciones que \(R\) (y con los mismos elemento neutro e identidad).

% CITEME: definciiones de Niederreiter (13)

\begin{definition}
  Un subconjunto \(S\) de un anillo \(R\) se denomina \textit{subanillo} si \(S\) contiene a los elementos neutros de la suma y el producto de \(R\), es cerrado bajo dichas operaciones y forma un anillo con ellas.
\end{definition}

\begin{example}
  Sea \(R\) el cuerpo \(\mathbb Q\) de los racionales.
  Entonces, el subconjunto \(\mathbb Z\) de los enteros es un subanillo, pues contiene a los elementos neutros de producto y suma —\(1\) y \(0\), respectivamente—, la suma de dos enteros es un entero y el producto de dos enteros es, de nuevo, un entero.
  \label{ex:subanillo-racionales}
\end{example}

\begin{definition}
  Sea \(J\) un subconjunto de un anillo \(R\).
  \begin{itemize}
    \item Se dice que \(J\) es un \emph{ideal por la izquierda} si es un subgrupo aditivo de \(R\) y verifica que para todo \(j \in J\) y para todo \(r \in R\), el elemento \(rj \in J\).
    \item De igual forma, se dice que \(J\) es un \emph{ideal por la derecha} si es un subgrupo aditivo de \(R\) y verifica que para todo \(j \in J\) y para todo \(r \in R\), el elemento \(jr \in J\).
    \item Finalmente, se dice que \(J\) es un \emph{ideal bilátero} si es ideal tanto por la izquierda como por la derecha.
  \end{itemize}
\end{definition}

Notamos que en un anillo conmutativo solo habrá ideales biláteros y por tanto nos referiremos a ellos como ideales a secas.
Veamos a continuación un par de ejemplos.

\begin{example}
  Continuando el ejemplo \ref{ex:subanillo-racionales}, el conjunto \(\mathbb Z\) de los enteros no es un ideal pues, por ejemplo, \(1 \in \mathbb Z, 1/2 \in \mathbb Q\), pero \(1/2 \cdot 1 = 1/2 \notin \mathbb Z\).
\end{example}

\begin{example}
  Sea \(R\) el anillo de los enteros \(\mathbb Z\) y consideremos el subconjunto de los números enteros \(J = \{2n : n \in \mathbb Z\}\).
  Entonces \(J\) es un ideal (bilátero), pues todo producto de un número par es otro número par.
\end{example}

\begin{proposition}
  \label{prop:ideal-unidad}
  Sea \(J\) un ideal de un anillo \(R\). Si \(1 \in J\) entonces \(J = R\).
\end{proposition}

\begin{definition}
  Sea \(R\) un anillo. Un ideal por la izquierda \(J\) de \(R\) se dice que es \textit{principal} si existe un elemento \(a \in R\) tal que \(J = Ra = \{ra : r \in R\}\).
  De forma análoga, un ideal por la derecha \(J\) de \(R\) es \textit{principal} si existe un elemento \(a \in R\) tal que \(J = aR = \{ar : r \in R\}\).
  Un ideal bilátero será principal si verifica cualquiera de las dos condiciones.
\end{definition}

% Cociente de anillos

Dado un ideal bilátero \(I\) de un anillo \(R\) es posible definir una relación de equivalencia, dada por \(a \sim b\) si y solo si \(a - b \in I\).
Esta relación nos permite definir el conjunto \(R/I\), de las clases de equivalencia obtenidas a partir de ella.
Para un \(a \in R\) su clase de equivalencia viene dada por
\[
  [a] = a + I = \{a + x : x \in I\}.  
\]
Pero es más, \(R/I\) también tiene estructura de anillo y se conoce como \emph{anillo cociente} de \(R\) por \(I\), cuyas operaciones —bien definidas— vienen dadas por
\[
  (a + I) + (b + I) = (a + b) + I \quad\text{y}\quad (a + I)(b + I) = ab + I.
\]

A continuación vamos a definir algunos otros conceptos relativos a anillos.
Decimos que un ideal \(J\) es \textit{minimal} si no existe ningún otro ideal entre \(\{0\}\) y \(J\).
Podemos clasificar los elementos de un anillo distintos de cero en dos tipos: \textit{divisores de cero} y \textit{regulares}.
Tomemos un elemento \(a \neq 0\).
Si existe \(b \neq 0\) tal que \(ab\) o \(ba\) es cero, entonces \(a\) es un elemento \textit{divisor de cero}, y en caso contrario, un elemento \textit{regular}.
A partir de esta clasificación podemos hablar de anillos \textit{enteros} —aquellos que no son triviales y no tienen divisores de cero— y de \textit{dominios de integridad} —anillos enteros conmutativos—.

Una propiedad importante de los elementos regulares —y que es tan natural que estamos plenamente acostumbrados a ella— es la llamada ley de cancelación.
\begin{proposition}
  Si \(c\) es un elemento regular de un anillo \(R\) entonces para cada \(a, b \in R\), tales que, o bien \(ca = cb\), o bien \(ac = bc\), se tiene que \(a = b\).
\end{proposition}

Para cerrar esta sección vamos a introducir dos definiciones: una será una propiedad de los anillos, mientras que la otra será una propiedad que verificarán algunos elementos de los anillos.

\begin{definition}
  Sea \(R\) un anillo. La \textit{característica} del anillo es el menor natural \(n\) tal que \(n1 = 0\).
  Si no existe tal número, la característica del anillo es \(0\).  
\end{definition}

\begin{definition}
  Un elemento \(e\) de un anillo tal que \(e^2 = e\) se dice que es \textit{idempotente}.
\end{definition}

\section{Cuerpos finitos}

En esta sección vamos a introducir el concepto de cuerpo, junto a otros conceptos relevantes, para posteriormente centrarnos en los cuerpos finitos.
Estudiaremos además los anillos de polinomios que podemos definir sobre ellos y cómo a partir del cociente de estos podemos construir cuerpos finitos.

\begin{definition}
  Un \textit{cuerpo} es un anillo de división conmutativo.
  Se dice que un cuerpo es \textit{finito}\footnote{Los cuerpos finitos también se suelen conocer como \textit{cuerpos de Galois} en honor a Évariste Galois, uno de los primeros matemáticos en trabajar con ellos.} si tiene un número finito de elementos, al que llamamos \textit{orden} del cuerpo.
\end{definition}

% MAYBE: todo dominio de integridad finito es un cuerpo (Niederreiter, 12 [21])

Sea \(F\) un cuerpo. Un subconjunto \(K\) de \(F\) que es por sí mismo un cuerpo bajo las operaciones de \(F\) se denomina \textit{subcuerpo} de \(F\).
También podemos referirnos a ello al revés, diciendo que \(F\) es una \textit{extensión} de \(K\), lo que notaremos como \(F/K\).
Observamos que \(F\) es un espacio vectorial sobre \(K\), esto es, los elementos de \(F\) pueden ser vistos como vectores sobre el cuerpo de escalares \(K\), con las operaciones de suma \(\alpha + \beta\), para \(\alpha, \beta \in F\) y multiplicación por escalares \(a\alpha\), para \(a \in K\) y \(\alpha \in F\), dadas por las propias operaciones de suma y multiplicación en \(K\).
Todas las nociones que hemos definido para anillos —como la característica— son válidas para los cuerpos, pues un cuerpo no deja de ser un anillo.

Habitualmente notaremos a los cuerpos finitos por \(\mathbb F_p\), donde \(p\) denota el orden del cuerpo.
Trabajaremos habitualmente con cuerpos finitos, pues son los que vamos a utilizar de forma prominente cuando trabajemos con códigos.

\begin{theorem}
  % Cohn Vol 2, 63 [74]
  \label{th:cuerpo-subcuerpo-primo-caracteristica}
  Todo cuerpo \(F\) tiene al menos un subcuerpo \(P\), el subcuerpo primo de \(F\), que está contenido en cada subcuerpo de \(F\).
  O bien \(F\) tiene característica \(0\) y \(P \cong \mathbb Q\), o bien \(F\) tiene característica \(p\), un número primo, y entonces \(P \cong \mathbb F_p\).
\end{theorem}


% CITEME ? Cohn Algebra Vol 2


\begin{lemma}
  Un espacio vectorial \(n\)-dimensional sobre \(\mathbb F_p\) tiene \(p^n\) elementos.
\end{lemma}

\begin{proof}
  Sea \(V\) un espacio vectorial \(n\)-dimensional sobre \(\mathbb F_p\).
  Si los elementos \(u_1, \dots, u_n\) forman una base de \(V\), entonces cada elemento de \(V\) se escribe de forma única en la forma \(\sum \alpha_iu_i\), donde \(\alpha_i \in \mathbb F_p\).
  Como cada coeficiente puede tener hasta \(p\) valores distintos, obtenemos un total de \(p^n\) elementos.
\end{proof}

Concluimos destacando que en virtud del teorema \ref{th:cuerpo-subcuerpo-primo-caracteristica} todo cuerpo finito \(F\) tiene característica \(p\) (un primo) y su subcuerpo primo es \(\mathbb F_p\).

\subsection{Anillos de polinomios sobre cuerpos finitos}

Para cualquier anillo \(R\) podemos definir un anillo de polinomios en \(x\) con coeficientes en \(R\), que denotamos \(R[x]\) y que está compuesto por el conjunto
\[
  \{a_0 + a_1x + \dots + a_nx^n : a_0, a_1, \dots, a_n \in R\}.
\]
Dados \(f(x) = a_0 + a_1x + \dots + a_nx^n\) y \(g(x) = b_0 + b_1x + \dots + b_mx^m\) las operaciones de suma y multiplicación de \(R[x]\) (suponiendo \(m \leq n\)) vienen dadas por:
\[
  f(x) + g(x) = a_0 + b_0 + (a_1 + b_1)x + \dots + (a_m + b_m)x^m + \dots + a_nx^n,
\]
\begin{align*}
  f(x)g(x) = &a_0b_0 + (a_0b_0 + a_1b_0)x + (a_0b_2 + a_1b_1 + a_2b_0)x^2\\
    &+ \dots + a_nb_mx^{m+n}.
\end{align*}
El elemento neutro para la suma es el \(0 \in R\), y el del producto, el \(1 \in R\).
Es fácil comprobar que efectivamente \(R[x]\) así definido satisface todas las propiedades de los anillos.
Veamos a continuación algunos conceptos básicos sobre polinomios.
El \textit{grado} de un polinomio es el mayor grado de cualquier término con coeficiente distinto de cero.
% \begin{proposition}
%   Grado de sumas y productos
%   % MAYBE: (Cohn, 149 [159])
% \end{proposition}
El coeficiente del término de mayor grado se denomina \textit{coeficiente líder}.
Un polinomio es \textit{mónico} si su coeficiente líder es 1.
Sean \(f(x)\) y \(g(x)\) polinomios en \(R[x]\).
Decimos que \(f(x)\) \textit{divide a} \(g(x)\), denotado por \(f(x) | g(x)\), si existe un polinomio \(h(x) \in \mathbb R[x]\) tal que \(g(x) = f(x)h(x)\).
El polinomio \(f(x)\) se llama \textit{divisor} o \textit{factor} de \(g(x)\).
%Un polinomio \(f(x)\) se dice que es \emph{irreducible} cuando no es posible expresarlo como producto de dos polinomios no constantes y distintos entre sí.

Como ya hemos anticipado vamos a centrarnos en el estudio de los anillos de polinomios en cuerpos finitos pues son los que necesitaremos para la teoría de códigos.
Siguiendo la notación que hemos establecido, denotaremos el anillo de le los polinomios con coeficientes en \(\mathbb F_p\) por \(\mathbb F_p[x]\).
Es un anillo conmutativo con las operaciones habituales de suma y multiplicación de polinomios que acabamos de describir.
Es, de hecho, un dominio de integridad.

Un polinomio en \(\mathbb F_p[x]\) viene dado por \(f(x) = \sum_{i=0}^n a_ix^i\), donde \(a_i\) son los coeficientes del término de grado \(i\) y pertenecen a \(\mathbb F_p\).
Dados \(f(x), g(x) \in \mathbb F_p[x]\) el \textit{máximo común divisor} de \(f(x)\) y \(g(x)\), siendo al menos uno de ellos distinto de cero, es el polinomio mónico de \(\mathbb F_p[x]\) de mayor grado que divida tanto a \(f(x)\) como a \(g(x)\).
Lo denotamos por \(\operatorname{mcd}(f(x), g(x))\).
Decimos que dos polinomios son \textit{primos relativos} si su máximo común divisor es 1.

El siguiente resultado nos proporciona la existencia y unicidad de divisores de un polinomio en \(\mathbb F_p[x]\).

% Algoritmo de división (cambiado nombre del teorema para quitar algoritmo de división por sugerencia de Gabriel)

\begin{theorem}
  \label{th:algoritmo-division}
  Sean \(f(x)\) y \(g(x)\) polinomios de \(\mathbb F_p[x]\), siendo \(g(x)\) distinto de cero.
  \begin{enumerate}
    \item \label{thi:algoritmo-division-division} Existen polinomios únicos, \(h(x)\), \(r(x) \in \mathbb F_p[x]\) tales que \[
      f(x) = g(x)h(x) + r(x), \quad \text{donde } \operatorname{gr} r(x) < \operatorname{gr} g(x) \text{ o } r(x) = 0. 
    \]
    \item \label{thi:algoritmo-division-mcd} Si \(f(x) = g(x)h(x) + r(x)\), entonces \[\operatorname{mcd}(f(x), g(x)) = \operatorname{mcd}(g(x), r(x)).\]
  \end{enumerate}
\end{theorem}

Podemos utilizar este resultado para hallar el máximo común divisor de dos polinomios.
Este procedimiento se conoce como \textit{algoritmo de Euclides} y es muy parecido a su homólogo para números enteros.

\begin{theorem}[Algoritmo de Euclides]
  Sean \(f(x)\) y \(g(x)\) dos polinomios en \(\mathbb F_p[x]\) con \(g(x)\) distinto de cero.
  \begin{enumerate}
    \item Realiza los siguientes pasos hasta que \(r_n(x) = 0\) para algún \(n\):
    \begin{align*}
      f(x) &= g(x)h_1(x) + r_1(x), \qquad \text{donde } \operatorname{gr} r_1(x) < \operatorname{gr} g(x),\\
      g(x) &= r_1(x)h_2(x) + r_2(x), \qquad \text{donde } \operatorname{gr} r_2(x) < \operatorname{gr} r_1(x),\\
      r_1(x) &= r_2(x)h_3(x) + r_3(x), \qquad \text{donde } \operatorname{gr} r_2(x) < \operatorname{gr} r_3(x),\\
        &\,\vdots \\
      r_{n-3}(x) &= r_{n-2}(x)h_{n-1}(x) + r_{n-1}(x), \ \text{donde } \operatorname{gr} r_{n-1}(x) < \operatorname{gr} r_{n-2}(x),\\
      r_{n-2}(x) &= r_{n-1}(x)h_n(x) + r_n(x), \qquad \text{donde } r_n(x) = 0.
    \end{align*} 
    Entonces, \(\operatorname{mcd}(f(x), g(x)) = cr_{n-1}(x)\), donde \(c \in \mathbb F_p\) se escoge para que \(cr_{n-1}(x)\) sea mónico.
    \item Existen polinomios \(a(x), b(x) \in \mathbb F_p[x]\) tales que 
    \[
      a(x)f(x) + b(x)g(x) = \operatorname{mcd}(f(x), g(x)).
    \]
  \end{enumerate}
  \label{th:algoritmo-euclides}
\end{theorem}

La secuencia de pasos descrita termina porque en cada paso el grado del resto se reduce al menos en 1.
A continuación vamos a comentar un par de resultados que nos serán útiles en el futuro.

\begin{proposition}
  \label{prop:k-divisor-f-g}
  Sean \(f(x)\) y \(g(x)\) polinomios en \(\mathbb F_p[x]\).
  \begin{enumerate}
    \item Si \(k(x)\) es un divisor de \(f(x)\) y de \(g(x)\), entonces \(k(x)\) es divisor del polinomio \(a(x)f(x) + b(x)g(x)\) para todo \(a(x), b(x) \in \mathbb F_p[x]\).
    \item Si \(k(x)\) es un divisor de \(f(x)\) y de \(g(x)\) entonces \(k(x)\) es divisor del polinomio \(\operatorname{mcd}(f(x), g(x))\).
  \end{enumerate}
\end{proposition}

\begin{proposition}
  \label{prop:raices-factores-pol-Fp}
  Sea \(f(x)\) un polinomio en \(\mathbb F_p[x]\) de grado \(n\). 
  Entonces: 
  \begin{enumerate}
    \item Si \(\alpha \in \mathbb F_p\) es una raíz de \(f(x)\) entonces \(x - \alpha\) es un factor de \(f(x)\).
    \item El polinomio \(f(x)\) tiene al menos \(n\) raíces en cualquier cuerpo que contenga a \(\mathbb F_p\).
  \end{enumerate}
\end{proposition}

Un polinomio no constante \(f(x) \in \mathbb F_p[x]\) es \textit{irreducible sobre} \(\mathbb F_p\) si no es posible factorizarlo como producto de dos polinomios de \(\mathbb F_p[x]\) de grado menor.
\begin{theorem}
  Sea \(f(x)\) un polinomio no constante. Entonces, 
  \[
    f(x) = p_1(x)^{a_1}p_2(x)^{a_2}\dots p_k(x)^{a_k},
  \]
  donde cada \(p_i(x)\) es irreducible, los polinomios \(p_i(x)\) son únicos salvo el orden en el que aparecen y producto por unidades, y los elementos \(a_i\) son únicos.
\end{theorem}
Este resultado nos dice que \(\mathbb F_p[x]\) es lo que se conoce como \textit{dominio de factorización única}.
Se puede comprobar que es, además, un dominio de ideales principales.

% 3.3 Primitive elements

% 3.4 Constructing finite fields

\subsection{Construcción de cuerpos finitos}

En esta subsección vamos a construir cuerpos finitos de característica arbitraria —siempre que sea un primo, por supuesto— a partir del cociente de anillos de polinomios por polinomios irreducibles.
%\begin{proof}
  % Me ha dicho Gabriel que no es necesario
  % NOPE: demostración de que F_q[x] es un dominio de ideales principales.
  % Ejercicio 153 (p. 102 [121]), F_q[x] es un anillo conmutativo con unidad y un dominio de integridad
  % Ejercicio 166 (p. 106 [125]), cada ideal de F_q[x] es un ideal principal
%\end{proof}

\begin{proposition}
  Sea \(p\) un número primo y \(f(x) \in \mathbb F_p[x]\) un polinomio irreducible sobre \(\mathbb F_p\) de grado \(m\).
  Entonces, el anillo cociente dado por \(\mathbb F_p[x]/(f(x))\)  es un cuerpo de característica \(p\) con \(q = p^m\) elementos.
\end{proposition}

% \begin{proof}
%   Tenemos que probar que el anillo cociente \(\mathbb F_p[x]/(f(x))\) es un anillo de división conmutativo.
%   Sabemos que es conmutativo, pues \(\mathbb F_p[x]\) lo es.
%   Resta comprobar que es un anillo de división, es decir, que todo elemento tiene inverso.
%   Consideremos una clase lateral de este cociente distinta de cero, que será de la forma \(g(x) + (f(x))\), donde el polinomio \(g(x)\) es único y con grado como mucho \(m-1\).
%   Por definición \(g(x)\) no es múltiplo de \(f(x)\).
%   Como \(f(x)\) es irreducible, \(\operatorname{mcd}(f(x), g(x))\) es, o bien \(1\), o bien \(f(x)\).
%   Como \(f(x)\) no divide a \(g(x)\) necesariamente \(\operatorname{mcd}(f(x), g(x)) = 1\).
%   El algoritmo de Euclides \ref{th:algoritmo-euclides} nos da polinomios \(a(x)\) y \(b(x)\) tales que \(g(x)a(x) + b(x)f(x) = 1\).
%   Observamos que entonces \(a(x)\) representa a un inverso de la clase \(g(x) + (f(x))\).
%   Esto implica que la clase lateral de \(g(x)\) es invertible, y por tanto, todos los elementos no nulos de \(\mathbb F_p[x]/(f(x))\) lo son, por lo que es un anillo de división conmutativo, como queríamos.
%   Vemos que además tiene caracterítica \(p\), pues dado \(g(x)\), 
%   \[
%     \underbrace{g(x) + g(x) + \dots + g(x)}_{p \text{ términos}} = (1 + 1 + \dots + 1)g(x) = p \cdot g(x) = 0.
%   \]
%   Finalmente, para ver que tiene \(q = p^m\) elementos
% \end{proof}

Puede probarse que existen polinomios irreducibles de cualquier grado, lo que nos lleva a afirmar el siguiente teorema.

\begin{theorem}
  Para cualquier primo \(p\) y cualquier entero positivo \(m\), existe un cuerpo finito, único salvo isomorfismos, con \(q = p^m\) elementos.
\end{theorem}

A la hora de trabajar con el cuerpo \(\mathbb F_p[x]/(f(x))\) escribiremos cada clase lateral \(g(x) + (f(x))\) como un vector en \(\mathbb F_p^m\) siguiendo la correspondencia:
\[
  g_{m-1}x^{m-1} + g_{m-2}x^{m-2}+ \dots + g_1x + g_0 + (f(x)) \iff g_{m-1}g_{m-2}\dots g_1g_0.
\]
Esta notación vectorial nos permite realizar la suma en el cuerpo utilizando la suma habitual de los vectores.
Multiplicar es una tarea a priori más complicada.
Para multiplicar \(g_1(x) + (f(x))\) por \(g_2(x) + (f(x))\) primero utilizamos al algoritmo de división para escribir
\[
  g_1(x)g_2(x) = f(x)h(x) + r(x),
\]
donde como sabemos o bien \(\operatorname{gr} r(x) \leq m -1\) o bien \(r(x) = 0\).
Puesto que estamos en el anillo cociente \(\mathbb F_p[x]/(f(x))\) nos queda
\[
  (g_1(x) + (f(x)))(g_2(x) + (f(x))) = r(x) + (f(x)).
\]

Notamos rápidamente que esta notación es engorrosa, por lo que vamos a tratar de encontrar otra mejor que nos permita multiplicar con más facilidad.
Para ello vamos a tomar una raíz \(\alpha\) de \(f(x)\), es decir, un elemento tal que \(f(\alpha) = 0\) y operaremos en \(\alpha\) vez de en \(x\).
Así, \(g_1(\alpha)g_2(\alpha) = r(\alpha)\) y podemos extender la correspondencia vectorial a
\[
  g_{m-1}g_{m-2}\dots g_1g_0 \iff g_{m-1}\alpha^{m-1} + g_{m-2}\alpha^{m-2}+ \dots + g_1\alpha + g_0.
\]
En consecuencia, multiplicamos los polinomios en \(\alpha\) de la forma habitual y utilizamos la ecuación \(f(\alpha) = 0\) para reducir las potencias de \(\alpha\) de grado mayor a \(m-1\) a polinomios en \(\alpha\) de grado menor que \(m\).
Veamos un ejemplo.

\begin{example}
  El polinomio \(f(x) = x^2 + 1\) es irreducible sobre \(\mathbb F_3\).
  Vamos a construir el cuerpo finito \(\mathbb F_9\) como \(\mathbb F_3[x]/(x^2 + 1)\).
  Sea \(\alpha\) una raíz del polinomio \(f(x)\).
  Utilizando la correspondencia clases laterales-vectores-polinomios en \(\alpha\) obtenemos los elementos descritos en la tabla.
  \begin{table}[h]
    \centering
    \sffamily
    \begin{tabular}{lcc}
      \toprule
      Clases laterales & Vectores & Polinomio en \(\alpha\)\\
      \midrule
      \(0 + (f(x))\) & \(00\) & \(0\)\\
      \(1 + (f(x))\) & \(01\) & \(1\)\\
      \(2 + (f(x))\) & \(02\) & \(2\)\\
      \(x + (f(x))\) & \(10\) & \(\alpha\)\\
      \(x + 1 + (f(x))\) & \(11\) & \(\alpha + 1\)\\
      \(x + 2 + (f(x))\) & \(12\) & \(\alpha + 2\)\\
      \(2x + (f(x))\) & \(20\) & \(2\alpha\)\\
      \(2x + 1 + (f(x))\) & \(21\) & \(2\alpha + 1\)\\
      \(2x + 2 + (f(x))\) & \(22\) & \(2\alpha + 2\)\\
      \bottomrule
    \end{tabular}
  \end{table}

  Veamos un par de ejemplos de operaciones: \begin{itemize}
    \item Sumemos \((x + 2 + (f(x))) + (2x + 1 + (f(x)))\). Lo haremos con la notación vectorial, sumando \(12 + 21 = 00 = 0\).
    \item Multipliquemos \((2x + 1 + (f(x)))(2x + 2 + (f(x)))\). Lo haremos con la expresión del polinomio en \(\alpha\):
    \[
      (2\alpha + 1)(2\alpha + 2) = 4\alpha^2 + 6\alpha + 2 = \alpha^2 + 2,
    \]
    y usando que \(\alpha^2 = 2\), 
    \[
      \alpha^2 + 2 = 2 + 4 = 1.
    \]
  \end{itemize}
\end{example}

Decimos \(\mathbb F_q\) es la extensión generada por una una raíz \(\alpha\) de \(f(x)\) a partir de \(\mathbb F_p\), lo que se nota como \(\mathbb F_q = \mathbb F_p(\alpha)\).
Esta raíz viene dada formalmente por \(\alpha = x + (f(x))\) en el anillo cociente \(\mathbb F_p[x]/(f(x))\).
Por tanto, ya hemos visto antes que \(g(x) + (f(x)) = g(\alpha)\) y \(f(\alpha) = f(x + (f(x))) = f(x) + (f(x)) = 0 + (f(x))\).

En el siguiente apartado veremos que multiplicar elementos de un cuerpo finito generado por una raíz \(\alpha\) es más sencillo cuando esta raíz es un tipo de elemento concreto.

\subsection{Elementos primitivos}

Vamos a buscar otra forma de expresar los elementos de un cuerpo \(\mathbb F_q\) —con \(q = p^m\) para \(p\) primo— de tal forma que podamos conectarla con nuestra notación como tuplas y que haga que la multiplicación sea más sencilla de expresar.
Esta expresión vendrá dada por lo que vamos a denominar elementos primitivos.

Comenzamos viendo que el conjunto \(\mathbb F_q^*\) —de los elementos de \(\mathbb F_q\) distintos de cero— es un grupo.

\begin{theorem}
  \label{th:Fq-ast-cilcico}
  Se verifican las siguientes afirmaciones.
  \begin{enumerate}
    \item El grupo \(\mathbb F_q^*\) es cíclico de orden \(q - 1\) con la multiplicación de \(\mathbb F_q\).
    \item Si \(\gamma\) es un generador de este grupo cíclico entonces
    \[
      \mathbb F_q = \{0, 1 = \gamma^0, \gamma, \gamma^2, \dots, \gamma^{q-2}\},
    \] y se tiene que \(\gamma^i = 1\) si y solo si \((q-1) \mid i\).
  \end{enumerate}
\end{theorem}

Cada generador \(\gamma\) de \(\mathbb F_q^*\) se llama \textit{elemento primitivo} de \(\mathbb F_q\).
Cuando los elementos distintos de cero de un cuerpo finito se expresan como potencias de \(\gamma\) podemos multiplicar de forma sencilla teniendo en cuenta que \(\gamma^i\gamma^j = \gamma^{i+j} = \gamma^s\), donde \(0 \leq s \leq q-2\) e \(i + j \equiv s \bmod q - 1\).

\begin{theorem}
  \label{th:el-Fq-raices-pol}
  Los elementos de \(\mathbb F_q\) son las raíces del polinomio \(x^q - x\).
\end{theorem}

\begin{proof}
  Sea \(\gamma\) un elemento primitivo de \(\mathbb F_q\).
  Entonces, \(\gamma^{q-1} = 1\) por definición.
  Por tanto, \((\gamma^i)^{q-1} = 1\) para todo \(i\) tal que \(0 \leq i \leq q - 2\).
  En consecuencia, los elementos de \(\mathbb F_q^*\) son las raíces de \(x^{q-1}-1 \in \mathbb F_p[x]\) y en consecuencia, de \(x^q - x\).
  Como \(0\) es raíz de \(x^q - x\), por la proposición \ref{prop:raices-factores-pol-Fp} sabemos que los elementos de \(\mathbb F_q\) son las raíces de \(x^q - x\), como queríamos.
\end{proof}

Un elemento \(\xi \in \mathbb F_q\) es una raíz enésima de la unidad si \(\xi^n = 1\), y es una raíz enésima primitiva de la unidad si además \(\xi^s \neq 1\) para todo s tal que \(0 < s < n\).
Un elemento primitivo \(\gamma\) de \(\mathbb F_q\) es por tanto una raíz \((q-1)\)-ésima de la unidad.
Se deduce del teorema \ref{th:Fq-ast-cilcico} que el cuerpo \(\mathbb F_q\) contiene una raíz enésima primitiva de la unidad si y solo si \(n \mid (q - 1)\), en cuyo caso \(\gamma^{(q-1)/n}\) es dicha raíz.


%El conjunto \(\{0\alpha^{m-1} + 0\alpha^{m-2} + \dots + 0\alpha + a_0 \mid a_0 \in \mathbb F_p\} = \{a_0 \mid a_0 \in \mathbb F_p\}\) es el subcuerpo primo de \(\mathbb F_q\).



Un polinomio irreducible sobre \(\mathbb F_p\) de grado \(m\) es \textit{primitivo} si tiene una raíz que es un elemento primitivo de \(\mathbb F_q = \mathbb F_{p^m}\).
Cuando construimos un cuerpo finito con un polinomio \emph{primitivo}, podemos expresar sus elementos como potencias de la raíz primitiva correspondiente, lo que simplificará las operaciones de multiplicación.
Veamos un ejemplo.

\begin{example}
  El polinomio \(f(x) = x^2 + x + 2\) es irreducible sobre \(\mathbb F_3\).
  Vamos a construir el cuerpo finito \(\mathbb F_9\) como \(\mathbb F_3[x]/(f(x))\).
  Sea \(\alpha\) una raíz del polinomio \(f(x)\).
  Utilizando de nuevo la correspondencia clases laterales-vectores-polinomios en \(\alpha\) obtenemos la misma descripción de los elementos, pero usando la correspondencia \(\alpha^2 = -\alpha - 2 = 2\alpha + 1\) podemos expresarlos en forma de potencias de \(\alpha\), como vemos en la tabla.
  \begin{table}[h]
    \centering
    \sffamily
    \begin{tabular}{lccc}
      \toprule
      Clases laterales & Vectores & Polinomio en \(\alpha\) & Potencia de \(\alpha\)\\
      \midrule
      \(0 + (f(x))\) & \(00\) & \(0\) & \(0\)\\
      \(1 + (f(x))\) & \(01\) & \(1\) & \(1 = \alpha^0\)\\
      \(2 + (f(x))\) & \(02\) & \(2\) & \(\alpha^4\)\\
      \(x + (f(x))\) & \(10\) & \(\alpha\) & \(\alpha\)\\
      \(x + 1 + (f(x))\) & \(11\) & \(\alpha + 1\) & \(\alpha^7\)\\
      \(x + 2 + (f(x))\) & \(12\) & \(\alpha + 2\) & \(\alpha^6\)\\
      \(2x + (f(x))\) & \(20\) & \(2\alpha\) & \(\alpha^5\)\\
      \(2x + 1 + (f(x))\) & \(21\) & \(2\alpha + 1\) & \(\alpha^2\)\\
      \(2x + 2 + (f(x))\) & \(22\) & \(2\alpha + 2\) & \(\alpha^3\)\\
      \bottomrule
    \end{tabular}
  \end{table}

  Así, si queremos multiplicar \((2x + 1 + (f(x)))(2x + 2 + (f(x)))\) podemos hacer simplemente \(\alpha^2 \alpha^3 = \alpha^5 = 2\alpha = 2x + (f(x))\).
\end{example}

% 3.7 Cyclotomic cosets and minimal polynomials

\subsection{Clases ciclotómicas y polinomios minimales}
\label{subsec:clases-ciclotomicas}

Si consideramos la extensión de cuerpos \(\mathbb F_{q^t}/\mathbb F_q\) sabemos por el teorema \ref{th:el-Fq-raices-pol} que cada elemento de \(\mathbb F_{q^t}\) es raíz del polinomio \(\cramped{x^{q^t}} - x\).
Existe por tanto un polinomio mónico \(M_{\alpha}\) en \(\mathbb F-q[x]\) de grado mínimo que tiene a \(\alpha\) como raíz.
Este polinomio se conoce como \emph{polinomio minimal} de \(\alpha\) sobre \(\mathbb F_q\).
El siguiente teorema nos detalla algunas de las propiedades de los polinomios minimales.

\begin{theorem}
  \label{th:pol-minimal}
  Sea \(\mathbb F_{q^t}/\mathbb F\) una extensión de cuerpos y sea \(\alpha\) un elemento de \(\mathbb F_{q^t}\) cuyo polinomio minimal es \(M_{\alpha} \in \mathbb F_q[x]\).
  Se verifica:
  \begin{enumerate}
    \item El polinomio \(M_{\alpha}(x)\) es irreducible sobre \(\mathbb F_q\).
    \item Si \(g(x)\) es cualquier polinomio en \(\mathbb F_q[x]\) tal que \(g(\alpha) = 0\) entonces \(M_{\alpha}(x) \mid g(x)\).
    \label{thi:pol-minimal-div}
    \item El polinomio \(M_{\alpha}(x)\) es único.
  \end{enumerate}
\end{theorem}

Si partimos de \(f(x)\), un polinomio irreducible sobre \(\mathbb F_q\) de grado \(r\), podemos considerar la extensión generada por una de las raíces de \(f(x)\) y obtendremos el cuerpo \(\mathbb F_{q^r}\).
De hecho, el siguiente teorema afirma que en ese caso todas las raíces de \(f(x)\) estarán en \(\mathbb F_{q^r}\).

\begin{theorem}
  Sea \(f(x)\) un polinomio irreducible mónico sobre \(\mathbb F_q\) de grado \(r\).
  Entonces:
  \begin{enumerate}
    \item Todas las raíces de \(f(x)\) están en \(\mathbb F_{q^r}\) y en cualquier extensión de cuerpos de \(\mathbb F_q\) generada por una de sus raíces.
    \label{thi:pol-irr-raices}
    \item Podemos expresar \(f(x)\) como \(f(x) = \prod_{i=1}^r (x - \alpha_i)\), donde \(\alpha_i \in \mathbb F_{q^r}\) para \(1 \leq i \leq r\).
    \item El polinomio \(f(x)\) divide a \(x^{q^r} - x\).
    \label{thi:pol-irr-factor}
  \end{enumerate}
  \label{th:pol-irr}
\end{theorem}

\begin{proof}
  Veamos la demostración por partes.
  \begin{enumerate}
    \item Sea \(\alpha\) una raíz de \(f(x)\) y consideramos el cuerpo \(\mathbb F_{q^r} = \mathbb F_{q}(\alpha)\), la extensión de \(\mathbb F_q\) generada por \(\alpha\).
    Sea \(\beta\) otra raíz de \(f(x)\) y supongamos que no está en \(\mathbb F_{q}(\alpha)\).
    Entonces, será raíz de algún factor irreducible de \(f(x)\) sobre \(\mathbb F_{q}(\alpha)\).
    Consideramos ahora la extensión \(\mathbb F_{q}(\alpha, \beta)\), la extensión de \(\mathbb F_{q}(\alpha)\) generada por \(\beta\).
    Dentro de \(\mathbb F_{q}(\alpha, \beta)\) hay un subcuerpo \(\mathbb F_{q}(\beta)\), que es la extensión de \(\mathbb F_q\) generada por \(\beta\).
    Además, \(\mathbb F_{q}(\beta)\) ha de tener \(q^r\) elementos, pues \(f(x)\) es un irreducible de grado \(r\) sobre \(\mathbb F_q\).
    Como tanto \(\mathbb F_{q}(\alpha)\) como \(\mathbb F_{q}(\beta)\) son subcuerpos del mismo tamaño, han de ser iguales (cita). % TODO: mencionar el teorema (no incluido) de que dos subcuerpos de F_p del mismo tamaño son iguales
    Por tanto, todas las raíces de \(f(x)\) están en \(\mathbb F_{q^r}\).
    Finalmente, todo cuerpo conteniendo a \(\mathbb F_q\) y una raíz de \(f(x)\) contiene a \(\mathbb F_{q^r}\).
    \item Se deduce de lo anterior y de la proposición \ref{prop:raices-factores-pol-Fp}.
    \item Se deduce del apartado anterior y del hecho de que por el teorema \ref{th:el-Fq-raices-pol}, \(x^{q^r} - x = \prod_{\alpha \in \mathbb F_{q^r}}(x - \alpha)\).\qedhere
  \end{enumerate}
\end{proof}

En particular este teorema se verifica para los polinomios minimales \(M_{\alpha}(x)\) sobre \(\mathbb F_q\), pues son mónicos irreducibles.

\begin{theorem}
  \label{th:prop-pol-minimal}
  Sea \(\mathbb F_{q^t}/\mathbb F_q\) una extensión de cuerpos y sea \(\alpha\) un elemento de \(\mathbb F_{q^t}\) con polinomio minimal \(M_{\alpha}\) en \(\mathbb F_q[x]\).
  Se verifican las siguientes afirmaciones.
  \begin{enumerate}
    \item El polinomio \(M_{\alpha}(x)\) divide a \(x^{q^t} - x\).
    \label{thi:prop-pol-minimal-div}
    \item El polinomio \(M_{\alpha}(x)\) tiene raíces distintas dos a dos, todas en \(\mathbb F_{q^t}\).
    \item El grado de \(M_{\alpha}(x)\) divide a \(t\).
    \item Podemos expresar \(x^{q^t}- x = \prod_{\alpha}M_{\alpha}(x)\), donde \(\alpha\) varía entre los elementos de un subconjunto de \(\mathbb F_{q^t}\) de forma que enumera los polinomios minimales de todos los elementos de \(\mathbb F_{q^t}\) una sola vez.
    \item Podemos expresar \(x^{q^t}- x = \prod_{f}f(x)\), donde \(f\) varía entre todos los mónicos irreducibles sobre \(\mathbb F_q\) cuyo grado divide a \(t\).
  \end{enumerate}
\end{theorem}

\begin{proof}
  Veamos la demostración por apartados.
  \begin{enumerate}
    \item Se deduce del teorema \ref{th:pol-minimal}(\ref{thi:pol-minimal-div}), pues por el teorema \ref{th:el-Fq-raices-pol}, \(\alpha^{q^t} - \alpha = 0\).
    \item Las raíces del polinomio \(x^{q^t} - x\) son los \(q^t\) elementos de \(\mathbb F_{q^t}\), luego este polinomio tiene raíces distintas dos a dos en \(\mathbb F_{q^t}\) por el teorema \ref{th:pol-irr}(\ref{thi:pol-irr-raices}).
    Ahora, como por \ref{thi:prop-pol-minimal-div} \(M_{\alpha}(x)\) divide a \(x^{q^t} - x\), todas sus raíces también son distintas dos a dos y están en \(\mathbb F_{q^t}\).
    \item Si el grado de \(M_{\alpha}(x)\) es \(r\), la extensión \(\mathbb F_{q}(\alpha)\) genera el subcuerpo \(\mathbb F_{q^r} = \mathbb F_{p^{mr}}\) de \(\mathbb F_{q^t} = \mathbb F_{p^{mt}}\), de forma que \(mr | mt\) por el teorema (cita) y por tanto el grado de \(M_{\alpha}(x)\) divide a \(t\). % TODO: mencionar el teorema (no incluido) de que dos subcuerpos de F_p del mismo tamaño son iguales
    \item Como \(x^{q^t} - x\) tiene raíces distintas dos a dos, sus factores \(p_i\)(x) también lo son, y puesto que es mónico, podemos asumir que también lo son.
    Por tanto, \(p_{i}(x) = M_{\alpha}(x)\) para cualquier \(\alpha \in \mathbb F_{q^t}\) tal que \(p_i(\alpha) = 0\), obteniendo así los factores del producto que buscamos.
    \item Este apartado se deduce del anterior, si demostramos que todo polinomio mónico irreducible de grado \(r | t\) sobre \(\mathbb F_q\) es un factor de \(x^{q^t} - x\).
    Pero \(f(x) | (x^{q^r} - x)\) por el teorema \label{th:pol-irr}(\label{thi:pol-irr-factor}).
    Como \(mr | mt\), \((x^{q^r} - x) | (x^{q^t} - x)\) por el lema (cita).\qedhere
    % TODO: citar lema 3.5.2 del libro (no incluido)
  \end{enumerate}
\end{proof}

Dos elementos de \(\mathbb F_{q^t}\) que tienen el mismo polinomio minimal en \(\mathbb F_q[x]\) se llaman \textit{conjugados sobre} \(\mathbb F_q\).
Es importante encontrar todos los conjugados de \(\alpha \in \mathbb F_q\), es decir, todas las raíces de \(M_{\alpha}(x)\).
Sabemos por el teorema \ref{th:prop-pol-minimal} que las raíces de \(M_{\alpha}(x)\) son todas distintas dos a dos y que se encuentran en el cuerpo \(\mathbb F_{q^t}\).
Podemos encontrar estas raíces con ayuda del siguiente teorema.

\begin{theorem}
  Sea \(f(x)\) un polinomio en \(\mathbb F_q[x]\) y sea \(\alpha\) una raíz de \(f(x)\) en una extensión \(\mathbb F_{q^t}/\mathbb F_q\).
  Entonces se verifican las siguientes afirmaciones.
  \begin{enumerate}
    \item Evaluando el polinomio obtenemos que \(f(x^q) = f(x)^q\).
    \item El elemento \(\alpha^q\) es también una raíz de \(f(x)\) en \(\mathbb F_q\).
  \end{enumerate}
\end{theorem}

\begin{proof}
  % TODO (ECC teorema 3.7.4, p114 [133])
\end{proof}

Si aplicamos este teorema de forma consecutiva podremos obtener todas las raíces de \(M_{\alpha}(x)\), que serán de la forma \(\alpha, \alpha^q, \alpha^{q^2},\)\,etc.; secuencia que terminará tras \(r\) términos, cuando \(\alpha^{q^r} = \alpha\).

Supongamos ahora que \(\gamma\) es un elemento primitivo de \(\mathbb F_{q^t}\).
Entonces sabemos que \(\alpha = \gamma^s\) para algún \(s\).
Por tanto, \(\alpha^{q^r} = \alpha\) si y solo si \(\gamma^{sq^r - s} = 1\).
Por el teorema \ref{th:Fq-ast-cilcico} se tiene que \(sq^r \equiv s \bmod q^t - 1\).
Basándonos en esta idea podemos definir la \textit{clase q-ciclotómica de s módulo} \(q^t - 1\) como el conjunto
\[
  C_s = \{s, sq, \dots, sq^{r-1}\} \bmod q^t - 1, 
\]
donde \(r\) es el menor entero positivo tal que \(sq^r \equiv s \bmod q^t - 1\).
Los conjuntos \(C_s\) dividen el conjunto de enteros \(\{0, 1, 2, \dots, q^t - 2\}\) en conjuntos disjuntos.

% TODO: justificación de este teorema

\begin{theorem}
  \label{th:pol-minimal-el-primitivo}
  Si \(\gamma\) es un elemento primitivo de \(\mathbb F_{q^t}\) entonces el polinomio minimal de \(\gamma^s\) sobre \(\mathbb F_q\) es
  \[
    M_{\gamma^s}(x) = \prod_{i \in C_s}(x - \gamma^i).
  \]
\end{theorem}

\section{Automorfismos de cuerpos finitos}

En esta sección vamos a describir someramente los automorfismos de cuerpos finitos, que nos serán necesarios cuando trabajemos con anillos de polinomios de Ore.

Dado un cuerpo finito \(\mathbb F_p\) un automorfismo \(\sigma\) de \(\mathbb F_q\) es una aplicación biyectiva \(\sigma: \mathbb F_p \to \mathbb F_p\) tal que \(\sigma(\alpha + \beta) = \sigma(\alpha) + \sigma(\beta)\) y \(\sigma(\alpha \beta) = \sigma(\alpha)\sigma(\beta)\) para todo \(\alpha, \beta \in \mathbb F_p\).
Vamos a describir en el siguiente ejemplo un automorfismo al que nos referiremos en el futuro.

\begin{example}
  Sea \(\mathbb F_q\) un cuerpo finito, donde \(q = p^m\) con \(p\) primo. 
  La aplicación biyectiva dada por \(\sigma_p : \mathbb F_q \to \mathbb F_q\) tal que \(\sigma_p(\alpha) = \alpha^p\) para todo \(\alpha \in \mathbb F_q\) es un automorfismo, conocido como \emph{automorfismo de Frobenius}.
\end{example}

Al grupo \(G\) que forman los automorfismos de un cuerpo finito \(\mathbb F_p\) lo llamamos \emph{grupo de Galois} de \(\mathbb F_p\).
Así, definomos el \emph{orden} de un automorfismo como el menor \(n\) tal que \(\sigma^n(\alpha) = \alpha\) para todo \(\alpha \in \mathbb F_p\).

\begin{theorem}
  El grupo de Galois de un cuerpo finito \(\mathbb F_q\), con \(q = p^m\) y \(p\) primo es cíclico de orden \(m\) y está generado por el automorfismo de Frobenius \(\sigma_p\).
\end{theorem}

Decimos que un elemento \(\alpha \in \mathbb F_p\) queda \emph{fijo} por un automorfismo \(\sigma\) si \(\sigma(\alpha) = \alpha\).
El conjunto de los elementos de \(\mathbb F_p\) que quedan fijos por un automorfismo \(\sigma\) forma un subcuerpo de \(\mathbb F_p\) y se denomina \emph{subuerpo fijo de} \(\mathbb F_p\) por \(\sigma\).
Lo denotamos por \(\mathbb F_p^{\sigma}\).

\begin{definition}
  \label{def:base-normal}
  Sea \(\mathbb F_{p^m}/\mathbb F_p\) una extensión de cuerpos finitos y sea \(G\) el grupo de Galois de \(\mathbb F_{p^m}\). Una \emph{base normal} de \(\mathbb F_{p^m}\) sobre \(\mathbb F_p\) es una base de la forma \(\{\alpha, \alpha^p, \dots, \alpha^{p^m - 1}\}\) para algún \(\alpha \in \mathbb F_{p^m}\).
\end{definition}

En general las bases normales de los cuerpos finitos están formadas por las imágenes de un elemento por los distintos automorfismos de su grupo de Galois.
Pero como en este caso el grupo de Galois está generado de forma cíclica por el automorfismo de Frobenius, podemos expresarlo como en la definición \ref{def:base-normal}.
El siguiente resultado se conoce como \emph{teorema de la base normal}.

\begin{theorem}
  Existe una base normal para toda extensión de cuerpos.
\end{theorem}

%\section{Espacios vectoriales}