\chapter{Preliminares}

En este capítulo se detallan algunos conceptos básicos de Álgebra que son necesarios para la comprensión más adelante de la teoría de códigos. Según \parencite{cohn_algebra_1982} y \parencite{cohn_algebra_1989}.

\section{Monoides}

\begin{definition}
  Un \textit{monoide} es un conjunto \(S\) con un elemento \(e\) y una aplicación \(\mu: S^2 \to S\) tal que si \(\mu(x, y)\) es el resultado de aplicar \(\mu\) a la pareja de elementos \(x, y \in S\), se verifican: \begin{enumerate}
    \item \(\mu(x, \mu(y, z)) = \mu(\mu(x, y), z)\) para todo \(x, y, z \in S\).
    \item \(\mu(e, x) = \mu(x, e) = x\) para todo \(x \in S\).
  \end{enumerate}
\end{definition}

Obsérvese que, por definición, un monoide siempre tiene al menos un elemento. A la aplicación \(\mu\) que actúa sobre parejas de elementos se le llama \textit{operación binaria} y al elemento \(e\), elemento neutro de \(\mu\).

\section{Anillos}

\begin{definition}
  Un \textit{anillo}.
\end{definition}