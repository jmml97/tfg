\chapter{Preliminares}

En este capítulo se detallan algunos conceptos básicos de Álgebra que son necesarios para la comprensión más adelante de la teoría de códigos. Según \parencite{cohn_algebra_1982} y \parencite{cohn_algebra_1989} y el libro de Jacobson.

\section{Anillos}

\begin{definition}
  Un \textit{anillo} es un conjunto junto a dos operaciones: la suma \((+)\) y la multiplicación \((\cdot)\), que verifican las siguientes propiedades.
  \begin{itemize}[itemsep=0pt]
    \item Propiedad asociativa:
    \[(x+y)+z = x + (y+z), \qquad (xy)z = x(yz).\]
    \item Propiedad conmutativa para la suma:
    \[x + y = y + x.\]
    \item Existencia de elemento neutro:
    \[x + 0 = x, \qquad x1 = x.\]
    \item Existencia de elemento inverso para la suma:
    \[x + (-x) = 0.\]
    \item Propiedad distributiva para la multiplicación sobre la suma:
    \[x(y+z) = xy + xz.\]
  \end{itemize}

  Si un anillo verifica la propiedad conmutativa para la multiplicación, es decir, \(xy = yx\), se dice que es un \textit{anillo conmutativo}.
\end{definition}

Dos anillos son \textit{isomorfos} si hay un \textit{isomorfismo} entre ellos, es decir, existe una biyección que preserva todas las operaciones.
Decimos que los anillos isomorfos son idénticos, pues intrínsecamente son iguales.

En  cualquier anillo \(R\) se verifica que \(0x = x0 = 0\) para todo \(x \in R\), ya que \(x0 = x(0+0) = x0 + x0\), de donde concluimos que \(x0 = 0\) e igualmente, \(0x = 0\).

El \textit{anillo trivial} es aquel que solo tiene un elemento. 
Necesariamente entonces \(1 = 0\), pero este es el único caso en el que ocurre. 
Supongamos que \(1 = 0\). 
Entonces, para cada elemento \(x\) del anillo se tiene que \(x = x1 = x0 = 0\), luego tiene un único elemento.

% MAYBE: abreviar na = a + a + a (Cohn 137 [147])

Un elemento \(a\) de un anillo se dice que es \textit{invertible} si existe un elemento \(a'\) en el anillo tal que \(aa' = a'a = 1\).
A este elemento, que es único lo llamamos \textit{elemento inverso} de \(a\) y lo denotamos por \(a^{-1}\).
% TODO: por qué es único (anillo está formado por monoide multiplicativo de R)
El elemento \(0\) no puede tener inverso porque ya hemos visto que siempre que se multiplique por él se obtiene el \(0\).
Los anillos en los que todo elemento distinto de 0 es invertible se llaman \textit{anillos de división}.

% Dados dos anillos \(R\) y \(S\), decimos que \(S\) es un \textit{subanillo} de \(R\) si \(S\) está contenido en \(R\) y forma un anillo con las mismas operaciones que \(R\) (y con los mismos elemento neutro e identidad).

% CITEME: definciiones de Niederreiter (13)

\begin{definition}[Subanillo]
  Un subconjunto \(S\) de un anillo \(R\) se denomina \textit{subanillo} si \(S\) es cerrado bajo las operaciones de suma y producto de \(R\) y forma un anillo con estas operaciones.
\end{definition}

\begin{definition}[Ideal]
  Un subconjunto \(J\) de un anillo \(R\) se denomina \textit{ideal} si \(J\) es un subanillo de \(R\) y para todo \(a \in J\) y \(r \in R\) se verifica que \(ar \in J\) y \(ra \in J\).
\end{definition}

\begin{example}\hfill
  \begin{itemize}
    \item Sea \(R\) el cuerpo \(\mathbb Q\) de números racionales. Entonces, el conjunto \(\mathbb Z\) de los enteros es un subanillo de \(\mathbb Q\) pero no es un ideal porque, por ejemplo, \(1 \in \mathbb Z, 1/2 \in \mathbb Q\), pero \(1/2 \cdot 1 = 1/2 \notin \mathbb Z\). 
    \item Sea \(R\) un anillo conmutativo, \(a \in R\) y sea \(J = \{ra : r \in R\}\). Entonces, \(J\) es un ideal.
    % TODO: algún ejemplillo más?
  \end{itemize}
\end{example}

\begin{definition}
  Sea \(R\) un anillo conmutativo. Un ideal \(J\) de \(R\) se dice que es \textit{principal} si existe un elemento \(a \in R\) tal que \(J = (a) = \{ra : r \in R\}\).
\end{definition}

% MAYBE: si el anillo no es conmutativo, ideales por la izquierda y por la derecha.

Dado un elemento del anillo distinto de cero, podemos clasificarlo en dos tipos. 
Sea \(a \neq 0\). 
Si existe \(b \neq 0\) tal que \(ab\) o \(ba\) es cero, entonces \(a\) es un elemento \textit{divisor de cero}, y en caso contrario, un elemento \textit{regular}.

Un anillo no trivial sin divisores de cero se dice que es \textit{entero}, un anillo entero conmutativo se denomina \textit{dominio de integridad}.

Una propiedad importante de los elementos regulares es la ley de cancelación.
\begin{proposition}
  Si \(c\) es un elemento regular de un anillo \(R\) entonces para cada \(a, b \in R\), tales que \(ca = cb\) o bien \(ac = bc\), se tiene que \(a = b\).
\end{proposition}

\begin{definition}
  Sea \(R\) un anillo. La \textit{característica} del anillo es el menor natural \(n\) tal que \(n1 = 0\).
  Si no existe tal número, la característica del anillo es \(0\).  
\end{definition}

\section{Cuerpos finitos}

\begin{definition}
  Un \textit{cuerpo} es un anillo de división conmutativo.
  Se dice que un cuerpo es \textit{finito} si tiene un número finito de elementos, al que llamamos \textit{orden} del cuerpo.
\end{definition}

% MAYBE: todo dominio de integridad finito es un cuerpo (Niederreiter, 12 [21])

Sea \(F\) un cuerpo. Un subconjunto \(K\) de \(F\) que es por sí mismo un cuerpo bajo las operaciones de \(F\) se denomina \textit{subcuerpo} de \(F\).
También podemos decir que \(F\) es una extensión de \(K\).

De hecho todas las nociones que hemos definido para anillos (característica, ...) son válidas para los cuerpos, pues un cuerpo no deja de ser un anillo.

\begin{theorem}
  % Cohn Vol 2, 63 [74]
  \label{th:cuerpo-subcuerpo-primo-caracteristica}
  Todo cuerpo \(F\) tiene al menos un subcuerpo \(P\), el subcuerpo primo de \(F\) que está contenido en cada subcuerpo de \(F\).
  O bien \(F\) tiene característica 0 y \(P \cong \mathbb Q\) o bien \(F\) tiene característica \(p\), un número primo, y entonces \(P \cong \mathbb F_p\).
\end{theorem}


% CITEME ? Cohn Algebra Vol 2

Mención especial merecen los cuerpos finitos.

Un cuerpo con un número finito de elementos se denomina \textit{cuerpo finito} o \textit{cuerpo de Galois}, por su descubridor.

Sea \(V\) un espacio vectorial \(n\)-dimensional sobre \(\mathbb F_p\), el cuerpo de \(p\) elementos.
Si \(u_1, \dots, u_n\) es una base de \(V\), entonces cada elemento de \(V\) se escribe de forma única en la forma \(\sum \alpha_iu_i\), donde \(\alpha_i \in \mathbb F_p\).
Como cada coeficiente puede tener hasta \(p\) valores distintos, obtenemos un total de \(p^n\) elementos.

\begin{lemma}
  Un espacio vectorial \(n\)-dimensional sobre \(\mathbb F_p\) tiene \(p^n\) elementos.
\end{lemma}

Todo cuerpo finito \(F\) tiene claramente característica \(p\) en virtud del teorema \ref{th:cuerpo-subcuerpo-primo-caracteristica} y su subcuerpo primo es \(\mathbb F_p\).

\section{Anillos de polinomios}

Para cualquier anillo \(R\) podemos definir un anillo de polinomios en \(x\) con coeficientes en \(x\).

Trabajaremos con anillos de polinomios en cuerpos finitos.

Denotamos el anillo de le los polinomios con coeficientes en \(\mathbb F_q\) por \(\mathbb F_q[x]\).
Es un anillo conmutativo con las operaciones habituales de suma y multiplicación de polinomios.
De hecho, es un dominio de integridad.

Un polinomio en \(\mathbb F_q[x]\) viene dado por \(f(x) = \sum_{i=0}^n a_ix^i\), donde \(a_i\) son los coeficientes del término de grado \(i\) y pertenecen a \(\mathbb F_q\).

El grado de un polinomio es el mayor grado de cualquier término con coeficiente distinto de cero.

\begin{proposition}
  Grado de sumas y productos
  % TODO: (Cohn, 149 [159])
\end{proposition}

El coeficiente del término de mayor grado se denomina \textit{coeficiente líder}.

Un polinomio es \textit{mónico} si su coeficiente líder es 1.
Sean \(f(x)\) y \(g(x)\) polinomios en \(\mathbb F_q[x]\).
Decimos que \(f(x)\) \textit{divide a} \(g(x)\), denotado por \(f(x) | g(x)\), si existe un polinomio \(h(x) \in \mathbb F_q[x]\) tal que \(g(x) = f(x)h(x)\).
El polinomio \(f(x)\) se llama \textit{divisor} o \textit{factor} de \(g(x)\).
El \textit{máximo común divisor} de \(f(x)\) y \(g(x)\), siendo al menos uno de ellos distinto de cero, es el polinomio mónico de \(\mathbb F_q[x]\) de mayor grado que divida tanto a \(f(x)\) como a \(g(x)\).
Lo denotamos por \(\operatorname{mcd}(f(x), g(x))\).
Dos polinomios son \textit{primos relativos} si su máximo común divisor es 1.

% TODO: polinomios irreducibles

% TODO: 

\begin{theorem}
  Sean \(f(x)\) y \(g(x)\) polinomios de \(\mathbb F_q[x]\), siendo \(g(x)\) distinto de cero.
  \begin{enumerate}
    \item Existen polinomios únicos, \(h(x)\), \(r(x) \in \mathbb F_q[x]\) tales que \[
      f(x) = g(x)h(x) + r(x), \quad \text{donde } \operatorname{gr} r(x) < \operatorname{gr} g(x) \text{ o } r(x) = 0. 
    \]
    \item Si \(f(x) = g(x)h(x) + r(x)\), entonces \[\operatorname{mcd}(f(x), g(x)) = \operatorname{mcd}(g(x), r(x)).\]
  \end{enumerate}
\end{theorem}

% 3.4 Constructing finite fields

Un polinomio no constante \(f(x) \in \mathbb F_q[x]\) es \textit{irreducible sobre} \(\mathbb F_q\) si no es posible factorizarlo como producto de dos polinomios de \(\mathbb F_q[x]\) de grado menor.

\begin{theorem}
  Sea \(f(x)\) un polinomio no constante. Entonces, 
  \[
    f(x) = p_1(x)^{a_1}p_2(x)^{a_2}\dots p_k(x)^{a_k},
  \]
  donde cada \(p_i(x)\) es irreducible, los polinomios \(p_i(x)\) son únicos salvo producto por escalares y los elementos \(a_i\) son únicos.
\end{theorem}

Esto nos dice que \(\mathbb F_q[x]\) es lo que se conoce como \textit{dominio de factorización única}.
Pero es además un dominio de ideales principales.

\begin{proof}
  % TODO: demostración de que F_q[x] es un dominio de ideales principales.
  % Ejercicio 153 (p. 102 [121]), F_q[x] es un anillo conmutativo con unidad y un dominio de integridad
  % Ejercicio 166 (p. 106 [125]), cada ideal de F_q[x] es un ideal principal
\end{proof}

Para construir un cuerpo de característica \(p\) comenzamos con un polinomio \(f(x) \in \mathbb F_q[x]\) que es irreducible sobre \(\mathbb F_q\) y que tiene grado \(m\).
Usando el algoritmo de euclides podemos demostrar que el anillo cociente dado por \(\mathbb F_p[x]/(f(x))\) es un cuerpo, y de hecho, un cuerpo finito con \(q = p^m\) elementos.

\begin{proof}
  % TODO: demostración de que F_p[x]/(f(x)) es un cuerpo de q = p^m elementos
  % Ejercicio 167 (p. 107 [126])
\end{proof}

Cada elemento de dicho anillo cociente es una clase lateral de la forma \(g(x) + (f(x))\), donde \(g(x)\) es único y tiene grado ocmo mucho \(m-1\).

Escribiremos cada clase lateral como un vector en \(\mathbb F_p^m\) siguiendo la correspondencia:
\[
  g_{m-1}x^{m-1} + g_{m-2}x^{m-2}+ \dots + g_1x + g_0 + (f(x)) \iff g_{m-1}g_{m-2}\dots g_1g_0.
\]

Esta notación de vectorial nos permite realizar la suma en el cuerpo utilizando la suma habitual de los vectores.
Multiplicar es una tarea a priori más complicada.
Para multiplicar \(g_1(x) + (f(x))\) por \(g_2(x) + (f(x))\) primero utilizamos al algoritmo de división para escribir
\[
  g_1(x)g_2(x) = f(x)h(x) + r(x),
\]
donde como sabemos o bien \(\operatorname{gr} r(x) \leq m -1\) o bien \(r(x) = 0\).
Puesto que estamos en el anillo cociente \(\mathbb F_p[x]/(f(x))\) nos queda
\[
  (g_1(x) + (f(x)))(g_2(x) + (f(x))) = r(x) + (f(x)).
\]
Esta notación es engorrosa, por lo que habitualmente operaremos en \(\alpha\) en vez de en \(x\) suponiendo que \(f(\alpha) = 0\).
Así, \(g_1(\alpha)g_2(\alpha) = r(\alpha)\).
% TODO: correspondencia vectorial de los polinomios en alpha (3.4)
En consecuencia, multiplicamos los polinomios en \(\alpha\) de la forma habitual y utilizamos la ecuación \(f(\alpha) = 0\) para reducir las potencias de \(\alpha\) de grado mayor a \(m-1\) a polinomios en \(\alpha\) de grado menor que \(m\).

El conjunto \(\{0\alpha^{m-1} + 0\alpha^{m-2} + \dots + 0\alpha + a_0 \mid a_0 \in \mathbb F_p\} = \{a_0 \mid a_0 \in \mathbb F_p\}\) es el subcuerpo primo de \(\mathbb F_q\).

Decimos que obtenemos \(\mathbb F_q\) a partir de \(\mathbb F_p\) yuxtaponiendo una raíz \(\alpha\) de \(f(x)\) a \(\mathbb F_p\).
Esta raíz viene dada formalmente por \(\alpha = x + (f(x))\) en el anillo cociente \(\mathbb F_p[x]/(f(x))\).
Por tanto, ya hemos visto antes que \(g(x) + (f(x)) = g(\alpha)\) y \(f(\alpha) = f(x + (f(x))) = f(x) + (f(x)) = 0 + (f(x))\).

Un polinomio irreducible sobre \(\mathbb F_p\) de grado \(m\) es \textit{primitivo} si tiene una raíz que es un elemento primitivo de \(\mathbb F_q = \mathbb F_{p^m}\).
Puede probarse que existen polinomios irreducibles de cualquier grado.
% TODO: probar? referencia prueba? mejorar afirmación?

\begin{theorem}
  Para cualquier primo \(p\) y cualquier entero positivo \(m\), existe un cuerpo finito, único salvo isomorfismos, con \(q = p^m\) elementos.
\end{theorem}

% 3.7 Cyclotomic cosets and minimal polynomials


\section{Automorfismos}

% \section{Monoides}

% \begin{definition}
%   Un \textit{monoide} es un conjunto \(S\) con un elemento \(e\) y una aplicación \(\mu: S^2 \to S\) tal que si \(\mu(x, y)\) es el resultado de aplicar \(\mu\) a la pareja de elementos \(x, y \in S\), se verifican: \begin{enumerate}
%     \item \(\mu(x, \mu(y, z)) = \mu(\mu(x, y), z)\) para todo \(x, y, z \in S\).
%     \item \(\mu(e, x) = \mu(x, e) = x\) para todo \(x \in S\).
%   \end{enumerate}
% \end{definition}

% Obsérvese que, por definición, un monoide siempre tiene al menos un elemento. A la aplicación \(\mu\) que actúa sobre parejas de elementos se le llama \textit{operación binaria} y al elemento \(e\), elemento neutro de \(\mu\).

% \section{Anillos}

% \begin{definition}
%   Un \textit{anillo}.
% \end{definition}