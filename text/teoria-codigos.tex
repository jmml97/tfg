\chapter{Fundamentos de teoría de códigos}

La teoría de códigos 

\section{Códigos lineales}

Vamos a comenzar nuestro estudio con los códigos lineales, pues son los más sencillos de comprender. Consideremos el espacio vectorial de todas las \(n\)-tuplas sobre el cuerpo finito \(\mathbb F_q\), al que denotaremos en lo que sigue como \(\mathbb F_q^n\). Los elementos \((a_1, \dots, a_n)\) de \(\mathbb F_q^n\) los notaremos usualmente como \(a_1\!\cdots a_n\).

\begin{definition}
  Un \((n, M)\) \textit{código} \(\mathcal C\) sobre el cuerpo \(\mathbb F_q\) es un subconjunto de \(\mathbb F_q^n\) de tamaño \(M\). A los elementos de \(\mathcal C\) los llamaremos \textit{palabras codificadas} —o \textit{codewords} en inglés—.
\end{definition}

Es necesario añadir más estructura a los códigos para que sean de utilidad.

\begin{definition}
  Decimos que un código \(\mathcal C\) es \([n, k]\) \textit{lineal} si es un subespacio vectorial de \(\mathbb F_q^n\) de dimensión \(k\).
\end{definition}

Un código lineal \(\mathcal C\) tiene \(q^k\) palabras codificadas.
