\chapter{Fundamentos de teoría de códigos}

La teoría de códigos tal y cual. Las definiciones aquí ... según lo descrito en \parencite[1-48]{huffman-pless-2003}.

\section{Códigos lineales}

Vamos a comenzar nuestro estudio con los códigos lineales, pues son los más sencillos de comprender. Consideremos el espacio vectorial de todas las \(n\)-tuplas sobre el cuerpo finito \(\mathbb F_q\), al que denotaremos en lo que sigue como \(\mathbb F_q^n\). Los elementos \((a_1, \dots, a_n)\) de \(\mathbb F_q^n\) los notaremos usualmente como \(a_1\!\cdots a_n\).

\begin{definition}
  Un \((n, M)\) \textit{código} \(\mathcal C\) sobre el cuerpo \(\mathbb F_q\) es un subconjunto de \(\mathbb F_q^n\) de tamaño \(M\). A los elementos de \(\mathcal C\) los llamaremos \textit{palabras codificadas} —o \textit{codewords} en inglés—.
\end{definition}

Es necesario añadir más estructura a los códigos para que sean de utilidad.

\begin{definition}
  Decimos que un código \(\mathcal C\) es \([n, k]\) \textit{lineal} si es un subespacio vectorial de \(\mathbb F_q^n\) de dimensión \(k\).
\end{definition}

Un código lineal \(\mathcal C\) tiene \(q^k\) palabras codificadas.

\begin{definition}
  Una \textit{matriz generadora} para un \([n, k]\) código \(\mathcal C\) es una matriz \(k \times n\) cuyas filas conforman una base de \(\mathcal C\).
\end{definition}

Para cada conjunto \(k\) de columnas independientes de una matriz generadora \(G\) el conjunto de coordenadas correspondiente se denomina \textit{conjunto de información} para un código \(\mathcal C\). Las \(r = n - k\) coordenadas restantes se llaman \textit{conjunto redundante} y \(r\), la \textit{redundancia} de \(\mathcal C\).

Si las primeras \(k\) coordenadas forman un conjunto de información el código tiene una única matriz generadora de la forma \([I_k \mid A]\), donde \(I_k\) es la matriz identidad \(k \times k\). Esta matriz generadora se dice que está en \textit{forma estándar}.

Como un código lineal es el subespacio de un espacio vectorial, es el núcleo de una transformación lineal. En particular, existe una matriz \(H\) de dimensiones \((n - k) \times n\), llamada \textit{matriz de comprobación de paridad} para un \([n, k]\) código \(\mathcal C\) definida por \begin{equation}
  \mathcal C = \left\{x \in \mathbb F_q^n : H \boldsymbol x^T = 0 \right\}.
\end{equation}
