\chapter{Fundamentos de teoría de códigos}

La teoría de códigos (...). Las definiciones y los resultados comentados en esta sección seguirán lo descrito en \parencite[1-48]{huffman_fundamentals_2003} y \parencite{podesta_introduccion_2006}.

\section{Códigos lineales}

Vamos a comenzar nuestro estudio con los códigos lineales, pues son los más sencillos de comprender. Consideremos el espacio vectorial de todas las \(n\)-tuplas sobre el cuerpo finito \(\mathbb F_q\), al que denotaremos en lo que sigue como \(\mathbb F_q^n\). A los elementos \((a_1, \dots, a_n)\) de \(\mathbb F_q^n\) los notaremos usualmente como \(a_1\!\cdots a_n\).

\begin{definition}
  Un \((n, M)\) \textit{código} \(\mathcal C\) sobre el cuerpo \(\mathbb F_q\) es un subconjunto de \(\mathbb F_q^n\) de tamaño \(M\). A los elementos de \(\mathcal C\) los llamaremos \textit{palabras codificadas} —o \textit{codewords} en inglés—.
\end{definition}

Es necesario añadir más estructura a los códigos para que sean de utilidad.

\begin{definition}
  Decimos que un código \(\mathcal C\) es un código \textit{lineal de longitud \(n\) y rango \(k\)} —abreviado como \([n, k]\) \textit{lineal}— si dicho código es un subespacio vectorial de \(\mathbb F_q^n\) de dimensión \(k\).
\end{definition}

Un código lineal \(\mathcal C\) tiene \(q^k\) palabras codificadas.

\begin{definition}
  Una \textit{matriz generadora} para un \([n, k]\) código \(\mathcal C\) es una matriz \(k \times n\) cuyas filas conforman una base de \(\mathcal C\).
\end{definition}

Veamos un ejemplo de matriz generadora. Consideremos la matriz \(G = \begin{psmallmatrix}
  1 & 1 & 0 \\ 0 & 1 & 1
\end{psmallmatrix} \in \mathcal M_{2 \times 3}(\mathbb F_2)\). Dicha matriz genera un \([3, 2]\) código binario, pues dado \((x_1, x_2)\), se tiene que \[(x_1, x_2) \begin{pmatrix}
  1 & 1 & 0 \\ 0 & 1 & 1
\end{pmatrix} = (x_1, x_1 + x_2, x_2),\] y por tanto este código codifica de la forma \[00 \to 000, \quad 01 \to 011,\quad 10 \to 110,\quad 11 \to 101.\]

\begin{definition}
  Para cada conjunto \(k\) de columnas independientes de una matriz generadora \(G\) el conjunto de coordenadas correspondiente se denomina \textit{conjunto de información} para un código \(\mathcal C\). Las \(r = n - k\) coordenadas restantes se llaman \textit{conjunto redundante}, y el número \(r\), la \textit{redundancia} de \(\mathcal C\).
\end{definition}

Si las primeras \(k\) coordenadas forman un conjunto de información el código tiene una única matriz generadora de la forma \([I_k \mid A]\), donde \(I_k\) es la matriz identidad \(k \times k\) y \(A\) es una matriz \(k \times r\). Esta matriz generadora se dice que está en \textit{forma estándar}.

%Como un código lineal es el subespacio de un espacio vectorial, es el núcleo de una transformación lineal. En particular, existe una matriz \(H\) de dimensiones \(r \times n\), llamada \textit{matriz de comprobación de paridad} para un \([n, k]\) código \(\mathcal C\) definida por \begin{equation}
%  \mathcal C = \left\{x \in \mathbb F_q^n : H \boldsymbol x^T = 0 %\right\}.
%\end{equation}

Como un código lineal \(\mathcal C\) es un subespacio de un espacio vectorial, podemos calcular el ortogonal a dicho subespacio, obteniendo lo que llamaremos el \textit{código dual} \(\mathcal C^{\perp}\).

\begin{definition}
  El \textit{código dual} \(\mathcal C^{\perp}\) de un código \(\mathcal C\) viene dado por \[\mathcal C^{\perp} = \left\{x \in \mathbb F_q^n : x \cdot c = 0 \quad \forall c \in \mathcal C\right\}.\]
\end{definition}

\begin{definition}
  Sea \(\mathcal C\) un \([n, k]\) código lineal. Una matriz \(H\) se dice que es \textit{matriz de paridad} si es una matriz generadora de \(\mathcal C^{\perp}\).
\end{definition}

\begin{proposition}
  Sea \(H\) la matriz de paridad de un \([n, k]\) código lineal \(\mathcal C\). Entonces, \[\mathcal C = \left\{x \in \mathbb F_q^n : xH^T = 0\right\} = \left\{x \in F_q^n : Hx^T = 0\right\}.\]
\end{proposition}

\begin{proof}
  Sea \(c \in \mathcal C\) una palabra codificada. Sabemos que la podemos expresar como \(c = uG\), donde \(u \in \mathbb F_q^k\) y \(G\) es la matriz generadora de \(\mathcal C\). Tenemos entonces que \(c\cramped{H^T} = uG\cramped{H^T}\) y como \(G\cramped{H^T} = 0\) —por ser H matriz generadora del subespacio ortogonal \(\mathcal C\)— se tiene que \[\mathcal C \subset S_H = \left\{x \in \mathbb F_q^n : Hx^T = 0\right\},\] que es el espacio solución de un sistema de \(n - k\) ecuaciones con \(n\) incógnitas y de rango \(n - k\). Como \(\dim(S_H) = n - (n - k) = k = \dim L\), concluimos que \[L = S_H = \left\{x \in \mathbb F_q^n : Hx^T = 0\right\}.\qedhere\]
\end{proof}

El resultado anterior, junto a la definición anterior, nos conducen al siguiente teorema. 

\begin{theorem}
  Si \(G = [I_k \mid A]\) es una matriz generadora para un \([n, k]\) código \(\mathcal C\) en forma estándar entonces \(H = [-A \mid I_{n-k}]\) es una matriz de paridad para \(\mathcal C\).
\end{theorem}

Un código se dice \textit{autoortogonal} cuando \(\mathcal C \subseteq \mathcal C^{\perp}\), y \textit{autodual} cuando \(\mathcal C = \mathcal C^{\perp}\).

\subsection{Distancias y pesos}

\begin{definition}
  La \textit{distancia de Hamming} \(\operatorname{d}(\symbf{x}, \symbf{y})\) entre dos vectores \(\symbf{x}, \symbf{y} \in \mathbb F_q^n\) se define como el número de coordenadas en las que difieren \(\symbf{x}\) e \(\symbf{y}\).
\end{definition}

\begin{theorem}
  La función de distancia \(\operatorname{d}(\symbf{x}, \symbf{y})\) verifica las siguientes propiedades.
  \begin{enumerate}
    \item No negatividad: \(\operatorname{d}(\symbf{x}, \symbf{y}) \geq 0\) para todo \(\symbf{x}, \symbf{y}\in \mathbb F_q^n\).
    \item \(\operatorname{d}(\symbf{x}, \symbf{y}) = 0\) si y solo si \(\symbf{x} = \symbf{y}\).
    \item Simetría: \(\operatorname{d}(\symbf{x}, \symbf{y}) = \operatorname{d}(\symbf{y}, \symbf{x})\) para todo \(\symbf{x}, \symbf{y}\in \mathbb F_q^n\).
    \item Desigualdad triangular: \(\operatorname{d}(\symbf{x}, \symbf{z}) \leq \operatorname{d}(\symbf{x}, \symbf{y}) + \operatorname{d}(\symbf{y}, \symbf{z})\) para todo elemento \(\symbf{x}, \symbf{y}, \symbf{z}\in \mathbb F_q^n\).
  \end{enumerate}
\end{theorem}

La \textit{distancia (mínima)} de un código \(\mathcal C\) es la menor distancia posible entre dos palabras codificadas distintas. 
Es importante a la hora de determinar la capacidad de corrección de errores del código \(\mathcal C\), pues como veremos más tarde, a mayor distancia mínima, mayor número de errores en el código se pueden corregir.

\begin{definition}
  El \textit{peso de Hamming} \(\operatorname{wt}(\symbf{x})\) de un vector \(\symbf{x}\) es el número de coordenadas distintas de cero de \(\symbf{x}\).
\end{definition}

\begin{theorem}
  Si \(\symbf{x}, \symbf{y} \in \mathbb F_q^n\), entonces \(\operatorname{d}(\symbf{x}, \symbf{y}) = \operatorname{wt}(\symbf{x} - \symbf{y})\).
  Si \(\mathcal C\) es un código lineal, la distancia mímina es igual al peso mínimo de las palabras codificadas de \(\mathcal C\) distintas de cero.
\end{theorem}

Como consecuencia de este teorema —para códigos lineales— la distancia mínima también se llama \textit{peso mínimo} del código.
Si el peso mínimo \(d\) de un \([n,k]\) código es conocida, nos referiremos a él como un \([n,k,d]\) código.

\begin{definition}
  Sea \(A_i(\mathcal C)\) —que abreviaremos \(A_i\)— el número de palabras codificadas de peso \(i\) en \(\mathcal C\).
  Para cada \(0 \leq i \leq n\), la lista \(A_i\) se denomina \textit{distribución de peso} o \textit{espectro de peso} de \(\mathcal C\).
\end{definition}

\begin{example}
  Sea \(\mathcal C\) el código binario con matriz generadora
  \[
    G = \begin{pmatrix}
      1 & 1 & 0 & 0 & 0 & 0\\
      0 & 0 & 1 & 1 & 0 & 0 \\
      0 & 0 & 0 & 0 & 1 & 1
    \end{pmatrix}.
  \]
  Dado \((x_1, x_2, x_3)\), se tiene que \[(x_1, x_2, x_3) \begin{pmatrix}
    1 & 1 & 0 & 0 & 0 & 0\\
      0 & 0 & 1 & 1 & 0 & 0 \\
      0 & 0 & 0 & 0 & 1 & 1
  \end{pmatrix} = (x_1, x_1, x_2, x_2, x_3, x_3),\] y por tanto este código codifica de la forma 
  \[
    000 \to 000000, \quad 
    001 \to 000011,\quad 
    010 \to 001100,\quad 
    011 \to 001111,
  \]
  \[
    100 \to 110000, \quad 
    101 \to 110011,\quad 
    110 \to 111100,\quad 
    111 \to 111111.
  \]
  Luego la distribución de peso de \(\mathcal C\) es \(A_0 = A_6 = 1\) y \(A_2 = A_4 = 3\).
  Usualmente solo se listan los \(A_i\) que son distintos de cero.
\end{example}

\begin{theorem}
  Sea \(\mathcal C\) un \([n,k,d]\) código sobre \(\mathbb F_q\).
  Entonces, \begin{enumerate}
    \item \(A_0(\mathcal C) + A_1(\mathcal C) + \cdots + A_n(\mathcal C) = q^k\).
    \item \(A_0(\mathcal C) = 1\) y \(A_1(\mathcal C) = A_2(\mathcal C) = \cdots = A_{d-1}(\mathcal C) = 0\).
  \end{enumerate}
\end{theorem}

% TODO: teoremas sobre códigos que cumplen ciertas condiciones, no sé si merece Huffman 9-11



\section{Códigos cíclicos}

\section{Algoritmo de Peterson-Gorenstein-Zierler}