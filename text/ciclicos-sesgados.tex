\chapter{Códigos cíclicos sesgados}

En este capítulo definiremos los códigos cíclicos sesgados sobre un cuerpo finito \(\mathbb F_q\).
Las principales fuentes de este capítulo son \parencite{gomez-torrecillas_new_2016}, \parencite{gomez-torrecillas_petersongorensteinzierler_2018}, \parencite{shi_codes_2017} y \parencite{lam_vandermonde_1988}.

Consideremos el anillo de polinomios de Ore \(R = \mathbb F_q[x; \sigma]\).
Supondremos que el orden del automorfismo \(\sigma\) es \(n\).
Entonces por el teorema \ref{th:anillos-ore-centro} el polinomio \(x^n - 1\) es central en \(R\) y por tanto el ideal \((x^n - 1)\) es bilátero, por lo que podemos definir el anillo cociente \(\mathcal R = \mathbb F_q[x; \sigma]/(x^n - 1)\).
Este cociente \(\mathcal R\) es isomorfo a \(\mathbb F_q^n\) mediante la aplicación de coordenadas \(\mathfrak v : \mathcal R \to \mathbb F_q^n\).

\begin{definition}
  Un \emph{código cíclico sesgado} sobre \(\mathbb F_q\) es un subespacio vectorial \(\mathcal C \subseteq \mathbb F_q^n\) tal que \(\mathfrak v^{-1}(\mathcal C)\) es un ideal por la izquierda de \(\mathcal R\).
  Equivalentemente, es un subespacio vectorial \(\mathcal C \subseteq \mathbb F_q^n\) tal que si \((a_0, \dots, a_{n-2}, a_{n-1}) \in \mathcal C\) entonces \((\sigma(a_{n-1}), \sigma(a_0), \dots, \sigma(a_{n-2})) \in \mathcal C\).
\end{definition}


Sabemos que todo ideal por la izquierda de \(\mathcal R\) es principal, por lo que todo código cíclico sesgado está generado por un polinomio en \(R\).
De forma análoga a como ocurría con los códigos cíclicos, este generador es un divisor por la derecha de \(x^n - 1\), por lo que nos interesa conocer de nuevo la descomposición de \(x^n - 1\) en factores, esta vez sobre \(R\).

No existe un algoritmo de factorización completo para los polinomios de Ore, por lo que vamos a utilizar a continuación un método concreto para \(x^n - 1\), descrito en \parencite{gomez-torrecillas_new_2016}.
Por el teorema de la base normal podemos tomar un elemento \(\alpha \in \mathbb F_q\) tal que \(\{\alpha, \sigma(\alpha), \dots, \sigma^{n-1}(\alpha)\}\) sea una base de  \(\mathbb F_q\) como \(\mathbb F_q^{\sigma}\)-espacio vectorial.
Fijamos en lo que sigue \(\beta = \alpha^{-1}\sigma(\alpha)\).

\begin{lemma}
  \label{lem:pol-t-beta}
  Para cada subconjunto \(\{t_1, t_2, \dots, t_m\} \subseteq \{0, 1, \dots, n - 1\}\) el polinomio 
  \[
    g = \left[x - \sigma^{t_1}(\beta), x - \sigma^{t_{2}}(\beta), \dots, x - \sigma^{t_m}(\beta)\right]_{i}
  \]
  tiene grado \(m\).
  Por tanto, si \(x - \sigma^s(\beta) \mid_d g\) entonces \(s \in T\).
\end{lemma}

\begin{proof}
  Vamos a proceder por reducción al absurdo, suponiendo que \(\deg(g) < m\), de forma que \(g = \sum_{i=0}^{m-1}g_ix^{i}\).
  Como \(g\) es un múltiplo por la izquierda de \(x - \sigma^{t_1}(\beta )\) para todo \(1 \leq j \leq m\) por la proposición \ref{prop:norma-divisor} se tiene que
  \[
    \sum_{i=0}^{m-1}g_iN_i(\sigma^{t_j}(\beta)) = 0 \quad\text{para todo } 1 \leq j \leq m.
  \]
  Estas ecuaciones definen un sistema homogéneo, cuya matriz de coeficientes es la transpuesta de 
  \[
    M = \begin{pmatrix}
      N_0(\sigma^{t_1}(\beta)) & N_0(\sigma^{t_2}(\beta)) & \dots & N_0(\sigma^{t_m}(\beta))\\
      N_1(\sigma^{t_1}(\beta)) & N_1(\sigma^{t_2}(\beta)) & \dots & N_1(\sigma^{t_m}(\beta))\\
      N_2(\sigma^{t_1}(\beta)) & N_2(\sigma^{t_2}(\beta)) & \dots & N_2(\sigma^{t_m}(\beta))\\
      \vdots & \vdots & \ddots & \vdots \\
      N_{m-1}(\sigma^{t_1}(\beta)) & N_{m-1}(\sigma^{t_2}(\beta)) & \dots & N_{m-1}(\sigma^{t_m}(\beta))\\
    \end{pmatrix}
  \]
  Pero por (\ref{eq:norma-beta}) \(|M| = 0\) si y solo si el determinante de la matriz
  \begin{align*}
    M' &= \begin{pmatrix}
      \sigma^{t_1}(\alpha) & \sigma^{t_2}(\alpha) & \dots & \sigma^{t_m}(\alpha)\\
      \sigma^{t_1+1}(\alpha) & \sigma^{t_2+1}(\alpha) & \dots & \sigma^{t_m+1}(\alpha)\\
      \sigma^{t_1+2}(\alpha) & \sigma^{t_2+2}(\alpha) & \dots & \sigma^{t_m+2}(\alpha)\\
      \vdots & \vdots & \ddots & \vdots \\
      \sigma^{t_1+m-1}(\alpha) & \sigma^{t_2+m-1}(\alpha) & \dots & \sigma^{t_m+m-1}(\alpha)\\
    \end{pmatrix}\\
      &= \begin{pmatrix}
        \sigma^{t_1}(\alpha) & \sigma^{t_2}(\alpha) & \dots & \sigma^{t_m}(\alpha)\\
        \sigma(\sigma^{t_1}(\alpha)) & \sigma(\sigma^{t_2}(\alpha)) & \dots &\sigma(\sigma^{t_m}(\alpha))\\
        \sigma^2(\sigma^{t_1}(\alpha)) & \sigma^2(\sigma^{t_2}(\alpha)) & \dots &\sigma^2(\sigma^{t_m}(\alpha))\\
        \vdots & \vdots & \ddots & \vdots \\
        \sigma^{m-1}(\sigma^{t_1}(\alpha)) & \sigma^{m-1}(\sigma^{t_2}(\alpha)) & \dots &\sigma^{m-1}(\sigma^{t_m}(\alpha))\\
      \end{pmatrix}
  \end{align*}
  es cero.
  Pero por \parencite[Lema 2.1]{gomez-torrecillas_petersongorensteinzierler_2018} \(|M'| \neq 0\), por lo que la única solución del sistema anterior es \(g_0 = \dots = g_{m-1} = 0\), lo que es una contradicción.
  Por tanto, \(\deg(g) = m\).
\end{proof}

\begin{corollary}
  Se tiene que
  \[
  x^n - 1 = \left[x - \beta, x - \sigma(\beta), \dots, x - \sigma^{n-1}(\beta)\right]_i
  \]
\end{corollary}

\begin{proof}
  Como consecuencia del lema \ref{lem:pol-t-beta} se tiene que 
  \[
    \left[x - \beta, x - \sigma(\beta), \dots, x - \sigma^{n-1}(\beta)\right]_i
  \]
  tiene grado \(n\).
  Pero además, por (\ref{eq:norma-beta}) se tiene que
  \[
  N_n(\sigma^k(\beta)) = \sigma^k(\alpha)^{-1}\sigma^{k+n}(\alpha) = \sigma^k(\alpha^{-1}\alpha) = 1
  \]
  y por tanto 
  \[
    -1N_0(\sigma^k(\beta)) + 1N_n(\sigma^k(\beta)) = -1 + 1 = 0,
  \]
  por lo que por la proposición \ref{prop:norma-divisor} cada \(x - \sigma^k(\beta)\) divide a \(x^n -1\) por la derecha para todo \(0 \leq k \leq n -1\).
  Estas dos afirmaciones nos conducen a que 
  \[
  x^n - 1 = \left[x - \beta, x - \sigma(\beta), \dots, x - \sigma^{n-1}(\beta)\right]_i,
  \]
  como queríamos. 
\end{proof}

Este corolario nos permite afirmar que dados \(\{t_1, \dots, t_k\} \subset \{0, 1, \dots, n - 1\}\) el polinomio \(g = [x - \sigma^{t_1}(\beta), \dots, x - \sigma^{t_k}(\beta)]_i\) genera un ideal por la izquierda \(\mathcal Rg\) tal que \(\mathfrak v(\mathcal Rg)\) es un código cíclico sesgado de dimensión \(n - k\).

Antes de proceder a definir el tipo de códigos cíclicos sesgados que vamos a utilizar es necesario que comentemos algunos resultados que necesitaremos más adelante.
Comenzamos indicando que llamaremos \(\beta\)-raíces a los elementos del conjunto \(\{\beta, \sigma(\beta), \dots, \sigma^{n-1}(\beta)\}\).
Por la proposición \ref{prop:norma-divisor} y (\ref{eq:norma-beta}) se tiene que, dado un polinomio \(f = \sum_{i=0}^{n-1}a_ix^i \in \mathcal R\),
\begin{equation}
  \label{eq:equivalencias-divisor-sigma-b}
  x - \sigma^j(\beta) \mid_d f \iff \sum_{i=0}^{n-1}a_iN_i(\sigma^j(\beta)) = 0 \iff \sum_{i=0}^{n-1}a_i\sigma^{i+j}(\alpha) = 0.
\end{equation}
Sea \(N\) la matriz formada por las normas de las \(\beta\)-raíces:
\[
  N = \begin{pmatrix}
    N_0(\beta) & N_0(\sigma(\beta)) & \cdots & N_0(\sigma^{n-1}(\beta))\\
    N_1(\beta) & N_1(\sigma(\beta)) & \cdots & N_1(\sigma^{n-1}(\beta))\\
    \vdots & \vdots & \ddots & \vdots\\
    N_{n-1}(\beta) & N_{n-1}(\sigma(\beta)) & \cdots & N_{n-1}(\sigma^{n-1}(\beta))\\
  \end{pmatrix}.
\]
Las componentes de \(\mathfrak v(f)N = (a_1, \dots, a_{n-1})N\) son las evaluaciones por la derecha de \(f\) en el conjunto de las \(\beta\)-raíces, es decir, el vector compuesto por los restos por la izquierda obtenidos al dividir \(f\) por los polinomios \(x - \sigma^i(\beta)\) para \(i = 0, \dots, n - 1\).
Por tanto, tenemos que el diagrama es conmutativo
\begin{center}
  \begin{tikzcd}[column sep=large, row sep=large]
    \mathcal R \arrow[d, "\mathfrak v"] \arrow[rd, bend left, "ev"] & \\
    \mathbb F_q^n \arrow[r, "\cdot N"] & \mathbb F_q^n
  \end{tikzcd}
\end{center}
en el que \(ev\) representa una aplicación que lleva cada polinomio \(f\) en la \(n\)-tupla formada por los restos por la izquierda de dividir \(f\) por \(x - \sigma^i(\beta)\), para \(i = 0, \dots, n - 1\).
Es posible probar que la matriz \(N\) es no singular \parencite[ver][Lema 2.1]{gomez-torrecillas_petersongorensteinzierler_2018}, por lo que es un cambio de base de \(\mathbb F_q^n\).

Dado un polinomio \(f\) llamamos \emph{conjunto de}\(\beta\)\emph{-raíces} del polinomio \(f\) al conjunto formado por las \(\beta\)-raíces \(\gamma\) que cumplen \(x - \gamma \mid_d f\), es decir, a aquellas correspondientes a las coordenadas nulas de \((a_0, \dots, a_{n-1})N\).
Decimos que un divisor por la derecha no constante \(f \mid_d x^n - 1\) \(\beta\)\emph{-descompone totalmente} si existen \(\{t_1, \dots, t_m\} \subseteq \{0, 1, \dots, n-1\}\) tales que
\[
  f = \left[x - \sigma^{t_1}(\beta), \dots, x - \sigma^{t_m}(\beta)\right]_{i}.
\]
Sabemos por el lema \ref{lem:pol-t-beta} que \(\deg f = m\), el cardinal del conjunto de las \(\beta\)-raíces de \(f\).

\begin{lemma}
  \label{lem:matriz-mf-base}
  Sea \(f = \sum_{i=0}^m a_ix^i \in \mathcal R\) con \(a_m \neq 0\) y
  \[
    M_f = \begin{pmatrix}
      a_0 & a_1 & \cdots & a_m & 0 & \cdots & 0 \\
      0 & \sigma(a_0) & \cdots & \sigma(a_{m-1}) & \sigma(a_m) & \cdots & 0 \\
       &  & \ddots &  &  & \ddots & 0 \\
      0 & \cdots & 0 & \sigma^{n-m-1}(a_0) & \cdots & \cdots & \sigma^{n-m-1}(a_m) \\
    \end{pmatrix}_{(n-m) \times n}.
  \]
  Entonces las filas de \(M_f\) son la base de \(\mathfrak v(\mathcal Rf)\) como un \(\mathbb F_q\)-espacio vectorial.
  Es más, \(f\) \(\beta\)-descompone totalmente si y solo si 
  \[
    \operatorname{mepf}(M_jN) = \left( \begin{array}{@{}c@{}}
      \varepsilon_{i_1}\\\hline
      \vdots\\\hline
      \varepsilon_{i_{m-n}}
    \end{array}\right)
  \]
  para algunos \(0 \leq i_1 < \dots < i_{n-m} \leq n -1\), donde \(\operatorname{mepf}\) denota una matriz escalonada por filas y \(\varepsilon_i\) es un vector canónico de longitud \(n\)\footnote{Recordemos que un vector canónico \(\varepsilon_i\) de longitud \(n\) es de la forma \((0, \dots, 1, \dots, 0)\), una tupla de \(n\) elementos formada por \(0\) en todas las posiciones salvo en la \(i\)-ésima, que es \(1\)}.
\end{lemma}
\begin{proof}
  Una \(\mathbb F_q\)-base de \(\mathcal Rf\) es \(\{f, xf, \dots, x^{n-m-1}f\}\) cuyas coordenadas corresponden precisamente a las filas de \(M_f\).
  Teniendo esto en cuenta sabemos que \(f = \left[x - \sigma^{t_1}(\beta), \dots, x - \sigma^{t_m}(\beta)\right]_i\) si y solo si cada múltiplo por la izquierda de \(f\) es también múltiplo por la izquierda de \(x - \sigma^{t_i}(\beta)\) para \(1 \leq i \leq m\).
  Ahora bien, recordemos que el producto matricial \(M_fN\) representa en cada fila los restos por la izquierda obtenidos al dividir el polinomio correspondiente a los coeficientes de \(M_f\) en esa fila por los polinomios \(x - \sigma^i(\beta)\) para \(i = 0, \dots, n - 1\).
  Es evidente entonces que \(f = \left[x - \sigma^{t_1}(\beta), \dots, x - \sigma^{t_m}(\beta)\right]_i\) si y solo si las \(t_i\)-ésimas columnas de \(M_fN\) son cero para \(i = 1, \dots, m\).
  Como \(M_fN\) tiene \(n - m\) filas, rango \(n - m\) y \(n - m\) columnas distintas de cero, la forma escalonada por filas de la matriz \(M_jN\) será justamente la que hemos descrito.
\end{proof}

En otras palabras, la proposición \ref{lem:matriz-mf-base} nos dice que matriz \(M_f\) ahí definida es una matriz generadora del código \(\mathfrak v(\mathcal R f)\).

\begin{lemma}
  \label{lem:b-descomposicion-mcm-mcd}
  Sean \(f, g \in \mathcal R\) polinomios que pueden \(\beta\)-descomponerse totalmente.
  Entonces \((f, g)_d\) y \([f, g]_i\) también pueden \(\beta\)-descomponerse totalmente.
\end{lemma}

\begin{proof}
  Como \(f\) y \(g\) pueden \(\beta\)-descomponerse totalmente existen subconjuntos \(T_1, T_2 \subseteq \{0, \dots, n - 1\}\) tales que 
  \[
    f = \left[\{x - \sigma^i(\beta)\}_{i \in T_1}\right]_i \quad \text{y} \quad g = \left[\{x - \sigma^i(\beta)\}_{i \in T_2}\right]_i.
  \]
  Se deduce rápidamente entonces que
  \[
    [f, g]_i = \left[\{x - \sigma^i(\beta)\}_{i \in T_1 \cup T_2}\right]_i.
  \]
  Por otro lado es evidente que 
  \[
    \left[\{x - \sigma^i(\beta)\}_{i \in T_1 \cap T_2}\right]_i \mid_d (f, g)_d,
  \]
  pero como \(\deg(f) + \deg(g) = \deg((f, g)_d) + \deg([f, g]_i)\) por el lema \ref{lem:pol-t-beta} se tiene la igualdad.
\end{proof}

Con todas estas herramientas ya podemos definir la clase de códigos que vamos a utilizar en el siguiente capítulo.

\begin{definition}
  Bajo las condiciones y notación de este capítulo, un \emph{código RS (Reed-Solomon) sesgado} de distancia mínima prevista \(\delta\) es un código cíclico \(\mathcal C\) tal que \(\mathfrak v^{-1}(\mathcal C)\) está generado por 
  \[
    \left[x - \sigma^r(\beta), x - \sigma^{r+1}(\beta), \dots, x - \sigma^{r+\delta-2}(\beta)\right]_i
  \] 
  para algún \(r \geq 0\).
\end{definition}

\begin{theorem}
  \label{th:distancia-skew-rs}
  Un código RS sesgado de distancia mínima prevista \(\delta\) tiene distancia mínima \(\delta\).
\end{theorem}

\begin{proof}
  Vamos a denotar por
  \[
    g = \left[x - \sigma^r(\beta), x - \sigma^{r+1}(\beta), \dots, x - \sigma^{r+\delta-2}(\beta)\right]_i
  \]
  al generador de \(\mathcal C\).
  Una matriz de paridad de este código es
  \[
    H = \begin{pmatrix}
      N_0(\sigma^{r}(\beta)) & N_0(\sigma^{r+1}(\beta)) & \dots & N_0(\sigma^{r+\delta-2}(\beta))\\
      N_1(\sigma^{r}(\beta)) & N_1(\sigma^{r+1}(\beta)) & \dots & N_1(\sigma^{r+\delta-2}(\beta))\\
      N_2(\sigma^{r}(\beta)) & N_2(\sigma^{r+1}(\beta)) & \dots & N_2(\sigma^{r+\delta-2}(\beta))\\
      \vdots & \vdots & \ddots & \vdots \\
      N_{n-1}(\sigma^{r}(\beta)) & N_{n-1}(\sigma^{r+1}(\beta)) & \dots & N_{n-1}(\sigma^{r+\delta-2}(\beta))\\
    \end{pmatrix}
  \]
  puesto que las columnas son la evaluaciones por la izquierda en las raíces.
  Por el corolario \ref{cor:peso-minimo-columnas-dependientes} tenemos que comprobar que todo menor de \(H\) de tamaño \(\delta - 1\) es distinto de cero.
  Recordemos que \(N_i(\sigma^{k}(\beta)) = \sigma^k(N_i(\beta)) = \sigma^{k}(\alpha^{-1})\sigma^{i+k}(\alpha)\) para enteros \(i\) y \(k\).
  Por tanto, dada una submatriz de orden \(\delta - 1\),
  \[
    M = \begin{pmatrix}
      N_{k_1}(\sigma^{r}(\beta)) & N_{k_1}(\sigma^{r+1}(\beta)) & \dots & N_{k_1}(\sigma^{r+\delta-2}(\beta))\\
      N_{k_2}(\sigma^{r}(\beta)) & N_{k_2}(\sigma^{r+1}(\beta)) & \dots & N_{k_2}(\sigma^{r+\delta-2}(\beta))\\
      N_{k_3}(\sigma^{r}(\beta)) & N_{k_3}(\sigma^{r+1}(\beta)) & \dots & N_{k_3}(\sigma^{r+\delta-2}(\beta))\\
      \vdots & \vdots & \ddots & \vdots \\
      N_{k_{\delta - 1}}(\sigma^{r}(\beta)) & N_{k_{\delta - 1}}(\sigma^{r+1}(\beta)) & \dots & N_{k_{\delta - 1}}(\sigma^{r+\delta-2}(\beta))\\
    \end{pmatrix}
  \]
  con \(\{k_1 < k_2 < \dots < k_{\delta - 1}\} \subset \{0, 1, \dots, n - 1\}\),
  el determinante \(|M| = 0\) si y solo si \(|M'| = 0\), siendo \(M'\) la matriz 
  \begin{align*}
    &\phantom{={}} \begin{pmatrix}
      \sigma^{k_1 + r}(\alpha) & \sigma^{k_1 + r + 1}(\alpha) & \dots & \sigma^{k_1 + r + \delta - 2}(\alpha)\\ 
      \sigma^{k_2 + r}(\alpha) & \sigma^{k_2 + r + 1}(\alpha) & \dots & \sigma^{k_2 + r + \delta - 2}(\alpha)\\ 
      \sigma^{k_3 + r}(\alpha) & \sigma^{k_3 + r + 1}(\alpha) & \dots & \sigma^{k_3 + r + \delta - 2}(\alpha)\\ 
      \vdots & \vdots & \ddots & \vdots\\ 
      \sigma^{k_{\delta - 1} + r}(\alpha) & \sigma^{k_{\delta - 1} + r + 1}(\alpha) & \dots & \sigma^{k_{\delta - 1} + r + \delta - 2}(\alpha)\\ 
     \end{pmatrix}\\
     &= \begin{pmatrix}
      \sigma^{k_1 + r}(\alpha) & \sigma(\sigma^{k_1 + r}(\alpha)) & \dots & \sigma^{\delta - 2}(\sigma^{k_1 + r}(\alpha))\\ 
      \sigma^{k_2 + r}(\alpha) & \sigma(\sigma^{k_2 + r}(\alpha)) & \dots & \sigma^{\delta - 2}(\sigma^{k_2 + r}(\alpha))\\ 
      \sigma^{k_3 + r}(\alpha) & \sigma(\sigma^{k_3 + r}(\alpha)) & \dots & \sigma^{\delta - 2}(\sigma^{k_3 + r}(\alpha))\\ 
      \vdots & \vdots & \ddots & \vdots\\ 
      \sigma^{k_v + r}(\alpha) & \sigma(\sigma^{k_v + r}(\alpha)) & \dots & \sigma^{\delta - 2}(\sigma^{k_v + r}(\alpha))\\ 
     \end{pmatrix}.
  \end{align*}
  Como \(\{\alpha, \sigma(\alpha), \dots, \sigma^{n-1}(\alpha)\}\) es una base de \(\mathbb F_q\) como espacio vectorial del subcuerpo fijo \(\mathbb F^{\sigma}\), por \parencite[Lema 2.1]{gomez-torrecillas_petersongorensteinzierler_2018} \(|M'| \neq 0\).
\end{proof}